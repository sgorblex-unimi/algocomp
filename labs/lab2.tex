\chapter{Laboratorio 2: il problema dello zaino}


\begin{minted}[mathescape, fontsize=\small, xleftmargin=0.5em]{python}
import numpy as np
import random
\end{minted}


Definiamo una funzione helper per creare istanze arbitrarie del problema: 


\begin{minted}[mathescape, fontsize=\small, xleftmargin=0.5em]{python}
# n is the number of objects to generate
# maxv is the maximum value
def generateInstance(n, maxv = 100):
    w = random.sample(range(1, maxv), n)
    v = random.sample(range(1, maxv), n)
    wBound = max(int(random.random() * maxv), max(w))
    return (list(zip(w, v)), wBound)
\end{minted}


Definiamo quindi il primo metodo basato sulla matrice dei valori $vOPT$. 

\begin{minted}[mathescape, fontsize=\small, xleftmargin=0.5em]{python}
# wv is a list of pairs (w,v)
# wBound is the capacity of the knapsack
# returns a pair (I, v) where v is the optimal value of a solution
# and I is the solution (set of indices)
def knapsackVopt(wv, wBound):
    n = len(wv)
    vOpt = np.zeros((n + 1, wBound + 1), int)
    for i in range(1, n + 1):
        vOpt[i][0] = 0
        for w in range(1, wBound + 1):
            currentItemWv = wv[i-1]
            currentW = currentItemWv[0]
            currentV = currentItemWv[1]
            if w >= currentW:
                vOpt[i][w] = max(vOpt[i-1][w], vOpt[i-1][w-currentW]+currentV)
            else:
                vOpt[i][w] = vOpt[i-1][w]
    I = []
    i = n
    w = wBound
    while i > 0:
        if vOpt[i-1,w] != vOpt[i, w]:
            I.append(i-1)
            w -= wv[i-1][0]
        i -= 1
    return (I,vOpt[n][wBound])
\end{minted}



\begin{minted}[mathescape, fontsize=\small, xleftmargin=0.5em]{python}
a = generateInstance(5, 10)
print(a)
print(knapsackVopt(*a))
\end{minted}
\begin{minted}[fontsize=\small, xleftmargin=0.5em, mathescape, frame = leftline]{text}
([(3, 4), (9, 8), (4, 9), (2, 6), (1, 3)], 9)
([3, 2, 0], 19)
\end{minted}


Definiamo il secondo metodo, basato sulla matrice di pesi $wOPT$. 

\begin{minted}[mathescape, fontsize=\small, xleftmargin=0.5em]{python}
# wv is a list of pairs (w,v)
# wBound is the capacity of the knapsack
# returns a pair (I, v) where v is the optimal value of a solution
# and I is the solution (set of indices)
def knapsackWopt(wv, wBound):
    n = len(wv)
    v = [wv[i][1] for i in range(n)]
    w = [wv[i][0] for i in range(n)]
    vMax = max(v)
    nvMax = n * vMax
    # Dynamic programming matrix
    wOpt = np.zeros((n + 1, nvMax + 1))
    print("Dimensione tabella: ", wOpt.nbytes)
    # Initialization
    for a in range(1, nvMax + 1):
        wOpt[0][a] = float('inf')
    # Filling
    for i in range(1, n + 1):
        wOpt[i][0] = 0
        for a in range(1, nvMax + 1):
            wOpt[i][a] = min(
                    wOpt[i - 1][a], 
                    wOpt[i - 1][max(a - v[i - 1], 0)] + w[i - 1])
    # Find the solution value a (on the last row)
    for a in range(nvMax, 0, -1):
        if wOpt[n][a] <= wBound:
            break
    # Reconstruct the solution
    finalV = a
    I = []
    i = n
    b = a
    while i > 0:
        if wOpt[i-1][b] != wOpt[i][b]:
            I.append(i-1)
            b -= v[i-1]
        i -= 1
    return (I,finalV)
\end{minted}


L'algoritmo $\mathbf{FPTAS}$ utilizza proprio quest'ultimo algoritmo appena definito: 

\begin{minted}[mathescape, fontsize=\small, xleftmargin=0.5em]{python}
from math import ceil
def knapsackFPTAS(wv, wBound, epsilon):
    n = len(wv)
    v = [wv[i][1] for i in range(n)]
    w = [wv[i][0] for i in range(n)]
    vMax = max(v)
    theta = max(epsilon * vMax / (2 * n), 1.0)
    vHat = [int(ceil(v[i] / theta)) for i in range(n)]
    print("Prima dell'arrotondamento: ", wv, wBound)
    print("Dopo l'arrotondamento: ", list(zip(w, vHat)), wBound)
    I, opt = knapsackWopt(list(zip(w, vHat)), wBound)
    vOpt = sum([v[i] for i in I])
    return I, vOpt
\end{minted}



\begin{minted}[mathescape, fontsize=\small, xleftmargin=0.5em]{python}
a = generateInstance(15, 10000)
Iexact, vexact = knapsackWopt(*a)
I, v = knapsackFPTAS(*a, 0.3)
print(Iexact, vexact)
print(I, v)
\end{minted}
\begin{minted}[fontsize=\small, xleftmargin=0.5em, mathescape, frame = leftline]{text}
Dimensione tabella:  17850368
Prima dell'arrotondamento:  [(1042, 3555), (9804, 9297), (376, 2780),
(2421, 277), (6697, 5289), (1202, 4197), (7569, 1164), (1896, 1082),
(1183, 7622), (1546, 1473), (7089, 6509), (3244, 4382), (8391, 1358),
(1373, 7811), (1454, 8170)] 9804
Dopo l'arrotondamento:  [(1042, 39), (9804, 100), (376, 30), (2421,
3), (6697, 57), (1202, 46), (7569, 13), (1896, 12), (1183, 82), (1546,
16), (7089, 71), (3244, 48), (8391, 15), (1373, 85), (1454, 88)] 9804
Dimensione tabella:  192128
[14, 13, 11, 8, 5, 0] 35737
[14, 13, 11, 8, 5, 0] 35737
\end{minted}

