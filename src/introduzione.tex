\chapter{Introduzione}
Questo corso esplora alcune tra le classi di complessità non trattate nei corsi di base di algoritmi;
tratteremo gli algoritmi di ottimizzazione, gli algoritmi randomici e le strutture succinte.
L'obiettivo è studiare delle aree importanti per gli informatici, presentando ``oggetti'' alla base delle
tecniche informatiche. Iniziamo con un'introduzione alle notazioni matematiche utilizzate.

\section{Notazione matematica}
\subsection{Insiemi}
Useremo insiemi numerici come i numeri naturali $\mathbb{N}$,
i numeri interi $\mathbb{Z}$ e i numeri reali $\mathbb{R}$ e le rispettive
``versioni positive'' $\mathbb{N}^+$,  $\mathbb{Z}^+$ e $\mathbb{R}^+$.

\subsection{Monoidi e stringhe}
Avremo spesso a che fare con i {\bf monoidi liberi}. Essi sono delle strutture
algebriche che rispettano gli assiomi di {\bf chiusura} (un monoide è un \textit{gruppoide}),
di {\bf associatività} (un monoide è un \textit{semigruppo}) e di esistenza dell'elemento neutro.
Per ciò che ci interessa, ``istanzieremo'' i monoidi su degli alfabeti finiti $\Sigma$, costruendo
un monoide \textit{libero}\textit{} $\Sigma^*$ generato da $\Sigma$, ossia un insieme dotato
di un'operazione binaria associativa $\cdot$ e un elemento neutro $\epsilon$:
indicheremo il monoide con la tripla $(\Sigma^*, \cdot, \epsilon)$\footnote{
	Per un'introduzione alla relazione tra strutture algebriche e linguaggi (e la loro
	chiusura di Kleene) consultare \cite{Sakarovitch:09:automata}.}.
Data una stringa $w \in \Sigma$, possiamo indicarne la lunghezza con
$|w|$ e definiamo $w = w_0 w_1\cdots w_n$.

\subsection{Funzioni}
Dati due insiemi $A$ e $B$ si definisce
$$
	B^A = \{f | f: A \rightarrow B\}
$$
l'insieme di tutte le funzioni che hanno dominio $A$ e codominio $B$.
Se $k$ è un numero intero, si definisce, introducendo una lieve ambiguità,
$$
	k = \{0,1,\cdots,k-1\}
$$
l'insieme con cardinalità $k$ contenente i naturali da $0$ a $k-1$;
ad esempio, $0 = \emptyset$, $1 = \{0\}$ e così via.

Risulta quindi che $2^*$ è il monoide libero basato su tutte le stringhe binarie -
solitamente equipaggiato con $\cdot$ come operazione di concatenazione e $\epsilon$ la
stringa vuota. Definiamo
$$
	2^A = \{f | f: A \rightarrow \{0,1\}\}
$$
Possiamo interpretare i valori delle immagini di queste funzioni come dei booleani
che descrivono l'appartenenza ad $A$: per esempio, se
$$
	\forall a \in A ~~ f_A (a) = 1
$$
$f_A$ è la \textit{funzione caratteristica} dell'insieme $A$. Allargando il
ragionamento a tutte le $f \in 2^A$, possiamo definire quest'ultimo come l'insieme
delle funzioni caratteristiche di tutti i possibili sottoinsiemi di $A$.

Seguendo la definizione, abbiamo inoltre che
$$
	A^2 = \{f| f: \{0,1\} \rightarrow A\}
$$
ogni funzione associa a $0$ un elemento di $A$ e a $1$ un elemento di $A$ - a meno di isomorfismi,
l'insieme delle immagini delle funzioni in $A^2$ è $A \times A$.
Come ultimo esempio, abbiamo che $2^{2^*}$ è l'insieme di tutti i linguaggi
binari.
