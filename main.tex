\documentclass[12pt, a4paper,titlepage]{book}
\usepackage[a4paper,left=2cm,right=2cm,top=2.5cm,bottom=2.5cm]{geometry}
\usepackage[italian]{babel}
\usepackage[
    type={CC},
    modifier={by-nc-sa},
    version={3.0},
]{doclicense}
\usepackage{
	wrapfig,
	graphicx,
	titlesec,
	fancyhdr,
	tikz,
	pgfplots,
	amsmath,
	amsthm,
	subcaption,
	multirow,
	listings,
	fancybox,
	xcolor,
	environ,
	amssymb,
	minted
}

% -- Font settings
\usepackage[math-style=ISO]{unicode-math}
\setmainfont{EB Garamond}%You should have installed the font
\setmathfont{Garamond-Math.otf}[StylisticSet={7,9}]

\DeclareMathAlphabet{\mathcal}{OMS}{cmsy}{m}{n}
\setminted[python]{mathescape,
	linenos,
	numbersep=5pt,
	frame=lines,
	framesep=2mm}
\usetikzlibrary{shapes.geometric}
\usetikzlibrary{calc}
\usetikzlibrary{patterns}
\usetikzlibrary{arrows.meta}
\usetikzlibrary{positioning, shapes.geometric}
\usepackage[italiano,linesnumbered,ruled,vlined]{algorithm2e}
\DontPrintSemicolon
\usepackage{tkz-euclide}
\tikzset{%
	pics/sema/.style args={#1/#2/#3}{code={%
					\ifstrequal{#2}{0}{%
						\node[circle,minimum width=2mm,draw,fill=#1] {};
					}{%
						\tkzDefPoint(0,0){O}
						\tkzDrawSector[R,fill=#1](O,2mm)(90,90-#2)
						\tkzDrawSector[R,fill=#3](O,2mm)(90-#2,90-360)
					}
				}},
}
\SetKwInput{KwInput}{Input}
\SetKwInput{KwOutput}{Output}

\usepackage[utf8]{inputenc}
\usepackage[debugshow]{tabularx}
\usepackage[hidelinks, linktoc=all]{hyperref}
\usepackage[capitalize, italian]{cleveref}
\usetikzlibrary{matrix,chains,positioning,decorations.pathreplacing,arrows,angles,quotes}
\let\oldforall\forall
\renewcommand{\forall}{~~\oldforall}
\usepgfplotslibrary{fillbetween}
\pgfplotsset{compat=1.17}
\newcommand{\pgfenv}{
	\pgfplotsset{
		standard/.style={%Axis format configuration
				axis x line=middle,
				axis y line=middle,
				enlarge x limits=0.15,
				enlarge y limits=0.15,
				every axis x label/.style={at={(current axis.right of origin)},anchor=north west},
				every axis y label/.style={at={(current axis.above origin)},anchor=north east},
				every axis plot post/.style={mark options={fill=white}}
			}
	}
}

\setlength{\headheight}{16pt}
\setcounter{tocdepth}{2}
\setcounter{secnumdepth}{2}


% Chapter Titling: Chapter [0-9] LEFT, Chapter Title RIGHT
\newcommand*{\justifyheading}{\raggedleft}
\titleformat{\chapter}[display]
{\normalfont\large}{\MakeUppercase\chaptertitlename \ \ \thechapter}
{20pt}{\Huge\bfseries\justifyheading}

% -- Definizione di problemi
\newcommand{\prob}[4]{
	\begin{tabular}{ll}
		\multicolumn{2}{l}{\textsc{#1}} \\
		\textbf{Input}:    & #2         \\
		\textbf{Output}:   & #3         \\
		\textbf{Problema}: & #4
	\end{tabular}
}


\newcommand{\popt}[7]{
	\begin{tabular}{ll}
		\multicolumn{2}{l}{\textsc{#1}} \\
		\textbf{Input}:       & #2      \\
		\textbf{Output}:      & #3      \\
		\textbf{Problema}:    & #4      \\
		\textbf{Ammissibili}: & #5      \\
		\textbf{Tipo}:        & #6      \\
		\textbf{Costo}:       & #7
	\end{tabular}
}

% -- Teoremi, lemmi, corollari, osservazioni
\newtheorem{theorem}{Teorema}
\numberwithin{theorem}{chapter}
\newtheorem{lemma}[theorem]{Lemma}
\newtheorem{corollario}[theorem]{Corollario}
\newtheorem{oss}[theorem]{Osservazione}

\makeatletter
\newenvironment{CenteredBox}{% 
	\begin{Sbox}}{% Save the content in a box
	\end{Sbox}\centerline{\parbox{\wd\@Sbox}{\TheSbox}}}% And output it centered
\makeatother

\begin{document}
% Forewords + TOC Page header Style
% pageNumber -- Chapter Title -------- | ------- Chapter Title -- pageNumber
\pagestyle{fancy}
\renewcommand{\headrulewidth}{0pt} % to remove line on header
\renewcommand{\footrulewidth}{0pt} % to remove line on footer
\renewcommand{\chaptermark}[1]{\markboth{#1}{}}
\fancyhead[LE]{\thepage \ \ }
\fancyhead[RO]{\MakeUppercase\leftmark \ \ \thepage}
\fancyfoot[C] {\thepage}


\frontmatter
\begin{titlepage}
	\centering
	\vspace*{4cm}
	\Huge{Algoritmi e complessità}

	\vspace*{0.5cm}

	\Large{Appunti delle lezioni tenute dal Prof. Paolo Boldi}

	\vspace*{2.5cm}

	\Large{Edoardo Marangoni}

	\vspace*{2cm}

	\small{Università Statale di Milano}

	\small{Dipartimento di Informatica}

	\small{\today}

	\vspace*{1cm}
	\begin{figure}[h]
	    \centering
		\begin{tikzpicture}[thick]
			\foreach \i in {-30,-28,...,0}{

					\node[draw,
						isosceles triangle,
						isosceles triangle apex angle=60,
						minimum size=-2*\i mm,
						rotate=\i,inner sep =0pt] at (0,0){};
				}

		\end{tikzpicture}
	\end{figure}
%	\includegraphics[width=0.2\textwidth]{images/unimi.png}
\end{titlepage}


\vspace*{\fill}
% most of the used packages use LPPL licenses; some others use MIT and 
% GPL. For now, CC-BY-NC-SA is ok, however we shall use GPL or GFDL.
\doclicenseThis
\vspace*{\fill}

\tableofcontents
\listoffigures
\listoftables

\mainmatter
% Corpus Header Style
% pageNumber -- ChapterTitle ----- Chapter | Chapter ------ Section -- pageNumber
\pagestyle{fancy}
\renewcommand{\headrulewidth}{0pt}
\renewcommand{\chaptermark}[1]{\markboth{#1}{}}
\fancyhf{}
\fancyhead[LE]{\thepage \ \ \MakeUppercase\leftmark}
\fancyhead[RE, LO]{\MakeUppercase\chaptertitlename \ \ \thechapter}
\fancyhead[RO]{\rightmark \ \ \thepage}
\fancyfoot[C]{\thepage}


% TEX root = ../main.tex

% lezione 1 - 29/09/2021
\chapter{Introduzione}
Questo corso esplora alcune tra le classi di complessità non trattate nei corsi di base di algoritmi;
tratteremo gli algoritmi di ottimizzazione, gli algoritmi randomici e le strutture succinte.
L'obiettivo è studiare delle aree importanti per gli informatici, presentando ``oggetti'' alla base delle
tecniche informatiche. Iniziamo con un'introduzione alle notazioni matematiche utilizzate.

\section{Notazione matematica}
\subsection{Insiemi}
Useremo insiemi numerici come i numeri naturali $\mathbb{N}$,
i numeri interi $\mathbb{Z}$ e i numeri reali $\mathbb{R}$ e le rispettive
``versioni positive'' $\mathbb{N}^+$,  $\mathbb{Z}^+$ e $\mathbb{R}^+$.

\subsection{Monoidi e stringhe}
Avremo spesso a che fare con i {\bf monoidi liberi}. Essi sono delle strutture
algebriche che rispettano gli assiomi di {\bf chiusura} (un monoide è un \textit{gruppoide}),
di {\bf associatività} (un monoide è un \textit{semigruppo}) e di esistenza dell'elemento neutro.
Per ciò che ci interessa, ``istanzieremo'' i monoidi su degli alfabeti finiti $\Sigma$, costruendo
un monoide \textit{libero}\textit{} $\Sigma^*$ generato da $\Sigma$, ossia un insieme dotato
di un'operazione binaria associativa $\cdot$ e un elemento neutro $\epsilon$:
indicheremo il monoide con la tripla $(\Sigma^*, \cdot, \epsilon)$\footnote{
	Per un'introduzione alla relazione tra strutture algebriche e linguaggi (e la loro
	chiusura di Kleene) consultare \cite{sakarovitch_2009}.}.
Data una stringa $w \in \Sigma$, possiamo indicarne la lunghezza con
$|w|$ e definiamo $w = w_0 w_1\cdots w_n$.

\subsection{Funzioni}
Dati due insiemi $A$ e $B$ si definisce
$$
	B^A = \{f | f: A \rightarrow B\}
$$
l'insieme di tutte le funzioni che hanno dominio $A$ e codominio $B$.
Se $k$ è un numero intero, si definisce, introducendo una lieve ambiguità,
$$
	k = \{0,1,\cdots,k-1\}
$$
l'insieme con cardinalità $k$ contenente i naturali da $0$ a $k-1$;
ad esempio, $0 = \emptyset$, $1 = \{0\}$ e così via.

Risulta quindi che $2^*$ è il monoide libero basato su tutte le stringhe binarie -
solitamente equipaggiato con $\cdot$ come operazione di concatenazione e $\epsilon$ la
stringa vuota. Definiamo
$$
	2^A = \{f | f: A \rightarrow \{0,1\}\}
$$
Possiamo interpretare i valori delle immagini di queste funzioni come dei booleani
che descrivono l'appartenenza ad $A$: per esempio, se
$$
	\forall a \in A ~~ f_A (a) = 1
$$
$f_A$ è la \textit{funzione caratteristica} dell'insieme $A$. Allargando il
ragionamento a tutte le $f \in 2^A$, possiamo definire quest'ultimo come l'insieme
delle funzioni caratteristiche di tutti i possibili sottoinsiemi di $A$.

Seguendo la definizione, abbiamo inoltre che
$$
	A^2 = \{f| f: \{0,1\} \rightarrow A\}
$$
ogni funzione associa a $0$ un elemento di $A$ e a $1$ un elemento di $A$ - a meno di isomorfismi,
l'insieme delle immagini delle funzioni in $A^2$ è $A \times A$.
Come ultimo esempio, abbiamo che $2^{2^*}$ è l'insieme di tutti i linguaggi
binari.

\section{Problemi}
\subsection{Definizione formale}
Prima di definire cosa siano gli \textit{algoritmi}, è necessario definire
formalmente cosa sia un \textit{problema}; un problema $\Pi$ è definito da:
\begin{itemize}
	\setlength\itemsep{0pt}
	\item l'insieme degli input del problema $I_{\Pi}$;
	\item l'insieme degli output del problema $O_{\Pi}$; e
	\item una funzione $Sol_{\Pi}: I_{\Pi} \rightarrow \{2^{O_{\Pi}} \setminus \emptyset\}$,
	      interpretata come una funzione che associa ad ogni input i relativi
	      output corretti per il problema - in altre parole, la funzione calcola
	      un sottoinsieme non vuoto di $O_{\Pi}$ che risolve il problema per il dato input\footnote{
		      La funzione ha come codominio un insieme di funzioni, di conseguenza
		      si può vedere come una curryficazione di $Sol: I \times O \rightarrow 2$, che indica se,
		      effettivamente, un output sia valido un certo input.}.

\end{itemize}

\subsubsection{Esempi}
\paragraph{1}
\prob {NPrime} {$\mathbb{N}$} {$\{0, 1\} = 2$} {$n \in \mathbb{N}$ è primo?}

è un problema di \textbf{decisione}.

\paragraph{2}
\prob {MCD} {$\mathbb{N}\times\mathbb{N}$} {$\mathbb{N}$}
{Trova il massimo comun divisore tra due numeri.}

\paragraph{3}
\prob {SAT} {CNF ben formate} {$\{0, 1\} = 2$} {\`E possibile soddisfare la formula in input?}

è nuovamente un problema di decisione.

\section{Rappresentazioni}
% @TODO: Riguardare questa lezione, la spiegazione qui sotto non è 
% affatto chiara. 
In modo da poter definire formalmente gli algoritmi prendendo come modello
di riferimento le macchine di Turing, assumiamo

$$
	I_{\Pi} \subseteq 2^2
$$
e
$$
	O_{\Pi} \subseteq 2^2
$$
Assumiamo di dover scrivere in binario i due numeri $3$ e $5$, ossia $11$ e $101$.
Per dare i due input ``in pasto'' alla macchina di Turing non possiamo
semplicemente concatenare le due stringhe: $11101$ sarebbe un altro numero
($29$, in particolare). Possiamo utilizzare un trucco, ossia raddoppiamo ogni
bit: $1111$, $110011$. Si nota facilmente che non compare mai la coppia $01$ o
$10$ leggendo i bit due a due; si potranno utilizzare quindi questi marker per
segnalare la fine di un numero e l'inizio di un altro.

Avremo spesso a che fare con input più complicati
(si pensi al \textsc{TSP}: matrici di incidenza, liste di adiacenza, ...);
se ci sono molti modi diversi per \textbf{codificare} l'input, parlare informalmente
dei problemi causa problemi nel definire (e implementare, ovviamente)
gli algoritmi, addirittura arrivando a cambiare la complessità dell'algoritmo
in base alla codifica utilizzata.

Per il livello di dettaglio al quale noi vogliamo scendere nello studio della
complessità, sarà sufficiente non dare molto peso in termini di differenza di complessità
alle rappresentazioni, introducendo una leggera imprecisione. In termini pratici,
come \textit{regola del pollice}, si può notare che la distanza indotta,
in termini di complessità, dal cambio di rappresentazione è polinomialmente
limitata: si prenda l'esempio della rappresentazione a matrici di incidenza
per i grafi sparsi; nonostante essi siano \textit{meno efficienti} delle liste
di adiacenza, ci si accorge facilmente che la distanza è limitata polinomialmente.

Tuttavia, non è sempre questo il caso: si prenda per esempio la rappresentazione
binaria di un numero, e.g. \texttt{10100}, e la sua rappresentazione unaria
\texttt{000000000000000000001}. \`E chiaro che il rapporto tra le due
rappresentazioni non sia polinomiale: il confine tra polinomiale e ``probabilmente
non polinomiale'' contiene dei problemi che hanno una complessità esponenziale
se l'input è in rappresentazione binaria e ``diventano'' polinomiali se l'input
è unario.

Gonfiando artificialmente l'input, il costo in tempo dell'algoritmo - che è
rappresentato in termini della lunghezza dell'input - necessariamente decresce.
In questo senso, se l'algoritmo è polinomiale nel \textit{valore} dell'input
ma non necessariamente nella sua lunghezza, esso è detto \textbf{pseudo-polinomiale}.

\section {Algoritmi}
Un algoritmo $A$ per un problema $\Pi$ è una funzione
$$
	A: I_{\Pi} \rightarrow O_{\Pi}
$$
O, localmente,
$$
	x \mapsto y
$$
tale che  $y \in Sol_{\Pi}(x)$ (o, alternativamente, $Sol_{\Pi}(x)(y) = 1$),
ossia una soluzione \textit{corretta} per $x$.

In termini formali, un algoritmo rappresenta una \textit{macchina di Turing}.
Tuttavia, non scenderemo mai ad un livello di dettaglio tale per cui dovremo descrivere,
effettivamente, un programma in termini di MdT, che richiederebbe uno sforzo
non indifferente; utilizzeremo invece una notazione relativamente informale
basata sullo \textit{pseudocodice}.

Ciò che ci interessa degli algoritmi è studiare la loro \textbf{complessità}.
Quando si parla di complessità, si possono adottare due accezioni:
complessità \textbf{algoritmica} e complessità \textbf{strutturale}.

\subsection{Complessità algoritmica}
Chiedersi se un determinato problema $\Pi$ può essere risolto
non è banale, affatto (basta seguire un qualsiasi corso di informatica teorica
per rendersene conto) - e anche per una semplice argomentazione di cardinalità\footnote{
	Tra tanti, due testi che trattano la calcolabilità sono \cite{hopcroft_1979} e
	\cite{kfoury_1982}.
}
ci possiamo rendere conto che esiste una quantità più che numerabile di problemi
che può essere risolta da un numero numerabile di algoritmi. La teoria della
calcolabilità è l'area che si occupa di questi problemi.

Per quanto riguarda il nostro corso, ci terremo dall'altra parte della barricata,
ossia tratteremo solo problemi che sappiamo essere calcolabili, ossia problemi
$\Pi$ per i quali esiste un algoritmo $A$ che lo risolve.
Ma non tutti gli algoritmi sono equivalenti: ci interessa
infatti studiare ``quanto costa'' un algoritmo, e dovremo di conseguenza
adottare una misura di costo (spazio sul nastro, istruzioni eseguite,...)

\subsubsection{Costo}
Definiamo quindi una funzione di costo:
$$
	T_A : I_{\Pi} \rightarrow \mathbb{N}
$$
che dipende da ciò che vogliamo calcolare; così com'è, però, è difficile da
utilizzare con l'obiettivo di confrontare due algoritmi. \`E preferibile
lavorare per ``taglia'', ossia per dimensione dell'input;
definiamo quindi una funzione
$$
	t_A:\mathbb{N} \rightarrow \mathbb{N} ~~~ \text{dove} ~~~ t_A(n) = max\{T_A(x)|x \in I_{\Pi} \land |x| = n\}
$$
che è chiamata \textit{semplificazione del caso peggiore}; chiaramente, si ha
che $t_A$ è una valutazione pessimista del costo: per esempio, se $t_A(100) = 7500$
significa che su input di grandezza $100$ il costo \textit{massimo} è $7500$.
La complessità \textbf{algoritmica} utilizza proprio queste funzioni
per confrontare due algoritmi. Dati $A_1$ e $A_2$ possiamo disegnare $t_{A_1}$
e $t_{A_2}$ come in figura \ref{fig:t1t2algcomp}.


\begin{figure}[ht]
	\centering
	\begin{tikzpicture}
		\begin{axis}[
				axis lines = middle,
				xlabel = {$n$},
				ylabel = {$T$},
				yticklabels={,,},
				xticklabels={,,}
			]

			\addplot [name path = A,
				-,
				domain = 0:4.5,
				samples = 1000] {x^2}
			node [near end, right] {$t_{A_1}$};

			\addplot [name path = B,
				-,
				domain = 0:4.5] {x^(1/2)}
			node [very near end, above] {$t_{A_2}$};

		\end{axis}
	\end{tikzpicture}
	\caption{Complessità algoritmica semplificata nel caso peggiore per due funzioni $t_{A_1}$ e $t_{A_2}$.}
	\label{fig:t1t2algcomp}
\end{figure}

A questo punto, possiamo interessarci alle fasce di grandezza e capire, in un
certo range, quale algoritmo scegliere, oppure fare un'assunzione asintotica
scegliendo l'algoritmo che asintoticamente cresce di meno.

\paragraph{Upper e lower bound, ottimalità}
Supponiamo di aver trovato un algoritmo $A$ la cui complessità è $O(n^{2.37})$.
Questo valore è un \textit{upper bound}. Siamo sicuri di non poter fare di meglio?
Raramente si è certi: qui interviene un tipo di ragionamento completamente
diverso, il cui obiettivo è dimostrare che più di tanto per un certo problema
non si può fare, dimostrando quindi dei \textit{lower bound}. In questo contesto
si cercano dimostrazioni, non algoritmi. Trovando, per esempio, che il lower
bound teorico per il problema è $\Omega(n^2)$, non si sanulla del range tra
$n^2$ e $n^{2.37}$ e, idealmente, si continuerà a cercare un algoritmo finché si arriva ad un
algoritmo $O(n^2)$, che è (asintoticamente) ottimale. Pochissimi problemi
godono di un algoritmo ottimale: uno dei pochi è l'ordinamento di array, che ha
complessità ottimale $\Theta(n\cdot log(n))$. Tuttavia, questo non significa
che \textit{heapsort}, per esempio, sia l'algoritmo \textit{migliore} in toto:
la pratica spesso smentisce queste possibilità. Un esempio è l'algoritmo
di Danzig, in teoria esponenziale e in pratica migliore di altri algoritmi
polinomiali.

\subsection{Complessità strutturale}
L'obiettivo finale della complessità algoritmica è trovare un algoritmo
ottimo per ogni problema. Questo obiettivo è, chiaramente, quasi sempre
irraggiungibile. Immaginiamo, per un momento, di conoscere tutte le complessità
ottimali dei problemi: la complessità strutturale parte dal presupposto che
per ogni problema si possa definire la \textit{sua} complessità, in modo da
poter collocare ogni problema in una precisa \textbf{classe}.

\subsubsection{Classi di complessità}
Solitamente, la complessità strutturale si occupa esclusivamente di problemi 
di decisione, i quali sono analoghi al problema dell'appartenenza di una 
\textit{parola} ad un \textit{linguaggio}; pertanto tutti i problemi di decisione 
sono dei sottoinsiemi di $2^{2^*}$ e un sottoinsieme in particolare è 
l'insieme di \textit{tutti i problemi decidibili in tempo polinomiale} $\mathbf{P}$. 

Per molti problemi vorremmo sapere se esso appartiene o meno a 
$\mathbf{P}$ ma, al momento, non abbiamo modo di saperlo: 
un esempio è SAT. Allo scopo di arrivare ad una risposta a questa domanda, 
è stato ``inventata'' la classe di complessità $\mathbf{NP}$\footnote{Per approfondire queste 
tematiche consultare \cite{arora_2009} e \cite{complexity_zoo}.}.

\begin{figure}[ht]
	\centering
	\begin{tikzpicture}[scale=1.0]
		\draw[black,rounded corners=10,thick]
		(0,0) rectangle (10,5.5);
		\draw[black,rounded corners=10,thick]
		(4.9,5.1) node {\bf Problemi decidibili};

		\draw[black,rounded corners=10,thick]
		(0.5, 0.5) rectangle (9.5,4.7);
		\draw[black,rounded corners=10,thick]
		(5,4.4) node {$\mathbf{\mathbf{NP}}$};

		\draw[black,rounded corners=10,thick]
		(1,1) rectangle (5,4);
		\draw[black,rounded corners=10,thick]
		(3,2.5) node {$\mathbf{\mathbf{P}}$};

		\draw[black,rounded corners=10,thick]
		(5.1,1) rectangle (9,4);
		\draw[black,rounded corners=10,thick]
		(7,3) node {\bf $\mathbf{NP-completi}$};
		\draw[black,rounded corners=10,thick]
		(7,2) node {$SAT$};

	\end{tikzpicture}
	\caption{Classi di complessità strutturale.}
	\label{fig:structcomplclass}
\end{figure}

\subsubsection{Riducibilità}
Il concetto di \textbf{riducibilità} (in tempo polinomiale) gioca una parte
fondamentale nella teoria della complessità strutturale: si dice che un problema $\Pi_1$
è polinomialmente riducibile ad un problema $\Pi_2$ se e solo se
$$
	\exists f: 2^* \rightarrow 2^*
$$
tale che:
\begin{itemize}
    \setlength\itemsep{0pt}
	\item $f$ è calcolabile polinomialmente
	\item $\forall x \in I_{\Pi_1}  ~~ f(x) \in I_{\Pi_2}$
	\item $\forall x ~~ Sol_{\Pi_2}(x) = 1 \iff Sol_{\Pi_1}(x) = 1$
\end{itemize}

Questa definizione ha come conseguenza il seguente lemma.
\begin{lemma}\label{lem:poli_red}
	Se $\Pi_2 \in \mathbf{P}$ e $ \Pi_1 \leq_{p} \Pi_2$,
	ossia $\Pi_1$ è riducibile polinomialmente a $\Pi_2$, allora $\Pi_1 \in \mathbf{P}$.
\end{lemma}
La classe dei problemi $\mathbf{NP-completi}$ è quindi definita come
$$
	\mathbf{NP-completi} = \{\Pi \in \mathbf{NP} | \forall \Pi'\in \mathbf{NP} ~~ \Pi' \leq_{p} \Pi\}
$$
\begin{theorem}[di Cook]
	$SAT \in \mathbf{NP-completi}$.
\end{theorem}


%%% Local Variables:
%%% TeX-master: "../main"
%%% End:

%  TEX root = ../main.tex
% lezione 2 - 01/10/2021
\chapter{Problemi di ottimizzazione}
\section {Introduzione}
Un problema di \textbf{ottimizzazione} è caratterizzato da: 
\begin{enumerate}
  \item l'insieme degli input $I_{\Pi}$;
  \item l'insieme degli output $O_{\Pi}$;
  \item  una funzione che ad ogni input associa una famiglia di output:
    $Amm_{\Pi} : I_{\Pi} \rightarrow (2^{O_{\Pi}} \setminus \emptyset)$;
  \item il tipo del problema $Tipo_{\Pi} \in \{Min, Max\}$; e
  \item una funzione da una coppia input-output ai naturali
    $$
	c_{\Pi}: I_{\Pi} \times O_{\Pi} \rightarrow \mathbb{N}
    $$
    che formalizza il concetto di \textbf{costo} di una soluzione - per un problema 
    di minizzazione, l'obiettivo sarà scegliere l'output con costo minore. 

\end{enumerate}
Le proprietà $3$, $4$ e $5$ formalizzano ulteriormente l'insieme $Sol_{\Pi}$ definito in precedenza. 
\subsection{Esempi}
\subsubsection{MaxSat}

\popt{MaxSAT}
{CNF ben formate}
{$\mathbb{N}$} 
{Qual è il numero massimo di clausole che si possono verificare?} 
{Assegnamenti di valori di verità coerenti} 
{$Max$}
{Numero di clausole rese vere}

\noindent
Le istanze di questo problema sono formule ben formate in forma normale congiunta
(ossia CNF); le soluzioni ammissibili sono assegnamenti di valori di verità; il
costo (o \textit{funzione obiettivo}) è il numero di clausole rese vere.
\textsc{MaxSat} ha chiaramente $Tipo_{\Pi} = Max$, in quanto l'obiettivo è
massimizzare il numero di clausole verificate.

In alcuni frangenti potrebbe causarsi una certa ambiguità: l'algoritmo cerca
\textit{il valore} della soluzione ottimale o \textit{la soluzione} ottimale
stessa? Nel nostro corso diciamo che cerchiamo la soluzione stessa, in quanto il
suo valore è calcolabile con la funzione di costo, e la indicheremo con la
notazione $y^*(x)$; inoltre, indicheremo il costo della soluzione ottimale con
$c^*(x) = c_{\pi}(y^*(x))$

\section{Rapporto di prestazioni}
Dato un input $x \in I_{\Pi}$ e $y \in Amm_{\Pi}$, possiamo sempre affermare che 
$$
\begin{cases}
  c_{\Pi}(x,y) \geq c_{\Pi}(x,y^*(x)) = c_{\Pi}^*(x) & \text{ per i problemi di minimo} \\
  c_{\Pi}(x,y) \leq c_{\Pi}(x,y^*(x)) = c_{\Pi}^*(x) & \text{ per i problemi di massimo}
\end{cases}
$$

\subsection{Rapporto di approssimazione}
Definiamo \textbf{rapporto di approssimazione} la quantità
$$
R_{\Pi}(x,y) = \max\{\frac{c_{\Pi}(x,y)}{c_{\Pi}(x, y^*(x))}, \frac{c_{\Pi}(x,y^*(x))}{c_{\Pi}(x, y)}\}
$$
questo valore ci permette di ``dimenticare'' se stiamo trattando un problema di 
minimizzazione o massimizzazione, in quanto sarà sempre $R_{\Pi} \geq 1$.

\subsubsection{$\alpha$-approssimazione}
Se, per esempio, $R_{\Pi} = 1$, la soluzione $y$ è in realtà $y = y^*(x)$; se
$R_{\Pi} = 2$, per un problema di minimo significa che il costo di $y$ è il
doppio del costo di $y^*(x)$, mentre per un problema di massimo significa che il
costo di $y$ è la metà del costo di $y^*(x)$. In generale, dato un problema di
approssimazione tenteremo di progettare un algoritmo che preso un input $x \in
I_{\Pi}$ produca un output $y(x) \in Amm_{\Pi}$: se si riesce a dimostrare che
l'algoritmo costruito trova una soluzione che, per ogni input $x$, è tale per
cui $R(x,y(x)) \leq \alpha$ si definisce l'algoritmo $\alpha$-approssimato.

\section{Classi di complessità per l'ottimizzazione}
Considereremo sempre algoritmi che terminano in tempo polinomiale; ovviamente, 
vorremmo trovare un $\alpha$ il più piccolo possibile - l'obiettivo quindi non sarà
più migliorare il polinomio, bensì migliorare il grado di approssimazione $\alpha$
trovando il più piccolo possibile. 

La classe dei problemi approssimabili in modo esatto ($\alpha = 1$) 
in tempo polinomiale è chiamata $\mathbf{PO}$; si noti che non è una classe 
ristretta ai problemi di decisione - questa è infatti l'analogo della 
classe $\mathbf{P}$ rispetto ai problemi di ottimizzazione. 
Allo stesso modo possiamo definire una classe dei problemi 
di ottimizzazione risolvibili con approssimazione $1$ in tempo nondetermistico 
polinomiale: $\mathbf{NPO}$ è la classe definita dai 
problemi $\Pi = (I_{\Pi}, O_{\Pi}, Amm_{\Pi}, c_{\Pi})$ tali per cui
\begin{enumerate}
  \item esiste un polinomio $Q$ tale che $\forall x \in I_{\Pi} \forall y \in Amm_{\Pi}(x) ~~ |y| \leq Q(|x|)$,
  \item dato $x \in I_{\Pi}$ e $y \in 2^*$ con $|y| \leq Q(|x|)$ è decidibile in
    tempo polinomiale se $y \in Amm_{\Pi}$, e
  \item $c_{\Pi}$ è calcolabile in tempo polinomiale
\end{enumerate}
Nonostante questa classe non sia definita in termini di MdT con modulo nondetermistico (in quanto 
questo modello è ``démodée'': la teoria della complessità moderna utilizza al loro 
posto il concetto di \textit{testimoni}) la definizione è completamente analoga e
riconducibile alle definizioni che ne fanno uso.

\subsection{Classe di problemi $\mathbf{NPO-completi}$}
Tra la classe $\mathbf{PO}$ e $\mathbf{NPO}$ sussiste la stessa relazione 
che c'è tra $\mathbf{P}$ e $\mathbf{NP}$ - effettivamente, possiamo anche definire 
i problemi $\mathbf{NPO-completi}$. Per arrivare a questa definizione occorre 
definire la nozione di \textit{problema di decisione associato}: dato un problema 
di ottimizzazione $\Pi$, definiamo un problema di decisione $\hat{\Pi}$ associato 
a $\Pi$ definendo 
$$
I_{\hat{\Pi}} = I_{\Pi} \times \mathbb{N}
$$ 
che formalizza la \textit{richiesta} ``esiste una soluzione ammissibile 
per $x$ con costo minore o uguale a $k$?'' (o maggiore o uguale per i problemi 
di massimizzazione). 

\begin{theorem}
 $$
 \Pi \in \mathbf{PO} \implies \hat{\Pi} \in \mathbf{P}
 $$
 $$
 \Pi \in \mathbf{NPO} \implies \hat{\Pi} \in \mathbf{NP}
 $$
\end{theorem}

\noindent
Analogamente, la classe dei problemi $\mathbf{NPO-completi}$ è 
$$
\mathbf{NPO-completi} = \{\Pi \in \mathbf{NPO} | \hat{\Pi} \in \mathbf{NP-completi}\}
$$
Ed è corretto aspettarsi che il problema di inclusione di $\mathbf{NP}$ in $\mathbf{P}$ 
venga mantenuto anche per i problemi di ottimizzazione:
\begin{theorem}
  Se $\Pi \in \mathbf{NPO-completi}$, allora $\Pi \notin \mathbf{PO}$ a meno che
$\mathbf{P} = \mathbf{NP}$.
\end{theorem}

\begin{proof}
Assumiamo $Tipo_{\Pi} = Max$. Per assurdo, supponiamo $\Pi \in \mathbf{PO}$.
Dato un input $(x,k) \in I_{\Pi} \times \mathbb{N}$ calcoliamo la soluzione
ottima $y^*(x)$ in tempo polinomiale usando il fatto che $\Pi \in \mathbf{PO}$.
Se $k \leq c_{\Pi}(x, y^*(x))$ rispondiamo \textit{sì}, altrimenti rispondiamo
\textit{no}. Questo algoritmo funziona in tempo polinomiale e decide il problema
di decisione associato a $\Pi$; in quanto $\hat{\Pi} \in \mathbf{NP-completi}$,
concludiamo $\mathbf{P} = \mathbf{NP}$.
\end{proof}

\subsection{Altre classi di complessità}
In base al rapporto di approssimazione e al comportamento dell'algoritmo dati
gli input e il rapporto di approssimazione stesso è possibile definire ulteriori
classi di complessità. Utilizziamo ora la notazione $A_{\Pi}$ per denotare un
algoritmo che risolve il problema $\Pi$.

\subsubsection{La classe {\bf APX} ({\it NPO-approximable})}
La classe dei problemi \textit{approssimabili} in tempo nondeterministico polinomiale:
$$
\mathbf{APX} = \{\Pi | \exists A_{\Pi}, \alpha: A_{\Pi} \text{ è } \alpha\text{-approssimante per } \Pi\}
$$
Abbiamo che $\mathbf{APX} \subsetneq \mathbf{NPO}$: vi sono infatti alcuni
problemi che non sono approssimabili. 

\subsubsection{La classe {\bf PTAS} ({\it Polynomial time approximation scheme})}
La seguente classe è parametrizzata dal valore del rapporto di approssimazione 
scelto:
$$
\mathbf{PTAS} = \{\Pi | \exists A_{\Pi},  (x, \rho) \in I_{\Pi} \times \mathbb{Q}^{\geq 1}:
A_{\Pi}(x) = y \in Amm_{\Pi}(x) \text{ in tempo } poly(x) \land R_{\Pi}(x,y) \leq \rho \} 
$$
Abbiamo che  $\mathbf{PTAS} \subsetneq \mathbf{APX}$, infatti vi sono problemi
che non possono essere approssimati al più di un certo valore. Si noti che i
problemi in $\mathbf{PTAS}$ sono risolti in tempo polinomiale
\textit{nell'input} ma non nel valore di approssimazione stesso.

\subsubsection{La classe {\bf FPTAS} ({\it Fully polynomial time approximation
scheme})}
Stringendo la restrizione di polinomialità anche sul rapporto di approssimazione 
otteniamo la seguente classe: 
$$
\mathbf{FPTAS} =  \{ \Pi | \Pi \in \mathbf{PTAS} \land A_{\Pi} \text{ termina in tempo } poly(x, \rho)\}
$$
Abbiamo che  $\mathbf{FPTAS} \subsetneq \mathbf{PTAS}$, infatti vi sono problemi 
che possono essere approssimati arbitrariamente solo utilizzando un tempo non 
polinomiale nel valore di approssimazione. 


\begin{figure}[h]
  \centering



\tikzset{every picture/.style={line width=0.75pt}} %set default line width to 0.75pt        

\begin{tikzpicture}[x=0.75pt,y=0.75pt,yscale=-1,xscale=1]
%uncomment if require: \path (0,477); %set diagram left start at 0, and has height of 477

%Shape: Ellipse [id:dp4588658322810296] 
\draw   (310.61,174.5) .. controls (376.13,87.79) and (491.77,17.5) .. (568.88,17.5) .. controls (645.98,17.5) and (655.37,87.79) .. (589.84,174.5) .. controls (524.31,261.21) and (408.68,331.5) .. (331.57,331.5) .. controls (254.46,331.5) and (245.08,261.21) .. (310.61,174.5) -- cycle ;
%Shape: Ellipse [id:dp4441062862246874] 
\draw   (327.15,184.82) .. controls (376.62,117.65) and (463.91,63.2) .. (522.11,63.2) .. controls (580.32,63.2) and (587.41,117.65) .. (537.94,184.82) .. controls (488.48,251.99) and (401.19,306.44) .. (342.98,306.44) .. controls (284.77,306.44) and (277.68,251.99) .. (327.15,184.82) -- cycle ;
%Shape: Ellipse [id:dp42532014972954124] 
\draw   (341.82,199.56) .. controls (376.94,147.86) and (438.92,105.95) .. (480.25,105.95) .. controls (521.58,105.95) and (526.61,147.86) .. (491.49,199.56) .. controls (456.37,251.26) and (394.39,293.17) .. (353.06,293.17) .. controls (311.73,293.17) and (306.7,251.26) .. (341.82,199.56) -- cycle ;
%Shape: Ellipse [id:dp9403889034822421] 
\draw   (350.81,210.25) .. controls (376.75,173.81) and (422.52,144.28) .. (453.04,144.28) .. controls (483.57,144.28) and (487.28,173.81) .. (461.34,210.25) .. controls (435.4,246.68) and (389.63,276.22) .. (359.11,276.22) .. controls (328.58,276.22) and (324.87,246.68) .. (350.81,210.25) -- cycle ;
%Shape: Ellipse [id:dp9056149450292196] 
\draw   (367.2,215.78) .. controls (380.43,196.64) and (406.86,181.13) .. (426.24,181.13) .. controls (445.62,181.13) and (450.6,196.64) .. (437.38,215.78) .. controls (424.15,234.91) and (397.71,250.42) .. (378.34,250.42) .. controls (358.96,250.42) and (353.97,234.91) .. (367.2,215.78) -- cycle ;
%Shape: Ellipse [id:dp18032210948172078] 
\draw  [dash pattern={on 4.5pt off 4.5pt}] (478.73,161.23) .. controls (518.48,135.18) and (572.44,114.06) .. (599.25,114.06) .. controls (626.06,114.06) and (615.57,135.18) .. (575.82,161.23) .. controls (536.07,187.29) and (482.11,208.41) .. (455.3,208.41) .. controls (428.48,208.41) and (438.98,187.29) .. (478.73,161.23) -- cycle ;
%Straight Lines [id:da10034962558485494] 
\draw    (638.18,160.5) -- (638.18,125.85) -- (593.29,127.26) ;
\draw [shift={(591.29,127.33)}, rotate = 358.2] [color={rgb, 255:red, 0; green, 0; blue, 0 }  ][line width=0.75]    (10.93,-3.29) .. controls (6.95,-1.4) and (3.31,-0.3) .. (0,0) .. controls (3.31,0.3) and (6.95,1.4) .. (10.93,3.29)   ;

% Text Node
\draw (574.07,41.64) node [anchor=north west][inner sep=0.75pt]   [align=left] {NPO};
% Text Node
\draw (514.9,80.38) node [anchor=north west][inner sep=0.75pt]   [align=left] {APX};
% Text Node
\draw (460.1,117.56) node [anchor=north west][inner sep=0.75pt]   [align=left] {PTAS};
% Text Node
\draw (416.67,155.6) node [anchor=north west][inner sep=0.75pt]   [align=left] {FPTAS};
% Text Node
\draw (394.14,202.63) node [anchor=north west][inner sep=0.75pt]   [align=left] {PO};
% Text Node
\draw (638.18,178) node   [align=left] {\begin{minipage}[lt]{51.43pt}\setlength\topsep{0pt}
\begin{center}
NPO-completi
\end{center}

\end{minipage}};


\end{tikzpicture}
\caption{Rappresentazione insiemistica delle classi di complessità.}
\label{fig:compsets}
\end{figure}



\section{Terminologia riguardante i problemi}
\subsection{Grafi}
I grafi non orientati sono $G=(V,E)$ (vertici e lati), dove 
$$
E \in {V\choose{2}}
$$
Il \textbf{grado} di un vertice $x$ $d(x)$ è il numero di lati incidenti su 
tale vertice. Il numero di vertici è $n = |V|$, mentre $m = |E|$. 
In un grafo non orientato un \textbf{cammino} di lunghezza $k$ 
$$
\pi = v_1, v_2, \cdots, v_k
$$
tale che $\forall v_{i} \in \pi :\exists \langle v_i, v_{i+1}\rangle \in E$. 
Un cammino senza ripetizioni di vertici è chiamato \textit{semplice}, altrimenti
è definito \textit{non semplice}. Un \textbf{circuito} è un cammino semplice
chiuso di lunghezza $\geq 3$. La \textbf{connessione} tra due vertici $x$ e $y$
è denotata $x \leadsto y$ e sussiste se esiste un cammino da $x$ a $y$ (e, di
conseguenza, da $y$ a $x$); questa nozione è inoltre una relazione di
equivalenza (totale), gode infatti di riflessività, transitività e simmetria.
Gli insiemi di vertici mutuamente connessi formano le \textbf{componenti
connesse} di un grafo.

\chapter{Algoritmi deterministici}



\section{\BiMaxMatching}\label{sec:BiMaxMatching}
Sia $G=(V,E)$ un grafo non orientato. Un \emph{matching} (matrimonio) in $G$ è una selezione $X$ di lati in $E$ tale che su nessun vertice in $V$ incide più di un lato di $X$.
\MaxMatching è il problema di trovare il matching di cardinalità massima in un dato grafo.

\BiMaxMatching è una versione di \MaxMatching su grafi bipartiti, cioè in cui i vertici sono divisi in due classi e ogni lato incide su un vertice per classe: $G=(V_1,V_2,E)$ dove $V_1\cap V_2=\emptyset$ e $E\subseteq V_1\times V_2$. $\BiMaxMatching\in\PO$.

\popt{\BiMaxMatching}
{Grafo bipartito $G=(V_1,V_2,E)$}
{$X\subseteq E$}
{Determinare il matching di cardinalità massima in $G$}
{$X\subseteq E:\forall e_1,e_2\in X,~ e_1\cap e_2\neq\emptyset\impl e_1=e_2$}
{$\MAX$}
{$\card X$}

\begin{figure}
	\centering
	\begin{tikzpicture}[every node/.style={draw,inner sep=0pt,minimum size=5pt,fill,circle},matching/.style={red,thick}]
	\node at (0,1) (a) {};
	\node at (0,2) (b) {};
	\node at (0,3) (c) {};
	\node at (0,4) (d) {};
	\node at (0,5) (e) {};

	\node at (3,2) (f) {};
	\node at (3,3) (g) {};
	\node at (3,4) (h) {};
	\node at (3,5) (i) {};

	\draw		(a) -- (i);
	\draw		(b) -- (i);
	\draw		(d) -- (i);
	\draw[matching]	(e) -- (i);
	\draw[matching]	(b) -- (f);
	\draw[matching]	(c) -- (g);
	\draw		(c) -- (h);
\end{tikzpicture}

	\caption{Esempio di grafo bipartito. I lati colorati rappresentano un possibile matching.}
	\label{fig:graphmatching}
\end{figure}

Fissato un matching $M\subseteq E$, un lato $l\in E$ si dice \emph{occupato} se $l\in M$ e \emph{libero} se $l\notin M$.
Un vertice si dice \emph{esposto} se e solo se su di esso incidono solo lati liberi. Un cammino semplice si dice \emph{aumentante} rispetto a un matching se alterna lati liberi e occupati e inizia e termina su vertici esposti.
Quando un matching ha un cammino aumentante si può fare un \flang{flip}, cioè invertire l'appartenenza al matching dei lati del cammino aumentante.

\begin{figure}
	\centering
	\begin{subfigure}[b]{0.4\textwidth}
		\centering
		\begin{tikzpicture}
	\node[draw,inner sep=0pt,minimum size=5pt,fill, circle] at (3, 1)  (a) {};
	\node[draw,inner sep=0pt,minimum size=5pt,fill, circle] at (0, 2)  (b) {};
	\node[draw,inner sep=0pt,minimum size=5pt,fill, circle] at (0, 3)  (c) {};
	\node[draw,inner sep=0pt,minimum size=5pt,fill, circle] at (3, 2)  (f) {};
	\node[draw,inner sep=0pt,minimum size=5pt,fill, circle] at (3, 3)  (g) {};
	\node[draw,inner sep=0pt,minimum size=5pt,fill, circle] at (0, 4)  (h) {};

	\draw (a) -- (b);
	\draw[red] (b) -- (f);
	\draw (f) -- (c);
	\draw[red] (c) -- (g);
	\draw (g) -- (h);
\end{tikzpicture}

		\subcaption{Prima del flip.}
	\end{subfigure}
	\begin{subfigure}[b]{0.4\textwidth}
		\centering
		\input{img/matching_aumentante_post.tikz}
		\subcaption{Dopo il flip.}
	\end{subfigure}
	\caption{Esempio di cammino aumentante in un grafo bipartito.}
	\label{fig:augpaths}
\end{figure}

Vale il seguente teorema relativo ai cammini aumentanti per matching su grafi:
\begin{theorem}
	Sia $M$ un matching per un grafo $G$. Allora:
	\begin{equation*}
		\text{Esiste un cammino aumentante per $M$} \iff \text{$M$ non è massimo per $G$.}
	\end{equation*}
\end{theorem}
\begin{proof}~
	\begin{description}
		\item[$\Rightarrow$)] Applicando un flip al cammino aumentante si aumenta il matching di $1$.
		\item[$\Leftarrow$)] Se $M$ non è massimo, sia $M'$ un matching massimo per $G$ e sia $X:=(M\setminus M')\cup(M'\setminus M)$ (la differenza simmetrica di $M$ e $M'$).
			Su ogni vertice di $G$ possono incidere al più $2$ lati di $X$ (uno per ciascuno dei due matching).
			Nel grafo indotto da $X$, in ogni circuito l'appartenenza dei lati a $M$ e $M'$ è alternata e quindi il circuito è composto dallo stesso numero di lati di $M$ e di $M'$.
			Siccome però $M'$ ha più lati di $M$, esiste almeno un cammino semplice nel grafo indotto da $X$ che ha più lati in $M'$.
			Tale cammino alterna lati di $M$ e di $M'$ ed è aumentante rispetto a $M$ in $G$.
	\end{description}
\end{proof}

% TODO: è necessario discutere di come viene effettuata la visita per FindAugmenting e quale sia la sua complessità.
\SetKwFunction{FindAugmenting}{FindAugmenting}
\SetKwFunction{Flip}{Flip}
L'algoritmo \ref{alg:BiMaxMatching} risolve \BiMaxMatching trovando l'ottimo in tempo polinomiale.
La procedura \FindAugmenting tiene traccia dei vertici esposti e fa una visita in profondità del grafo a partire da uno di essi, alternando lati liberi e occupati, identificando un cammino aumentante.
La procedura \Flip esegue un flip del matching dato.

\begin{algorithm}
	$M \asn \emptyset$\;
\While{}{
	$\pi\asn\FindAugmenting{G,M}$\;
	\If{$\pi=\bot$}{
		\Return $M$\;
	}\Else{
		$M\asn\Flip(M,\pi)$\;
	}
}

	\caption{Risoluzione polinomiale di \BiMaxMatching}
	\label{alg:BiMaxMatching}
\end{algorithm}

\begin{corollario}
	$\BiMaxMatching\in\PO$.
\end{corollario}

\PerfectMatching è il problema di decisione che si chiede se in un grafo bipartito esista un matching che coinvolge tutti i vertici.

\begin{corollario}
	$\PerfectMatching\in\P$.
\end{corollario}

% TODO: in relazione al todo precedente, spiegare perché l'algoritmo precedente non sia polinomiale su grafi generici e dire che in quel caso servono tecniche più avanzate per trovare cammini aumentanti.
Vale inoltre il seguente risultato, che non dimostriamo:
\begin{theorem}
	$\MaxMatching\in\PO$.
\end{theorem}



\section{\LoadBalancing}
Il problema \LoadBalancing (bilanciamento del carico) consiste nel dividere un insieme di task, ciascuno con la sua durata, tra le macchine di un insieme, in modo da minimizzare la durata complessiva (carico) impiegata dalla macchina più carica. Questo problema è \NPO-completo.

\popt
{\LoadBalancing}
{Durate $t_0,\dots,t_{n-1}\in\N^+$ per $n$ task, numero $m$ di macchine}
{Assegnamento di ciascun task a una macchina}
{Determinare l'assegnamento che minimizza la durata massima}
{Assegnamenti $x:n\to m$}
{$\MIN$}
{$L=\max_j L_j$, dove $L_j:=\sum_{i\in x^{-1}(j)} t_i$}


\subsection{\GreedyLoadBalancing}
Un algoritmo greedy può trovare una soluzione ammissibile per \LoadBalancing.
L'algoritmo esamina i task $t_0,t_1,\dots,t_{n-1}$ nell'ordine, assegnando ogni task alla macchina più scarica.
\GreedyLoadBalancing, se implementato con una coda con priorità che tenga conto della macchina dal carico minimo (effettuando $n$ operazioni di insert/update per i carichi di $m$ macchine), opera in tempo $O(n\log m)$.
L'algoritmo non ottiene, in generale, la soluzione ottima, ma ha la garanzia di non produrre più del doppio dell'ottimo, come dimostrato dal seguente teorema:

\begin{theorem}\label{thm:greedyloadbalancing}
	\GreedyLoadBalancing è un algoritmo 2-approssimante per \LoadBalancing.
\end{theorem}
Prima di dimostrare il teorema si consideri il seguente lemma:
\begin{lemma}\label{lem:load:ultimopasso}
	Sia $L\star$ il costo della soluzione ottima. Sia $\bar j$ tale che $L_{\bar j}=L\star$ e sia $\bar i$ l'ultimo task assegnato a $\bar j$. Allora:
	\begin{equation*}
		L_{\bar j}-t_{\bar i} \leq L\star
	\end{equation*}
\end{lemma}
\begin{proof}
	Il carico della macchina $\bar j$ prima dell'assegnamento $\bar i$ era $L_{\bar j}-t_{\bar i}$, il cui era minore di ogni altro carico per via di come l'algoritmo sceglie a chi assegnare.
	Indicato con $L\star_j$ il carico della macchina $j$ nella soluzione ottima, vale $L\star\geq\frac1m\sum_it_i$, essendo $mL\star\geq\sum_jL\star_j=\sum_it_i$.
	\begin{gather*}
		L_{\bar j}-t_{\bar i}\leq L_j \qquad\forall j \\
		\sum_j (L_{\bar j}-t_{\bar i})\leq\sum_j L_j=\sum_i t_i\\
		m(L_{\bar j}-t_{\bar i})\leq \sum_i t_i \\
		L_{\bar j}-t_{\bar i}\leq\frac1m\sum_j t_j\leq L\star
	\end{gather*}
\end{proof}
Si può ora procedere con la dimostrazione del teorema \ref{thm:greedyloadbalancing}:
\begin{proof}
	Si osservi che $L\star\geq\max_it_i$. Applicando il lemma \ref{lem:load:ultimopasso}:
	\begin{gather*}
		L=L_{\bar j}=\underbrace{L_{\bar j}-t_{\bar i}}_{\leq L\star}+\underbrace{t_{\bar i}}_{\leq L\star}\leq 2L\star \\[1ex]
		\frac L{L\star}\leq 2
	\end{gather*}
\end{proof}

\begin{corollario}
	$\LoadBalancing\in\gAPX2$.
\end{corollario}

Si dimostra che questo risultato sull'analisi di \GreedyLoadBalancing è tight:
\begin{theorem}
	Per ogni $\varepsilon>0$ esiste un input di \LoadBalancing su cui \GreedyLoadBalancing produce una soluzione $L$ tale che
	\begin{equation*}
		2-\varepsilon\leq\frac L{L\star}
	\end{equation*}
\end{theorem}
\begin{proof}
	Scegliamo un numero di macchine $m\geq\frac1\varepsilon$ e un numero di task $n=m(m-1)+1$. Questi task consistono, nell'ordine, in $n-1$ task di durata $1$ e un task da $m$. Naturalmente la soluzione ottima assegna unicamente il task da $m$ a una macchina, che risulta la più carica, quindi $L\star=m$.

	Per assegnare i task, l'algoritmo assegna un task da $1$ per ogni macchina ciclicamente finché arriva il task da $m$, che viene assegnato a una macchina di carico $m-1$ producendo un costo finale di $2m-1$. In questo caso si ha
	\begin{equation*}
		\frac L{L\star}=\frac{2m-1}{m}=2-\frac1m\geq2-\varepsilon
	\end{equation*}
\end{proof}


\subsection{\SortedGreedyBalance}
Un algoritmo \SortedGreedyBalance può ordinare i task in ordine decrescente prima di assegnarli come \GreedyLoadBalancing. Questo algoritmo ha costo in tempo di $O(n\log n+n\log m)$.

\begin{theorem}
	\SortedGreedyBalance è un algoritmo $\frac32$-approssimante per \LoadBalancing.
\end{theorem}
\begin{proof}
	Se $n\leq m$, a ciascuna macchina viene assegnato un solo task, quindi $L=L\star=\max_i t_i$. La tesi, pertanto, è ovvia.

	Se $n>m$, esiste una macchina che riceve due task.
	Il primo task assegnato a una macchina già carica è $m$.
	Poiché esistono $m$ task di durata superiore a $t_m$, anche nella soluzione ottima due di essi devono essere assegnati alla stessa macchina (principio della piccionaia), quindi vale:
	\begin{equation*}
		L\star\geq 2t_m \text.
	\end{equation*}

	Sia $\bar j$ una macchina con carico massimo al termine dell'esecuzione e $\bar i$ l'ultimo task assegnatovi.
	Se $\bar i$ è l'unico task di $\bar j$, allora la soluzione è ottima e la tesi è ovvia.
	In ogni altro caso si ha $\bar i \geq m$, da cui $t_{\bar i}\leq t_m$ e quindi $t_{\bar i}\leq t_m\leq \frac12 L\star$.
	\begin{gather*}
		L=\underbrace{L_{\bar j}-t_{\bar i}}_{\leq L\star}+\underbrace{t_{\bar i}}_{\leq \frac12 L\star}\leq L\star+\frac12 L\star=\frac32L\star
	\end{gather*}
\end{proof}

Valgono inoltre i seguenti teoremi riguardanti \LoadBalancing, che non dimostriamo.
\begin{theorem}~
	\begin{itemize}
		\item \SortedGreedyBalance è un algoritmo $\frac43$-approssimante per \LoadBalancing \cite{Graham:69:sortedgreedybalance};
		\item $\LoadBalancing\in\PTAS$;
		\item $\P\neq\NP\impl\LoadBalancing\notin\FPTAS$.
	\end{itemize}
\end{theorem}



\section{\CenterSelection}
Nel problema \CenterSelection è dato uno spazio metrico $\tuple{S,d}$.
Tra i punti si $S$ si selezionano centri $C\subseteq S$: ogni punto ha un centro di riferimento che consiste nel più vicino.
La massima distanza tra un punto di $S$ e il suo centro di riferimento è detta raggio di copertura ed è il valore che si vuole minimizzare.
\CenterSelection è un problema \NPO-completo.

\popt{\CenterSelection}
{Spazio metrico $S$ con distanza $d$, numero massimo di centri $k\in\N^+$}
{$C\subseteq S$}
{Selezionare al più $k$ centri che minimizzino il raggio di copertura}
{$C\subseteq S\mid \card C\leq k$}
{$\MIN$}
{$\rho(C) = \max_{x \in S} d(x, C)$}

Uno spazio metrico è un insieme $\Omega$ di punti su cui vale una funzione distanza $d:\Omega\times\Omega\to\R^{\geq0}$ tale che:
\begin{itemize}
	\item $\forall x\in S\quad d(x,x)=0$
	\item $\forall x,y\in S\quad d(x,y)=d(y,x)$
	\item $\forall x,y,z\in S\quad d(x,y)\leq d(x,z)+d(z,y)$ (disuguaglianza triangolare)
\end{itemize}
Sia $C\subseteq\Omega$ e $s\in\Omega$. Con abuso di notazione si indica con $d(s,C)$ la distanza minima tra il punto $s$ e i punti appartenenti a $C$, ossia $d(s,C):=\min_{c\in C} d(s,c)$.

\begin{defin}[tassellatura di Voronoi]
	Dato uno spazio metrico $\tuple{\Omega,d}$, sia $S\subseteq\Omega$ finito e $C\subseteq S$ un insieme di centri.
	Dato un punto $s\in S$ il centro di riferimento di $s$ è $\text{VC}_C(s):=\arg\min_{c\in C} d(s,c)$.
	Per ogni $c\in C$, la cella di Voronoi di centro $c$ è l'insieme $\text{VC}_C^{-1}(c)$ degli elementi di $S$ mappati in $c$ tramite $\text{VC}_C$.
\end{defin}

\begin{equation*}
	\text{VC}_C(s)=\arg\min_{c\in C} d(s,c)
\end{equation*}

\CenterSelection è il problema che riceve in input $S\subseteq_\text{fin}\Omega$ e $k\in\N^+$, ha come soluzioni ammissibili le selezioni di centri $C\subseteq S$ con $\card C\leq k$ e come obiettivo da minimizzare:
\begin{equation*}
	\rho(C)=\max_{s\in S} d(s,\text{VC}_C(s))
\end{equation*}

\subsection{\CenterSelectionPlus}
L'algoritmo \ref{alg:CenterSelectionPlus} non risolve \CenterSelection in quanto richiede un input $r$ aggiuntivo rispetto a quelli previsti dal problema, che consiste in una stima del raggio di copertura ottimo $\rho\star$.

\begin{algorithm}[ht]
	\caption{\CenterSelectionPlus}
	\label{alg:CenterSelectionPlus}
	\SetKwFunction{ExtractRandomPoint}{ExtractRandomPoint}

\KwInput{$S$, $k$, $d$, $r$}

$C\asn\emptyset$\;
\While{$S \neq \emptyset$}{
	$\bar s\asn\ExtractRandomPoint(S)$\;
	$C\asn C\cup \set{\bar s}$\;
	$S\asn S\setminus\set{x\mid d(x,\bar s)\leq 2r}$\;
}
\If{$\card C\leq k$}{
	\Return $C$\;
}\Else{
	Impossibile\;
}

\end{algorithm}

\begin{theorem}
	Valgono le seguenti proprietà per \CenterSelectionPlus:
	\begin{enumerate}[(a)]
		\item \label{itm:centerselection:plus1} se l'algoritmo emette un output $C$, allora $C$ è una soluzione ammissibile con $\rho(C)\leq 2r$;
		\item \label{itm:centerselection:plus2} se $r\geq\rho\star$ allora l'algoritmo emette un output.
		\item \label{itm:centerselection:plus3} se $r<\frac{\rho\star}{2}$ allora l'algoritmo non produce un output.
	\end{enumerate}
\end{theorem}
\begin{proof}
	\ref{itm:centerselection:plus1} L'output $C$ è ammissibile in quanto composto da elementi di $S$ e $\card C\leq k$.
	Poiché l'algoritmo termina quando sono stati eliminati tutti i punti di $S$, e ogni punto $s\in S$ viene eliminato quando per un centro $\bar s$ selezionato vale $d(s,\bar s)\leq 2r$, il raggio massimo da un punto al centro più vicino è minore o uguale a $2r$;

	\ref{itm:centerselection:plus2} Sia $C\star$ una soluzione ottima. Si consideri $\bar s\in S$ e sia $\bar c:=\text{VC}_{C\star}(\bar s)$ e $X_{\bar c}:=\text{VC}_{C\star}^{-1}(\bar c)$.
	Per qualunque punto $s\in X_{\bar c}$ vale, per disuguaglianza triangolare:
	\begin{equation*}
		d(s,\bar s) \leq d(s,\bar c)+d(\bar c,\bar s) \leq \rho\star + \rho\star = 2\rho\star \leq 2r \text.
	\end{equation*}
	Quindi dopo l'inserimento di $\bar s$ non rimane in $S$ alcun elemento di $X_{\bar c}$.
	Poiché la soluzione ottima contiene al più $k$ centri, dopo al più $k$ iterazioni tutti i punti di $S$ sono stati rimossi, e $\card C\leq k$.

	\ref{itm:centerselection:plus3} Se l'algoritmo emettesse un output $C$ per $r<\frac{\rho\star}{2}$, allora per il punto \ref{itm:centerselection:plus1} si avrebbe $\rho(C)\leq 2r<\rho\star$, il che è assurdo se $\rho\star$ è l'ottimo.
\end{proof}


Come rappresentato in figura \ref{fig:csplus_r_beh}, l'algoritmo ricevendo $r\geq\rho\star$ produce una soluzione ammissibile con approssimazione $\frac{2r}{\rho\star}$; se $r<\frac{\rho\star}{2}$ l'algoritmo non produce output; nei casi rimanenti il comportamento dell'algoritmo non è consistente.
Un algoritmo del genere può essere utilizzato sfruttando un'opportuna applicazione della ricerca dicotomica per la determinazione del valore ottimo di $r$.

\begin{figure}[ht]
	\centering
	\begin{tikzpicture}[x=0.75pt,y=0.75pt,yscale=-1,xscale=1]
	\draw    (121,120.5) -- (490,120.5) ;
	\draw [shift={(493,120.5)}, rotate = 180] [fill={rgb, 255:red, 0; green, 0; blue, 0 }  ][line width=0.08]  [draw opacity=0] (8.93,-4.29) -- (0,0) -- (8.93,4.29) -- cycle    ;
	\draw    (230.33,115) -- (230.33,127) ;
	\draw    (330.33,115) -- (330.33,127) ;
	\draw    (430.33,115) -- (430.33,127) ;
	\draw    (140.75,115) -- (140.75,127) ;
	\draw    (140,140) .. controls (141.67,138.33) and (143.33,138.33) .. (145,140) .. controls (146.67,141.67) and (148.33,141.67) .. (150,140) .. controls (151.67,138.33) and (153.33,138.33) .. (155,140) .. controls (156.67,141.67) and (158.33,141.67) .. (160,140) .. controls (161.67,138.33) and (163.33,138.33) .. (165,140) .. controls (166.67,141.67) and (168.33,141.67) .. (170,140) .. controls (171.67,138.33) and (173.33,138.33) .. (175,140) .. controls (176.67,141.67) and (178.33,141.67) .. (180,140) .. controls (181.67,138.33) and (183.33,138.33) .. (185,140) .. controls (186.67,141.67) and (188.33,141.67) .. (190,140) .. controls (191.67,138.33) and (193.33,138.33) .. (195,140) .. controls (196.67,141.67) and (198.33,141.67) .. (200,140) .. controls (201.67,138.33) and (203.33,138.33) .. (205,140) .. controls (206.67,141.67) and (208.33,141.67) .. (210,140) .. controls (211.67,138.33) and (213.33,138.33) .. (215,140) .. controls (216.67,141.67) and (218.33,141.67) .. (220,140) .. controls (221.67,138.33) and (223.33,138.33) .. (225,140) .. controls (226.67,141.67) and (228.33,141.67) .. (230,140) -- (233,140) -- (233,140) ;
	\draw    (334.67,137.33) .. controls (336.34,135.66) and (338,135.66) .. (339.67,137.33) .. controls (341.34,139) and (343,139) .. (344.67,137.33) .. controls (346.34,135.66) and (348,135.66) .. (349.67,137.33) .. controls (351.34,139) and (353,139) .. (354.67,137.33) .. controls (356.34,135.66) and (358,135.66) .. (359.67,137.33) .. controls (361.34,139) and (363,139) .. (364.67,137.33) .. controls (366.34,135.66) and (368,135.66) .. (369.67,137.33) .. controls (371.34,139) and (373,139) .. (374.67,137.33) .. controls (376.34,135.66) and (378,135.66) .. (379.67,137.33) .. controls (381.34,139) and (383,139) .. (384.67,137.33) .. controls (386.34,135.66) and (388,135.66) .. (389.67,137.33) .. controls (391.34,139) and (393,139) .. (394.67,137.33) .. controls (396.34,135.66) and (398,135.66) .. (399.67,137.33) .. controls (401.34,139) and (403,139) .. (404.67,137.33) .. controls (406.34,135.66) and (408,135.66) .. (409.67,137.33) .. controls (411.34,139) and (413,139) .. (414.67,137.33) .. controls (416.34,135.66) and (418,135.66) .. (419.67,137.33) .. controls (421.34,139) and (423,139) .. (424.67,137.33) -- (427.67,137.33) -- (427.67,137.33) ;

	\draw (483,129) node [anchor=north west][inner sep=0.75pt]   [align=left] {$\displaystyle r$};
	\draw (134.5,93) node [anchor=north west][inner sep=0.75pt]   [align=left] {$\displaystyle 0$};
	\draw (219,67) node [anchor=north west][inner sep=0.75pt]   [align=left] {$\displaystyle \frac{\rho ^{*}}{2}$};
	\draw (327.5,76.5) node [anchor=north west][inner sep=0.75pt]   [align=left] {$\displaystyle \rho ^{*}$};
	\draw (426.5,84) node [anchor=north west][inner sep=0.75pt]   [align=left] {$\displaystyle \rho_{\max}$};
	\draw (142.67,152.67) node [anchor=north west][inner sep=0.75pt]   [align=left] {Impossibile};
	\draw (337.33,152) node [anchor=north west][inner sep=0.75pt]   [align=left] {$\rho(C)\leq 2r$};
	\draw (274.67,152) node [anchor=north west][inner sep=0.75pt]   [align=left] {??};
\end{tikzpicture}

	\caption{Comportamento di \CenterSelectionPlus.}
	\label{fig:csplus_r_beh}
\end{figure}


\subsection{\GreedyCenterSelection}
L'algoritmo \ref{algo:GreedyCenterSelection} è un algoritmo greedy per \CenterSelection.

\begin{algorithm}
	\caption{\GreedyCenterSelection}
	\SetKwFunction{ExtractRandomPoint}{ExtractRandomPoint}

\KwInput{$S$, $k$, $d$}

\If{$\card S\leq k$}{
	\Return $S$\;
}
$s\asn\ExtractRandomPoint(S)$\;
$C\asn C\cup \set{s}$\;
\While{$\card C\leq k$}{
	$\bar s\asn\arg\max_{s\in S} d(s,C)$\;
	$C\asn C\cup\set{\bar s}$\;
}
\Return $C$\;

	\label{algo:GreedyCenterSelection}
\end{algorithm}

\begin{theorem}
	\GreedyCenterSelection è un algoritmo $2$-approssimante per \CenterSelection.
\end{theorem}
\begin{proof}
	Si consideri una variante di \CenterSelectionPlus che, invece di eliminare punti di $S$, prenda in considerazione nella selezione di nuovi centri solo punti $s$ tali che $d(s,C)>2r$.
	Questo algoritmo \CenterSelectionPlus' è del tutto equivalente a \CenterSelectionPlus in quanto, selezionato un centro, i punti entro una distanza $2r$ da esso vengono ignorati per il resto della computazione.

	Per assurdo, supponiamo che l'output $C$ sia tale che $\rho(C)>2\rho\star$, ossia esiste $\hat s\in S$ tale che $d(\hat s,C)> 2\rho\star$.
	Consideriamo l'$i$-esima iterazione dell'algoritmo: sia $C_i$ l'insieme dei centri all'inizio dell'$i$-esima iterazione e $\bar s_i$ il centro da inserire.
	Per il funzionamento dell'algoritmo e poiché $C_i\subseteq C$, vale:
	\begin{equation*}
		d(\bar s_i,C_i)\geq d(\hat s,C_i)\geq d(\hat s, C)>2\rho\star
	\end{equation*}
	Quindi, il centro viene scelto tra quelli distanti almeno $2\rho\star$ dai centri attuali.
	L'algoritmo equivale quindi a un'esecuzione di \CenterSelectionPlus' con $r=\rho\star$.
	Ma per $r=\rho\star$ \CenterSelectionPlus' emette una $\frac{2\rho\star}{\rho\star}=2$-approssimazione, il che contraddice l'ipotesi di assurdo.
\end{proof}


\subsection{Inapprossimabilità di \CenterSelection}
\begin{theorem}
	Se $\P\neq\NP$, non esiste $\alpha<2$ tale che $\CenterSelection\in\gAPX\alpha$.
\end{theorem}
\begin{proof}
	Si consideri il problema di decisione \NP-completo \DominatingSet. Dato un grafo non orientato $G=(V,E)$, un dominating set per $G$ è un insieme di vertici $D\subseteq V$ se e solo se ogni vertice non appartenente a $D$ ha un adiacente in $D$: $\forall x\in(V\setminus D),\exists y\in D\mid \set{x,y}\in E$.

	\pdec{\DominatingSet}
	{Grafo non orientato $G=(V,E)$, $k\in\N^+$}
	{Determinare se esiste un dominating set per $G$ di cardinalità minore o uguale a $k$}

	Data un'istanza $((V,E),k)$ di \DominatingSet, costruiamo una istanza di \CenterSelection che ha per spazio l'insieme dei vertici, per numero massimo di centri il limite di cardinalità $k$ per il dominating set e per metrica la funzione
	\begin{equation*}
		d(x,y) =
		\begin{cases}
			0 \quad & \text{se } x = y                      \\
			1 \quad & \text{se } x\neq y\land\set{x,y}\in E \\
			2 \quad & \text{altrimenti}
		\end{cases}
	\end{equation*}
	Si noti che $d$ soddisfa la definizione di metrica.

	In una istanza di \CenterSelection derivata in questo modo, $\rho\star=2$ oppure $\rho\star=1$.
	Nel secondo caso l'insieme di centri scelti è un dominating set nell'istanza originale del problema.
	Infatti, sia $D\subseteq S$ la soluzione ottima per tale istanza. Vale:
	\begin{align*}
		\rho\star=1 & \coimpl\forall x\in(V\setminus D)\quad d(x,D)=1                             \\
		            & \coimpl\forall x \in (V \setminus D)\quad \exists y\in D\mid d(x,y)=1       \\
		            & \coimpl\forall x \in (V \setminus D)\quad \exists y\in D\mid \set{x,y}\in E \\
		            & \coimpl D\text{ è un dominating set}
	\end{align*}

	Per assurdo, supponiamo che esista un algoritmo polinomiale $\alpha$-approssimante per \CenterSelection, con $\alpha < 2$.
	Eseguendo tale algoritmo su un'istanza così costruita si ottiene un output $D$ tale che:
	\begin{gather*}
		1\leq\frac{\rho(D)}{\rho\star}\leq\alpha<2 \\
		\rho\star\leq\rho(D)\leq\alpha\cdot\rho\star<2\rho\star
	\end{gather*}

	Dato che $\rho\star=1\lor\rho\star=2$, allora vale esattamente una delle seguenti proposizioni:
	\begin{equation*}
		\begin{cases}
			1\leq\rho(D)<2\quad & \text{se } \rho\star=1 \\
			2\leq\rho(D)<4\quad & \text{se } \rho\star=2
		\end{cases}
	\end{equation*}
	Quindi, eseguito l'algoritmo, se $\rho(D)<2$ allora $\rho\star=1$ e la decisione per l'istanza originale di \DominatingSet è positiva; se $\rho(D)\geq2$ allora $\rho\star=2$ e la decisione è negativa.

	Poiché l'algoritmo agisce in tempo polinomiale, allora $\DominatingSet\in\P$, il che è assurdo se $\P\neq\NP$.
\end{proof}
Questa tecnica dimostrativa, consistente nel ridurre un problema \NP-completo e poterlo decidere in tempo polinomiale in ipotesi di assurdo discriminando di che intervallo fa parte la soluzione, è detta tecnica di riduzione con produzione di gap (\flang{gap-producing reduction}).



\section{\MinSetCover}\label{sec:SetCover}
Si definisce funzione armonica la funzione $H:\N^+\to\R$ tale che
\begin{equation*}
	H(n)=\sum_{k=1}^n \frac 1k
\end{equation*}

% TODO: dimostrare in un'appendice (vedi vecchi appunti)
Vale la seguente proprietà per la funzione armonica:
\begin{theorem}
	$\ln(n+1)\leq H(n)\leq 1+\ln n$.
\end{theorem}

Nel problema \MinSetCover sono dati insiemi, ciascuno con il suo peso, la cui unione è un insieme universo.
Il problema consiste nel trovare una selezione degli insiemi la cui unione copra l'intero insieme universo e il cui costo complessivo sia minimo.
\MinSetCover è un problema \NPO-completo.

\popt{\MinSetCover}
{$S_1,S_2,\dots,S_m\subseteq U$ tali che $\cup_{i=1}^m S_i=U$ e pesi $w_1,\dots,w_m$ con $w_i \in\Q^+~\forall i$}
{$C\subseteq\set{S_1,\dots,S_n}$}
{Determinare una selezione di insiemi che minimizzi il costo complessivo}
{$C$ tale che $\cup_{i\in C}S_i=U$}
{$\MIN$}
{$w:=\sum_{i:S_i\in C} w_i$}


\subsection{\GreedySetCover}
\begin{algorithm}[ht]
	\caption{\GreedySetCover}
	\label{algo:greedysetcover}
	\KwInput{$S_i, U$}

$R\asn U$\;
$C\asn\emptyset$\;
\While{$R\neq\emptyset$} {
	$i=\arg\min_i \set{\frac{w_i}{\card{S_i \cap R}}}$\;
	$C\asn C\cup\set{S_i}$\;
	$R\asn R\setminus S_i$\;
}
\Return{$C$}\;

\end{algorithm}
L'algoritmo \ref{algo:greedysetcover} costruisce polinomialmente una soluzione per \MinSetCover, scegliendo a ogni iterazione il sottoinsieme di input che minimizza il rapporto tra il suo peso e il numero di elementi che esso aggiunge all'output parziale.

Ogni elemento $s\in U$ viene inserito nell'output parziale in qualche iterazione $j$ con l'aggiunta di un sottoinsieme $S_j$. Definiamo quindi
\begin{equation*}
	c_u = \frac{w_j}{\card{S_j\cap R_j}}
\end{equation*}
il costo della copertura del singolo elemento di $U$, avvenuta tramite l'aggiunta di $S_j$ durante la $j$-esima iterazione.

\begin{lemma}\label{lem:gsetcov_w_sum_c_u}
	\begin{equation*}
		w=\sum_{u\in U} c_u
	\end{equation*}
\end{lemma}
\begin{proof}
	Si noti che gli insiemi $S_j\cap R_j$, dove $S_j$ è l'insieme di input scelto al passo $j$ e $R_j$ è l'insieme degli elementi dell'universo rimasti da selezionare al passo $j$, costituiscono una partizione di $U$. Infatti, l'algoritmo termina solo dopo aver esaurito gli elementi di $U$, e ogni insieme $S_j\cap R_j$ aggiunge unicamente nuovi elementi.

	Sia $w_j$ il costo dell'insieme $S_j$ aggiunto al passo $j$. Allora
	\begin{equation*}
		w = \sum_j w_j=\sum_j\sum_{s\in S_j\cap R_j} c_s=\sum_{u\in U} c_u
	\end{equation*}
\end{proof}
\begin{lemma}\label{lem:gsetcov_cu_leq_harmoskwk}
	\begin{equation*}
		\forall k\in\set{1,\dots,m} \quad\sum_{s\in S_k} c_s\leq H(\card{S_k}) \cdot w_k
	\end{equation*}
\end{lemma}
\begin{proof}
	Sia $S_k=\set{s_1,s_2,\dots,s_d}$, dove gli elementi sono elencati in ordine di copertura.

	Prima che un elemento $s_i$ venga coperto dall'inserimento di un insieme $S_{k'}$, gli elementi di $S_k$ ancora da inserire spaziano almeno da $s_i$ a $s_d$, quindi:
	\begin{equation*}
		\card{S_k\cap R_j}\geq d-i+1 \text.
	\end{equation*}
	Quindi
	\begin{equation*}
		c_{s_i}=\frac{w_{k'}}{\card{S_{k'}\cap R_j}}
		\leq\frac{w_k}{\card{S_k\cap R_j}}
		\leq\frac{w_k}{d-i+1} \text.
	\end{equation*}
	E, di conseguenza
	\begin{align*}
		\sum_{s\in S_k} c_s & =c_{s_1}+c_{s_2}+c_{s_3}\dots+c_{s_d}                            \\
		                    & \leq \frac{w_k}{d-1+1}+\frac{w_k}{d-2+1}+\dots+\frac{w_k}{d-d+1} \\
		                    & \leq \frac{w_k}{d}+\frac{w_k}{d-1}+\dots+\frac{w_k}{1}           \\
		                    & = w_k\left(1 + \frac{1}{2} + \dots + \frac{1}{d}\right)          \\
		                    & = w_k\cdot H(\card{S_k})
	\end{align*}
\end{proof}

\begin{theorem}
	Sia $M=\max_i\card{S_i}$. \GreedySetCover è $H(M)$-approssimante per \MinSetCover.
\end{theorem}
\begin{proof}
	Sia $w\star:=\sum_{i:S_i\in C\star} w_i$.
	Eseguito l'algoritmo, in virtù del lemma \ref{lem:gsetcov_cu_leq_harmoskwk} vale, per qualunque $i$:
	\begin{equation*}
		w_i\geq\frac{\sum_{s\in S_i} c_s}{H(\card{S_i})}\geq\frac{\sum_{s\in S_i} c_s}{H(M)}
	\end{equation*}
	Essendo $C\star$ una copertura e applicando il lemma \ref{lem:gsetcov_w_sum_c_u}:
	\begin{equation*}
		\sum_{S_i\in C\star}\sum_{s\in S_i} c_s \geq \sum_{s\in U} c_s = w
	\end{equation*}
	Applicando queste due osservazioni:
	\begin{gather*}
		w\star = \sum_{i:S_i\in C\star} w_i \geq \sum_{i:S_i\in C\star} \frac{\sum_{s\in S_i} c_s}{H(M)} \geq \frac{w}{H(M)} \\
		\frac{w}{w\star} \leq H(M)
	\end{gather*}
\end{proof}

Inoltre vale:
\begin{equation*}
	H(M)\leq H(\card U) = O(\log \card U)
\end{equation*}
Ergo:
\begin{corollario}
	\GreedySetCover è un algoritmo $O(\log n)$-approssimante per \MinSetCover, dove $n$ è la cardinalità dell'insieme universo.
\end{corollario}

Per quanto riguarda l'ottimalità di questo bound:
\begin{theorem}
	Per ogni $\varepsilon>0$, \GreedySetCover non è $(O(\log n)-\varepsilon)$-approssimante per \MinSetCover.
\end{theorem}
\begin{proof}
	% TODO: è necessario essere più precisi in questa costruzione: si può fare con qualunque n? È necessaria una potenza di 2? Gli insiemi S finali possono contenere un solo elemento o sempre almeno 2? Etc.
	Fissati $\varepsilon$ e $n$ (sia per semplicità $n=2^k$ per qualche $k\in\N^+,k>2$), si consideri l'input per \MinSetCover mostrato in figura \ref{fig:setcover_tightness}.
	L'input è costituito da due insiemi disgiunti $A$ e $B$ di costo $1+\varepsilon$ e cardinalità $n/2$; e $\log_2 n$ insiemi disgiunti $S_1,S_2,\dots,S_{\log_2 n}$, di cardinalità rispettive $n/2,n/4,\dots$ e costo $1$. In ciascun insieme $S_i$, metà degli elementi è contenuta in $A$ e l'altra metà in $B$.

	\begin{figure}[ht]
		\centering
		\begin{tikzpicture}[x=0.75pt,y=0.75pt,yscale=-1,xscale=1]
	\draw   (150,155) .. controls (150,80.44) and (174.62,20) .. (205,20) .. controls (235.38,20) and (260,80.44) .. (260,155) .. controls (260,229.56) and (235.38,290) .. (205,290) .. controls (174.62,290) and (150,229.56) .. (150,155) -- cycle ;
	\draw   (190,245) .. controls (190,236.72) and (196.72,230) .. (205,230) .. controls (213.28,230) and (220,236.72) .. (220,245) .. controls (220,253.28) and (213.28,260) .. (205,260) .. controls (196.72,260) and (190,253.28) .. (190,245) -- cycle ;
	\draw   (190,185) .. controls (190,176.72) and (196.72,170) .. (205,170) .. controls (213.28,170) and (220,176.72) .. (220,185) .. controls (220,193.28) and (213.28,200) .. (205,200) .. controls (196.72,200) and (190,193.28) .. (190,185) -- cycle ;
	\draw   (190,125) .. controls (190,116.72) and (196.72,110) .. (205,110) .. controls (213.28,110) and (220,116.72) .. (220,125) .. controls (220,133.28) and (213.28,140) .. (205,140) .. controls (196.72,140) and (190,133.28) .. (190,125) -- cycle ;
	\draw   (270,155) .. controls (270,80.44) and (294.62,20) .. (325,20) .. controls (355.38,20) and (380,80.44) .. (380,155) .. controls (380,229.56) and (355.38,290) .. (325,290) .. controls (294.62,290) and (270,229.56) .. (270,155) -- cycle ;
	\draw   (310,245) .. controls (310,236.72) and (316.72,230) .. (325,230) .. controls (333.28,230) and (340,236.72) .. (340,245) .. controls (340,253.28) and (333.28,260) .. (325,260) .. controls (316.72,260) and (310,253.28) .. (310,245) -- cycle ;
	\draw   (310,185) .. controls (310,176.72) and (316.72,170) .. (325,170) .. controls (333.28,170) and (340,176.72) .. (340,185) .. controls (340,193.28) and (333.28,200) .. (325,200) .. controls (316.72,200) and (310,193.28) .. (310,185) -- cycle ;
	\draw   (310,125) .. controls (310,116.72) and (316.72,110) .. (325,110) .. controls (333.28,110) and (340,116.72) .. (340,125) .. controls (340,133.28) and (333.28,140) .. (325,140) .. controls (316.72,140) and (310,133.28) .. (310,125) -- cycle ;
	\draw   (130,215) .. controls (130,184.62) and (190.44,160) .. (265,160) .. controls (339.56,160) and (400,184.62) .. (400,215) .. controls (400,245.38) and (339.56,270) .. (265,270) .. controls (190.44,270) and (130,245.38) .. (130,215) -- cycle ;
	\draw   (130,125) .. controls (130,111.19) and (190.44,100) .. (265,100) .. controls (339.56,100) and (400,111.19) .. (400,125) .. controls (400,138.81) and (339.56,150) .. (265,150) .. controls (190.44,150) and (130,138.81) .. (130,125) -- cycle ;

	\draw (119,102) node [anchor=north west][inner sep=0.75pt]   [align=left] {$1$};
	\draw (121,181) node [anchor=north west][inner sep=0.75pt]   [align=left] {$1$};
	\draw (159,32) node [anchor=north west][inner sep=0.75pt]   [align=left] {$1$};
	\draw (369,32) node [anchor=north west][inner sep=0.75pt]   [align=left] {$1$};
	\draw (188.33,2.67) node [anchor=north west][inner sep=0.75pt]   [align=left] {$1+\epsilon$};
	\draw (306.67,3) node [anchor=north west][inner sep=0.75pt]   [align=left] {$1+\epsilon$};
\end{tikzpicture}

		\caption{Esempio di input "cattivo" per $n=8$}
		\label{fig:setcover_tightness}
	\end{figure}

	L'algoritmo sceglie nell'ordine gli insiemi $S_i$ in quanto il costo di aggiungerli è, per ogni iterazione $j$, $\frac{1}{n/2^j}$, contro un costo di $\frac{1+\varepsilon}{n/2^j}$ per scegliere $A$ o $B$.
	Questo porta un costo complessivo di $\log_2 n$. La soluzione ottima tuttavia è naturalmente quella composta dagli insiemi $A$ e $B$, che ha un costo di $2+2\varepsilon$. Il rapporto tra le due è necessariamente logaritmico.
\end{proof}



\section{\VertexCover}
Una copertura per un grafo non orientato $G=(V,E)$ è un insieme di vertici $X\subseteq V$ tale che ogni lato di $E$ incide su un vertice in $X$.
Il problema \VertexCover associa a ogni vertice un peso e cerca la copertura di costo complessivo minimo.
\VertexCover è \NPO-completo.

\popt{\VertexCover}
{Grafo non orientato $G=(V,E)$ e pesi $w_1,\dots,w_n$, con $n=\card V$ e $w_i\in\Q^+~\forall i$}
{$X\subseteq V$}
{Determinare una copertura di $G$ di costo minimo}
{$X\subseteq V$ tale che $\forall e\in E ~ e\cap X\neq\emptyset$}
{$\MIN$}
{$w=\sum_{i\in X} w_i$}

Si considerino i problemi di decisione associati a \VertexCover e \MinSetCover.
Un'istanza $\tuple{G=(V,E),\angle{w_i}_{i\in V}}$ di \VertexCover può essere convertita in tempo polinomiale in una di \MinSetCover scegliendo come insiemi gli insiemi dei lati incidenti su ogni vertice $i$:
\begin{equation*}
	S_i=\set{e\in E\mid i\in e}
\end{equation*}
L'insieme universo è $E$ e i pesi sono quelli del vertice relativo a ogni insieme.

\begin{theorem}
	Sia $D$ il grado massimo di un grafo di input a \VertexCover. \VertexCover è $H(D)$-approssimabile.
\end{theorem}


\subsection{\PricedVertexCover}
Si consideri un'istanza di \VertexCover formata dal grafo $G=(V,E)$ e i pesi $\angle{w_i}_{i\in V}$.
$\angle{P_e}_{e\in E}$ è un \emph{assegnamento di prezzi} sui lati.
Un assegnamento $\angle{P_e}_{e\in E}$ si dice \emph{equo} se e solo se
\begin{equation*}
	\forall i\in V\quad\sum_{e\ni i} P_e\leq w_i \text.
\end{equation*}
Un assegnamento si dice \emph{stretto} su un vertice $i$ se e solo se
\begin{equation*}
	\forall i\in V\quad\sum_{e\ni i} P_e = w_i \text.
\end{equation*}

\begin{lemma}\label{lem:vcov_pricing_eq_sum_p_e_w_opt}
	Se $\angle{P_e}_{e\in E}$ è equo allora
	\begin{equation*}
		\sum_{e\in E} P_e \leq w\star
	\end{equation*}
	dove $w\star$ il costo ottimo per l'istanza di \VertexCover.
\end{lemma}
\begin{proof}
	Sia $X\star\subseteq V$ una soluzione ottima. Poiché $\angle{P_e}_{e\in E}$ è equo:
	\begin{equation*}
		\sum_{i\in X\star} \sum_{e\ni i} P_e \leq \sum_{i\in X\star} w_i = w\star \text.
	\end{equation*}
	Ogni lato del grafo incide su un vertice di $X\star$, quindi
	\begin{equation*}
		\sum_{e\in E} P_e \leq \sum_{i\in X\star} \sum_{e\ni i} P_e \leq w\star \text.
	\end{equation*}
\end{proof}

L'algoritmo \ref{algo:PricedVertexCover} usa una tecnica di pricing per costruire una soluzione per \VertexCover.
\begin{algorithm}
	\caption{\PricedVertexCover}
	\label{algo:PricedVertexCover}
	\KwInput{$G=(V,E)$, $\angle{w_i}_{i\in V}$}

\For{$e \in E$}{
	$P_e \asn 0$\;
}
\While{esiste un lato $\bar e=\set{\bar i,\bar j}$ tale che $\angle{P_{\bar e}}$ non è stretto né su $\bar i$ né su $\bar j$}{
	$\Delta \asn \min\set{w_{\bar i}-\sum_{e\ni\bar i} P_e , w_{\bar j}-\sum_{e\ni\bar j} P_e}$\;
	$P_{\bar e} \asn P_{\bar e} + \Delta$\;
}
$S \asn \set{v\in V \mid \angle{P_e} \text{ è stretto su } v}$\;
\Return {$S$}\;

\end{algorithm}

\begin{lemma}\label{lem:pvcov_w_le_w_sum_P_e}
	L'algoritmo \ref{algo:PricedVertexCover} produce una soluzione $S$ ammissibile per \VertexCover. Inoltre al termine dell'esecuzione vale:
	\begin{equation*}
		w \leq 2 \sum_{e \in E} P_e \text.
	\end{equation*}
\end{lemma}
\begin{proof}
	All'uscita dal ciclo, non esistono lati $\set{i,j}$ tali che $\angle{P_e}$ non è stretto né su $i$ né su $j$.
	Quindi l'insieme $S$ dei vertici su cui $\angle{P_e}$ è stretto è una copertura.

	Essendo $S$ una soluzione ammissibile, per definizione:
	\begin{equation*}
		w = \sum_{i\in S} w_i \text.
	\end{equation*}
	Poiché $S$ contiene solo vertici su cui $\angle{P_e}$ è stretto:
	\begin{equation*}
		\forall i\in S \quad w_i = \sum_{e\ni i} P_e \text.
	\end{equation*}
	Ergo
	\begin{equation*}
		w = \sum_{i\in S} \sum_{e\ni i} P_e \text.
	\end{equation*}
	Poiché un lato compare nella somma al più $2$ volte:
	\begin{equation*}
		w \leq 2 \sum_{e\in E} P_e \text.
	\end{equation*}
\end{proof}

\begin{theorem}
	\PricedVertexCover è un algoritmo $2$-approssimante per \VertexCover.
\end{theorem}
\begin{proof}
	\begin{equation*}
		\frac{w}{w\star} \underset{\text{lemma \ref{lem:pvcov_w_le_w_sum_P_e}}}{\leq}
		\frac{2\sum_{e\in E} P_e}{w\star} \underset{\text{lemma \ref{lem:vcov_pricing_eq_sum_p_e_w_opt}}}{\leq}
		\frac{2\sum_{e\in E} P_e}{\sum_{e\in E} P_e} = 2 \text.
	\end{equation*}
\end{proof}


\subsection{\VertexCover tramite arrotondamento di \LinearProgramming}
Si consideri il problema di programmazione lineare e la sua versione vincolata all'integrità della soluzione, il problema di programmazione lineare intera.

\popt{\LinearProgramming}
{Sistema $Ax\geq b$, con $A\in\Q^{m\times n},b\in\Q^m$ ($x$ incognito), vettore $c\in\Q^n$.}
{Assegnamenti per $x$ in $\Q^n$}
{Determinare il vettore $x$ che minimizza la funzione obiettivo}
{$x\in\Q^n\mid Ax\geq b$}
{$\MIN$}
{Funzione obiettivo $c\trans x$}

\popt{\IntegerLinearProgramming}
{Sistema $Ax\geq b$, con $A\in\Q^{m\times n},b\in\Q^m$ ($x$ incognito), vettore $c\in\Q^n$.}
{Assegnamenti per $x$ in $\Z^n$}
{Determinare il vettore $x$ che minimizza la funzione obiettivo}
{$x\in\Z^n\mid Ax\geq b$}
{$\MIN$}
{Funzione obiettivo $c\trans x$}

Per lo stesso input, una soluzione ottima di \IntegerLinearProgramming è ammissibile, ma non necessariamente ottima, per \LinearProgramming.
Viceversa una soluzione ottima per \LinearProgramming può non essere ammissibile per un'istanza di \IntegerLinearProgramming dallo stesso input.

\LinearProgramming è un problema appartenente a \PO (l'ottimo può essere trovato polinomialmente ad esempio con l'algoritmo di Karmarkar \cite{Karmarkar:84:LP}), mentre \IntegerLinearProgramming è \NPO-completo.

Data un'istanza di \VertexCover $\tuple{G=(V,E),\angle{w_i}_{i\in V}}$, con $n:=\card V$ e $m:=\card E$, si consideri l'istanza di \IntegerLinearProgramming in cui il sistema di input è così costruito:
\begin{equation*}
	\begin{cases}
		x_i\geq 0     & \qquad \forall i\in V         \\
		x_i\leq 1     & \qquad \forall i\in V         \\
		x_i+x_j\geq 1 & \qquad \forall \set{i,j}\in E \\
	\end{cases}
\end{equation*}
e la funzione obiettivo è
\begin{equation*}
	w = \min\sum_{i\in V} w_i x_i
\end{equation*}

Una soluzione $x$ dell'istanza di \IntegerLinearProgramming costruita si può interpretare come una soluzione di \VertexCover in cui $x_i=1\iff i\in X$.

Si consideri l'istanza di \LinearProgramming ottenuta rilassando il vincolo di integrità dell'istanza precedente.
Una soluzione ottima $x$ di tale istanza può essere calcolata in tempo polinomiale, ma non è, in generale, ammissibile per la sua versione intera.
Si consideri il vettore $r$, ottenuto dall'arrotondamento di $x$, ossia, per ciascun $i\in n$:
\begin{equation*}
	r_i:= \begin{cases}
		1 & \quad x_i\geq\frac{1}{2} \\
		0 & \quad x_i<\frac{1}{2}
	\end{cases}
\end{equation*}

\begin{lemma}\label{lem:ilp_r_ammiss}
	Il vettore $r$ è una soluzione ammissibile di \IntegerLinearProgramming.
\end{lemma}
\begin{proof}
	Per definizione, $0\leq r\leq 1$. Se non fosse $r_i+r_j\geq 1~\forall \set{i,j}\in E$, siano $\bar i,\bar j$ tali che $r_{\bar i}+r_{\bar j}<1$.
	Allora $r_{\bar i}=r_{\bar j}=0$. Per definizione di $r$ si ha $x_{\bar i}<\frac 12$ e $x_{\bar j}<\frac 12$.
	Ma allora $x_{\bar i}+x_{\bar j}<1$, il che contraddice l'ammissibilità di $x$.
\end{proof}

\begin{lemma}\label{lem:ilp_r_i_leq_2_x_i}
	\begin{equation*}
		\forall i\in V \qquad r_i \leq 2x_i
	\end{equation*}
\end{lemma}
\begin{proof}
	Se $r_i=0$ la disuguaglianza è ovvia;
	se $r_i=1$ allora, $x_i\geq \frac 12$ e $2x_i\geq 1=r_i$.
\end{proof}

\begin{theorem}\label{lem:ilp_appr}
	L'insieme $\set{i\in V\mid r_i=1}$ è una $2$-approssimazione per \VertexCover.
\end{theorem}
\begin{proof}
	Sia $w:=\sum{i\in V} w_i r_i$ il costo della soluzione di \VertexCover indotta dall'istanza arrotondata di \LinearProgramming, e sia $w\star$ la soluzione ottima. Si denoti con $w\star_{\text{LP}}$ il costo ottimo dell'istanza di \LinearProgramming e $w\star_{\text{ILP}}$ quello di \IntegerLinearProgramming.
	Applicando il lemma \ref{lem:ilp_r_i_leq_2_x_i}:
	\begin{equation*}
		w = \sum_{i\in V} w_i r_i \leq 2\sum_{i\in V} w_i x_i = 2w\star_{\text{LP}} \leq 2w\star_{\text{ILP}} = w\star
	\end{equation*}
\end{proof}





\section{\DisjointPaths}
Dato un grafo orientato su cui sono selezionati un numero di vertici \flang{source} e un numero di rispettivi vertici \flang{target} (potenzialmente con molteplicità), il problema \DisjointPaths si pone l'obiettivo di massimizzare il numero di coppie source-target connettibili da un cammino usando ogni arco un numero massimo di $c$ volte, dove $c$ è un dato parametro detto di congestione.
Il problema prende il nome dalla sua variante con $c=1$, in cui i cammini non hanno archi in comune.
\DisjointPaths è \NPO-completo.

\popt{\DisjointPaths}
{Grafo orientato $G=(V,E)$, vertici $\angle{s_i}_{i\in k}$ e $\angle{t_i}_{i\in k}$ e un parametro $c\in\N^+$}
{$I\subseteq k$, cammini $\angle{\pi_i}_{i\in I}$, con $\pi_i:=s_i\leadsto t_i$}
{Determinare il massimo numero di coppie $s_i,t_i$ che si possono connettere con un cammino, usando un arco al più $c$ volte complessivamente}
{$I\subseteq k$, cammini $\angle{\pi_i}_{i\in I}$ tali che nessun arco $e\in E$ appartenga ai cammini più di $c$ volte}
{$\MAX$}
{$\card I$}


\subsection{\PricedDisjointPaths}
L'algoritmo \ref{algo:PricedDisjointPaths} usa una tecnica di pricing in cui viene definita una funzione di costo $l:E\to\Q^+$ per gli archi, estendibile ai cammini $\tuple{x_0,x_1\dots,x_{t-1},x_t}$ con $l(\tuple{x_0,x_1,\dots,x_{t-1},x_t}):=l((x_0,x_1))+\dots+l((x_{t-1},x_t))$.
L'algoritmo fa inoltre uso di un valore $\beta$ che, come si vedrà, può essere calcolato in modo da ottimizzare il risultato.
\SetKwFunction{MinPath}{MinPath}
La procedura \MinPath restituisce, in tempo polinomiale, un cammino di costo minimo e l'indice $i$ dei vertici $s_i$ e $t_i$ che collega, con $i\notin I$. Se un cammino del genere non esiste, la procedura restituisce un cammino vuoto.
L'algoritmo produce una soluzione ammissibile per \DisjointPaths, dal momento che $P$ contiene solo cammini con archi utilizzati al più $c$ volte (dopo $c$ volte un arco viene eliminato) e che collegano, per definizione di \MinPath, vertici non ancora collegati.
\PricedDisjointPaths ha un costo polinomiale: utilizzando ad esempio Floyd-Warshall la procedura \MinPath può essere implementata in $O(\card V^3)$ e viene ripetuta un massimo di $k$ volte, per un costo totale in tempo di $O(k\card V^3)$.

\begin{algorithm}
	\caption{\PricedDisjointPaths}
	\label{algo:PricedDisjointPaths}
	\KwInput{$G=(V,E), \angle{s_i}_{i\in k}, \angle{t_i}_{i\in k}, c\in\N^+, b\in\Q^+$}

$I \asn \emptyset$\;
$P \asn \emptyset$\;
\For{$e\in E$}{
	$l(e) \asn 1$\;
}
\While{true}{
	$\pi,i \asn \MinPath(G,l)$\;
	\If{$\pi=\langle\rangle$}{
		\Return{$I,P$}\;
	}
	$I \asn I\cup\set{i}$\;
	$P \asn P\cup\set{\pi}$\;

	\tcc{Aggiorna i costi ed elimina gli archi già usati $c$ volte}
	\For{$e\in\pi$}{
		$l(e) = l(e) \cdot \beta $\;
		\If{$l(e) = \beta^c$}{
			elimina $e$ \;
		}
	}
}


\end{algorithm}

A una data iterazione dell'algoritmo, un cammino $\pi$ si dice \emph{corto} se e solo se $l(\pi)<\beta^c$.
Un cammino $\pi$ si dice \emph{utile} se e solo se collega una coppia $i\notin I$.

Finché esistono cammini corti e utili, l'algoritmo seleziona uno di essi a ogni iterazione.
Quando nessun cammino è corto e utile, l'esecuzione si ferma oppure iniziano a venire selezionati cammini lunghi (i.e. non corti).
Si consideri la prima iterazione $\bar t$ in cui non esistono cammini corti e utili, o il termine dell'esecuzione se tale iterazione non esiste.
Sia $\bar l$ la funzione di costo in tale iterazione e $\bar I$ l'insieme degli indici dei vertici collegati da cammini.

\begin{lemma}\label{lem:priceddpaths_non_included_non_short}
	Se all'iterazione $\bar t$ la coppia $\tuple{s_i,t_i}$ non è stata collegata dalla soluzione corrente, allora il costo del relativo cammino ottimo $\pi\star_i$ è maggiore o uguale a $\beta^c$:
	\begin{equation*}
		\bar l(\pi\star_i)\geq\beta^c\qquad\forall i\notin I
	\end{equation*}
\end{lemma}
\begin{proof}
	Se fosse $\bar l(\pi\star_i)<\beta^c$, allora $\pi\star$ sarebbe corto e utile, pertanto sarebbe stato selezionato prima dell'iterazione $\bar t$.
\end{proof}

\begin{lemma}\label{lem:priceddpaths_sum_l_a_leq_bc_i_m}
	Sia $m:=\card E$.
	\begin{equation*}
		\sum_{e\in E}\bar l(e) \leq \beta^{c+1}\card{\bar I} + m
	\end{equation*}
\end{lemma}
\begin{proof}~
	\begin{itemize}
		\item Alla prima iterazione, $\sum_{e\in E} l_0(e) = \sum_{e\in E} 1 = m$.
		\item Al termine di ogni iterazione $j<\bar t$, si modificano i valori $l_j$ in valori $l_{j+1}$ così scelti:
		      \begin{equation*}
			      l_{j+1}(e) =
			      \begin{cases}
				      l_j(e)            & \quad\text{se } e\notin\pi_i \\
				      \beta\cdot l_j(e) & \quad\text{se } e\in\pi_i
			      \end{cases}
		      \end{equation*}
		      Si consideri la differenza tra i pesi complessivi all'iterazione $j$ e quelli all'iterazione $j+1$:
		      \begin{align*}
			      \sum_{e\in E} l_{j+1}(e) - \sum_{e\in E} l_j(e) & = \sum_{e\in E} (l_{j+1}(e)-l_j(e))   \\
			                                                      & = \sum_{e\in\pi}(\beta l_j(e)-l_j(e)) \\
			                                                      & = \sum_{e\in\pi} (\beta-1)l_j(e)      \\
			                                                      & \leq \beta\sum_{e\in\pi}l_j(e)
		      \end{align*}
		      Tale valore è al più uguale a $\beta^{c+1}$, essendo il cammino $\pi$ corto perché $j<\bar t$.
	\end{itemize}
	Quindi, all'inizio dell'iterazione $\bar t$, a un costo iniziale di $m$ sono state aggiunte $\card{\bar I}$ variazioni (una per ogni iterazione e quindi aggiunta di cammini a $I$) di al più $\beta^c$ l'una, ergo:
	\begin{equation*}
		\sum_{e\in E} \bar l(e) \leq \beta^{c+1}\card{\bar I}+m \text.
	\end{equation*}
\end{proof}

\begin{corollario}\label{cor:priceddpaths_cor_1}
	\begin{equation*}
		\sum_{i\in I\star\setminus I} \bar l(\pi_i\star) \geq \beta^c \card{I\star\setminus I} \text.
	\end{equation*}
\end{corollario}
\begin{proof}
	Ottenuto dal lemma \ref{lem:priceddpaths_non_included_non_short} sommando per i valori in $I\star\setminus I$.
\end{proof}

\begin{corollario}\label{cor:priceddpaths_cor_2}
	\begin{equation*}
		\sum_{i\in I\star\setminus I} \bar l(\pi\star_i) \leq c(\beta^{c+1}\card{\bar I}+m) \text.
	\end{equation*}
\end{corollario}
\begin{proof}
	\begin{align*}
		\sum_{i\in I\star\setminus I} \bar l(\pi\star_i) & \leq \sum_{i\in I\star} \bar l(\pi\star_i)                                                                                     \\
		                                                 & \leq c \sum_{e\in E} \bar l(e)             &  & \text{ogni arco è usato al più $c$ volte nella soluzione ammissibile $I\star$} \\
		                                                 & \leq c (\beta^{c+1}\card{\bar I}+m)        &  & \text{dal lemma \ref{lem:priceddpaths_sum_l_a_leq_bc_i_m}}
	\end{align*}
\end{proof}

\begin{theorem}\label{thm:priceddpaths_approx}
	\PricedDisjointPaths è un algoritmo $(1+c(\beta+\beta^{-c}m))$-approssimante per \DisjointPaths.
	Se $\beta=m^{\frac{1}{c+1}}$, \PricedDisjointPaths fornisce una $(1+2cm^{\frac{1}{c+1}})$-approssimazione.
\end{theorem}
\begin{proof}
	Sia $I\star$ la soluzione ottima e $I$ la soluzione prodotta da \PricedDisjointPaths.
	\begin{align*}
		\beta^c\card{I\star} & = \beta^c\card{I\star\cap I}+\beta^c\card{I\star\setminus I}                                                                                  \\
		                     & \leq \beta^c\card{I\star\cap I} + \sum_{i\in I\star\setminus I} \bar l(\pi\star_i) &  & \text{per il corollario \ref{cor:priceddpaths_cor_1}} \\
		                     & \leq \beta^c\card I + \sum_{i\in I\star\setminus I} \bar l(\pi\star_i)                                                                        \\
		                     & \leq \beta^c\card I + c(\beta^{c+1}\card{\bar I}+m)                                &  & \text{per il corollario \ref{cor:priceddpaths_cor_2}} \\
		                     & \leq \beta^c\card I + c(\beta^{c+1}\card I+m)                                      &  & \text{essendo $\bar I\subseteq I$} \text.
	\end{align*}
	Dividendo per $\beta^c$:
	\begin{align*}
		\card{I\star} & \leq \card I+c\beta\card I+c\beta^{-c}m                                                   \\
		              & \leq \card I+c\beta\card I+c\beta^{-c}m\card I &  & \text{essendo $\card I\geq 1$} \text.
	\end{align*}
	Dividendo per $\card I$:
	\begin{equation*}
		\frac{\card{I^*}}{\card I} \leq 1+c\beta+c\beta^{-c}m = 1+c(\beta+\beta^{-c}m)
	\end{equation*}
	% TODO: dimostrare in un'appendice
	Questo valore è minimizzato per $\beta=m^{\frac{1}{c+1}}$, ottenendo:
	\begin{align*}
		\frac{\card{I^*}}{\card I} & \leq 1+c\left(m^{\frac{1}{c+1}}+m^{\frac{-c}{c+1}}m\right) \\
		                           & = 1+c\left(m^{\frac{1}{c+1}}+m^{\frac{-c+c+1}{c+1}}\right) \\
		                           & = 1+2cm^{\frac{1}{c+1}}
	\end{align*}
	L'analisi dimostra che le sole prime $\bar t$ iterazioni producono una $(1+2cm^{\frac{1}{c+1}})$-approssimazione. Dal momento che, come mostrato, $\bar I\subseteq I$, l'algoritmo al termine dell'esecuzione produce una soluzione quantomeno non peggiore.
\end{proof}



\section{\TravelingSalesman}
Il problema del commesso viaggiatore, o \TravelingSalesman (problem, abbreviato in TSP), è uno dei problemi più famosi della teoria dei grafi.
Il problema consiste nel trovare un circuito hamiltoniano di costo minimo in un grafo pesato sugli archi.
Dato un grafo non orientato, un \emph{circuito hamiltoniano} è un circuito che passa per ogni vertice del grafo una e una sola volta.
\TravelingSalesman è \NPO-completo.

\popt{\TravelingSalesman}
{Grafo non orientato $G=(V,E)$, pesi $\angle{\delta_e}_{e\in E}$}
{$\pi\subseteq E$ ordinato}
{Determinare il circuito hamiltoniano di minor costo}
{$\pi$ forma un circuito hamiltoniano}
{$\MIN$}
{$\sum_{e\in\pi} \delta_e$}

Con un abuso di notazione useremo $\delta$ per indicare il costo del suo argomento, inteso come la somma dei pesi degli archi che lo compongono.

L'algoritmo di Christofides, che descriveremo a breve, mette insieme una serie di risultati, problemi e algoritmi e trova un'approssimazione della soluzione ottima di \TravelingSalesman. Introduciamo ora tali nozioni.


\subsection{Requisiti}

\subsubsection{Circuiti euleriani}
Dato un grafo non orientato, un \emph{circuito euleriano} è un circuito che include ogni lato del grafo una e una sola volta. Si noti che possono esistere circuiti euleriani che includono un vertice multiple volte.

Il problema di trovare un circuito euleriano in un grafo (o multigrafo) è stato formalizzato da Eulero a partire dal problema dei ponti di Könisberg, che chiedeva se fosse possibile attraversare tutti i ponti della città una volta e tornare al punto di partenza.
Eulero dimostra una condizione necessaria e sufficiente per l'esistenza di un cammino euleriano in un multigrafo:
\begin{theorem}[di Eulero]\label{thm:eulero}
	Un multigrafo ammette un circuito euleriano se e solo se è connesso e tutti i suoi vertici hanno grado pari.
\end{theorem}

Un risultato fondamentale inerente alla parità del grado dei vertici è il lemma delle strette di mano (\flang{handshaking lemma}):
\begin{lemma}[delle strette di mano]\label{lem:handshaking}
	In ogni grafo, il numero di vertici di grado dispari è pari.
\end{lemma}
\begin{proof}
	Dal momento che ogni arco aumenta il grado di due vertici, la somma dei gradi di tutti i vertici è pari. Ma una somma di interi è pari se e solo se il numero di addendi dispari è pari.
\end{proof}

\subsubsection{Il TSP metrico su cricca}\label{subsub:tsp:criccametrica}

\paragraph{Il TSP su cricca} A partire da un'istanza di \TravelingSalesman composta dal grafo $G=(V,E)$ e i pesi $\angle{\delta_e}_{e\in E}$, si consideri l'istanza composta dalla cricca $K=\left(V,\binom V 2\right)$ e dai pesi $\angle{\bar\delta}_{e\in\binom V 2}$ così definiti:
\begin{equation*}
	\bar\delta_e = \begin{cases}
		\delta_e                 & \quad e \in E    \\
		1+\sum_{e\in E} \delta_e & \quad e \notin E
	\end{cases}
\end{equation*}
Trovando la soluzione ottima di \TravelingSalesman su questa istanza, è facile determinare se il circuito hamiltoniano trovato è una soluzione per il grafo originale.
Poiché ogni arco viene percorso al più una volta (altrimenti i vertici verrebbero ripetuti nel circuito), se il costo totale è al più $\sum_{e\in E} \delta_e$ allora nessun arco assente nel grafo originale è stato usato, e la soluzione è ottima anche per l'istanza originale.
Viceversa, se il costo totale è almeno $1+\sum_{e\in E} \delta_e$ allora non è stato possibile trovare un circuito euleriano che non usasse un arco assente nell'istanza originale, e quindi questa non ha soluzione.
Si può quindi semplificare la risoluzione di \TravelingSalesman senza perdita di generalità limitandosi a risolvere il problema nel caso di cricche.

\paragraph{Il TSP metrico} Semplifichiamo ora il problema, imponendo una caratteristica "metrica" alla funzione dei pesi, cioè la disuguaglianza triangolare:
\begin{equation*}
	\forall i,j,k\in V \quad \delta_{\set{i,j}} \leq \delta_{\set{i,k}} + \delta_{\set{k,j}} \text.
\end{equation*}
Come vedremo, solo grazie a questa assunzione è possibile costruire algoritmi approssimanti per \TravelingSalesman.

\subsubsection{L'albero ricoprente minimo}
Dato un grafo non orientato connesso, un albero ricoprente è un sottografo che sia un albero e mantenga tutti i vertici.
Se il grafo è pesato sui lati, un albero ricoprente minimo minimizza la somma dei costi dei lati che mantiene.
\MinimumSpanningTree (MST) è il problema di ottimizzazione che cerca un albero ricoprente minimo in un grafo.
Il problema è risolvibile esattamente in tempo polinomiale, ad esempio dall'algoritmo di Kruskal in tempo $O(m\log n)$.

\popt{\MinimumSpanningTree}
{Grafo connesso $G=(V,E)$, pesi $\angle{\delta_e}_{e\in E}$}
{Albero $T=(V',E')$}
{Trovare un albero ricoprente minimo per $G$}
{$T$ è un albero tale che $V'=V$}
{$\MIN$}
{$\sum_{e\in E'} \delta_e$}

\subsubsection{\MinimumWeightPerfectMatching}
Dato un grafo non orientato con un numero pari di vertici, un \emph{matching perfetto} è un matching (vedi \ref{sec:BiMaxMatching}) che coinvolge tutti i vertici.
In un grafo pesato sugli archi, \MinimumWeightPerfectMatching è il problema di trovare il matching perfetto a costo complessivo minimo.
\MinimumWeightPerfectMatching è risolvibile esattamente in tempo polinomiale, ad esempio con il \flang{blooming algorithm} in tempo $O(m\log n)$.

\popt{\MinimumWeightPerfectMatching}
{Grafo $G=(V,E)$ tale che $\card V$ è pari, pesi $\angle{\delta_e}_{e\in E}$}
{Matching $M\subseteq E$}
{Determinare un matching perfetto di costo minimo}
{$M$ è un matching perfetto su $G$}
{$\MIN$}
{$\sum_{e\in M} \delta_e$}


\subsection{L'algoritmo di Christofides}
L'algoritmo di Christofides per TSP metrico su cricca applica i concetti precedentemente descritti manipolando diverse strutture fino a ottenere un cammino hamiltoniano. Ricevendo in input una cricca $G=(V,E)$ e pesi $\angle{\delta_e}_{e\in E}$, i passi dell'algoritmo sono i seguenti:
\begin{enumerate}
	\item identificare un albero ricoprente minimo $T$ per $G$;
	\item sia $D$ il grafo indotto su $G$ dall'insieme dei suoi vertici che hanno grado dispari in $T$.
	      Questi sono in numero pari per il lemma \ref{lem:handshaking}. È quindi possibile identificare un matching perfetto minimo $M$ su $D$;
	\item sia $H$ il multigrafo indotto su $G$ dall'unione disgiunta dei lati di $T$ ed $M$. In $H$ tutti i vertici hanno grado pari, poiché quelli che avevano grado dispari in $T$ hanno un nuovo arco grazie a $M$. In virtù del lemma \ref{thm:eulero} si può trovare un circuito euleriano $\pi$ in $H$;
	\item trasformare il circuito euleriano $\pi$, valido anche per $G$, in un hamiltoniano $\tilde\pi$.
	      Ogni volta che $\pi$ passa su un vertice $v$ una seconda volta tramite i lati $\set{a,v}$ e $\set{v,b}$, è sufficiente "saltare" tale vertice sostituendo in $\tilde\pi$ i due lati con il lato $\set{a,b}$ (esistente in quanto si sta lavorando sulla cricca originale), con conseguente diminuzione del costo in virtù della disuguaglianza triangolare.
\end{enumerate}

L'analisi dell'algoritmo ci porta a dimostrare un risultato di approssimazione rispetto al TSP metrico su cricca.
\begin{lemma}\label{lem:chri_spanning}
	In una cricca $G$ pesata metricamente sui lati, il costo $\delta(T)$ di un albero ricoprente minimo $T$ non è peggiore del costo minimo $\delta\star$ di un cammino hamiltoniano su $G$:
	\begin{equation*}
		\delta(T) \leq \delta\star
	\end{equation*}
\end{lemma}
\begin{proof}
	Sia $\pi\star$ un circuito hamiltoniano ottimo e sia $e\in\pi\star$.
	Il grafo indotto da $\pi\star\setminus e$ è un albero ricoprente. Pertanto:
	\begin{equation*}
		\delta(T) \leq \delta(\pi\star\setminus e) \leq\delta\star \text.
	\end{equation*}
\end{proof}

\begin{lemma}\label{lem:chri_matching}
	\begin{equation*}
		\delta(M) \leq \frac 1 2 \delta\star
	\end{equation*}
\end{lemma}
\begin{proof}
	Si consideri un circuito hamiltoniano ottimo $\pi\star$. Poiché è hamiltoniano, questo comprende i vertici di $D$. Cortocircuitando $\pi\star$ sui soli vertici di $D$, si ottiene un circuito di peso complessivo minore o uguale a quello di $\pi\star$ (per la disuguaglianza triangolare) e che ha un numero di lati pari (poiché i vertici di $D$ sono pari). Alternando i lati di questo circuito, si formano due matching $M_1$ e $M_2$ perfetti su $D$. Ma poiché $M$ è il matching perfetto di costo minimo per $D$, vale:
	\begin{equation*}
		\delta(\pi\star)\geq\delta(M_1)+\delta(M_2)\geq 2\delta(M) \text. \qedhere
	\end{equation*}
\end{proof}

\begin{theorem}
	L'algoritmo di Christofides è $\frac 3 2$-approssimante per il TSP metrico su cricca.
\end{theorem}
\begin{proof}
	Per la costruzione del circuito hamiltoniano $\tilde\pi$ a partire dal circuito euleriano $\pi$ e grazie ai lemmi sopra dimostrati:
	\begin{equation*}
		\delta(\tilde\pi)\leq\delta(\pi) = \delta(T)+\delta(M)\leq
		\underbrace{\delta\star}_{\text{lemma \ref{lem:chri_matching}}} + \underbrace{\frac{\delta\star}{2}}_{\text{lemma \ref{lem:chri_spanning}}} =
		\frac 3 2 \delta\star \text.
	\end{equation*}
\end{proof}

\begin{theorem}
	Per ogni $\varepsilon>0$ esiste un input del TSP metrico su cricca su cui l'algoritmo di Christofides produce una soluzione $\pi$ tale che
	\begin{equation*}
		\frac 3 2-\varepsilon \leq \frac{\delta(\pi)}{\delta\star}
	\end{equation*}
\end{theorem}
\begin{proof}
	Dato $n$ pari e $\varepsilon\in(0,1)$, si consideri il grafo in figura \ref{fig:christotight}, in cui gli archi sono etichettati con i loro pesi. Si estenda il grafo a una cricca $G$, in cui ogni lato $\set{u,v}$ non rappresentato ha come peso il peso del cammino minimo tra $u$ e $v$ sul grafo originale.

	\begin{figure}[ht]
		\centering
		\begin{tikzpicture}[vert/.style={minimum size=15pt,draw,circle}]
	\node[vert]		(1) {$v_1$};
	\node[vert,right=of 1]	(2) {$v_2$};
	\node[vert,right=of 2]	(3) {$v_3$};
	\node[vert,right=of 3]	(4) {$v_4$};
	\node[vert,right=of 4]	(5) {$v_{n-2}$};
	\node[vert,right=of 5]	(6) {$v_{n-1}$};
	\node[vert,right=of 6]	(7) {$v_{n}$};

	\draw (1) to node [auto] {$1$} (2);
	\draw (2) to node [auto] {$1$} (3);
	\draw (3) to node [auto] {$1$} (4);
	\draw (5) to node [auto] {$1$} (6);
	\draw (6) to node [auto] {$1$} (7);

	\draw[dotted]		(4) to (5);

	\draw[bend right=40]	(1) to node	[below]	{$1+\epsilon$} (3);
	\draw[bend left=41]	(2) to node	[auto]	{$1+\epsilon$} (4);
	\draw[bend left=37]	(5) to node	[auto]	{$1+\epsilon$} (7);
	\draw[bend right=30]	(3) edge node	[below]	{$1+\epsilon$} (5.5,-1);
	\draw[bend left=30]	(6) edge node	[auto]	{$1+\epsilon$} (7.5,-1);
\end{tikzpicture}

		\caption{Esempio di input "cattivo" per l'algoritmo di Christofides.}
		\label{fig:christotight}
	\end{figure}

	L'algoritmo di Christofides identifica l'albero ricoprente $T$ indotto da tutti i lati di peso $1$, quindi $\delta(T)=n-1$.
	L'algoritmo seleziona poi $D=\set{v_1,v_n}$ e su di esso il matching $M$ composto dal solo lato $\set{v_1,v_n}$ di peso $\delta(M)=(1+\varepsilon)\left(\frac n2-1\right)+1$.
	Sul grafo indotto dall'unione $H$ di $T$ e $M$ l'algoritmo costruisce il circuito euleriano, nonché hamiltoniano, composto dall'unione degli archi di peso $1$ e l'arco che compone $M$.
	Il costo $\delta$ del circuito costruito è quindi:
	\begin{equation*}
		\delta = n-1 + (1+\varepsilon)\left(\frac n2-1\right)+1 = \frac32n+\varepsilon\left(\frac n2-1\right)-1 \text.
	\end{equation*}
	Ma il circuito hamiltoniano ottimo è quello che, a partire da $v_1$, percorre tutti gli archi di peso $1+\varepsilon$ che uniscono vertici di indice dispari, l'arco $\set{v_{n-1},v_n}$, tutti gli archi di peso $1+\varepsilon$ di indice pari e infine l'arco $\set{v_1,v_2}$.
	Il costo $\delta\star$ di tale circuito è:
	\begin{equation*}
		\delta\star = 2(1+\varepsilon)\left(\frac n2-1\right) + 2 = n+\varepsilon(n-2) \text.
	\end{equation*}
	Quindi
	\begin{equation*}
		\frac{\delta}{\delta\star} = \frac{\frac32n+\varepsilon\left(\frac n2-1\right)-1}{n+\varepsilon(n-2)}
	\end{equation*}
	Che è definitivamente\footnote{Precisamente, la disequazione è vera per $n\geq 2+\frac{1-2\varepsilon}{\varepsilon^2}$.} maggiore o uguale a $\frac 3 2-\varepsilon$.

\end{proof}


\subsection{Inapprossimabilità di \TravelingSalesman}
Abbandonando l'ipotesi di disuguaglianza triangolare della funzione $\delta$, il problema \TravelingSalesman è inapprossimabile.
Al fine di dimostrare ciò ricordiamo che decidere se un grafo contiene un cammino hamiltoniano è un problema \NP-completo.

\begin{theorem}
	Se $\P\neq\NP$, non esiste alcun $\alpha>1$ tale che \TravelingSalesman sia $\alpha$-approssimabile.
\end{theorem}
\begin{proof}
	Fissato $\alpha>1$, sia $G=(V,E)$ un grafo non orientato in cui tutti i lati hanno peso $1$.
	Sia $G'$ la sua estensione a cricca, ottenuta con il metodo citato nel paragrafo \ref{subsub:tsp:criccametrica}, ossia in cui la funzione peso $\delta$ è così definita:
	\begin{equation*}
		\delta(x,y) = \begin{cases}
			1          & \quad \set{x,y}\in E    \\
			\alpha n+1 & \quad \set{x,y}\notin E
		\end{cases}
	\end{equation*}
	Se $G$ ammette un circuito hamiltoniano, allora tale circuito è ammesso anche in $G'$ e ha costo $n<\alpha n$.
	Viceversa, se $G$ non ammette un circuito hamiltoniano allora ogni circuito hamiltoniano in $G'$ ha costo di almeno $\alpha n+1$.

	Se esiste un algoritmo $\alpha$-approssimante per \TravelingSalesman, questo trova in tempo polinomiale un circuito hamiltoniano in $G'$ di costo al più $\alpha n$. Ma un tale circuito esiste se e solo se esiste un cammino hamiltoniano in $G$. L'algoritmo decide quindi in tempo polinomiale se un grafo $G$ ammette un cammino hamiltoniano, il che è impossibile se $\P\neq\NP$.
\end{proof}



\section{\texorpdfstring{$2$}{2}-\LoadBalancing}
$2$-\LoadBalancing è una specializzazione di \LoadBalancing in cui il numero di macchine $m$ è uguale a $2$.
Questo problema è anche chiamato \MinimumPartition in quanto è equivalente al bilanciare una partizione insiemistica di due elementi.
Il problema è \NPO-completo.

L'algoritmo \ref{algo:partitionbalance} è un algoritmo \PTAS-caratterizzante per $2$-\LoadBalancing.
L'algoritmo opera in tempo polinomiale rispetto alla dimensione dell'input ed esponenziale rispetto a $\varepsilon$.

\begin{algorithm}
	\caption{Algoritmo \PTAS per $2$-\LoadBalancing.}
	\label{algo:partitionbalance}
	\KwInput{$m_1,m_2,t_1,\dots,t_n,\varepsilon$}

\If{$\epsilon\geq 1$}{
	assegna i task $\set{t_1,\dots,t_n}$\;
	\Return\;
}
Ordina in ordine decrescente di costo i task $t_1,\dots,t_n$\;
$k \asn \ceil{\frac{1}{\epsilon}-1}$\;
Cerca esaustivamente l'assegnamento ottimo dei task $t_1,\dots,t_k$ e applicalo\;
Assegna in modo greedy (i.e. alla macchina più scarica) i task $t_{k+1},\dots,t_n$\;

\end{algorithm}

\begin{theorem}
	Dato $\varepsilon>0$, l'algoritmo \ref{algo:partitionbalance} produce in tempo polinomiale in $n$ una $(1+\varepsilon)$-approssimazione per $2$-\LoadBalancing.
\end{theorem}
\begin{proof}
	Sia $L:=\frac12\sum_{i=1}^n t_i$.
	Il costo ottimo è almeno $L$, perciò se $\varepsilon\geq1$, assegnando tutti i task alla stessa macchina si ottiene un costo di $\sum_{i=1}^n t_i=2L\leq(1+\varepsilon)L\star$. Qualunque altro assegnamento produrrebbe una soluzione migliore, e quindi altrettanto buona, per questo caso.

	Se $\varepsilon<1$, si consideri lo stato delle macchine al termine dell'esecuzione e sia $L_1\geq L_2$ senza perdita di generalità.
	Sia $t_h$ l'ultimo task assegnato alla macchina $1$.
	\begin{itemize}
		% TODO: inserire considerazioni della dispensa aggiuntiva del prof
		\item Se $h\leq k$, $t_h$ è stato assegnato nella fase esaustiva e pertanto la soluzione costruita è ottima.
		\item Se $h>k$, allora $t_h$ è assegnato nella fase greedy, quindi, se $L_2'$ è il carico della macchina $2$ all'assegnamento di $t_h$, vale:
		      \begin{equation*}
			      L_1-t_h\leq L_2'\leq L_2
		      \end{equation*}
		      Quindi, ricordando che $L_1+L_2=2L$:
		      \begin{align}
			      L_1-t_h  & \leq L_2 \nonumber                    \\
			      2L_1-t_h & \leq L \nonumber                      \\
			      L_1      & \leq L+\frac{t_h}{2} \label{eq:2lb:1}
		      \end{align}
		      Inoltre vale
		      \begin{equation}\label{eq:2lb:2}
			      2L = \underbrace{t_0+t_1+\dots+t_k}_{\geq t_hk}+\underbrace{\dots+t_h+\dots+t_{n-1}}_{\geq t_h} \geq t_h(k+1)
		      \end{equation}
		      Ergo:
		      \begin{align*}
			      \frac{L_1}{L\star} & \leq \frac{L+\frac{t_h}{2}}{L\star}        &  & \text{per la \ref{eq:2lb:1}} \\
			                         & \leq \frac{L+\frac{t_h}{2}}{L}             &  & L\star\geq L                 \\
			                         & = 1+\frac{t_h}{2L}                                                           \\
			                         & \leq 1+\frac{t_h}{t_h(k+1)}                &  & \text{per la \ref{eq:2lb:2}} \\
			                         & = 1+\frac{1}{k+1}                                                            \\
			                         & \leq 1+\frac{1}{\frac{1}{\varepsilon}-1+1} &  & k\geq\frac{1}{\varepsilon}-1 \\
			                         & = \varepsilon+1 \text.
		      \end{align*}
	\end{itemize}
	Considerate le complessità delle operazioni di ordinamento, della ricerca esaustiva delle $2^k$ combinazioni per la parte ottima, e della coda per la parte greedy, l'algoritmo ha tempo d'esecuzione $O(n\log n+2^{\frac{1}{\varepsilon}}n)$.
\end{proof}



\section{\Knapsack}
\Knapsack, il problema dello zaino, è un celebre problema di ottimizzazione.
Dati uno zaino di capacità $W$ e degli oggetti, ciascuno con un valore e un volume (o peso), il problema consiste nel selezionare un gruppo di oggetti in modo che il loro volume non ecceda la capacità dello zaino ma il loro valore complessivo sia massimo.
\Knapsack è un problema \NPO-completo.

% TODO: bisogna pensare a un modo furbo di indicizzare gli oggetti di modo che la colonna i gestisca l'oggetto di indice i. Occhio però alla notazione angle, che non supporta facilmente gli indici da 1
\popt{\Knapsack}
{$n$ oggetti con valori $\angle{v_i}_{i\in n}\in\N^+$ e pesi $\angle{w_i}_{i\in n}\in\N^+$, capacità $W\in N$}
{$I\subseteq n$}
{Determinare l'insieme di oggetti di valore complessivo maggiore il cui peso complessivo non superi $W$}
{$\sum_{i\in I}w_i\leq W$}
{$\MAX$}
{$\sum_{i\in I} v_i$}


\subsection{\DynamicKnapsack}
Il problema può essere risolto in modo esatto dall'algoritmo \DynamicKnapsack, basato su programmazione dinamica.
L'algoritmo costruisce una matrice in cui costruisce le soluzioni ottime per istanze di complessità minore di quella in input, usandole per calcolare la soluzione ottima complessiva.

L'elemento $A[i,w]$ della matrice $A\in{\N^+}^{(n+1)\times (W+1)}$ è associato semanticamente al massimo valore ottenibile con i primi $i$ oggetti con una capacità massima di $w$.
Gli elementi $A[0,w]$, per ogni $w\in W+1$, sono vincolati al valore $0$ dal momento che nessun oggetto può essere utilizzato, mentre i restanti valori sono così calcolati:
\begin{equation*}
	A[i+1,w] =
	\begin{cases}
		A[i,w]                       & \quad w_i>w     \\
		\max(A[i,w], A[i,w-w_i]+v_i) & \quad w_i\leq w
	\end{cases}
\end{equation*}
L'algoritmo valuta quindi se l'oggetto $i+1$ ha volume compatibile con la capacità massima: in caso negativo reitera la soluzione che non usa l'ultimo oggetto, in caso positivo sceglie la migliore tra tale soluzione e quella che fa uso dell'ultimo oggetto (costruita aggiungendolo alla soluzione ottima dell'istanza con capacità $w-w_i$).

Riempita la tabella, l'algoritmo può ricostruire la soluzione che produce il valore ottimo $A[n,W]$.
L'algoritmo opera in tempo pseudopolinomiale, in quanto opera su una matrice di dimensione lineare nel valore di $W$, che è esponenziale nella lunghezza dell'input.


\subsection{\WeightDynamicKnapsack}
Una variante dell'algoritmo \DynamicKnapsack costruisce una matrice $B\in{\N^+}^{(n+1)\times (\sum_{i\in n} v_i)}$ in cui l'elemento $B[i,v]$ rappresenta il minimo peso necessario per raggiungere un valore di $v$ usando i primi $i$ oggetti.

L'elemento $B[0,0]$ è posto uguale a $0$. Gli elementi $B[0,v]$, per ogni $v\in\sum_{i\in n} v_i$, sono vincolati al valore $+\infty$ dal momento che il valore $v$ è irraggiungibile. I restanti valori sono così calcolati:
\begin{equation*}
	B[i+1,v] = \min(B[i,v], w_i+B[i,\max(v-v_i,0)])
\end{equation*}
L'algoritmo sceglie quindi la migliore soluzione tra quella che non fa uso dell'oggetto $i+1$-esimo e quella che lo aggiunge alla soluzione che utilizza i primi $i$.

Un algoritmo che usa questa matrice può trovare la soluzione ottima cercando sulla riga $n$ la colonna di indice massimo il cui valore non supera $W$. Un tale algoritmo sarebbe tuttavia poco efficiente per via dell'utilizzo di una matrice con un tale numero di colonne.

L'algoritmo può essere migliorato, a costo di perdere l'esattezza, introducendo una normalizzazione che ridimensiona i valori in base al valore massimo $v_{\max}$.
L'algoritmo risultante approssima a piacere in tempo polinomiale.

Sia $\pi=\tuple{\angle{v_i}_{i\in n},\angle{w_i}_{i\in n},W}$ un'istanza di \Knapsack. Si rimuovano eventuali pesi $i$ tali che $w_i>W$, che non contribuiscono ad alcuna soluzione ammissibile.
Dato un valore $\varepsilon\in(0,1]$, si definisce il valore di scala $\theta$ come
\begin{equation*}
	\theta:=\frac{\varepsilon v_{\max}}{2n} \text.
\end{equation*}
Si considerino le istanze $\bar\pi:=\tuple{\angle{\bar v_i=\ceil{\frac{v_i}{\theta}}\theta}_{i\in n},\angle{w_i}_{i\in n},W}$ e $\hat\pi:=\tuple{\angle{\hat v_i=\ceil{\frac{v_i}{\theta}}}_{i\in n},\angle{w_i}_{i\in n},W}$, che scalano in modo diverso i valori e lasciano invariati pesi e capacità.
Siano $\bar I\star$ e $\hat I\star$ le rispettive soluzioni ottime e $\bar v\star$ e $\hat v\star$ i loro valori.

\begin{lemma}\label{lem:knap_barhat}
	$\bar I\star = \hat I\star \land \bar v\star = \theta\hat v\star$.
\end{lemma}
\begin{proof}
	La funzione obiettivo del problema $\bar\pi$ non è altro che una dilatazione di fattore $\theta$ di quella del problema $\hat\pi$, pertanto le soluzioni ottime sono le stesse e i valori ottimi subiscono una dilatazione.
\end{proof}

\begin{lemma}
	Sia $I$ una soluzione ammissibile per $\pi$. Allora
	\begin{equation*}
		(1+\varepsilon)\sum_{i\in\hat I\star} v_i \geq \sum_{i\in I} v_i
	\end{equation*}
\end{lemma}
\begin{proof}
	\begin{align*}
		\sum_{i\in I} v_i & \leq \sum_{i\in I} \bar v_i                                          &  & \quad \text{arrotondamento per eccesso}      \\
		                  & \leq \sum_{ i\in\bar I\star} \bar v_i                                                                                  \\
		                  & = \sum_{ i\in\hat I\star} \bar v_i                                   &  & \quad \text{per lemma \ref{lem:knap_barhat}} \\
		                  & \leq \sum_{i\in\hat I\star} (v_i+\theta)                             &  & \quad \text{in quanto
		$\bar v_i=\ceil{\frac{v_i}{\theta}}\theta\leq\left(\frac{v_i}{\theta}+1\right)\theta=v_i+\theta$}                                          \\
		                  & \leq n\theta+\sum_{i\in\hat I\star} v_i                                                                                \\
		                  & = \frac{\varepsilon v_{\max}}{2} + \sum_{i\in\hat I\star} v_i \text.
	\end{align*}
	Quindi
	\begin{equation}\label{eq:knap:1}
		\sum_{i\in I} v_i \leq \frac{\varepsilon v_{\max}}{2} + \sum_{i\in\hat I\star} v_i
	\end{equation}

	\noindent Si consideri la soluzione ammissibile composta da un solo peso di valore massimo. Applicando (\ref{eq:knap:1}):
	\begin{gather}
		v_{\max} \leq \frac{\varepsilon v_{\max}}{2} + \sum_{i\in\hat I\star} v_i \leq \frac{v_{\max}}{2} + \sum_{i\in\hat I\star} v_i \nonumber \\
		\sum_{i\in\hat I\star} v_i \geq \frac{v_{\max}}{2} \label{eq:knap:2}
	\end{gather}
	Tornando alla generica soluzione ammissibile $I$ e applicando (\ref{eq:knap:1}) e (\ref{eq:knap:2}):
	\begin{gather*}
		\sum_{i\in I} v_i \leq \frac{\varepsilon v_{\max}}{2} + \sum_{i\in\hat I\star} v_i
		\leq \varepsilon\sum_{i\in\hat I\star} v_i + \sum_{i\in\hat I\star} v_i = (1+\varepsilon)\sum_{i\in\hat I\star} v_i
	\end{gather*}
\end{proof}
\begin{corollario}\label{corol:knapapprox}
	$\displaystyle(1+\varepsilon)\sum_{i\in\hat I\star} v_i \geq v\star$
\end{corollario}

\begin{theorem}
	$\Knapsack\in\FPTAS$.
\end{theorem}
\begin{proof}
	Risolvendo il problema $\hat\pi$ ottimamente con $\WeightDynamicKnapsack$ si trova una soluzione $\hat I\star$ che ha valore, nella funzione obiettivo originale, di $\sum_{i\in\hat I\star} v_i$. In virtù del corollario \ref{corol:knapapprox} tale valore produce un rapporto di approssimazione di $1+\varepsilon$.

	Per verificare che questo metodo sia effettivamente polinomiale in $n$ e $\frac{1}{\varepsilon}$, valutiamo il numero di colonne della matrice $B$.
	Nel caso peggiore tutti i pesi sono uguali a $\hat v_{\max}$, pertanto esso sarà $n \hat v_{\max}$.
	Ma vale
	\begin{equation*}
		\hat v_{\max}=\ceil{\frac{v_{\max}}{\theta}}=\ceil{\frac{2nv_{\max}}{\varepsilon v_{\max}}}=\ceil{\frac{2n}{\varepsilon}}\leq\frac{2n}{\varepsilon}+1 \text,
	\end{equation*}
	quindi la dimensione complessiva della matrice è
	\begin{equation*}
		n\cdot n\hat v_{\max}\leq n^2\left(\frac{2n}{\varepsilon}+1\right)=\frac{2n^3}{\varepsilon}+n^2
	\end{equation*}
\end{proof}

Non si può applicare alla variante originale di \DynamicKnapsack un approccio analogo a quello sopra descritto, in quanto la compressione porterebbe a un'approssimazione, invece che nei valori, nei pesi, perdendo quindi la garanzia di ammissibilità delle soluzioni ottenute.

% lezione 10 - 03/11/2021
\chapter{Algoritmi probabilistici}
Fino ad ora abbiamo considerato il modello di calcolo delle macchine di Turing per
scrivere algoritmi per problemi di ottimizzazione, in particolare algoritmi di
approssimazione:
% disegno: input x -> MdT A -> output y
\begin{figure}[h]


	\centering

	\tikzset{every picture/.style={line width=0.75pt}} %set default line width to 0.75pt        

	\begin{tikzpicture}[x=0.75pt,y=0.75pt,yscale=-1,xscale=1]
		%uncomment if require: \path (0,437); %set diagram left start at 0, and has height of 437

		%Rounded Rect [id:dp7665142760923608] 
		\draw   (280,152) .. controls (280,145.37) and (285.37,140) .. (292,140) -- (358,140) .. controls (364.63,140) and (370,145.37) .. (370,152) -- (370,188) .. controls (370,194.63) and (364.63,200) .. (358,200) -- (292,200) .. controls (285.37,200) and (280,194.63) .. (280,188) -- cycle ;
		%Straight Lines [id:da12014265713573546] 
		\draw    (170,170) -- (277,170) ;
		\draw [shift={(280,170)}, rotate = 180] [fill={rgb, 255:red, 0; green, 0; blue, 0 }  ][line width=0.08]  [draw opacity=0] (8.93,-4.29) -- (0,0) -- (8.93,4.29) -- cycle    ;
		%Straight Lines [id:da500409813700509] 
		\draw    (370,170) -- (477,170) ;
		\draw [shift={(480,170)}, rotate = 180] [fill={rgb, 255:red, 0; green, 0; blue, 0 }  ][line width=0.08]  [draw opacity=0] (8.93,-4.29) -- (0,0) -- (8.93,4.29) -- cycle    ;

		% Text Node
		\draw (311,162) node [anchor=north west][inner sep=0.75pt]   [align=left] {MdT};
		% Text Node
		\draw (214,152) node [anchor=north west][inner sep=0.75pt]   [align=left] {input};
		% Text Node
		\draw (404,152) node [anchor=north west][inner sep=0.75pt]   [align=left] {output};


	\end{tikzpicture}
	\caption{Macchina di Turing deterministica}
	\label{fig:mdtdet}
\end{figure}
gli algoritmi sono quindi \textbf{deterministici}, nonostante in alcune situazioni
gli algoritmi possano fare scelte ``arbitrarie'', che formalizziamo con la nozione
di gradi di libertà - abbiamo tuttavia dimostrato che le proprietà sono
indipendenti da queste scelte arbitrarie.

Dobbiamo riservare il termine di \textbf{non determinismo} per definire
classi di complessità per modelli di calcolo \textit{non realistici}. Estendiamo
ora il modello: abbiamo una MdT che è anche in grado di leggere un nastro su
cui sono scritti dei valori casuali, chiamato \textbf{sorgente aleatoria}.
%          sorgente aleatoria one way
%                      |
%                      v 
% disegno: input x -> MdT A -> output y
\begin{figure}[h]
	\centering



	\tikzset{every picture/.style={line width=0.75pt}} %set default line width to 0.75pt        

	\begin{tikzpicture}[x=0.75pt,y=0.75pt,yscale=-1,xscale=1]
		%uncomment if require: \path (0,437); %set diagram left start at 0, and has height of 437

		%Rounded Rect [id:dp7665142760923608] 
		\draw   (280,152) .. controls (280,145.37) and (285.37,140) .. (292,140) -- (358,140) .. controls (364.63,140) and (370,145.37) .. (370,152) -- (370,188) .. controls (370,194.63) and (364.63,200) .. (358,200) -- (292,200) .. controls (285.37,200) and (280,194.63) .. (280,188) -- cycle ;
		%Straight Lines [id:da12014265713573546] 
		\draw    (170,170) -- (277,170) ;
		\draw [shift={(280,170)}, rotate = 180] [fill={rgb, 255:red, 0; green, 0; blue, 0 }  ][line width=0.08]  [draw opacity=0] (8.93,-4.29) -- (0,0) -- (8.93,4.29) -- cycle    ;
		%Straight Lines [id:da500409813700509] 
		\draw    (370,170) -- (477,170) ;
		\draw [shift={(480,170)}, rotate = 180] [fill={rgb, 255:red, 0; green, 0; blue, 0 }  ][line width=0.08]  [draw opacity=0] (8.93,-4.29) -- (0,0) -- (8.93,4.29) -- cycle    ;
		%Straight Lines [id:da29880468042507347] 
		\draw    (330,62.29) -- (330,137) ;
		\draw [shift={(330,140)}, rotate = 270] [fill={rgb, 255:red, 0; green, 0; blue, 0 }  ][line width=0.08]  [draw opacity=0] (8.93,-4.29) -- (0,0) -- (8.93,4.29) -- cycle    ;

		% Text Node
		\draw (311,162) node [anchor=north west][inner sep=0.75pt]   [align=left] {MdT};
		% Text Node
		\draw (214,152) node [anchor=north west][inner sep=0.75pt]   [align=left] {input};
		% Text Node
		\draw (404,152) node [anchor=north west][inner sep=0.75pt]   [align=left] {output};
		% Text Node
		\draw (347,72) node [anchor=north west][inner sep=0.75pt]   [align=left] {sorgente aleatoria};
		% Text Node
		\draw    (178,37) -- (491,37) -- (491,62) -- (178,62) -- cycle  ;
		\draw (181,41) node [anchor=north west][inner sep=0.75pt]   [align=left] { 0 1 0 1 1 1 0 0 1 1 1 1 1 1 0 0 1 0 1 0 1 1 1 1 ...};


	\end{tikzpicture}
	\caption{Macchina di Turing probabilistica}
	\label{fig:mdtprob}
\end{figure}

Un algoritmo così costruito è definito \textbf{probabilistico}, in quanto l'output
sarà in funzione dell'input e del seme casuale. L'algoritmo possiede quindi
una certa distribuzione associata
$$
	P[Y = y | X = x]
$$
ossia la probabilità di avere l'output $y$ per un input $x$. Gli algoritmi
probabilistici si dividono in due famiglie:
gli algoritmi \textbf{Monte-Carlo} in cui l'output è probabilistico e gli algoritmi
\textbf{Las Vegas}, in cui l'output è deterministico, ma il tempo di esecuzione
è probabilistico. In particolare, studieremo la prima famiglia.

Questo modello di calcolo
si può applicare sia a problemi di decisione che di ottimizzazione; per applicare
questi algoritmi, vi sono due varianti: nel primo caso l'algoritmo mira ad
ottenere l'ottimo con una certa probabilità, idealmente alta - essi possono
tuttavia fallire arbitrariamente male. Alternativamente, l'algoritmo può
mirare a ottenere un'\textit{approssimazione} dell'ottimo con una certa
probabilità.

\section{Problema del taglio minimo}
\popt {MinimumCut} {$G = (V,E)$} {Sottoinsieme $X \subseteq V$}
{Qual è il \textit{taglio} minore?}
{
	$X \subseteq V$ tale che il numero di lati che hanno un vertice in $X$ e un vertice
	in $X^c$ (\textit{tagliati}) è minimo;
	definiamo
	$E_X = \{e | e \cap X \neq \emptyset \land e \cap X^* \neq \emptyset\}$
}
{$Min$}{$|E_x|$}


Le soluzioni banali sono $X = V$ e $X = \emptyset$; inoltre, una soluzione
sempre possibile è scegliere $X = \{ v \}$ per un qualsiasi $v \in V$.

\begin{theorem}
	\textsc{MinimumCutProblem} $\in \mathbf{NPO-completi}$
\end{theorem}

\subsection{Algoritmo di Karger}
L'algoritmo di Karger utiliza l'operazione di \textit{contrazione}: dato un
grafo $G$, l'operazione $G \downarrow e$ su un lato $e = \{u, v\} \in E$
unisce i due vertici $u$ e $v$, rimuovendo $e$.
\begin{figure}[h]
	\begin{center}
		\begin{tikzpicture}[scale=1, transform shape]
			\node[draw,inner sep=0pt,minimum size=5pt,fill, circle] at (0, 2)  (u1) {};
			\node[draw,inner sep=0pt,minimum size=5pt,fill, circle] at (-1, 2)  (u2) {};
			\node[draw,inner sep=0pt,minimum size=5pt,fill, circle] at (1, 2)  (u3) {};
			\node[draw,inner sep=0pt,minimum size=5pt,fill, circle, label={0:u}] at (0, 0)  (a) {};
			\node[draw,inner sep=0pt,minimum size=5pt,fill, circle, label={0:v}] at (0, 1)  (b) {};
			\node[draw,inner sep=0pt,minimum size=5pt,fill, circle] at (0, -1)  (l1) {};
			\node[draw,inner sep=0pt,minimum size=5pt,fill, circle] at (-1, -1)  (l2) {};
			\node[draw,inner sep=0pt,minimum size=5pt,fill, circle] at (1, -1)  (l3) {};

			\draw (u1) -- (b);
			\draw (u2) -- (b);
			\draw (u3) -- (b);

			\draw (l1) -- (a);
			\draw (l2) -- (a);
			\draw (l3) -- (a);


			\node at (1.75,0.5) (to) {$\implies$};
			\draw (a) edge node [left] {e} (b);
			\node[draw,inner sep=0pt,minimum size=5pt,fill, circle] at (3.5, 1.5)  (lu1) {};
			\node[draw,inner sep=0pt,minimum size=5pt,fill, circle] at (2.5, 1.5)  (lu2) {};
			\node[draw,inner sep=0pt,minimum size=5pt,fill, circle] at (4.5, 1.5)  (lu3) {};
			\node[draw,inner sep=0pt,minimum size=5pt,
			fill, circle, label={0:{u,v}}] 			at (3.5, 0.5)  (la) {};
			\node[draw,inner sep=0pt,minimum size=5pt,fill, circle] at (3.5, -0.5)  (ll1) {};
			\node[draw,inner sep=0pt,minimum size=5pt,fill, circle] at (2.5, -0.5)  (ll2) {};
			\node[draw,inner sep=0pt,minimum size=5pt,fill, circle] at (4.5, -0.5)  (ll3) {};
			\draw (lu1) -- (la);
			\draw (lu2) -- (la);
			\draw (lu3) -- (la);
			\draw (ll1) -- (la);
			\draw (ll2) -- (la);
			\draw (ll3) -- (la);
		\end{tikzpicture}
	\end{center}
	\caption{Contrazione $G\downarrow e$.}
	\label{fig:contrazione}
\end{figure}

\`E necessario che $G$ sia
un multigrafo, poiché unendo $u$ e $v$ è possibile che un terzo vertice $y$
fosse collegato sia a $u$ che a $v$; rimarranno quindi dei lati paralleli.
In caso il lato contratto fosse uno dei lati paralleli tra $u$ e $v$ anche
i lati paralleli rimanenti vengono contratti.

\begin{algorithm}[h]
	\caption{\textsc{KargerMinimumCut}}
	\label{algo:Karger}
	\KwInput{$G = (V,E)$}

	\If{$\neg G.isConnected()$}
	{
		\Return{$findConnectedComponent(G)$}
	}
	\While{$|V| > 2$}
	{
		\tcc*{Estrai un lato con una distribuzione uniforme e contrailo}
		$e = uniformExtraction(E)$

		$G = G\downarrow e$

	}

	\tcc*{Restituisci uno dei due vertici rimanenti}
	\Return{$chooseOne(V)$}

\end{algorithm}

Chiaramente, l'output dipende dalla scelta dei lati da contrarre, che è casuale.
Sia quindi $S^*$ il taglio minimo, $k^*$ il numero di lati tagliati da $S^*$ e
la serie $G_1, \cdots, G_i$ la sequenza di grafi ottenuti per ogni contrazione
operata dall'algoritmo.
\begin{oss}\label{oss:kargercontraction}
	$\forall i ~~ |G_i(V)| = n - i + 1 \land |G_i(E)| \leq m - i + 1$
\end{oss}
\begin{oss}\label{oss:kargercuts}
	Per ogni $i$, ogni taglio in $G_i$ è un taglio in $G$ dello stesso costo.
\end{oss}
\begin{oss}\label{oss:kargermindeg}
	Il grado minimo in $G_i$ è maggiore o uguale a $k^*$.
\end{oss}

\begin{lemma}\label{lem:kargeredges}
	$m - i  +1 \geq \frac{(n - i + 1) \cdot k^*}{2}$
\end{lemma}
\begin{proof}
$$
	2(m - i +1) \geq 2 \cdot |G_i(E)| = \sum_{v \in G_i(V)} d_{G_i}(v) \geq k^* (n - i + 1) \implies m - i  +1 \geq \frac{(n - i + 1) \cdot k^*}{2}
$$
\end{proof}

\begin{lemma}\label{lem:kargerprob_ei}
	Sia $E_i$ l'evento ``al passo $i$-esimo un lato $e \notin E_{S^*}$ viene contratto''.
	$$
		\forall i ~~ P[E_i | E_1, \cdots, E_{i-1}] \leq \frac{n-i-1}{n-i+1}
	$$
\end{lemma}
\begin{proof}
	\begin{align*}
		P[E_i | E_1, \cdots, E_{i -1}] & = 1 - P[\neg E_i | E_1, \cdots, E_{i -1}]           		\\
		 & = 1 - \frac{k^*}{|G_i(V)|}  																\\
		 & \geq 1 - \frac{k^*}{m - i + 1} && \text{per \cref{oss:kargercontraction}} 				\\
		 & \geq 1 - \frac{2 \cdot k^*}{k^* \cdot (n - i  + 1)} && \text{per \cref{lem:kargeredges}} \\
		 & = 1 - \frac{2}{n - i + 1} = \frac{n - i -  1}{n - i + 1}
	\end{align*}
\end{proof}

\begin{theorem}
	L'algoritmo di Karger emette l'ottimo con probabilità $p \geq \frac{1}{{n\choose{2}}}$.
\end{theorem}
\begin{proof}
	Una esecuzione dell'algoritmo che emette l'ottimo vuol dire che, ad ogni
	passo, ha selezionato un lato da contrarre $e \notin E_{S^*}$. Siamo dunque
	interessati alla probabilità dell'evento
	$E_1 \land E_2 \land \cdots \land E_{n-2}$:
	\begin{align*}
		P[E_1 \land E_2 \land \cdots \land E_{n-2}] & = P[E_1] \cdot P[E_2|E_1] \cdot \cdots \cdot P[E_{n-2}| E_{n-3}, \cdots, E_1] 				\\
		& \geq \frac{n-1-1}{n-1+1} \cdot \frac{n-2-1}{n-2+1} \cdot \cdots \cdot \frac{n-(n-2)-1}{n-(n-2)+1} & \text{applico \cref{lem:kargerprob_ei}}  \\
		& = \frac{n-2}{n} \cdot \frac{n-3}{n-1} \cdot \cdots \cdot \frac{1}{3}																		\\
		& = \frac{\Pi_{i = 1}^{n-2} i}{\Pi_{i = 3}^{n} i} = \frac{2 \cdot 1}{n \cdot (n-1)} = \frac{2}{n (n-1)} = \frac{1}{{n\choose{2}}}
	\end{align*}
\end{proof}

\begin{oss}
	Si esegua l'algoritmo di Karger ${n\choose{2}} \log(n)$ volte.
	L'ottimo si ottiene con probabilità $p \geq 1 - \frac{1}{n}$.
\end{oss}
\begin{proof}
	Ad ogni esecuzione dell'algoritmo, la probabilità di non trovare l'ottimo è
	$$
		p \leq \left(1 - \frac{1}{{n \choose{2}}}\right)
	$$
	Di conseguenza, se eseguiamo l'algoritmo ${n \choose{2}} log(n)$ volte,
	la probabilità che nessuna di queste esecuzioni trovi l'ottimo è
	$$
		p \leq \left(1 - \frac{1}{{n \choose 2}}\right)^{{n \choose{2}}log(n)}
	$$
	\'E facilmente dimostrabile, osservando il grafico della funzione, che vale
	la seguente proprietà:
	$$
		\forall x \geq 1 \text{ interi } ~~ \frac{1}{4} \leq ( 1 - \frac{1}{x})^x \leq \frac{1}{e}
	$$
	Applicando la disuguaglianza alla nostra proprietà, con $x={n \choose{2}}$
	ed elevando entrambi i lati per $log(n)$, otteniamo
	$$
		p \leq \left( 1 - \frac{1}{{n \choose 2}}\right) ^{{n \choose 2} \log (n)} \leq \left(\frac{1}{e}\right)^{\log(n)} = \frac{1}{n}
	$$
	In questo caso $p$ è la probabilità di \textit{non} trovare l'ottimo,
	dunque la probabilità di trovarlo è $\geq 1 - \frac{1}{n}$
\end{proof}

\section{Problema della copertura d'insiemi}
Abbiamo già definito il \textsc{SetCoverProblem}:
dato una serie di insiemi
$$
	S_1, \cdots, S_m \subseteq U
$$
con pesi
$$
	w_1 \cdots, w_m \in \mathbb{Q}^+
$$
definiamo $n = |U|$; vogliamo trovare un $S \subseteq m$ tale che
$$
	\bigcup_{i \in S} S_i = U
$$
e il suo costo $ \sum_{i \in S} w_i$ sia minimo.

\subsection{Algoritmo probabilistico basato sull'arrotondamento}
Il problema può essere trasposto in un problema di programmazione lineare intera:
creiamo delle variabili
$$
	x_1, \cdots, x_m
$$
tali che
$$
	\begin{cases}
		x_j \leq 1                     & \forall j = 1, \cdots, m \\
		x_j \geq 0                     & \forall j = 1, \cdots, m \\
		\sum_{i: u \in S_i} x_i \geq 1 & \forall u \in U
	\end{cases}
$$
Come sappiamo, i problemi di programmazione lineare intera appartengono alla
classe \textbf{NP-Completi}. Consideriamo dunque il problema $\Pi$ nella
sua versione non intera, $\Pi_{PL}$.

\`E importante richiamare ora alcune proprietà statistiche.
\begin{theorem}[Disuguaglianza di Markov]\label{thm:markov}
	Per ogni variabile aleatoria $X$ non negativa e per ogni $\alpha > 0$
	$$
		P[X \geq \alpha] \leq \frac{E[X]}{\alpha}
	$$
\end{theorem}

\begin{theorem}[Union bound o disuguaglianza di Boole]\label{thm:boole}
	$$
		P[\bigcup_{i} E_i] \leq \sum_i P[E_i]
	$$
\end{theorem}

\begin{algorithm}[h]
	\caption{\textsc{ProbabilisticRoundingSetCover}}
	\label{algo:ProbRoundingSetCover}
	\KwInput{$S_1, \cdots, S_m$, $w_1, \cdots, w_m$  e un intero $k$}

	$\hat{x_1}, \cdots, \hat{x_m} = solve(\Pi_{PL}) $

	$S =  \emptyset$

	\For {$t = \{1, \lceil k + \log(n) \rceil \}$}
	{

		\For {$i = 1, ..., m$}
		{
			\tcc*{Inserisci $i$ in $S$ con probabilità $\hat{x_i}$}
			$S = probInsert(S, i, \hat{x_i})$
		}
	}

	\Return {$S$}
\end{algorithm}

L'\cref{algo:ProbRoundingSetCover} potrebbe trovare una soluzione non ammissibile. Dimostriamo
ora che \textit{spesso} è ammissibile e che quando è ammissibile è una soluzione molto buona.

\begin{theorem}
	La probabilità che l'\cref{algo:ProbRoundingSetCover} produca una soluzione ammissibile è
	$$
		p \geq 1 - e ^ {-k}
	$$
	parametrica in $k$, che determina il numero di tentativi di inserimento.
\end{theorem}
\begin{proof}
	$$
		P[\text{soluzione ammissibile}] = 1 - P[\text{almeno un elemento dell'universo non è coperto}]
	$$
	chiamiamo $B_u$ l'evento per cui $u$ non è coperto nella soluzione. Allora
	$$
		P[\text{soluzione ammissibile}] = 1 - P[\bigcup_{u \in U} B_u]
	$$
	Possiamo quindi usare il \cref{thm:boole}:
	$$
		1 - P[\bigcup_{u \in U} B_u] \geq 1 - \sum_{u \in U} P[B_u]
	$$
	L'elemento $u$ non è coperto quando nessuno degli elementi che contentono
	$u$ non è stato scelto. Ad ogni iterazione, che sono
	$\lceil k + \log(n) \rceil$ in totale, ogni insieme ha probabilità
	$\hat{x_i}$ di essere scelto:
	\begin{align*}
		1 - \sum_{u \in U} \prod_{i: u \in S_i} P[S_i \text{ non è stato scelto}] & = 1 - \sum_{u \in U} \prod_{i: u \in S_i} (1 - \hat{x_i})^{\lceil k + \log(n) \rceil}	\\
		& \geq 1 - \sum_{u \in U} \prod_{i : u \in S_i} e^{-\hat{x_i} (k + \log(n))} & \text{nota che } 1 - x \leq e^{-x}													\\
		& = 1 - \sum_{u \in U} \exp(- (k + \log (n)) \sum_{i: u \in S_i} \hat{x_i})
	\end{align*}
	Siccome $S_i$ è ammissibile, deve essere $\sum_{i: u \in S_i} \hat{x_i}  \geq 1$, quindi
	\begin{align*}
		\geq 1 - \sum_{u \in U} e^{- (k + \log(n))} = 1 - \sum_{u \in U} \frac{e^{-k}}{n}  = 1 - e^{-k} \frac{1}{n}|U| = 1 - e^{-k}
	\end{align*}
\end{proof}

\begin{theorem} \label{thm:ProbRoundingSetCoveralpha}
	$$\forall \alpha  ~~ 0 \leq \alpha \leq 1 ~~ P[\frac{v_{out}}{v^*}\geq \alpha (k + \log(n))] \leq \frac{1}{\alpha}$$
\end{theorem}
\begin{proof}
	Abbiamo che $\hat{v} = \sum_{i} w_i \hat{x_I} \leq v^*$; inoltre,
	la probabilità che $S_i$ venga scelto è

	\begin{align*}
		P[ S_i \text{ sia scelto}] & = P[ \bigcup_{t} S_i \text{ sia scelto durante l'iterazione } t] 	\\
		& \leq \sum_t P[S_i \text{ sia scelto durante l'iterazione } t] & \text{per \cref{thm:boole}}	\\ 
		& = (k + \log (n))\hat{x}_i
	\end{align*}

	Per poter utilizzare il \cref{thm:markov} calcoliamo il valore atteso di $v_{out}$:
	\begin{align*}
		E[v_{out}] & = \sum_{i} w_i P[S_i \text{ sia scelto}] 							\\
		& \leq \sum_i w_i (k + \log(n)) \hat{x_i} & \text{per osservazione precedente} 	\\
		& = \hat{v} (k + \log(n)) & \text{ricorda che } \hat{v}=\sum_i w_i \hat{x_i} 	\\
		& \leq v^* (k + \log(n))
	\end{align*}

	Infine utilizziamo questa osservazione per determinare la probabilità che
	la nostra soluzione approssimi bene il problema:
	\begin{align*}
		P [ \frac{v_{out}}{v^*} \geq \alpha (k + \log(n))] & = P [ v_{out} \geq v^* \alpha (k + \log(n)) ] 	\\
		& \leq \frac{E[v_{out}]}{\alpha (k + \log(n))v^*} & \text{applico Markov} 							\\
		& \leq \frac{v^* (k + \log(n))}{\alpha (k + \log(n))v^*} & \text{applico l'osservazione precedente}	\\ 
		& = \frac{1}{\alpha}		
	\end{align*}
\end{proof}

\begin{oss}
	Se si esegue l'\cref{algo:ProbRoundingSetCover} con $k = 3$ c'è il $45\%$ di probabilità di ottenere
	una soluzione ammissibile con fattore di approssimazione $\frac{v_{out}}{v^*}\leq 6 + 2 \log (n)$.
\end{oss}
\begin{proof}
	Sia $E_{ammissibile}$ l'evento per cui si ottiene una soluzione ammissible e
	$E_{buona}$ l'evento per cui l'output sia entro $6 + 2 \log(n)$ dall'ottimo.
	Ci interessa
	\begin{align*}
		 & P[E_{ammissibile} \land E_{buona}] = 1 - P[\neg E_{ammissible} \lor
		\neg E_{buona}] \geq 1 - P[\neg E_{ammissibile}] - P[\neg E_{buona}]                                  \\
		 & \geq 1 - e^{-3} - \frac{1}{2} \approx 0.45  ~~~  (\text{dal \cref{thm:ProbRoundingSetCoveralpha}})
	\end{align*}
\end{proof}

% lezione 13 - 10-11-2021 
\section{Problema MaxEkSat}
\textsc{MaxEkSat} è la versione $k$-indicizzata di \textsc{MaxSat}: date $n$ clausole
di $k$ letterali ciascuna, l'obiettivo è massimizzare il numero di clausole
soddisfatte.

Definiamo $x_1, \cdots, x_n$ le formule che compaiono nella formula
e $c_1, \cdots, c_t$ le clausole della formula.

\begin{theorem}\label{thm:maxeksatnp}
	\textsc{MaxEkSat} $\in \mathbf{NPO-completi}$ per $k \leq 3$.
\end{theorem}

\subsection{Algoritmo probabilistico}
\begin{theorem}\label{thm:probassgn}
	Un assegnamento casuale soddisfa, in media,
	$$
		E[T] = \frac{2^k-1}{k} t
	$$
	clausole.
\end{theorem}

Per la dimostrazione definiamo
$$
	X_i ~ Unif\{0,1\}
$$
il valore assegnato ad ogni $x_i$ probabilisticamente;
$$
	C_i =
	\begin{cases}
		0 & C_i \text{ non soddisfatto} \\
		1 & \text{ altrimenti}
	\end{cases}
$$
mentre $T$ è il numero di clausole soddisfatte.
E l'algoritmo si svolge, banalmente, estraendo un valore per ogni $x_i$ e restituendo il numero
di clausole soddisfatte.
Prima di procedere con la dimostrazione, ricordiamo
la seguende legge statistica:
\begin{theorem}\label{thm:leggevat}
	Sia $\{\mathcal{E}_i\}_{i=1..k}$ una partizione dell'universo degli eventi.
	Allora
	$$
		E[X] = \sum_{i = 1}^{k} E[X|\mathcal{E}_i]P[\mathcal{E}_i]
	$$
\end{theorem}

\begin{proof}
	La partizione $\{\mathcal{E}_i\}_i$ nel contesto di \textsc{MaxEkSat} è l'insieme
	dei possibili assegnamenti $X_1 = b_1, \cdots, X_n = b_n$; pertanto
	\begin{align*}
		E[T] & = \sum_{b_1 \in 2} \cdots \sum_{b_n\in 2} E[T|X_i = b_1, \cdots, X_n = b_n]P[X_1 = b_1, \cdots, X_n = b_n]   \\
		     & = \sum_{b_1 \in 2} \cdots \sum_{b_n\in 2} E[T|X_i = b_1, \cdots, X_n = b_n]P[X_1 = b_1] \cdots P[X_n = b_n]  \\
		     & = \frac{1}{2^n}\sum_{b_1 \in 2} \cdots \sum_{b_n\in 2} E[T|X_i = b_1, \cdots, X_n = b_n]                     \\
		     & = \frac{1}{2^n}\sum_{j = 1}^{t}(\sum_{b_1 \in 2} \cdots \sum_{b_n\in 2} E[C_j|X_i = b_1, \cdots, X_n = b_n]) \\
		     & = \frac{1}{2^n} \sum_{j = 1}^t (2^n - 2^{n -k}) = \frac{2^n - 2^{n-l}}{2^n}t =
		\frac{2^k - 1}{k}t
	\end{align*}
\end{proof}

\subsection{Algoritmo derandomizzato}
\begin{theorem}\label{thm:maxsatderandomexv}
	Per ogni $j  = 0, \cdots, n$ esistono $b_1, \cdots b_j \in 2$ tali che
	$$
		E[T|X_1 = b_1, \cdots, X_j = b_j] \geq \frac{2^k-1}{2^k}t
	$$
\end{theorem}
Questo significa che non solo il numero atteso di clausole è \textit{alto}, ma
possiamo inoltre fissare le prime $j$ variabili in modo che questa proprietà
continui ad essere preservata.
\begin{proof}
	Per induzione su $j$.
	Base: per $j = 0$ verificato dal \cref{thm:probassgn}.
	Passo induttivo: per ipotesi induttiva vale
	$$
		E[T|X_1 = b_1, \cdots, X_{j-1} = b_{j-1}] \geq \frac{2^k-1}{k}
	$$
	Per il \cref{thm:leggevat} vale
	\begin{align*}
		 & E[T|X_1 = b_1, \cdots, X_{j-1} = b_{j-1}] =                         \\
		 & = E[T | X_1 = b_1, \cdots X_{j-1} = b_{j-1}, X_j = 0]P[X_j = 0] +
		E[T | X_1 = b_1, \cdots X_{j-1} = b_{j-1}, X_j = 1]P[X_j = 1]          \\
		 & = \frac{1}{2} E[T | X_1 = b_1, \cdots X_{j-1} = b_{j-1}, X_j = 0] +
		\frac{1}{2} E[T | X_1 = b_1, \cdots X_{j-1} = b_{j-1}, X_j = 1]        \\
		 & = \frac{1}{2} \alpha_0 + \frac{1}{2} \alpha_1
	\end{align*}
	Deve essere che per qualche $i \in 2$
	$$
		\alpha_i \geq \frac{2^k -1 }{2^k} t
	$$
	supponiamo $\alpha_0$ e $\alpha_1$ minori: allora
	$$
		\frac{1}{2} \alpha_0 + \frac{1}{2} \alpha_1 < \frac{1}{2}\frac{2^k -1 }{2^k} t + \frac{1}{2} \frac{2^k -1 }{2^k} t = \frac{2^k -1 }{2^k} t
	$$
	contraddicendo l'ipotesi induttiva.
\end{proof}

\begin{algorithm}[h]
	\caption{\textsc{DerandomMaxEkSat}}
	\label{algo:DerandomMaxEkSat}

	$D = \emptyset$ \tcp*{indici delle clausole già determinate}

	\For{$i = 1,\cdots, n$}
	{
		$\Delta_0 = 0$

		$\Delta_1 = 0$

		$\Delta D_0 = \emptyset$

		$\Delta D_1 = \emptyset$

		\For {$ j = 1, \cdots, t$}
		{
			\If{$ j \in D$}
			{
				\textbf{continue}
			}
			\If { $x_i \notin c_j$}
			{
				\textbf{continue}
			}

			$h = findGeq(i, c_j)$ \tcp*{variabili con indice $\geq i$ in $c_j$}

			\If {$x_i \in c_j \land x_i = 1$}
			{

				$\Delta_0 = \Delta_0 - \frac{1}{2^h}$

				$\Delta_1 = \Delta_1 + \frac{1}{2^h}$

				$\Delta D_1 = \Delta D_1 \cup \{j\}$

			} \Else {
				$\Delta_0 = \Delta_0 + \frac{1}{2^h}$

				$\Delta_1 = \Delta_1 - \frac{1}{2^h}$

				$\Delta D_0 = \Delta D_0 \cup \{j\}$

			}

		}

		$ u = \arg \max_{0,1} \{\Delta_0, \Delta_1\}$

		$x_i = u$

		$D = D \cup \Delta D_u$
	}
\end{algorithm}

\begin{theorem}
	L'\cref{algo:DerandomMaxEkSat} trova un assegnamento che soddisfa
	$$
		\frac{2^k -1}{2^k} t
	$$
	clausole.
\end{theorem}
\begin{proof}
	Banale dal \cref{thm:maxsatderandomexv}.
\end{proof}
% lezione 14 - 12-11-2021
\begin{lemma}
	L'\cref{algo:DerandomMaxEkSat} garantisce che quando l'$i$-esima variabile
	viene assegnata vale
	$$
		E[T|X_1 = x[1], \cdots, X_{i-1}[x_{i-1}]] \geq \frac{2^k -1}{2^k}t
	$$
\end{lemma}
\begin{proof}
	Per $i = 0$, questo è vero per il \cref{thm:maxsatderandomexv}.
	Per il passo induttivo, supponiamo
	$E[T|X_1 = x_1, \cdots, X_{i-1}= x_{i-1}] \geq \frac{2^k -1}{2^k}t$.
	Che valore assegnare a $X_i$?
	Supponiamo che la clausola alla quale appartiene $X_i$ abbia $h$ variabili
	non assegnate $X_j$ con $j \geq i$ (le rimanenti hanno assegnato il valore $0$);
	ci sono allora $2^h$ possibili assegnamenti:
	di questi $2^h$, soltanto uno (tutte le $X_j$ false) rende falsa la clausola, mentre
	$\frac{2^h - 1}{2^h}$ rendono la clausola vera.
	Se si assegna $1$ a $X_i$ allora avrà contributo $1$. L'incremento di
	valore atteso sarà
	$$
		\Delta E[T] = 1 - \frac{2^h -1}{2^h} = \frac{1}{2^h}
	$$
	Viceversa, se si assegna a $X_i$ il valore $0$ la clausola rimarrà incerta e le
	variabili libere si riducono di uno: le possibili combinazioni che renderanno
	vera la clausola saranno $2^{h-1} -1 $. In questo caso, l'incremento nel valore
	atteso è
	$$
		\Delta E[T] = \frac{2^{h-1} -1}{2^{h-1}}  - \frac{2^h -1}{2^h} = - \frac{1}{2^h}
	$$
\end{proof}
\begin{corollario}
	L'\cref{algo:DerandomMaxEkSat} fornisce una $\frac{2^k}{2^k -1}$-approssimazione per
	\textsc{MaxEkSat}.
\end{corollario}
\begin{proof}
	Sia $t^*$ il numero ottimo di clausole soddisfatte. Si ha ovviamente $t^* \leq t$.
	Abbiamo, allora
	$$
		\frac{t^*}{t_{algo}} \leq \frac{t}{t_{algo}} \leq \frac{t}{\frac{2^k -1}{2^k} t}  = \frac{2^k}{2^k -1}.
	$$
\end{proof}

\section{Il teorema PCP}
Un punto fondamentale nell'analisi degli algoritmi proposta in questo corso sarà
l'analisi dell'inapprossimabilità di alcuni problemi. Per poter analizzare la questione
sarà necessario introdurre uno dei più importanti teoremi della teoria della complessità
dal teorema di Cook: il teorema PCP, \textit{probabilistically checkable proofs}.
Il punto di partenza per arrivare all'enunciato di PCP è descrivere un'estensione delle
macchine di Turing deterministiche.

\subsection{Macchine di Turing oracolari}
Le MdT oracolari ricevono in input un certo $x \in 2^*$ e in output restituiscono
un valore booleano, una risposta $True$ o $False$ per un certo problema di decisione.
Queste MdT hanno però anche accesso ad una \textit{stringa} o
\textit{nastro dell'oracolo} $w \in 2^*$;
se vuole accedere alla stringa dell'oracolo, la MdT ha un nastro speciale,
definito \textit{nastro di query} sul quale all'occorrenza può scrivere un numero
binario; una volta scritto tale numero, la macchina entra in uno stato speciale
che ``aspetta'' la risposta dell'oracolo: la MdT userà il numero scritto sul nastro
di query come posizione in cui leggere $w$, che conterrà un $1$ o uno $0$.
La MdT passerà ad uno stato relativo al numero trovato in $w$:
$$
	\text{stato speciale di interrogazione: } q?
	\begin{cases}
		w[n] = 1 & \rightarrow q_1 \\
		w[n] = 0 & \rightarrow q_0
	\end{cases}
$$
\begin{figure}[h]
	\begin{center}
		\tikzset{every picture/.style={line width=0.75pt}} %set default line width to 0.75pt        

		\begin{tikzpicture}[x=0.75pt,y=0.75pt,yscale=-1,xscale=1]
			%uncomment if require: \path (0,437); %set diagram left start at 0, and has height of 437

			%Rounded Rect [id:dp7665142760923608] 
			\draw   (280,152) .. controls (280,145.37) and (285.37,140) .. (292,140) -- (358,140) .. controls (364.63,140) and (370,145.37) .. (370,152) -- (370,188) .. controls (370,194.63) and (364.63,200) .. (358,200) -- (292,200) .. controls (285.37,200) and (280,194.63) .. (280,188) -- cycle ;
			%Straight Lines [id:da12014265713573546] 
			\draw    (170,168.67) -- (277,168.67) ;
			\draw [shift={(280,168.67)}, rotate = 180] [fill={rgb, 255:red, 0; green, 0; blue, 0 }  ][line width=0.08]  [draw opacity=0] (8.93,-4.29) -- (0,0) -- (8.93,4.29) -- cycle    ;
			%Straight Lines [id:da500409813700509] 
			\draw    (370,168.67) -- (477,168.67) ;
			\draw [shift={(480,168.67)}, rotate = 180] [fill={rgb, 255:red, 0; green, 0; blue, 0 }  ][line width=0.08]  [draw opacity=0] (8.93,-4.29) -- (0,0) -- (8.93,4.29) -- cycle    ;
			%Straight Lines [id:da6813515968969535] 
			\draw    (320.13,199.88) -- (320.13,214.71) -- (320.13,229.21) -- (454.13,229.85) -- (454.5,229.85) ;
			\draw [shift={(457.5,229.85)}, rotate = 180] [fill={rgb, 255:red, 0; green, 0; blue, 0 }  ][line width=0.08]  [draw opacity=0] (8.93,-4.29) -- (0,0) -- (8.93,4.29) -- cycle    ;
			%Straight Lines [id:da0793943852727933] 
			\draw    (500,242.31) -- (500,319.25) -- (226,319.25) -- (226,285.75) ;
			\draw [shift={(226,282.75)}, rotate = 90] [fill={rgb, 255:red, 0; green, 0; blue, 0 }  ][line width=0.08]  [draw opacity=0] (8.93,-4.29) -- (0,0) -- (8.93,4.29) -- cycle    ;

			% Text Node
			\draw (311,160.67) node [anchor=north west][inner sep=0.75pt]   [align=left] {MdT};
			% Text Node
			\draw (204.67,144) node [anchor=north west][inner sep=0.75pt]   [align=left] {$\displaystyle x\ \in 2^{*}$};
			% Text Node
			\draw (383.83,144.5) node [anchor=north west][inner sep=0.75pt]   [align=left] {$\displaystyle True/False$};
			% Text Node
			\draw    (179.33,257) -- (462.33,257) -- (462.33,281) -- (179.33,281) -- cycle  ;
			\draw (182.33,261.4) node [anchor=north west][inner sep=0.75pt]    {$1\ 0\ 1\ 1\ 0\ 0\ 0\ 0\ 1\ 1\ 1\ 1\ 1\ 0\ 0\ 0\ 0\ 1\ 1\ 1\ ...\ $};
			% Text Node
			\draw (262,284.83) node [anchor=north west][inner sep=0.75pt]   [align=left] {nastro dell'oracolo $\displaystyle w\in 2^{*}$};
			% Text Node
			\draw    (458,218) -- (557,218) -- (557,242) -- (458,242) -- cycle  ;
			\draw (461,222.4) node [anchor=north west][inner sep=0.75pt]    {$0\ 0\ \cdots \ 0\ 1\ 1$};
			% Text Node
			\draw (458,192) node [anchor=north west][inner sep=0.75pt]   [align=left] {nastro di query};
			% Text Node
			\draw (392.5,322) node [anchor=north west][inner sep=0.75pt]   [align=left] {$\displaystyle 3^{a}$ posizione};


		\end{tikzpicture}

	\end{center}
	\caption{La MdT è dotata di un nastro di query sulla quale scrive un numero quando necessario e, in base al numero scritto, accede al nastro dell'oracolo.}
	\label{fig:mdtoracle}
\end{figure}

Le MdT con oracolo sono il modo moderno per definire le classi nondeterministiche di macchine
di Turing.

\begin{theorem}
	Un linguaggio binario $L \subseteq 2^*$ appartiene a $\mathbf{NP}$ se e solo
	se esiste una MdT oracolare $v$ tale che:
	\begin{itemize}
		\item $v(x,w)$ termina in un numero polinomiale nella lunghezza $|x|$; e
		\item $\forall x \in 2^* ~ \exists w \in 2^* : v(w,x) = True$
		      se e solo se $x \in L$.
	\end{itemize}
\end{theorem}

\subsection{Probabilistic checkers}
Un'estensione delle MdT oracolari sono i \textit{probablistic checkers}:
anch'essi hanno accesso ad un oracolo e, in più, possono accedere ad una
\textit{sorgente di bit casuali} forniti su un nastro apposito. Nuovamente,
questo verificatore emetterà un valore tra $True$ e $False$; il suo comportamento,
ovviamente, dipenderà da $x$, $w$, e $r \in 2^*$, la stringa casuale.
\begin{figure}[h]
	\begin{center}
		\tikzset{every picture/.style={line width=0.75pt}} %set default line width to 0.75pt        

		\begin{tikzpicture}[x=0.75pt,y=0.75pt,yscale=-1,xscale=1]
			%uncomment if require: \path (0,437); %set diagram left start at 0, and has height of 437

			%Rounded Rect [id:dp7665142760923608] 
			\draw   (280,152) .. controls (280,145.37) and (285.37,140) .. (292,140) -- (358,140) .. controls (364.63,140) and (370,145.37) .. (370,152) -- (370,188) .. controls (370,194.63) and (364.63,200) .. (358,200) -- (292,200) .. controls (285.37,200) and (280,194.63) .. (280,188) -- cycle ;
			%Straight Lines [id:da12014265713573546] 
			\draw    (170,168.67) -- (277,168.67) ;
			\draw [shift={(280,168.67)}, rotate = 180] [fill={rgb, 255:red, 0; green, 0; blue, 0 }  ][line width=0.08]  [draw opacity=0] (8.93,-4.29) -- (0,0) -- (8.93,4.29) -- cycle    ;
			%Straight Lines [id:da500409813700509] 
			\draw    (370,168.67) -- (477,168.67) ;
			\draw [shift={(480,168.67)}, rotate = 180] [fill={rgb, 255:red, 0; green, 0; blue, 0 }  ][line width=0.08]  [draw opacity=0] (8.93,-4.29) -- (0,0) -- (8.93,4.29) -- cycle    ;
			%Straight Lines [id:da6813515968969535] 
			\draw    (320.13,199.88) -- (320.13,214.71) -- (320.13,229.21) -- (454.13,229.85) -- (454.5,229.85) ;
			\draw [shift={(457.5,229.85)}, rotate = 180] [fill={rgb, 255:red, 0; green, 0; blue, 0 }  ][line width=0.08]  [draw opacity=0] (8.93,-4.29) -- (0,0) -- (8.93,4.29) -- cycle    ;
			%Straight Lines [id:da0793943852727933] 
			\draw    (500,242.31) -- (500,319.25) -- (226,319.25) -- (226,285.75) ;
			\draw [shift={(226,282.75)}, rotate = 90] [fill={rgb, 255:red, 0; green, 0; blue, 0 }  ][line width=0.08]  [draw opacity=0] (8.93,-4.29) -- (0,0) -- (8.93,4.29) -- cycle    ;
			%Straight Lines [id:da08824377097646874] 
			\draw    (294.61,80.53) -- (294.61,111.03) -- (325.61,111.03) -- (325.61,137.03) ;
			\draw [shift={(325.61,140.03)}, rotate = 270] [fill={rgb, 255:red, 0; green, 0; blue, 0 }  ][line width=0.08]  [draw opacity=0] (8.93,-4.29) -- (0,0) -- (8.93,4.29) -- cycle    ;

			% Text Node
			\draw (311,160.67) node [anchor=north west][inner sep=0.75pt]   [align=left] {MdT};
			% Text Node
			\draw (204.67,144) node [anchor=north west][inner sep=0.75pt]   [align=left] {$\displaystyle x\ \in 2^{*}$};
			% Text Node
			\draw (383.83,144.5) node [anchor=north west][inner sep=0.75pt]   [align=left] {$\displaystyle True/False$};
			% Text Node
			\draw    (179.33,257) -- (462.33,257) -- (462.33,281) -- (179.33,281) -- cycle  ;
			\draw (182.33,261.4) node [anchor=north west][inner sep=0.75pt]    {$1\ 0\ 1\ 1\ 0\ 0\ 0\ 0\ 1\ 1\ 1\ 1\ 1\ 0\ 0\ 0\ 0\ 1\ 1\ 1\ ...\ $};
			% Text Node
			\draw (262,284.83) node [anchor=north west][inner sep=0.75pt]   [align=left] {nastro dell'oracolo $\displaystyle w\in 2^{*}$};
			% Text Node
			\draw    (458,218) -- (557,218) -- (557,242) -- (458,242) -- cycle  ;
			\draw (461,222.4) node [anchor=north west][inner sep=0.75pt]    {$0\ 0\ \cdots \ 0\ 1\ 1$};
			% Text Node
			\draw (458,192) node [anchor=north west][inner sep=0.75pt]   [align=left] {nastro di query};
			% Text Node
			\draw (392.5,322) node [anchor=north west][inner sep=0.75pt]   [align=left] {$\displaystyle 3^{a}$ posizione};
			% Text Node
			\draw    (129.94,55.78) -- (419.94,55.78) -- (419.94,79.78) -- (129.94,79.78) -- cycle  ;
			\draw (132.94,60.18) node [anchor=north west][inner sep=0.75pt]    {$1\ 1\ 1\ 1\ 0\ 0\ 1\ 1\ 0\ 1\ 0\ 1\ 1\ 0\ 1\ 1\ 1\ 0\ 0\ 0\ \cdots $};
			% Text Node
			\draw (129.11,31.28) node [anchor=north west][inner sep=0.75pt]   [align=left] {sorgente di bit random $\displaystyle r\in 2^{*}$};


		\end{tikzpicture}
	\end{center}
	\caption{I probabilistic checkers hanno accesso alla sorgente casuale e alle informazioni
		dell'oracolo.}
	\label{fig:probcheck}
\end{figure}

\subsubsection{Sottoclassi di PCP}
In particolare, ci interessano i PC che effettuano un numero massimo di
accessi alla stringa casuale e all'oracolo: definiamo
$\mathbf{PCP}[r,q]$ come la classe dei verificatori che leggono al massimo $r$ bit random
ed effettuano al massimo $q$ query all'oracolo. Utilizziamo inoltre la stessa
notazione per identificare i linguaggi accettabili dalle macchine così definite:
un linguaggio $L$ è  in $\mathbf{PCP}[r,q]$ se e solo se esiste una macchina
$v \in \mathbf{PCP}[r,q]$ tale che
\begin{enumerate}
	\item $v(x,R,W)$ funziona in tempo polinomiale in $|x|$,
	\item $v(x,R,W)$ effettua al massimo un numero proporzionale
	      a $q$ e $|x|$ query,
	\item $v(x,R,W)$ legge al massimo un numero proporzionale a
	      $r$ e $|x|$ bit casuali, infine
	\item se $x \in L$ esiste una $w \in 2^*$ tale che $v$ accetta $x$ con probabilità
	      $1$ -- cioè $v(x, -, W) = True$. Viceversa, se $x \notin L$, $v$ rifiuta
	      con probabilità $\geq \frac{1}{2}$ per qualunque $w$.
\end{enumerate}
In altre parole, fissando $x$ e $w$, a seconda di quale delle $2^{r(|x|)}$\footnote{Accadrà
	di utilizzare la notazione $r(|x|)$ o $q(|x|)$ nonostante $r$ e $q$ siano
	state definite come costanti e non come funzioni: l'interpretazione è considerarle come
	funzioni che restituiscono un naturale proporzionale sia a $x$ che a $r$  (o $q$).}
possibili stringhe casuali è a disposizione la macchina $v$ accetta o rifiuta:
la probablità di accettazione è il numero di sequenze random per cui per gli specifici
$x$ e $w$ si accetta sul numero totale di sequenze possibli.

Alcune sottoclassi sono interessanti: $\mathbf{PCP}[0,0]$ è una normale macchina deterministica,
pertanto la classe di linguaggi $\textbf{PCP}[0,0]$ è $\textbf{P}$.
$\textbf{PCP}[0, Poly]$ è una macchina nondeterministica senza stringhe casuali che riconosce
i linguaggi $\textbf{PCP}[0, Poly]$, ossia $\textbf{NP}$.

\subsection{Enunciato di PCP}
\begin{theorem}[Arora, Safra 1998: PCP]\label{thm:pcp}
	$\textbf{NP} = \textbf{PCP}[O(log(n)), O(1)]$.
\end{theorem}
\begin{proof}
	Omessa.
\end{proof}

In pratica,  dato un $L \in \mathbf{NP}$, si può costruire un probabilistic checker $v$
che necessita una quantità logaritmica in $|x|$ di bit casuali e che accede ad una stringa
oracolare di lunghezza finita che funziona ``come promesso'': se $x \in L$ esiste un $w$
per il quale $v$ accetterà con probabilità $1$, mentre
se $x \notin L$ la macchina rifiuterà con probabilità almeno $\frac{1}{2}$.
Questo determina il \textit{tradeoff} tra casualità e nondeterminismo: è più utile
avere informazione casuale piuttosto che la stessa ``quantità'' di nondeterminismo.

Si può inoltre notare che questo significa che l'albero di query nondeterministiche
ha un'altezza finita e nota aprioristicamente.

\subsubsection{Verificatori in forma normale}
Assumeremo, senza perdita di generalità, che
$v$ usi \textit{esattamente} $r(|x|)$ (che sarà sempre $O(\log n)$) bit
random e che effettui \textit{esattamente} $q \in \mathbb{N}$ query all'oracolo,
ossia i probabilistic checker che esamineremo saranno $v \in \mathbf{PCP}[r(n), q]$
con $r(n) \in O(\log n)$.
Inoltre, assumeremo anche che:
\begin{itemize}
	\item $v$ estrae tutti gli $r(|x|)$ bit random all'inizio;
	\item $v$, dopo aver estratto i bit random, effettua tutte le $q$ query all'oracolo;
	      le query, pertanto, non potranno essere adattive e il verificatore
	      dovrà effettuare, in caso, tutte le $2^q$ possibili chiamate all'inizio.
\end{itemize}
Un verificatore che si comporta in questo modo è definito \textit{verificatore in forma normale}.

% lezione 15 - 17-11-2021
% disegno matrice pagina 1 Alg17 Nov 2021
\subsubsection{Esemplificazione dei probabilistic checkers}
Sia $L \in \mathbf{PCP}[r(n), q(n)]$. Analizziamo cosa accade, secondo il
\cref{thm:pcp}, per una qualsiasi $x \in L$; ipotizziamo che $r(|x|) =  17$,
quindi vi saranno $2^{17}$ possibili $r$ e $q(|x|) = 15$, quindi vi saranno
$2^{15}$ possibili $w$.

\paragraph{$x \in L$}
In questo caso, tra le $2^{15}$ possibili, deve esistere una stringa $w$ tale per cui $v$, compiute le $15$ richieste, accetta con probabilità $1$.
Per ognuna di queste esistono $2^{17}$ possibili stringhe random\footnote{Descrivendo la forma normale abbiamo
	definito il comportamento del checker nella maniera esattamente opposta, ossia prima si estraggono informazioni
	dalla sorgente casuale e poi si effettuano le richieste all'oracolo; per ora, a scopo illustrativo,
	ipotizziamo tacitamente l'opposto.} e per ogni $w$ ognuna di queste può conferire una diversa probabilità
di accettazione, tranne nel caso di ``\textit{quella}'' stringa $w$ che conferisce la probabilità
di accettazione $1$.
\begin{table}[h]
	\centering
	\begin{tabular}{c|c|c|c|c|c}
		\textbf{risposte dell'oracolo} & $000 \cdots 000$                & $\cdots$ & $i_0i_1i_2\cdots i_{14}i_{15}i_{16}$ & $\cdots$ & $000\cdots000$            \\ \cline{1-1}
		\textbf{spazio dei bit random} & \tikz\pic{sema=white/90/black}; & $\cdots$ & \tikz\pic{sema=black/90/};           & $\cdots$ & \tikz\pic{sema=white/0/};
	\end{tabular}
	\caption{Rappresentazione delle associazioni $w$-$r$: nelle aree di  probabilità, la
		parte nera rappresenta le stringhe $r$ che causano l'accettazione, mentre la parte bianca rappresenta
		le stringhe $r$ che, tra tutte le $2^{17}$ possibili, causano la non accettazione.}
\end{table}


\paragraph{$x \notin L$}
In questo caso, tra le $2^{15}$ possibili, non può esistere una stringa $w$ tale per cui $v$, compiute le $15$ richieste, accetta con probabilità
maggiore di $1/2$.
\begin{table}[h]
	\centering
	\begin{tabular}{c|c|c|c|c|c}
		\textbf{risposte dell'oracolo} & $000 \cdots 000$                & $\cdots$ & $i_0i_1i_2\cdots i_{14}i_{15}i_{16}$ & $\cdots$ & $000\cdots000$                   \\ \cline{1-1}
		\textbf{spazio dei bit random} & \tikz\pic{sema=white/90/black}; & $\cdots$ & \tikz\pic{sema=white/45/black};      & $\cdots$ & \tikz\pic{sema=white/180/black};
	\end{tabular}
\end{table}

\section{Inapprossimabilità}
\subsection{MaxEkSat}
La prima applicazione del \cref{thm:pcp} che affronteremo è l'inapprossimabilità di
\textsc{MaxEkSat}. La conclusione alla quale arriveremo è che l'\cref{algo:DerandomMaxEkSat} derandomizzato
per \textsc{MaxEkSat} è \textit{ottimo}, ossia non si può fare meglio di così.
Il punto di partenza per questa dimostrazione è scegliere un linguaggio
$L \in \mathbf{NP-completi}$ (quindi anche $L \in \mathbf{PCP}[O(\log(n), O(1)]$ se $\mathbf{P} \neq \mathbf{NP}$).
Consideriamo la macchina $v \in \mathbf{PCP}[O(\log(n)), O(1)]$ e consideriamo un
certo input $z \in 2^*$, per il quale genereremo una sequenza
di $r(z)$ bit random -- le sequenze possibili, che denotano lo \textit{spazio probabilistico}
$\mathcal{R}$ su $z$, sono $2^{r(z)}$.
Per ogni specifica sequenza $R \in \mathcal{R}$, il verificatore produrrà $q$ query all'oracolo:
$$
	i_{1}^R, i_{2}^R, \cdots, i_{q}^R
$$
Per ognuna delle query il verificatore otterrà delle risposte in base alle
quali l'input verrà accettato o meno. Definiamo quindi
$$
	f^R(w_{i_{1}^R}, w_{i_{2}^R}, \cdots, w_{i_{q}^R})
$$

la funzione che, dati i bit alle posizioni $i_{j}^R$, restituisce \textit{True} se $v$
accetta dati $r = R \in \mathcal{R}$ e $w$ con le relative query oppure \textit{False}
in caso contrario:
$$
	f^R(w_{i_{1}^R}, w_{i_{2}^R}, \cdots, w_{i_{q}^R}) = True \iff v(z, r, w) = True
$$

L'idea è quindi descrivere il comportamento di $f$ (e di conseguenza di $v$) come una
formula booleana. In questo modo si dimostra che si può ridurre una qualsiasi istanza
del problema del riconoscimento $z \in L$ per un arbitrario $L$ in $\mathbf{NP-completi}$
ad un'istanza di \textsc{MaxEkSat};
introduciamo quindi $|w|$ variabili booleane $x_1, x_2, \cdots x_{|w|}$.
La funzione $f^R$ si può descrivere con una formula logica, che chiamiamo
$\varphi^R_z$ che utilizza come letterali
proprio le variabili $x_i$ che compongono $w$:
$$
	\varphi^R_z = (x_1 = 1 \lor x_2 = 0 \cdots) \land (x_ 1 = 1 \lor x_4 = 1 \lor x_8 = 0 \lor X_9 = 1) \land \cdots
$$
è importante notare che la CNF così descritta ha clausole con al massimo
$q$ letterali e possiamo assumere che sia esattamente così, ossia ogni clausola
abbia esattamente $q$ letterali. Complessivamente, il comportamento del verificatore è
esprimibile come la congiunzione di tutte le possibili $\varphi^R_z$:
$$
	\Phi_z = \bigwedge_{R \in \mathcal{R}} \varphi_{z}^R
$$
$\Phi_z$ ha una sottoformula per ogni possibile stringa random, quindi ci sono
$|\mathcal{R}|$ sottoformule ognuna con al più $2^q$ clausole, per un totale di
$|\mathcal{R}|2^q = 2^{r(|z|)}2^q = 2^{r(|z|) + q} = 2^{O(\log(|z|) + q} = O(|z|)$ clausole (al massimo).

In secondo luogo, notiamo che se $z \in L$ il verificatore deve accettare
con probabilità $1$ per un qualche $\bar{w}$ che quindi rende vera $\Phi_z$ e di conseguenza
anche ogni $\varphi_z^R$ - il che significa che $\Phi_z$ è soddisfacibile.
Al contrario, se $z \notin L$ per ogni possibile $w$ si soddisfa al più meno
della metà delle $\varphi_z^R$, ossia non è possibile che esista un $w$ che soddisfi
la metà o più delle possibili $\varphi_z^R$. Questo significa inoltre che delle
$|\mathcal{R}|2^q$ clausole di cui $\Phi_z$ è costituita, nel caso $z \notin L$,
ogni $w$ rende vere al massimo un numero di clausole minore o uguale a
$$
	\frac{|\mathcal{R}|}{2}2^q + \frac{|\mathcal{R}|}{2} (2^{q} - 1)
$$
\noindent
Il seguente teorema conduce all'inapprossimabilità di \textsc{MaxEkSat}.
\begin{theorem}
	Esiste $\epsilon > 0$ tale che \textsc{MaxSat} non è $(1+\epsilon)-$approssimabile in
	tempo polinomiale a meno che $\mathbf{P} = \mathbf{NP}$.
\end{theorem}
\begin{proof}
	Sia $L \in \mathbf{NP-completi}$. Per questo linguaggio esisterà una specifica
	funzione $r(|n|) \in O (\log(n))$ e uno specifico $q \in \mathbb{N}$ tale
	che $L \in \mathbf{PCP}[r,q]$. Definiamo
	$$
		\bar{\epsilon} = \frac{1}{2^{q+1}}
	$$
	e supponiamo che \textsc{MaxSat} sia $(1 + \epsilon)-$approssimabile.

	Per ogni input $z \in 2^*$ possiamo costruire $\Phi_z$ sulla quale potremo
	eseguire l'algoritmo $(1+\epsilon)$-approssimabile per \textsc{MaxSat}, il quale
	termina in tempo polinomiale. Se $z \in L$ sappiamo che $\Phi_z$ è soddisfacibile,
	cioè il vero valore calcolato risolvendo l'istanza di \textsc{MaxSat} è
	$t^*(\Phi_z) = |\mathcal{R}|2^q$, mentre se $z \notin L$ sappiamo che il valore restituito risolvendo
	l'istanza di \textsc{MaxSat} è $t^*(\Phi_z) = \frac{|\mathcal{R}|}{2}2^q + \frac{|\mathcal{R}|}{2}(2^{q} -1) = 2^q |\mathcal{R}| - \frac{|\mathcal{R}|}{2}$.

	Se $z \in L$  allora
	$$
		t(\Phi_z) \geq \frac{t^*(\Phi_z)}{1 + \bar{\epsilon}} = \frac{2^q |\mathcal{R}|}{1 + \frac{1}{2^{q+1}}} = A
	$$
	mentre se $z \notin L$
	$$
		t(\Phi_z) \leq t^*(\Phi_z) \leq 2^q|\mathcal{R}| - \frac{|\mathcal{R}|}{2}  = B
	$$
	Calcoliamo $A- B$:
	\begin{align*}
		A - B = & \frac{2^q|\mathcal{R}|}{1 + \frac{1}{2^{q+1}}} - 2^q |\mathcal{R}| + \frac{|\mathcal{R}|}{2}                            \\
		=       & |\mathcal{R}| \frac{2^{q +1} - 2^{q + 1}(1 + \frac{1}{2^{q+1}}) + (1 + \frac{1}{2 ^{q + 1}})}{2(1 + \frac{1}{2^{q+1}})} \\
		=       & |\mathcal{R}| \frac{2^{q +1} - 2^{q + 1} -1 + 1 + \frac{1}{2^{q+1}}}{2(1 + \frac{1}{2^{q+1}})}                          \\
		=       & |\mathcal{R}| \frac{\frac{1}{2^{q+1}}}{2(1 + \frac{1}{2^{q+1}})}                                                        \\
		>       & 0
	\end{align*}
	che implica $A > B$.
	Quindi tutta la catena polinomiale di calcoli può essere utilizzata per
	\textit{decidere} se $z$ appartiene o meno a $L$:
	$$
		z \in 2^* \overset{?}{=} L \in \mathbf{NP-completi}  \rightarrow  \text{riduci } (L \leadsto \Phi_z) \rightarrow \text{risolvi \textsc{MaxSat} su } \Phi_z \rightarrow t(\Phi_z)
		\begin{cases}
			> A    & z \in L    \\
			\leq B & z \notin L \\
		\end{cases}
	$$
	Assurdo se $\mathbf{P} \neq \mathbf{NP}$.
\end{proof}

\begin{theorem}
	$MaxE3Sat$ non è $(\frac{8}{7} - \epsilon)-$approssimabile per un qualche $\epsilon > 0$.
\end{theorem}
\begin{proof}
	Omessa.
\end{proof}
\begin{corollario}
	L'algoritmo randomizzato per $MaxE3Sat$ è ottimo.
\end{corollario}

\subsection{Problema dell'insieme indipendente}
\popt {IndependentSet} {$G = (V,E)$} {Sottoinsieme $X \subseteq V$}
{Qual è l'insieme indipendente maggiore?}
{
	$X \subseteq V$ tale che $X$ è un insieme indipendente, ossia
	tale che $\forall i,j \in X ~ (i,j) \notin E$
}
{$Max$}{$|X|$}

\begin{theorem}\label{thm:ind_set_inapprox}
	Per ogni $\epsilon > 0$ \textsc{IndependentSet} non è $(2-\epsilon)-$approssimabile
	in tempo polinomiale se $\mathbf{P} \neq \mathbf{NP}$.
\end{theorem}
\begin{proof}
	Sia $L \in \mathbf{NP-completi}$; per \cref{thm:pcp} è anche $L \in \mathbf{PCP}[r(n), q]$
	con $r(n) \in O(\log(n))$ e $q \in \mathbb{N}$.

	Per ogni $z \in 2^*$ si consideri l'insieme $\mathcal{R}_z$ sequenze
	di bit random di cardinalità $|\mathcal{R}_z| = 2 ^ {r(|z|)}$
	e, per ogni possibile $R \in \mathcal{R}$
	tutte le $Q_z^R$ possibili risposte dell'oracolo di cardinalità
	$|Q_z|^R = 2^q$; costruiamo quindi
	l'insieme $\mathcal{C}_z = \cup_{R \in \mathcal{R}_z} \{R\} \times Q_z^R$
	dei possibili input al probabilistic checker,
	ognuno dei quali porta ad una risposta \textit{sì} o \textit{no}.
	Definiamo quindi  $\mathcal{A}_z \subseteq \mathcal{C}_z$
	l'insieme delle configurazioni accettanti nella forma
	$$
		c = (R, \langle i_1^R: v_1, i_2^R: v_2, \cdots, i_q^R:v_q \rangle)
	$$
	abbiamo inoltre che
	$$
		|\mathcal{A}_z| \leq 2^{r(|z|)}2^q = 2^{O(\log(|z|))}2^q = O(|z|)
	$$
	Costruiamo un grafo $G_z = (\mathcal{A}_z, E_{\mathcal{A}_z})$ sulle configurazioni accettanti e
	inseriamo un arco tra due configurazioni
	$$
		(R, \langle i_1^R: v_1, i_2^R: v_2, \cdots, i_q^R:v_q\rangle) \rightarrow (R', \langle i_1^{R'}: v_1', i_2^{R'}: v_2', \cdots, i_q^{R'}:v_q'\rangle)
	$$
	se e solo se le configurazioni sono \textit{incompatibili}, ossia 
	$$
	R = R' \lor \exists k, k': i^R_k = i^{R'}_{k'} \land v_k \neq v'_{k'}
	$$
	definiamo questi lati come \textit{lati di incompatibilità}.
\begin{oss} \label{lem:ind_set_zinl}
	Se $z \in L$, $G_z$ ha un insieme indipendente di cardinalità maggiore o
	uguale a $2^{r(|z|)}$.
\end{oss}
% lezione 16 - 19-11-2021
\begin{proof}
	Siccome $z \in L$, deve esistere $\bar{w} \in 2^q$ tale che il verificatore
	accetta con probabilità $1$ - questo significa che tutte le
	configurazioni ottenute al variare delle possibili stringhe random in cui
	la seconda parte è compatibile con $\bar{w}$ sono accettanti: la quantità
	di queste configurazioni accettanti è $2^{r(|z|)}$, ossia l'enumerazione di
	tutte le possibili stringhe random compatibili con $\bar{w}$.
\end{proof}
\begin{oss}\label{lem:ind_set_znotinl}
	Se $z \notin L$ ogni insieme indipendente di $G_z$ ha cardinalità
	$\leq 2^{r(|z|)-1}$.
\end{oss}
\begin{proof}
    Si immagini un insieme di configurazioni compatibili: nelle configurazioni, la parte delle query può contenere 
    posizioni richieste più volte con risultati diversi, altrimenti non sarebbero compatibili (e di conseguenza 
    l'insieme non sarebbe indipendente). Dato un insieme indipendente, si può costruire una stringa $w$ che è 
    compatibile con tutte le risposte ottenute, dove nelle posizioni di query che non appaiono nell'insieme 
    indipendente si possono inserire valori arbitrari; tuttavia non ci possono essere contraddizioni. 
    Qualunque $w$ è adatta per far accettare $z$ dal probabilistic checker; se ci fosse un insieme indipendente 
    in $G_z$ di cardinalità maggiore di $2^{r(|z|) -1}$ verrebbe contraddetto il \cref{thm:pcp}. 
\end{proof}
Dall'\cref{lem:ind_set_zinl} e dall'\cref{lem:ind_set_znotinl} si arriva alla dimostrazione del \cref{thm:ind_set_inapprox}: 
qualunque algoritmo che sia in grado di dare un'approssimazione migliore di $2$ riesce a distinguere i due casi, 
sapendo quindi decidere se $z \in L$ o meno in tempo polinomiale.
\end{proof}


\chapter{Strutture succinte}
\section{Abstract data types}
Gli \textit{abstract data type} sono tipi di dati descritti dal loro comportamento: un esempio è
l'ADT \texttt{stack<T>}, il quale è dotato di alcune operazioni inerenti al tipo
stesso, chiamate \textbf{primitive}:
\begin{lstlisting}
  bool  isEmpty()
  T     top()
  void  pop()
  void  push(T)
\end{lstlisting}
Il comportamento cosa facciano i metodi si può descrivere in molti modi: si può
utilizzare un metodo discorsivo, spiegando a parole, o utilizzare un metodo
analitico:
$$
	\forall S, s.push(x).top() = x
$$
(nonostante la notazione impropria dovuta alla signatura delle funzioni);
$$
	\forall S, s.isEmpty() \implies S.push(x).pop().isEmpty()
$$
Una volta descritto un ADT è necessario implementarlo, ossia costruire effettivamente
una struttura che implementa le primitive rispettandone la descrizione. Chiaramente,
vi sono molte implementazioni diverse che soddisfano le richieste
ma hanno \textit{costi} diversi, sia in tempo che in spazio. Siamo interessati
ad alcuni ADT e relative implementazioni che utilizzano poco tempo e spazio.
Ogni ADT ha associato un concetto di \textit{taglia}, che rappresenta genericamente
la grandezza di un'istanza: nel caso dello stack, la taglia sarebbe il numero
di elementi presenti sulla pila.

\subsection{Teoria dell'informazione}
Per poter caratterizzare le implementazioni degli ADT in base allo
spazio che occupano è necessario introdurre alcuni concetti della teoria
dell'informazione, i quali discendono dai Teoremi di Shannon,
sommariamente riassumibili nel seguente teorema.

\begin{theorem}[della codifica della sorgente]
	\label{thm:shannon}
	Per codificare $v$ valori servono in media $\log_2(v)$ bit.
\end{theorem}

Per esempio, immaginiamo di dover codificare un'immagine $100\times100$ pixel
in bianco e nero. Le immagini possibili sono $2^{10000}$: per codificare
queste immagini servono in media $10000$ bit. In effetti, la
rappresentazione banale che rappresenta ogni pixel, utilizza esattamente
$10000$ bit e non potrebbe usarne di meno! Usandone, per esempio, solo
$9000$, alcune immagini diverse avrebbero la stessa rappresentazione.

Questo teorema vale anche per rappresentazioni di dimensione
variabile, ossia vale anche per codifiche: si supponga di avere un
algoritmo in grado di comprimere tre immagini ognuna in $100$ bit.
La conseguenza di questo teorema è che ci saranno delle altre immagini che
utlizzeranno più di $10000$ bit, in modo che la media rimanga $10000$.

In generale, dati $v$ valori rappresentabili con $x_1, x_2, \cdots, x_v$ bit
rispettivamente; il Teorema afferma che
$$
	\frac{\sum_{i} x_i}{v} \geq \log_2(v)
$$
%(
di così e che esiste un sistema di compressione che riesce a rappresentare
in realtà il \cref{thm:shannon} dice di più, ossia che non si può fare meglio
i $v$ valori utilizzando un numero di bit medio tra $[\log_2(v), 1 + \log_2(v))$,
assumendo che tutti i $v$ valori siano equiprobabili\footnote{
	\cite{shannon_1948} è il lavoro di C. Shannon che ha dato vita al campo della
	teoria dell'informazione; un approccio più moderno è \cite{cover_2006}. }.
%]

Ci confronteremo spesso con questo \textbf{information-theoretical lower bound}:
immaginiamo tutte le possibili istanze $v_i$ di ADT di taglia $i$;
per esempio, uno stack con valori in $\{0, 1, \cdots, 9\}$;
lo stack di taglia $0$ è lo stack vuoto, gli stack
di taglia $1$ sono le $10$ istanze stack che contengono solo $1$, solo $2$, e così via,
mentre gli stack di taglia $2$ sono $10^2$; in generale uno stack
di taglia $n$ ha $10^n$ valori. Il Teorema afferma che in media servono
$\log_2{10^n} = n \log_2(10) \approx 4n$ bit per rappresentare uno stack
con valori in $\{0, 1, \cdots, 9\}$: sappiamo quindi che
\textit{nessuna implementazione} può utilizzare, in media, meno di
$Z_n =n \log_2(10)$ bit, l'information-theoretical lower bound per
rappresentare stack di taglia $n$ con valori in $\{0, 1, \cdots, 9\}$.

Ipotizziamo di avere una struttura che utilizza in media $D_n \geq Z_n$ bit:
esiste un tradeoff tra quanto \textit{compatta} è la struttura e quanto
tempo è necessario per eseguire le funzioni primitive.
Esistono sistemi di compressione che ignorano completamente il problema:
ad esempio, comprimendo un oggetto con l'algoritmo \texttt{tz2}, non si può
utilizzare l'oggetto compresso come la sua rappresentazione non compressa
per eseguire le primitive su di esso! Noi siamo interessati a strutture
compresse - rappresentano i dati in maniera efficiente rispetto allo spazio -
e con primitive efficienti tanto quanto un'implementazione non compressa.

Definiamo quindi delle classificazioni delle implementazioni in base al rapporto
tra l'effettivo utilizzo di spazio (in media) e l'indice $Z_n$:
un'implementazione dell'ADT è chiamata \textbf{implicita} se occupa un numero
di bit $D_n = Z_n + O(1)$, \textbf{succinta} se occupa un numero di bit
$D_n = Z_n + o(Z_n)$ e \textbf{compatta} se $D_n = O(Z_n)$; tutto questo sempre
notando che le primitive devono essere efficienti tanto quanto quelle definite
su strutture non compresse.

\section{Strutture di rango e selezione}
Questi ADT sono definiti da un vettore $\mathbb{b} \in 2^n$ con due primitive:
$$
	\mathbf{rank}_b: \mathbb{N} \rightarrow \mathbb{N}
$$
$$
	\mathbf{select}_b: \mathbb{N} \rightarrow \mathbb{N}
$$
tali che:
$$
	\forall p \leq n ~~ \mathbf{rank}_b(p) = |\{i | i < p \land b_i = 1\}|
$$
$$
	\forall k \leq n ~~ \mathbf{select}_b(k) =\max \{p | \mathbf{rank}_b(p) \leq k\}
$$
Quindi, per esempio, per un $\mathbf{b} = [0 1 1 0 1 0 1]$ si hanno due
tabelle di rank e select come in \cref{table:rank_sel}.

\begin{table}[ht]
	\centering

	\begin{subtable}{0.45\textwidth}
		\centering
		\begin{tabular}{c|c}
			$p$ & $\mathbf{rank}_b(p)$ \\ \hline
			0   & 0                    \\
			1   & 0                    \\
			2   & 1                    \\
			3   & 2                    \\
			4   & 2                    \\
			5   & 3                    \\
			6   & 3                    \\
			7   & 4
		\end{tabular}
		\caption{$\mathbf{rank}_b(p)$}
	\end{subtable}
	\begin{subtable}{0.45\textwidth}
		\centering
		\begin{tabular}{c|c}
			$k$      & $\mathbf{select}_b(p)$ \\ \hline
			0        & 1                      \\
			1        & 2                      \\
			2        & 3                      \\
			3        & 6                      \\
			4        & 7                      \\
			5        & 7                      \\
			$\cdots$ & 7                      \\
		\end{tabular}
		\caption{$\mathbf{select}_b(p)$}
	\end{subtable}
	\caption{Tabelle per $\mathbf{b}$.}
	\label{table:rank_sel}
\end{table}
Rank e select sono funzioni inverse in un senso molto stretto, ciò vale che
$$
	\forall k ~~ \mathbf{rank}_b(\mathbf{select}_b(k)) = k
$$
mentre l'inverso, ossia applicare select a rank, si ottiene una proprietà
diversa:
$$
	\forall p ~~ \mathbf{select}_b(\mathbf{rank}_b(p)) \geq p
$$
proprio grazie a quest'ultima proprietà è possibile dedurre la struttura sottostante,
nel senso che è possibile capire dove siano gli $0$ e gli $1$ in $\mathbf{b}$.

\subsection{Struttura di Jacobson per il rango}
\subsubsection{Four-russians trick}
L'implementazione dell'ADT rank di Jacobson
utilizza il ``four-russians trick''. Si immagini di
voler rappresentare una matrice binaria: un modo per farlo potrebbe essere
dividere la matrice in blocchi chiamati \textit{piastrelle} ed enumerare
le possibili piastrelle. Chiaramente, se nella matrice appare ogni possibile
combinazione di piastrella, il guadagno del trucco sarà nullo.
Se, invece, la matrice è molto ripetitiva, le piastrelle possibili da ricordare saranno
poche e basterà utilizzare il numero associato alla piastrella per rappresentare
l'intera matrice. Un esempio è in \cref{fig:frtrick}.

% disegno.. 19-11-2021 ... 
\begin{figure}[ht]
	\centering
	\begin{subfigure}{0.32\textwidth}
		\centering
		\tikzset{every picture/.style={line width=0.75pt}} %set default line width to 0.75pt        

		\begin{tikzpicture}[x=0.75pt,y=0.75pt,yscale=-1,xscale=1]
			%uncomment if require: \path (0,300); %set diagram left start at 0, and has height of 300

			%Shape: Rectangle [id:dp005749332004390428] 
			\draw  [fill={rgb, 255:red, 195; green, 222; blue, 215 }  ,fill opacity=1 ] (10,30) -- (60,30) -- (60,70) -- (10,70) -- cycle ;
			%Shape: Rectangle [id:dp7752907448209945] 
			\draw   (60,30) -- (110,30) -- (110,70) -- (60,70) -- cycle ;
			%Shape: Rectangle [id:dp6434156098479539] 
			\draw   (110,30) -- (160,30) -- (160,70) -- (110,70) -- cycle ;
			%Shape: Rectangle [id:dp6511151383263759] 
			\draw   (10,70) -- (60,70) -- (60,110) -- (10,110) -- cycle ;
			%Shape: Rectangle [id:dp12594297400828225] 
			\draw  [color={rgb, 255:red, 0; green, 0; blue, 0 }  ,draw opacity=1 ][fill={rgb, 255:red, 239; green, 122; blue, 122 }  ,fill opacity=1 ] (60,70) -- (110,70) -- (110,110) -- (60,110) -- cycle ;
			%Shape: Rectangle [id:dp07170680173867716] 
			\draw  [fill={rgb, 255:red, 169; green, 198; blue, 234 }  ,fill opacity=1 ] (110,70) -- (160,70) -- (160,110) -- (110,110) -- cycle ;
			%Shape: Rectangle [id:dp016792670255308284] 
			\draw   (10,110) -- (60,110) -- (60,150) -- (10,150) -- cycle ;
			%Shape: Rectangle [id:dp40735665999968684] 
			\draw   (60,110) -- (110,110) -- (110,150) -- (60,150) -- cycle ;
			%Shape: Rectangle [id:dp5192196869257351] 
			\draw  [fill={rgb, 255:red, 195; green, 222; blue, 215 }  ,fill opacity=1 ] (110,110) -- (160,110) -- (160,150) -- (110,150) -- cycle ;
		\end{tikzpicture}
		\caption{Tabella iniziale. Ogni riquadro contiene $5$ bit.}
	\end{subfigure}
	\begin{subfigure}{0.32\textwidth}
		\centering

		\tikzset{every picture/.style={line width=0.75pt}} %set default line width to 0.75pt        

		\begin{tikzpicture}[x=0.75pt,y=0.75pt,yscale=-1,xscale=1]
			%uncomment if require: \path (0,300); %set diagram left start at 0, and has height of 300

			%Shape: Rectangle [id:dp7804092113436614] 
			\draw  [fill={rgb, 255:red, 195; green, 222; blue, 215 }  ,fill opacity=1 ] (200,50) -- (250,50) -- (250,90) -- (200,90) -- cycle ;
			%Shape: Rectangle [id:dp023528268951144793] 
			\draw  [fill={rgb, 255:red, 239; green, 122; blue, 122 }  ,fill opacity=1 ] (200,90) -- (250,90) -- (250,130) -- (200,130) -- cycle ;
			%Shape: Rectangle [id:dp43332731525643986] 
			\draw  [fill={rgb, 255:red, 169; green, 198; blue, 234 }  ,fill opacity=1 ] (200,130) -- (250,130) -- (250,170) -- (200,170) -- cycle ;
			%Shape: Rectangle [id:dp31985072077933197] 
			\draw   (250,50) -- (300,50) -- (300,90) -- (250,90) -- cycle ;

			%Shape: Rectangle [id:dp6876058724781219] 
			\draw   (250,90) -- (300,90) -- (300,130) -- (250,130) -- cycle ;

			%Shape: Rectangle [id:dp9303810523957489] 
			\draw   (250,130) -- (300,130) -- (300,170) -- (250,170) -- cycle ;

			% Text Node
			\draw (268,141) node [anchor=north west][inner sep=0.75pt]   [align=left] {3};
			% Text Node
			\draw (268,101) node [anchor=north west][inner sep=0.75pt]   [align=left] {2};
			% Text Node
			\draw (268,62) node [anchor=north west][inner sep=0.75pt]   [align=left] {1};
		\end{tikzpicture}
		\caption{Enumerazione di sottomatrici.}
	\end{subfigure}
	\begin{subfigure}{0.32\textwidth}
		\centering
		\tikzset{every picture/.style={line width=0.75pt}} %set default line width to 0.75pt        

		\begin{tikzpicture}[x=0.75pt,y=0.75pt,yscale=-1,xscale=1]
			%uncomment if require: \path (0,300); %set diagram left start at 0, and has height of 300

			%Shape: Rectangle [id:dp021688033849312616] 
			\draw   (295,78) -- (345,78) -- (345,118) -- (295,118) -- cycle ;
			%Shape: Rectangle [id:dp6593678571435228] 
			\draw   (345,78) -- (395,78) -- (395,118) -- (345,118) -- cycle ;
			%Shape: Rectangle [id:dp05373113383793526] 
			\draw  [fill={rgb, 255:red, 195; green, 222; blue, 215 }  ,fill opacity=0 ] (245,118) -- (295,118) -- (295,158) -- (245,158) -- cycle ;
			%Shape: Rectangle [id:dp881184835248234] 
			\draw  [fill={rgb, 255:red, 195; green, 222; blue, 215 }  ,fill opacity=0 ] (245,158) -- (295,158) -- (295,198) -- (245,198) -- cycle ;
			%Shape: Rectangle [id:dp40942537357979314] 
			\draw   (295,158) -- (345,158) -- (345,198) -- (295,198) -- cycle ;
			%Shape: Rectangle [id:dp4268873556029348] 
			\draw   (245,78) -- (295,78) -- (295,118) -- (245,118) -- cycle ;

			%Shape: Rectangle [id:dp9673334799284516] 
			\draw   (345,158) -- (395,158) -- (395,198) -- (345,198) -- cycle ;

			%Shape: Rectangle [id:dp06125848702433001] 
			\draw   (295,118) -- (345,118) -- (345,158) -- (295,158) -- cycle ;

			%Shape: Rectangle [id:dp6270788142678815] 
			\draw   (345,118) -- (395,118) -- (395,158) -- (345,158) -- cycle ;


			% Text Node
			\draw (363,129) node [anchor=north west][inner sep=0.75pt]   [align=left] {3};
			% Text Node
			\draw (313,129) node [anchor=north west][inner sep=0.75pt]   [align=left] {2};
			% Text Node
			\draw (363,170) node [anchor=north west][inner sep=0.75pt]   [align=left] {1};
			% Text Node
			\draw (263,90) node [anchor=north west][inner sep=0.75pt]   [align=left] {1};


		\end{tikzpicture}
		\caption{Matrice risultante dopo l'applicazione del \textit{four-russians trick}.}
	\end{subfigure}
	\caption{Trucco dei quattro russi.}
	\label{fig:frtrick}
\end{figure}

% lezione 17 - 24-11-2021
\subsubsection{Implementazione}
Il vettore $\mathbf{b}$  di $n$ bit viene quindi diviso in blocchi della
stessa lunghezza, chiamati \textit{superblocchi}, di lunghezza
$\log_2(n)^2$. Ogni superblocco viene
diviso a sua volta in blocchi più piccoli, di lunghezza $\frac{1}{2} \log_2(n)$,
come rappresentato in \cref{fig:jrank}.
\begin{figure}[ht]
	\centering
	\tikzset{every picture/.style={line width=0.75pt}} %set default line width to 0.75pt        

	\begin{tikzpicture}[x=0.75pt,y=0.75pt,yscale=-1,xscale=1]
		%uncomment if require: \path (0,300); %set diagram left start at 0, and has height of 300

		%Shape: Brace [id:dp6738156163571107] 
		\draw   (150,231) .. controls (149.99,235.67) and (152.32,238) .. (156.99,238.01) -- (199.99,238.07) .. controls (206.66,238.08) and (209.99,240.41) .. (209.98,245.08) .. controls (209.99,240.41) and (213.32,238.09) .. (219.99,238.1)(216.99,238.09) -- (262.99,238.16) .. controls (267.66,238.17) and (269.99,235.84) .. (270,231.17) ;
		%Straight Lines [id:da0175742846479654] 
		\draw    (150,150) -- (150,230) ;
		%Straight Lines [id:da9565831392918982] 
		\draw    (270,150) -- (270,230) ;
		%Shape: Rectangle [id:dp2322340296066201] 
		\draw   (150,130) -- (170,130) -- (170,150) -- (150,150) -- cycle ;
		%Shape: Rectangle [id:dp44447847120593675] 
		\draw   (170,130) -- (190,130) -- (190,150) -- (170,150) -- cycle ;
		%Shape: Rectangle [id:dp11830121656354553] 
		\draw   (190,130) -- (210,130) -- (210,150) -- (190,150) -- cycle ;
		%Shape: Rectangle [id:dp03816859758995417] 
		\draw   (210,130) -- (230,130) -- (230,150) -- (210,150) -- cycle ;
		%Shape: Rectangle [id:dp9408927207358553] 
		\draw   (230,130) -- (250,130) -- (250,150) -- (230,150) -- cycle ;
		%Shape: Rectangle [id:dp9557419136965325] 
		\draw   (250,130) -- (270,130) -- (270,150) -- (250,150) -- cycle ;
		%Shape: Rectangle [id:dp7039774887247845] 
		\draw   (270,130) -- (290,130) -- (290,150) -- (270,150) -- cycle ;
		%Shape: Rectangle [id:dp3777415956586846] 
		\draw   (290,130) -- (310,130) -- (310,150) -- (290,150) -- cycle ;
		%Shape: Rectangle [id:dp35581942359165963] 
		\draw   (150,150) -- (190,150) -- (190,170) -- (150,170) -- cycle ;
		%Shape: Rectangle [id:dp047912414707197204] 
		\draw   (190,150) -- (230,150) -- (230,170) -- (190,170) -- cycle ;
		%Shape: Brace [id:dp6450376307720747] 
		\draw   (190.22,170.82) .. controls (190.22,175.49) and (192.55,177.82) .. (197.22,177.82) -- (200.17,177.82) .. controls (206.84,177.82) and (210.17,180.15) .. (210.17,184.82) .. controls (210.17,180.15) and (213.5,177.82) .. (220.17,177.82)(217.17,177.82) -- (223.11,177.82) .. controls (227.78,177.82) and (230.11,175.49) .. (230.11,170.82) ;
		%Straight Lines [id:da007337321639370731] 
		\draw  [dash pattern={on 0.84pt off 2.51pt}]  (310,130) -- (330,130) ;
		%Straight Lines [id:da4459939594214508] 
		\draw  [dash pattern={on 0.84pt off 2.51pt}]  (310,150) -- (330,150) ;

		% Text Node
		\draw (179,250) node [anchor=north west][inner sep=0.75pt]   [align=left] {$\displaystyle log_{2}( n)^{2}$};
		% Text Node
		\draw (181,182) node [anchor=north west][inner sep=0.75pt]   [align=left] {$\displaystyle \frac{1}{2} log_{2}( n)$};
		% Text Node
		\draw (154,111.4) node [anchor=north west][inner sep=0.75pt]    {$0$};


	\end{tikzpicture}
	\caption{Divisione di $\textbf{b}$ in superblocchi e blocchi.}
	\label{fig:jrank}
\end{figure}
Per esempio, se $n = 256$, i superblocchi avranno lunghezza
$\log_2(256)^2 = 8^2 = 64$ bit e saranno $256/64 = 4$, mentre i
blocchi interni saranno $\frac{1}{2}\log_2(256) = \frac{1}{2}8 = 4$ bit e saranno
$64/4 = 16$ per superblocco, $16*4 = 64$ in totale.
In questo esempio, i possibili blocchi sono $2^4 = 16$ e, in generale,
siccome i blocchi hanno lunghezza $\frac{1}{2} \log_2(n)$, sono
$$
	2^{\frac{1}{2}\log_2(n)} = (2^{\log_2(n)})^{\frac{1}{2}} =  \sqrt{n}
$$
Se volessimo memorizzare la funzione di rank per un singolo blocco,
costruiremmo una tabella di $\frac{1}{2}\log_2(n)$ righe e per ognuna
di queste bisognerebbe salvare il numero di $1$ presenti nel blocco fino a quel punto,
utilizzando per ogni rank uno spazio $\log_2(\frac{1}{2}\log_2(n))$. Interamente, quindi,
la tabella di rank per un singolo blocco occupa spazio
$$
	\frac{1}{2}\log_2(n) \cdot \log_2(\frac{1}{2}\log_2(n)) \text{ bit}
$$
e, volendo memorizzare la tabella per ogni tipo di blocco, si consuma uno spazio
$$
	2^{\frac{1}{2}\log_2(n)} \cdot \frac{1}{2}\log_2(n) \cdot \log_2(\frac{1}{2}\log_2(n)) = \sqrt{n} \cdot \frac{1}{2}\log_2(n) \cdot \log_2(\frac{1}{2}\log_2(n))
	\leq \sqrt{n} \frac{1}{2}\log_2(n) \cdot \log_2(\log_2(n)) = o(n) \text{ bit}
$$
che significa che si può definire una tabella che enumera i tipi di blocco e per ognuno di
essi, come mostrato nella \cref{table:example_rank_block4}.

\begin{table}[!ht]
	\centering
	\begin{tabular}{|c|c|c|c|c|c|c|}
		\hline
		\multirow{2}{*}{$0000$} & $p$                & $0$ & $1$ & $2$ & $3$ & $4$ \\ \cline{2-7}
		                        & $\mathbf{rank}(p)$ & $0$ & $0$ & $0$ & $0$ & $0$ \\ \hline
		\multirow{2}{*}{$0001$} & $p$                & $0$ & $1$ & $2$ & $3$ & $4$ \\ \cline{2-7}
		                        & $\mathbf{rank}(p)$ & $0$ & $0$ & $0$ & $0$ & $1$ \\ \hline
		$\cdots$                &                    &     &     &     &     &     \\ \hline
		\multirow{2}{*}{$1111$} & $p$                & $0$ & $1$ & $2$ & $3$ & $4$ \\ \cline{2-7}
		                        & $\mathbf{rank}(p)$ & $0$ & $1$ & $2$ & $3$ & $4$ \\ \hline
	\end{tabular}
	\caption{Rank per ogni possibile blocco di lunghezza $4$.}
	\label{table:example_rank_block4}
\end{table}

Tutta questa struttura, benché sembra molto grande, è in realtà memorizzabile in $o(n)$, ossia
in una quantità di spazio che cresce meno rapidamente rispetto a $n$.
La prima idea è memorizzare queste strutture di rank per i blocchi. Dopo di che, per ogni superblocco
si memorizzano gli $1$ prima del superblocco, ossia per ogni superblocco $i$ si definisce
$$
	S_i = \mathbf{rank}(i[0])
$$
come rappresentato nella \cref{fig:superblock_i}.
\begin{figure}
	\centering
	\tikzset{every picture/.style={line width=0.75pt}} %set default line width to 0.75pt        
	\begin{tikzpicture}[x=0.75pt,y=0.75pt,yscale=-1,xscale=1]
		%uncomment if require: \path (0,300); %set diagram left start at 0, and has height of 300

		%Shape: Brace [id:dp6450376307720747] 
		\draw   (150.22,152.93) .. controls (150.22,157.6) and (152.55,159.93) .. (157.22,159.93) -- (180.17,159.93) .. controls (186.84,159.93) and (190.17,162.26) .. (190.17,166.93) .. controls (190.17,162.26) and (193.5,159.93) .. (200.17,159.93)(197.17,159.93) -- (223.11,159.93) .. controls (227.78,159.93) and (230.11,157.6) .. (230.11,152.93) ;
		%Shape: Brace [id:dp6867457932235534] 
		\draw   (150.62,189.98) .. controls (150.62,194.65) and (152.95,196.98) .. (157.62,196.98) -- (220.21,196.98) .. controls (226.88,196.98) and (230.21,199.31) .. (230.21,203.98) .. controls (230.21,199.31) and (233.54,196.98) .. (240.21,196.98)(237.21,196.98) -- (302.8,196.98) .. controls (307.47,196.98) and (309.8,194.65) .. (309.8,189.98) ;
		%Shape: Rectangle [id:dp04043578965235095] 
		\draw   (150,120) -- (310,120) -- (310,150) -- (150,150) -- cycle ;
		%Straight Lines [id:da7948124479327557] 
		\draw    (230,120) -- (230,150) ;
		%Straight Lines [id:da6507675221013395] 
		\draw    (150,140) .. controls (151.67,138.33) and (153.33,138.33) .. (155,140) .. controls (156.67,141.67) and (158.33,141.67) .. (160,140) .. controls (161.67,138.33) and (163.33,138.33) .. (165,140) .. controls (166.67,141.67) and (168.33,141.67) .. (170,140) .. controls (171.67,138.33) and (173.33,138.33) .. (175,140) .. controls (176.67,141.67) and (178.33,141.67) .. (180,140) .. controls (181.67,138.33) and (183.33,138.33) .. (185,140) .. controls (186.67,141.67) and (188.33,141.67) .. (190,140) .. controls (191.67,138.33) and (193.33,138.33) .. (195,140) .. controls (196.67,141.67) and (198.33,141.67) .. (200,140) .. controls (201.67,138.33) and (203.33,138.33) .. (205,140) .. controls (206.67,141.67) and (208.33,141.67) .. (210,140) .. controls (211.67,138.33) and (213.33,138.33) .. (215,140) .. controls (216.67,141.67) and (218.33,141.67) .. (220,140) .. controls (221.67,138.33) and (223.33,138.33) .. (225,140) .. controls (226.67,141.67) and (228.33,141.67) .. (230,140) -- (230,140) ;
		%Straight Lines [id:da816948397241291] 
		\draw    (150,130) .. controls (151.67,128.33) and (153.33,128.33) .. (155,130) .. controls (156.67,131.67) and (158.33,131.67) .. (160,130) .. controls (161.67,128.33) and (163.33,128.33) .. (165,130) .. controls (166.67,131.67) and (168.33,131.67) .. (170,130) .. controls (171.67,128.33) and (173.33,128.33) .. (175,130) .. controls (176.67,131.67) and (178.33,131.67) .. (180,130) .. controls (181.67,128.33) and (183.33,128.33) .. (185,130) .. controls (186.67,131.67) and (188.33,131.67) .. (190,130) .. controls (191.67,128.33) and (193.33,128.33) .. (195,130) .. controls (196.67,131.67) and (198.33,131.67) .. (200,130) .. controls (201.67,128.33) and (203.33,128.33) .. (205,130) .. controls (206.67,131.67) and (208.33,131.67) .. (210,130) .. controls (211.67,128.33) and (213.33,128.33) .. (215,130) .. controls (216.67,131.67) and (218.33,131.67) .. (220,130) .. controls (221.67,128.33) and (223.33,128.33) .. (225,130) .. controls (226.67,131.67) and (228.33,131.67) .. (230,130) .. controls (231.67,128.33) and (233.33,128.33) .. (235,130) .. controls (236.67,131.67) and (238.33,131.67) .. (240,130) .. controls (241.67,128.33) and (243.33,128.33) .. (245,130) .. controls (246.67,131.67) and (248.33,131.67) .. (250,130) .. controls (251.67,128.33) and (253.33,128.33) .. (255,130) .. controls (256.67,131.67) and (258.33,131.67) .. (260,130) .. controls (261.67,128.33) and (263.33,128.33) .. (265,130) .. controls (266.67,131.67) and (268.33,131.67) .. (270,130) .. controls (271.67,128.33) and (273.33,128.33) .. (275,130) .. controls (276.67,131.67) and (278.33,131.67) .. (280,130) .. controls (281.67,128.33) and (283.33,128.33) .. (285,130) .. controls (286.67,131.67) and (288.33,131.67) .. (290,130) .. controls (291.67,128.33) and (293.33,128.33) .. (295,130) .. controls (296.67,131.67) and (298.33,131.67) .. (300,130) .. controls (301.67,128.33) and (303.33,128.33) .. (305,130) .. controls (306.67,131.67) and (308.33,131.67) .. (310,130) -- (310,130) ;
		%Straight Lines [id:da005001855200690519] 
		\draw  [dash pattern={on 0.84pt off 2.51pt}]  (310,120) -- (350,120) ;
		%Straight Lines [id:da8158288414744241] 
		\draw  [dash pattern={on 0.84pt off 2.51pt}]  (310,150) -- (350,150) ;

		% Text Node
		\draw (181,166.4) node [anchor=north west][inner sep=0.75pt]    {$S_{1}$};
		% Text Node
		\draw (221,212.4) node [anchor=north west][inner sep=0.75pt]    {$S_{2}$};
	\end{tikzpicture}
	\caption{$S_2$ è il numero di $1$ presenti in $\mathbf{b}$ `sotto' la traccia più lunga, mentre $S_1$
		è il numero di $1$ sotto quella più corta. Va notato che $S_0 = 0$ benché non sia
		mostrato nella figura.}
	\label{fig:superblock_i}
\end{figure}
Per ogni blocco $l$ afferente al superblocco $i$ si definisce
$$
	B_l = \mathbf{rank}(l[0]) - S_i
$$
come rappresentato nella \cref{fig:block_i}.
\begin{figure}
	\centering



	\tikzset{every picture/.style={line width=0.75pt}} %set default line width to 0.75pt        

	\begin{tikzpicture}[x=0.75pt,y=0.75pt,yscale=-1,xscale=1]
		%uncomment if require: \path (0,300); %set diagram left start at 0, and has height of 300

		%Shape: Brace [id:dp6867457932235534] 
		\draw   (150.62,189.98) .. controls (150.62,194.65) and (152.95,196.98) .. (157.62,196.98) -- (220.21,196.98) .. controls (226.88,196.98) and (230.21,199.31) .. (230.21,203.98) .. controls (230.21,199.31) and (233.54,196.98) .. (240.21,196.98)(237.21,196.98) -- (302.8,196.98) .. controls (307.47,196.98) and (309.8,194.65) .. (309.8,189.98) ;
		%Shape: Rectangle [id:dp04043578965235095] 
		\draw   (150,120) -- (310,120) -- (310,150) -- (150,150) -- cycle ;
		%Straight Lines [id:da005001855200690519] 
		\draw  [dash pattern={on 0.84pt off 2.51pt}]  (310,120) -- (350,120) ;
		%Straight Lines [id:da8158288414744241] 
		\draw  [dash pattern={on 0.84pt off 2.51pt}]  (310,150) -- (350,150) ;
		%Shape: Brace [id:dp13835137879055492] 
		\draw   (150.09,150.38) .. controls (150.09,155.05) and (152.42,157.38) .. (157.09,157.38) -- (159.88,157.38) .. controls (166.55,157.38) and (169.88,159.71) .. (169.88,164.38) .. controls (169.88,159.71) and (173.21,157.38) .. (179.88,157.38)(176.88,157.38) -- (182.67,157.38) .. controls (187.34,157.38) and (189.67,155.05) .. (189.67,150.38) ;
		%Shape: Brace [id:dp46357313811514655] 
		\draw   (190.28,150.18) .. controls (190.28,154.85) and (192.61,157.18) .. (197.28,157.18) -- (200.07,157.18) .. controls (206.74,157.18) and (210.07,159.51) .. (210.07,164.18) .. controls (210.07,159.51) and (213.4,157.18) .. (220.07,157.18)(217.07,157.18) -- (222.86,157.18) .. controls (227.53,157.18) and (229.86,154.85) .. (229.86,150.18) ;
		%Shape: Brace [id:dp8791020491294269] 
		\draw   (230.47,150.18) .. controls (230.47,154.85) and (232.8,157.18) .. (237.47,157.18) -- (240.26,157.18) .. controls (246.93,157.18) and (250.26,159.51) .. (250.26,164.18) .. controls (250.26,159.51) and (253.59,157.18) .. (260.26,157.18)(257.26,157.18) -- (263.05,157.18) .. controls (267.72,157.18) and (270.05,154.85) .. (270.05,150.18) ;
		%Straight Lines [id:da2991531768387089] 
		\draw    (190,120) -- (190,150) ;
		%Straight Lines [id:da7604970738017156] 
		\draw    (230,120) -- (230,150) ;
		%Straight Lines [id:da9790584425339227] 
		\draw    (150,140) .. controls (151.67,138.33) and (153.33,138.33) .. (155,140) .. controls (156.67,141.67) and (158.33,141.67) .. (160,140) .. controls (161.67,138.33) and (163.33,138.33) .. (165,140) .. controls (166.67,141.67) and (168.33,141.67) .. (170,140) .. controls (171.67,138.33) and (173.33,138.33) .. (175,140) .. controls (176.67,141.67) and (178.33,141.67) .. (180,140) .. controls (181.67,138.33) and (183.33,138.33) .. (185,140) .. controls (186.67,141.67) and (188.33,141.67) .. (190,140) -- (190,140) ;
		%Straight Lines [id:da7640613174865398] 
		\draw    (150,130) .. controls (151.67,128.33) and (153.33,128.33) .. (155,130) .. controls (156.67,131.67) and (158.33,131.67) .. (160,130) .. controls (161.67,128.33) and (163.33,128.33) .. (165,130) .. controls (166.67,131.67) and (168.33,131.67) .. (170,130) .. controls (171.67,128.33) and (173.33,128.33) .. (175,130) .. controls (176.67,131.67) and (178.33,131.67) .. (180,130) .. controls (181.67,128.33) and (183.33,128.33) .. (185,130) .. controls (186.67,131.67) and (188.33,131.67) .. (190,130) .. controls (191.67,128.33) and (193.33,128.33) .. (195,130) .. controls (196.67,131.67) and (198.33,131.67) .. (200,130) .. controls (201.67,128.33) and (203.33,128.33) .. (205,130) .. controls (206.67,131.67) and (208.33,131.67) .. (210,130) .. controls (211.67,128.33) and (213.33,128.33) .. (215,130) .. controls (216.67,131.67) and (218.33,131.67) .. (220,130) .. controls (221.67,128.33) and (223.33,128.33) .. (225,130) .. controls (226.67,131.67) and (228.33,131.67) .. (230,130) -- (230,130) ;

		% Text Node
		\draw (221,212.4) node [anchor=north west][inner sep=0.75pt]    {$S_{0}$};
		% Text Node
		\draw (160,165.4) node [anchor=north west][inner sep=0.75pt]    {$B_{1}$};
		% Text Node
		\draw (200,165.4) node [anchor=north west][inner sep=0.75pt]    {$B_{2}$};
		% Text Node
		\draw (236,169.4) node [anchor=north west][inner sep=0.75pt]    {$\cdots $};
	\end{tikzpicture}
	\caption{$B_2$ è il numero di $1$ presenti in $\mathbf{b}$ `sotto' la traccia più lunga, mentre $B_1$
		è il numero di $1$ sotto quella più corta. Va notato che $B_0$, in questo frangente, è $0$,
		poiché $S_0$ è il primo superblocco di $\mathbf{b}$.}
	\label{fig:block_i}
\end{figure}

\noindent
Quindi, gli $S_i$ sono tanti quanti sono i superblocchi, ossia $\frac{n}{log_2(n)^2}$ e occupano spazio
$$
	\frac{n}{\log_2(n)^2} \underbrace{\log_2(n)}_{\text{spazio di un } S_I} = \frac{n}{\log_2(n)} = o(n) \text{ bit}
$$
mentre i $B_l$ sono tanti quanti sono i blocchi, ossia $\frac{n}{\frac{1}{2}\log_2(n)}$ e occupano spazio
$$
	\frac{n}{\frac{1}{2}\log_2(n)} \underbrace{\log_2(\log_2(n)^2)}_{\text{al massimo}} =
	\frac{2n}{\log_2(n)} 2 \log_2(\log_2(n)) = o(n) \text{ bit}
$$
Se si vuole conoscere il rango di uno specifico bit in un blocco bisogna
recuperare $S_i$ e $B_i$ e calcolare il quale sia effettivamente il rango, che
è stato memorizzato in una tabella $Tab$ utilizzando il four-russians trick
enumerando i possibili $\sqrt(n)$ blocchi e memorizzando il rango di ogni
bit del blocco. Complessivamente, quindi, tutte le tabelle necessarie occupano
$D_n = o(n)$ bit.

\begin{figure}[!h]
	\centering
	\tikzset{every picture/.style={line width=0.75pt}} %set default line width to 0.75pt        

	\begin{tikzpicture}[x=0.75pt,y=0.75pt,yscale=-1,xscale=1]
		%uncomment if require: \path (0,300); %set diagram left start at 0, and has height of 300

		%Shape: Rectangle [id:dp07152433688787241] 
		\draw   (100,100) -- (320,100) -- (320,120) -- (100,120) -- cycle ;
		%Straight Lines [id:da2936448051112307] 
		\draw    (120,100) -- (120,120) ;
		%Straight Lines [id:da027036583121563318] 
		\draw    (140,100) -- (140,120) ;
		%Straight Lines [id:da7810766858201694] 
		\draw    (160,100) -- (160,120) ;
		%Straight Lines [id:da5001340947135379] 
		\draw    (180,100) -- (180,120) ;
		%Straight Lines [id:da3843632481846373] 
		\draw    (200,100) -- (200,120) ;
		%Straight Lines [id:da03824030730667982] 
		\draw    (220,100) -- (220,120) ;
		%Straight Lines [id:da15646574434271243] 
		\draw    (240,100) -- (240,120) ;
		%Straight Lines [id:da846809534956415] 
		\draw    (260,100) -- (260,120) ;
		%Straight Lines [id:da4897590587430042] 
		\draw    (280,100) -- (280,120) ;
		%Straight Lines [id:da4069787777904943] 
		\draw    (300,100) -- (300,120) ;
		%Straight Lines [id:da05219000701716625] 
		\draw    (250,80) -- (250,98) ;
		\draw [shift={(250,100)}, rotate = 270] [color={rgb, 255:red, 0; green, 0; blue, 0 }  ][line width=0.75]    (10.93,-3.29) .. controls (6.95,-1.4) and (3.31,-0.3) .. (0,0) .. controls (3.31,0.3) and (6.95,1.4) .. (10.93,3.29)   ;
		%Shape: Brace [id:dp37474184805141453] 
		\draw   (100.25,150) .. controls (100.25,154.67) and (102.58,157) .. (107.25,157) -- (140.25,157) .. controls (146.92,157) and (150.25,159.33) .. (150.25,164) .. controls (150.25,159.33) and (153.58,157) .. (160.25,157)(157.25,157) -- (193.25,157) .. controls (197.92,157) and (200.25,154.67) .. (200.25,150) ;
		%Shape: Brace [id:dp9417201733674553] 
		\draw   (200.5,150.5) .. controls (200.5,155.17) and (202.83,157.5) .. (207.5,157.5) -- (240.5,157.5) .. controls (247.17,157.5) and (250.5,159.83) .. (250.5,164.5) .. controls (250.5,159.83) and (253.83,157.5) .. (260.5,157.5)(257.5,157.5) -- (293.5,157.5) .. controls (298.17,157.5) and (300.5,155.17) .. (300.5,150.5) ;
		%Straight Lines [id:da1555835238266523] 
		\draw  [dash pattern={on 0.84pt off 2.51pt}]  (100,100) -- (80,100) ;
		%Straight Lines [id:da0928939361477803] 
		\draw  [dash pattern={on 0.84pt off 2.51pt}]  (100,120) -- (80,120) ;
		%Straight Lines [id:da19581834915337515] 
		\draw  [dash pattern={on 0.84pt off 2.51pt}]  (340,120) -- (320,120) ;
		%Straight Lines [id:da4107689408293117] 
		\draw  [dash pattern={on 0.84pt off 2.51pt}]  (340,100) -- (320,100) ;
		%Shape: Brace [id:dp9863486277907854] 
		\draw   (240.25,120.5) .. controls (240.25,125.17) and (242.58,127.5) .. (247.25,127.5) -- (250.17,127.5) .. controls (256.84,127.5) and (260.17,129.83) .. (260.17,134.5) .. controls (260.17,129.83) and (263.5,127.5) .. (270.17,127.5)(267.17,127.5) -- (273.08,127.5) .. controls (277.75,127.5) and (280.08,125.17) .. (280.08,120.5) ;
		%Shape: Brace [id:dp8556326349440747] 
		\draw   (199.92,120.25) .. controls (199.92,124.92) and (202.25,127.25) .. (206.92,127.25) -- (209.83,127.25) .. controls (216.5,127.25) and (219.83,129.58) .. (219.83,134.25) .. controls (219.83,129.58) and (223.16,127.25) .. (229.83,127.25)(226.83,127.25) -- (232.75,127.25) .. controls (237.42,127.25) and (239.75,124.92) .. (239.75,120.25) ;

		% Text Node
		\draw (245,59.4) node [anchor=north west][inner sep=0.75pt]    {$p$};
		% Text Node
		\draw (104,102.4) node [anchor=north west][inner sep=0.75pt]    {$1$};
		% Text Node
		\draw (124,102.4) node [anchor=north west][inner sep=0.75pt]    {$0$};
		% Text Node
		\draw (144,102.4) node [anchor=north west][inner sep=0.75pt]    {$1$};
		% Text Node
		\draw (224,102.4) node [anchor=north west][inner sep=0.75pt]    {$1$};
		% Text Node
		\draw (164,102.4) node [anchor=north west][inner sep=0.75pt]    {$1$};
		% Text Node
		\draw (204,102.4) node [anchor=north west][inner sep=0.75pt]    {$0$};
		% Text Node
		\draw (240,170.4) node [anchor=north west][inner sep=0.75pt]    {$S_{n}$};
		% Text Node
		\draw (140,170.4) node [anchor=north west][inner sep=0.75pt]    {$S_{n-1}$};
		% Text Node
		\draw (248,132.4) node [anchor=north west][inner sep=0.75pt]    {$B_{k}$};
		% Text Node
		\draw (204,132.4) node [anchor=north west][inner sep=0.75pt]    {$B_{k-1}$};
		% Text Node
		\draw (184,102.4) node [anchor=north west][inner sep=0.75pt]    {$0$};
	\end{tikzpicture}
	\caption{Calcolo di $\mathbf{rank_b}(p)$.}
	\label{fig:example_rank_p}
\end{figure}

Quanto tempo impiega un'implementazione basata su queste strutture per
calcolare il rango di una posizione $p$? Per sapere quanti $1$ ci sono
prima della posizione $p$, bisogna innanzitutto calcolare il superblocco
di appartenenza di $p$, che è il superblocco numero $\frac{p}{\log_2(n)^2}$.
Vogliamo sapere quanti $1$ ci sono prima di quel superblocco - valore che abbiamo
memorizzato nei $S_i$. A questo punto, è necessario calcolare il numero di $1$
dall'inizio del superblocco all'inizio del blocco al quale appartiene $p$, valore salvato
in $B_i$. Rimane, quindi, da capire quanti $1$ ci sono dall'inizio del blocco al
quale appartiene $p$ fino alla posizione $p$ stessa. Per questo si può
utilizzare la tabella $Tab$ del four-russians trick:

$$
	\mathbf{rank}_{\mathbf{b}}(p) = S_{\frac{p}{\log_2(n)^2}} + B_{\frac{p}{\frac{1}{2}\log_2(n)}}
	+ Tab[t, p \mod \frac{1}{2}\log_2(n)]
$$
dove $t$ è il tipo blocco a cui $p$ appartiene secondo l'enumerazione dei blocchi di
$Tab$, calcolabile data la posizione d'inizio del blocco di $p$
$$
	x = \lfloor \frac{p}{\frac{1}{2}\log_2(n)}\rfloor (\frac{1}{2}\log_2(n))
$$
accedendo il vettore $\mathbf{b}$ e leggendo i valori del blocco\footnote{
	In termini pratici, come avviene questo accesso? Se avviene leggendo,
	uno per uno, tutti i valori del blocco al quale appartiene $p$ per poi
	accedere alla tabella $Tab$, non ha più senso leggere i valori da
	$\mathbf{b}[x]$ fino a $\mathbf{b}[p-1]$ contando gli $1$ visti?
}:
$$
	t = \mathbf{b}[x, x +1, \cdots, x + \frac{1}{2} \log_2(n) -1 ]
$$
Il calcolo del rank avviene quindi in tempo lineare,
poiché si tratta di accedere a $3$ tabelle e la struttura quindi
occupa lo spazio di $n + o(n)$ bit, poiché è necessario mantenere $b$ e
risponde alle query in tempo $O(1)$. Rispetto alla nostra classificazione,
questa struttura è \textbf{succinta} e ha la stessa efficienza della struttura naïve.

\subsection{Struttura di Clarke per la selezione}
La struttura di Clarke per la selezione è succinta teoricamente, ma in pratica
è raramente utilizzata perché la sua implementazione è molto complessa.
L'obiettivo è, dato un vettore $\mathbf{b}$ di valori binari fissato, calcolare
la funzione di selezione
$$
	\mathbf{select_b}(k) = \text{ posizione del } k-\text{esimo } 1
$$
La struttura di Clarke utilizza dei \textit{livelli}, simili a quelli utilizzati
dalla struttura di Jacobson.

\subsubsection{Primo livello}
Il primo livello della struttura di Clarke si è un insieme
di valori che rappresentano le posizioni degli $1$ di ordinalità multipla di
$\log_2(n) \cdot \log_2(\log_2(n))$, ossia
$$
	P_i =  \mathbf{select_b}(i \cdot \log_2(n) \cdot \log_2(\log_2(n)))
$$
ossia la posizione del $(i \cdot \log_2(n) \cdot \log_2(\log_2(n)))$-esimo $1$.

\paragraph{Memoria}
La grandezza di questa famiglia, ossia il numero di $P_i$, dipende dal
numero di $i$ che ci sono in $\mathbf{b}$, ma nel caso peggiore, il vettore
è composto unicamente da $1$ e vi saranno $\frac{n}{\log_2(n) \cdot \log_2(\log_2(n))}$ membri.
Ad ognuno di essi va associato un elemento, il quale richiede $\log_2(n)$ bit;
in totale, questo livello occupa
$$
	\frac{n}{\log_2(n) \cdot \log_2(\log_2(n))} \cdot \log_2(n) = o(n) \text{ bit}
$$

\subsubsection{Secondo livello}
Per le posizioni che non sono multiple di $\log_2(n) \cdot \log_2(\log_2(n))$ si utilizza
un secondo livello, che è costruito differentemente in base ad un indice calcolato
per ogni $P_i$:

$$
	\forall i ~ r_i = P_{i + 1} - P_i
$$
che rappresenta la distanza fra l'$(i \cdot \log_2(n) \cdot \log_2(\log_2(n)))$-esimo $1$
e il $((i+1) \cdot \log_2(n) \cdot \log_2(\log_2(n)))$-esimo. Questo
valore, $r_i$, sarà esattamente uguale a $\log_2(n) \cdot \log_2(\log_2(n))$ se e solo se
tra $P_{i}$ e $P_{i+1}$ ci sono unicamente $1$, e sarà maggiore se invece le
due posizioni sono più lontane.
% @TODO: serve un esempio, non si capisce bene. 
I due casi che si considerano dipendono dal valore di $r_i$.

\paragraph{Caso sparso}
Se $r_i \geq (\log_2(n) \cdot \log_2(\log_2(n)))^2$, significa che in $\mathbf{b}$ ci
sono pochi $1$ tra le posizioni $P_{i}$ e $P_{i+1}$; in questo caso
definiamo $S_i$ come la lista esplicita delle posizioni di tutti gli $1$
in $\mathbf{b}$ tra le due posizioni rappresentate come differenza tra $P_i$.

In questo caso, gli $S_i$ costruiti sono esattamente $\log_2(n)\cdot \log_2(\log_2(n))$,
in quanto stiamo contando gli $1$ in $\mathbf{b}$ tra il $(i \cdot \log_2(n) \cdot \log_2(\log_2(n))$-esimo
$1$ e il $((i+1) \cdot \log_2(n) \cdot \log_2(\log_2(n))$-esimo $1$, e ad ognuno
di essi si associa un numero che in grandezza è minore o uguale a
$\log_2(r_i)$, concludendo che per memorizzare un $S_i$ sono necessari
\begin{align*}
	\log_2(n)\cdot \log_2(\log_2(n)) \cdot \log_2(r_i) & = \frac{(\log_2(n)\cdot \log_2(\log_2(n)))^2}{\log_2(n)\cdot \log_2(\log_2(n))} \cdot \log_2(r_i) \\
	                                                   & \leq \frac{r_i}{\log_2(n)\cdot \log_2(\log_2(n))} \cdot \log_2(r_i)                               \\
	                                                   & \leq \frac{r_i}{\log_2(n)\cdot \log_2(\log_2(n))} \cdot \log_2(n)                                 \\
	                                                   & \leq \frac{r_i}{\log_2(\log_2(n))}                                                                \\
	                                                   & \leq \frac{n}{\log_2(\log_2(n))}   = o(n) \text{ bit}
\end{align*}

\paragraph{Caso denso}
Se $r_i \le (\log_2(n) \cdot \log_2(\log_2(n)))^2$, si memorizzano gli $1$
multipli di $\log_2(r_i)\log_2(\log_2(n))$, ossia partendo dalla posizione $P_i$ si salvano le
posizioni degli $j \cdot \log_2(r_i)\log_2(\log_2(n))$-esimi $1$ come differenze da $P_i$, ossia
$$
	S^i_j = \mathbf{select_b}(j \cdot \log_2(r_i)\log_2(\log_2(n))) - P_i
$$

In questo caso, gli $1$ memorizzati sono $\frac{\log_2(n)\cdot \log_2(\log_2(n))}{\log_2(r_i)\cdot \log_2(\log_2(n))}$.
Ad ognuno di questi si associa un valore in grandezza $\log_2(r_i)$, quindi lo spazio utilizzato
per memorizzare tutti i valori $S^i_j$ sono
$$
	\frac{\log_2(n)\cdot \log_2(\log_2(n))}{\log_2(r_i)\cdot \log_2(\log_2(n))} \log_2(r_i) =
	\frac{\log_2(n)\cdot \log_2(\log_2(n))}{\log_2(\log_2(n))} \leq \frac{r_i}{\log_2(\log_2(n))}
	\leq \frac{n}{\log_2(\log_2(n))} = o(n) \text{ bit}
$$


\subsubsection{Terzo livello}
Se nel secondo livello ci si trova nel caso denso, si utilizza un terzo livello.
Questo livello viene utilizzato esclusivamente per gli $1$ le cui posizioni non sono state
salvate nel secondo livello nel caso denso; pertanto, l'assunto è che
$$
	r_i < (\log_2(n) \log_2(\log_2(n)))^2
$$
Per ognuna di queste posizioni calcoliamo la differenza tra $S^i_j$:
$$
	\forall j \bar{r}^i_j = S^i_{j+1} - S^i_j
$$
Come nel caso precedente, abbiamo due possibilità in base al valore di $\bar{r}^i_j$, il quale
gode comunque della proprietà
$$
	\forall i, j ~~  \bar{r}^i_j \geq \log_2(r_i) \log_2(\log_2(n))
$$
\paragraph{Caso sparso}
Nel caso in cui $\bar{r}^i_j \geq \log_2(\bar{r}^i_j) \log_2(r_i) \log_2(\log_2(n))^2$,
si memorizzano esplicitamente tutte le posizioni degli $1$ tra $S^i_j$ e $S^i_{j+1}$
con dei valori $T^i_{j,k}$ come differenze tra $S^i_j$. In questo caso,
il consumo di memoria è
$$
	(\log_2(r_i)\cdot \log_2(\log_2(n)) \cdot \log_2(\bar{r}^i_j) \leq
	\frac{\log_2(r_i) \cdot \log_2(\log_2(n))^2 \log_2(\bar{r}^i_j)}{\log_2(\log_2(n))}
	\leq \frac{\bar{r}^i_j}{\log_2(\log_2(n))} = o(n) \text{ bit}
$$
\paragraph{Caso denso}
Nel caso in cui $\bar{r}^i_j \le \log_2(\bar{r}^i_j) \log_2(r_i) \log_2(\log_2(n))^2$,
significa che ci sono pochi $0$ tra $S^i_j$ e $S^i_{j+1}$ e
si utilizza il four-russians trick. Inizialmente osserviamo quanto segue.
\begin{oss}
	\begin{align*}
		\log_2(\bar{r}^i_j) \leq \log_2(r_i) & \leq \log_2(\log_2(n)\cdot \log_2(\log_2(n)))^2     \\
		                                     & = 2 \log_2(\log_2(n)) + 2 \log_2(\log_2(\log_2(n))) \\
		                                     & \leq 4 \log_2(\log_2(n))
	\end{align*}
\end{oss}
\begin{oss}
	$$
		\bar{r}^i_j < \log_2(\bar{r}^i_j) \cdot \log_2(r_i) \cdot (\log_2(\log_2(n))^2
		\leq 16(\log_2\log_2(n))^4
	$$
\end{oss}
Lo spazio necessario per utilizzare il four-russians trick è quanto segue.
Servono $2^{\bar{r}^i_j}$ enumerazioni di `sottovettori',
ossia la dimensione tra $S^i_j$ e $S^i_{j+1}$,
che è la parte che va memorizzata esplicitamente; per ognuna di queste
enumerazioni è necessario salvare la posizione di $\bar{r}^i_j$ $1$ utilizzando
memoria al massimo $\log_2(\bar{r}^i_j)$:
\begin{align*}
	2^{\bar{r}^i_j} \cdot \bar{r}^i_j \cdot \log_2(\bar{r}^i_j) & \leq
	2^{16(\log_2\log_2(n))^4} \cdot 16(\log_2\log_2(n))^4 \cdot \log_2(16(\log_2\log_2(n))^4)                                  \\
	                                                            & = 16(\log_2\log_2(n))^8 \log_2(16(\log_2\log_2(n))^4) = o(n)
\end{align*}


\subsubsection{Complessità totale in spazio}
In totale, per memorizzare $\mathbf{b}$ e i possibili tre livelli di struttura,
sono necessari $o(n)$ bit per il primo livello e, in base a come è fatto il secondo livello
$$
	\sum^{\frac{n}{\log_2(n)\cdot \log_2(\log_2(n))}} \frac{P_{i+1} - P_i}{\log_2(\log_2(n)}
	= \frac{P_n - P_0}{\log_2(\log_2(n))}
	\leq \frac{n}{\log_2(\log_2(n))} = o(n) \text{ bit}
$$
più $o(n)$ bit per il terzo livello. Quindi, la struttura di Clarke occupa
spazio $n + o(n)$ e ha accesso tempo di accesso costante, pertanto è
una struttura \textbf{succinta}.

\begin{figure}
	\centering
	\tikzset{every picture/.style={line width=0.75pt}} %set default line width to 0.75pt        

	\begin{tikzpicture}[x=0.75pt,y=0.75pt,yscale=-1,xscale=1]
		%uncomment if require: \path (0,300); %set diagram left start at 0, and has height of 300

		%Shape: Circle [id:dp9914836432536758] 
		\draw  [fill={rgb, 255:red, 0; green, 0; blue, 0 }  ,fill opacity=1 ] (70,101) .. controls (70,95.48) and (74.48,91) .. (80,91) .. controls (85.52,91) and (90,95.48) .. (90,101) .. controls (90,106.52) and (85.52,111) .. (80,111) .. controls (74.48,111) and (70,106.52) .. (70,101) -- cycle ;
		%Shape: Circle [id:dp09703813871749656] 
		\draw  [fill={rgb, 255:red, 0; green, 0; blue, 0 }  ,fill opacity=1 ] (150,61) .. controls (150,55.48) and (154.48,51) .. (160,51) .. controls (165.52,51) and (170,55.48) .. (170,61) .. controls (170,66.52) and (165.52,71) .. (160,71) .. controls (154.48,71) and (150,66.52) .. (150,61) -- cycle ;
		%Straight Lines [id:da8346853138498954] 
		\draw    (80,101) -- (148.21,66.89) ;
		\draw [shift={(150,66)}, rotate = 153.43] [color={rgb, 255:red, 0; green, 0; blue, 0 }  ][line width=0.75]    (10.93,-3.29) .. controls (6.95,-1.4) and (3.31,-0.3) .. (0,0) .. controls (3.31,0.3) and (6.95,1.4) .. (10.93,3.29)   ;
		%Shape: Circle [id:dp4565729376211175] 
		\draw  [fill={rgb, 255:red, 0; green, 0; blue, 0 }  ,fill opacity=1 ] (150,141) .. controls (150,146.52) and (154.48,151) .. (160,151) .. controls (165.52,151) and (170,146.52) .. (170,141) .. controls (170,135.48) and (165.52,131) .. (160,131) .. controls (154.48,131) and (150,135.48) .. (150,141) -- cycle ;
		%Straight Lines [id:da8127484960209145] 
		\draw    (80,101) -- (148.21,135.11) ;
		\draw [shift={(150,136)}, rotate = 206.57] [color={rgb, 255:red, 0; green, 0; blue, 0 }  ][line width=0.75]    (10.93,-3.29) .. controls (6.95,-1.4) and (3.31,-0.3) .. (0,0) .. controls (3.31,0.3) and (6.95,1.4) .. (10.93,3.29)   ;
		%Shape: Circle [id:dp546530222511162] 
		\draw  [fill={rgb, 255:red, 0; green, 0; blue, 0 }  ,fill opacity=1 ] (150,141) .. controls (150,135.48) and (154.48,131) .. (160,131) .. controls (165.52,131) and (170,135.48) .. (170,141) .. controls (170,146.52) and (165.52,151) .. (160,151) .. controls (154.48,151) and (150,146.52) .. (150,141) -- cycle ;
		%Shape: Circle [id:dp1691093585413369] 
		\draw  [fill={rgb, 255:red, 0; green, 0; blue, 0 }  ,fill opacity=1 ] (230,101) .. controls (230,95.48) and (234.48,91) .. (240,91) .. controls (245.52,91) and (250,95.48) .. (250,101) .. controls (250,106.52) and (245.52,111) .. (240,111) .. controls (234.48,111) and (230,106.52) .. (230,101) -- cycle ;
		%Straight Lines [id:da17118221437478087] 
		\draw    (160,141) -- (228.21,106.89) ;
		\draw [shift={(230,106)}, rotate = 153.43] [color={rgb, 255:red, 0; green, 0; blue, 0 }  ][line width=0.75]    (10.93,-3.29) .. controls (6.95,-1.4) and (3.31,-0.3) .. (0,0) .. controls (3.31,0.3) and (6.95,1.4) .. (10.93,3.29)   ;
		%Shape: Circle [id:dp8870903616648642] 
		\draw  [fill={rgb, 255:red, 0; green, 0; blue, 0 }  ,fill opacity=1 ] (230,181) .. controls (230,186.52) and (234.48,191) .. (240,191) .. controls (245.52,191) and (250,186.52) .. (250,181) .. controls (250,175.48) and (245.52,171) .. (240,171) .. controls (234.48,171) and (230,175.48) .. (230,181) -- cycle ;
		%Straight Lines [id:da2110125336208042] 
		\draw    (160,141) -- (228.21,175.11) ;
		\draw [shift={(230,176)}, rotate = 206.57] [color={rgb, 255:red, 0; green, 0; blue, 0 }  ][line width=0.75]    (10.93,-3.29) .. controls (6.95,-1.4) and (3.31,-0.3) .. (0,0) .. controls (3.31,0.3) and (6.95,1.4) .. (10.93,3.29)   ;

		% Text Node
		\draw (51,230) node [anchor=north west][inner sep=0.75pt]   [align=left] {I° livello};
		% Text Node
		\draw (131,231) node [anchor=north west][inner sep=0.75pt]   [align=left] {II° livello};
		% Text Node
		\draw (211,231) node [anchor=north west][inner sep=0.75pt]   [align=left] {III° livello};
		% Text Node
		\draw (141,162) node [anchor=north west][inner sep=0.75pt]   [align=left] {denso};
		% Text Node
		\draw (141,32) node [anchor=north west][inner sep=0.75pt]   [align=left] {sparso};
		% Text Node
		\draw (231,142) node [anchor=north west][inner sep=0.75pt]   [align=left] {four-russians trick};
	\end{tikzpicture}
	\caption{Struttura di Clarke per la selezione.}
\end{figure}



% lezione 19 - 01-12-2021
\section{Struttura per alberi binari}
Gli \textit{alberi} possono essere visti in due modi: per i
matematici, essi sono dei grafi connessi aciclici mentre per gli
informatici sono una \textit{struttura dati} radicata che ha una
\textit{radice}. In particolare, un albero \textit{binario} è un
albero tale per cui ogni nodo ha al più due figli; formalmente,
si definiscono induttivamente: l'albero con un solo nodo (che è anche
la radice), denotato $\emptyset$, è un albero binario;
se $T_1$ e $T_2$ sono due alberi binari,
allora anche l'albero radicato in un nuovo nodo che ha come figli
$T_1$ e $T_2$, denotato $(T_1, T_2)$ è un albero binario. In questa definizione,
ogni nodo ha o $0$ o $2$ figli. I nodi che non hanno figli
sono chiamati \textbf{nodi esterni} o \textbf{foglie}, mentre gli altri sono
chiamati \textbf{nodi interni}.


\begin{figure}[!hb]
	\begin{subfigure}{0.45\textwidth}
		\begin{center}
			\begin{tikzpicture}
				\node [circle, draw]{};
			\end{tikzpicture}
		\end{center}
		\caption{L'albero binario $\emptyset$.}
	\end{subfigure}
	\begin{subfigure}{0.45\textwidth}
		\begin{center}
			\begin{tikzpicture}
				\node [circle,draw]{}
				child {node [isosceles triangle,draw,shape border rotate=90] {$T_1$}}
				child {node [isosceles triangle,draw,shape border rotate=90] {$T_2$}};
			\end{tikzpicture}
		\end{center}
		\caption{L'albero binario $(T_1, T_2)$.}
	\end{subfigure}
	\caption{Definizione induttiva di albero binario.}
	\label{fig:btree_inductive}
\end{figure}

\begin{figure}
	\begin{center}
		\begin{tikzpicture}
			\node [circle,draw]{} [level distance=10mm,sibling distance=25mm]
			child { node [circle,draw]{} [level distance=10mm ,sibling distance=15mm]
					child {node [circle,draw] {}
							child {node [circle,fill] {}}
							child {node [circle,fill]{}}}
					child {node [circle,draw,fill]{}}
				}
			child {node [circle,draw] {} [level distance=10mm ,sibling distance=15mm]
					child {node [circle,draw,fill] {}}
					child {node [circle,draw,fill]{}}
				};
		\end{tikzpicture}
	\end{center}
	\caption{L'albero binario $(((\emptyset, \emptyset), \emptyset), (\emptyset, \emptyset))$.}
	\label{fig:btree_example}
\end{figure}

\noindent
Ora stiamo descrivendo alberi \textit{vuoti}, ossia che non contengono dati:
alternative sono gli alberi \textbf{ancillari}, i quali contengono dati solo nei nodi interni
o solo nelle foglie. Al netto di queste distinzioni, definiamo $E$ l'insieme dei nodi
esterni, $I$ l'insieme dei nodi interni di un albero binario e $n$ il numero di nodi
interni, ossia $n = |I|$. Inoltre, `equipaggiamo' gli alberi binari di due funzioni
$ext$ e $int$, che denotano l'insieme delle foglie interne e l'insieme delle foglie
interne di un albero binario.

\begin{theorem} \label{thm:btree_leaves}
	In ogni albero binario, il numero di foglie in un albero binario è uguale al numero di nodi esterni più uno:
	$$
		|E| = |I| + 1
	$$
\end{theorem}
\begin{proof}
	Per induzione.
	\begin{itemize}
		\item{\bf base}: $ext(\emptyset) = 1$ e $int(\emptyset) = 0$.
		\item{\bf induzione}: siano $L, R$ due alberi binari. Allora
		$$
			ext((L,R)) = ext(L) + ext(R) = int(L) + 1 + int(R) + 1 = int((L, R))  + 1
		$$
	\end{itemize}
\end{proof}

\begin{corollario}
	Ogni albero con $n$ nodi interni ha in totale $2n +1$ nodi.
\end{corollario}

\begin{theorem}[di Catalan]\label{thm:catalan}
	Il numero di alberi binari con $n$ nodi interni è
	$$
		C_n = \frac{1}{n+1}{2n \choose n}
	$$
\end{theorem}
\begin{proof}
	Omessa.
\end{proof}
\begin{corollario}
	$ \forall n ~ \log_2(C_n)  =2n + O(\log_2(n)) $
\end{corollario}
\begin{proof}
	Utilizzando l'approssimazione di Stirling
	$$
		x! \approx \sqrt{2\pi x} (\frac{x}{e})^x
	$$

	si ha che
	$$
		C_n = \frac{1}{n+1} \frac{(2n)!}{n! (2n - n)!} = \frac{1}{n+1}\frac{(2n)!}{(n!)^2} \approx
		\frac{1}{n+1} \frac{\sqrt{4 \pi n} (\frac{2n}{e})^{2n}}{2 \pi n (\frac{n}{e})^{2n}}
		= \frac{1}{n+1} \frac{1}{\sqrt{\pi n }} 2^{2n} \approx \frac{4^n}{\sqrt{\pi n^3}}
	$$
	(che può essere dimostrato asintoticamente corretto).
	Questo significa che
	$$
		\log_2(C_n) = n \log_2(4) - \frac{1}{2}\log_2(\pi n^3)
		= 2n - \frac{3}{2}\log_2(n) - \frac{1}{2}\log_2(\pi)
		= 2n + O(\log_2(n))
	$$
\end{proof}
\begin{corollario}
	Per memorizzare alberi binari con $n$ nodi interni sono necessari
	$$
		Z_n = 2n + O(\log_2(n)) \text{ bit}
	$$
\end{corollario}

\subsection{L'ADT albero binario}
L'ADT che vogliamo costruire per un albero binario è definito a partire da una definizione
di nodi interni $I$, nodi interni $E$, e relazioni genitore-figlio. Le operazioni
che vogliamo eseguire su questi oggetti sono:
$$
	\forall n \in (I \cup E) ~~ \mathbf{is\_leaf}(n) = 1 \iff n \in  E
$$

$$
	\forall n \in (I \cup E) ~~ \mathbf{left\_child}(n) = \{n_l | n_l \text{ è il figlio sinistro di } n\}
$$

$$
	\forall n \in (I \cup E) ~~ \mathbf{right\_child}(n) = \{n_l | n_l \text{ è il figlio destro di } n\}
$$

$$
	\forall n \in (I \cup E) ~~ \mathbf{parent}(n) = \{p | n \text{ è il figlio di } p\}
$$


\subsubsection{Rappresentazione}
Per rappresentare un albero binario con $n$ nodi interni, numeriamo seguendo una visita in
ampiezza i nodi dell'albero, come rappresentato in \cref{fig:btree_rappr_num}. I numeri assegnati
andranno da $0$ a $2n$ e introducono una biiezione tra l'insieme $\{0, \cdots, 2n\}$ e i nodi,
realizzato come una funzione $node: \{0, \cdots, 2n\} \rightarrow (E \cup I)$.

\begin{figure}[h]
	\centering
	\begin{subfigure}[t]{0.45\textwidth}
		\centering
		\begin{tikzpicture}[
				every node/.style={circle, draw, minimum size=4mm, inner sep=0.5mm},
				level distance=10mm,
				level 1/.style={sibling distance=25mm},
				level 2/.style={sibling distance=25mm}]
			\node [label=above:{0}]{}
			child {
			node [label=above:{1}] {}
			child {node[fill, label=above:{3}]{}}
			child {
			node[label=above:{4}]{}
			child {
			node [label=above:{5}]{}
			child {node[fill,label=above:{7}]{}}
			child {node[fill,label=above:{8}]{}}
			}
			child {
			node[label=above:{6}]  {}
			[level distance=10mm ,sibling distance=15mm]
			child {node[fill,label=above:{9}] {}}
			child {node[fill,label=above:{10}] {}}
			}
			}
			}
			child {node [fill, label=above:{2}]{}};
		\end{tikzpicture}

		\caption{L'albero binario $((\emptyset, ((\emptyset, \emptyset), (\emptyset, \emptyset))), \emptyset)$ numerato.}
		\label{fig:btree_rappr_num}

	\end{subfigure}
	\begin{subfigure}[t]{0.45\textwidth}
		\centering
		\tikzset{every picture/.style={line width=0.75pt}} %set default line width to 0.75pt        

		\begin{tikzpicture}[x=0.75pt,y=0.75pt,yscale=-1,xscale=1]
			%uncomment if require: \path (0,300); %set diagram left start at 0, and has height of 300

			%Shape: Rectangle [id:dp4589196649246813] 
			\draw   (160,120) -- (180,120) -- (180,140.67) -- (160,140.67) -- cycle ;
			%Shape: Rectangle [id:dp7482491218517459] 
			\draw   (180,120) -- (200,120) -- (200,140.67) -- (180,140.67) -- cycle ;
			%Shape: Rectangle [id:dp3055160865785119] 
			\draw   (200,120) -- (220,120) -- (220,140.67) -- (200,140.67) -- cycle ;
			%Shape: Rectangle [id:dp6460750266770654] 
			\draw   (220,120) -- (240,120) -- (240,140.67) -- (220,140.67) -- cycle ;
			%Shape: Rectangle [id:dp45180487257490853] 
			\draw   (240,120) -- (260,120) -- (260,140.67) -- (240,140.67) -- cycle ;
			%Shape: Rectangle [id:dp28629681352535363] 
			\draw   (260,120) -- (280,120) -- (280,140.67) -- (260,140.67) -- cycle ;
			%Shape: Rectangle [id:dp7936708770675233] 
			\draw   (280,120) -- (300,120) -- (300,140.67) -- (280,140.67) -- cycle ;
			%Shape: Rectangle [id:dp23915136536136594] 
			\draw   (300,120) -- (320,120) -- (320,140.67) -- (300,140.67) -- cycle ;
			%Shape: Rectangle [id:dp722227591081337] 
			\draw   (320,120) -- (340,120) -- (340,140.67) -- (320,140.67) -- cycle ;
			%Shape: Rectangle [id:dp6052306075858915] 
			\draw   (340,120) -- (360,120) -- (360,140.67) -- (340,140.67) -- cycle ;
			%Shape: Rectangle [id:dp5583026074292488] 
			\draw   (360,120) -- (380,120) -- (380,140.67) -- (360,140.67) -- cycle ;

			% Text Node
			\draw (121,122.4) node [anchor=north west][inner sep=0.75pt]    {$\mathbf{b} \ =\ $};
			% Text Node
			\draw (164,101.4) node [anchor=north west][inner sep=0.75pt]    {$0$};
			% Text Node
			\draw (184,101.4) node [anchor=north west][inner sep=0.75pt]    {$1$};
			% Text Node
			\draw (204,101.4) node [anchor=north west][inner sep=0.75pt]    {$2$};
			% Text Node
			\draw (224,101.4) node [anchor=north west][inner sep=0.75pt]    {$3$};
			% Text Node
			\draw (244,101.4) node [anchor=north west][inner sep=0.75pt]    {$4$};
			% Text Node
			\draw (264,101.4) node [anchor=north west][inner sep=0.75pt]    {$5$};
			% Text Node
			\draw (284,101.4) node [anchor=north west][inner sep=0.75pt]    {$6$};
			% Text Node
			\draw (304,101.4) node [anchor=north west][inner sep=0.75pt]    {$7$};
			% Text Node
			\draw (324,101.4) node [anchor=north west][inner sep=0.75pt]    {$8$};
			% Text Node
			\draw (344,101.46) node [anchor=north west][inner sep=0.75pt]    {$9$};
			% Text Node
			\draw (361,101.4) node [anchor=north west][inner sep=0.75pt]    {$10$};
			% Text Node
			\draw (164,121.4) node [anchor=north west][inner sep=0.75pt]    {$1$};
			% Text Node
			\draw (184,121.4) node [anchor=north west][inner sep=0.75pt]    {$1$};
			% Text Node
			\draw (204,121.4) node [anchor=north west][inner sep=0.75pt]    {$0$};
			% Text Node
			\draw (224,121.4) node [anchor=north west][inner sep=0.75pt]    {$0$};
			% Text Node
			\draw (244,121.4) node [anchor=north west][inner sep=0.75pt]    {$1$};
			% Text Node
			\draw (264,121.4) node [anchor=north west][inner sep=0.75pt]    {$1$};
			% Text Node
			\draw (284,121.4) node [anchor=north west][inner sep=0.75pt]    {$1$};
			% Text Node
			\draw (304,121.4) node [anchor=north west][inner sep=0.75pt]    {$0$};
			% Text Node
			\draw (324,121.4) node [anchor=north west][inner sep=0.75pt]    {$0$};
			% Text Node
			\draw (344,121.46) node [anchor=north west][inner sep=0.75pt]    {$0$};
			% Text Node
			\draw (364,121.4) node [anchor=north west][inner sep=0.75pt]    {$0$};

		\end{tikzpicture}
		\caption{Vettore $\mathbf{v}$ associato.}
		\label{fig:btree_rappr_vec}
	\end{subfigure}
	\caption{Un albero binario e il suo vettore associato.}
\end{figure}
\noindent
La concreta rappresentazione dell'albero si realizza con un vettore di bit di lunghezza
$2n + 1$ $\mathbf{v}$ tale che
$$
	\forall i \in \{0, \cdots, 2n\} ~~ \mathbf{v}[i] =
	\begin{cases}
		1 & node(i) \in I \\
		0 & node(i) \in E
	\end{cases}
$$
ottenendo quindi esattamente $n$ `$1$' nel vettore, come mostrato in \cref{fig:btree_rappr_vec}.

\subsubsection{Implementazione}
Per capire come implementare l'ADT utilizzando il vettore $\mathbf{v}$, ci poniamo
nella situazione di dover trovare, dato un nodo numerato $p$ in un albero
binario $T$, i suoi due figli, i quali sono numerati $q$ e $q+1$.
Sia quindi $T'$ un sottoalbero di $T$, radicato nella stessa radice di $T$,
che contiene tutti i nodi di $T$ numerati fino al nodo precedente
al nodo $q$, come rappresentato in \cref{fig:btree_rappr_step}.


\begin{figure}[h]
	\centering



	\tikzset{every picture/.style={line width=0.75pt}} %set default line width to 0.75pt        

	\begin{tikzpicture}[x=0.75pt,y=0.75pt,yscale=-1,xscale=1]
		%uncomment if require: \path (0,539); %set diagram left start at 0, and has height of 539

		%Shape: Trapezoid [id:dp3557987154319203] 
		\draw  [color={rgb, 255:red, 246; green, 237; blue, 156 }  ,draw opacity=1 ][fill={rgb, 255:red, 246; green, 237; blue, 156 }  ,fill opacity=1 ] (312.92,126.04) -- (317.75,115.9) -- (328.33,115.9) -- (333.17,126.04) -- cycle ;
		%Shape: Right Triangle [id:dp7862842021574125] 
		\draw  [color={rgb, 255:red, 246; green, 237; blue, 156 }  ,draw opacity=1 ][fill={rgb, 255:red, 246; green, 237; blue, 156 }  ,fill opacity=1 ] (346,114.3) -- (334.2,125.7) -- (334.2,114.3) -- cycle ;
		%Shape: Triangle [id:dp9413454407809129] 
		\draw  [color={rgb, 255:red, 246; green, 237; blue, 156 }  ,draw opacity=1 ][fill={rgb, 255:red, 246; green, 237; blue, 156 }  ,fill opacity=1 ] (360.11,21.08) -- (402.87,115.9) -- (317.34,115.9) -- cycle ;
		%Shape: Circle [id:dp7169355320482411] 
		\draw  [fill={rgb, 255:red, 0; green, 0; blue, 0 }  ,fill opacity=1 ] (360.11,106.08) .. controls (360.11,102.68) and (362.87,99.92) .. (366.28,99.92) .. controls (369.68,99.92) and (372.44,102.68) .. (372.44,106.08) .. controls (372.44,109.49) and (369.68,112.25) .. (366.28,112.25) .. controls (362.87,112.25) and (360.11,109.49) .. (360.11,106.08) -- cycle ;
		%Straight Lines [id:da11031799115386731] 
		\draw    (366.28,106.08) -- (356.04,122.86) ;
		%Straight Lines [id:da5314879825236313] 
		\draw    (366.28,106.08) -- (376.09,121.82) ;
		%Shape: Triangle [id:dp2636274720755136] 
		\draw   (360.11,21.08) -- (436.94,191.08) -- (283.28,191.08) -- cycle ;
		%Straight Lines [id:da07107046785599602] 
		\draw    (312.18,126.95) -- (334.47,126.95) ;
		%Straight Lines [id:da23996966123276942] 
		\draw    (345.08,116.14) -- (402.93,116.14) ;
		%Straight Lines [id:da8657897153449609] 
		\draw    (345.49,115.99) -- (333.99,127.24) ;
		%Shape: Circle [id:dp6390655913899737] 
		\draw  [fill={rgb, 255:red, 255; green, 255; blue, 255 }  ,fill opacity=1 ] (350.11,123.93) .. controls (350.11,120.53) and (352.87,117.76) .. (356.28,117.76) .. controls (359.68,117.76) and (362.44,120.53) .. (362.44,123.93) .. controls (362.44,127.34) and (359.68,130.1) .. (356.28,130.1) .. controls (352.87,130.1) and (350.11,127.34) .. (350.11,123.93) -- cycle ;
		%Shape: Circle [id:dp31674144349432576] 
		\draw  [fill={rgb, 255:red, 255; green, 255; blue, 255 }  ,fill opacity=1 ] (370.11,124.11) .. controls (370.11,120.71) and (372.87,117.95) .. (376.28,117.95) .. controls (379.68,117.95) and (382.44,120.71) .. (382.44,124.11) .. controls (382.44,127.52) and (379.68,130.28) .. (376.28,130.28) .. controls (372.87,130.28) and (370.11,127.52) .. (370.11,124.11) -- cycle ;
		%Shape: Right Triangle [id:dp4307506900461393] 
		\draw  [color={rgb, 255:red, 246; green, 237; blue, 156 }  ,draw opacity=1 ][fill={rgb, 255:red, 246; green, 237; blue, 156 }  ,fill opacity=1 ] (320.12,125.88) -- (333.92,109.62) -- (333.95,125.85) -- cycle ;

		% Text Node
		\draw (341.53,125.76) node [anchor=north west][inner sep=0.75pt]    {$q$};
		% Text Node
		\draw (364.58,131.05) node [anchor=north west][inner sep=0.75pt]    {$q+1$};
		% Text Node
		\draw (363.18,78.26) node [anchor=north west][inner sep=0.75pt]    {$p$};
		% Text Node
		\draw (353.2,51.2) node [anchor=north west][inner sep=0.75pt]    {$T'$};
		% Text Node
		\draw (414,30.4) node [anchor=north west][inner sep=0.75pt]    {$T$};


	\end{tikzpicture}
	\caption{Sottoalbero $T'$ di $T$ utilizzato per calcolare $q$ e $q+1$.}
	\label{fig:btree_rappr_step}
\end{figure}

Definiamo l'insieme dei nodi interni di $T$ con numerazione strettamente
minore di $p$
$$
	E_{T, p} = \{n \in int(T) | n < p \}
$$
e, intuitivamente, deve essere
$$
	q = |int(T')| + |ext(T')| = 2 |int(T')| + 1 = 2 |E_{T,p}| + 1 ) = 2 \mathbf{rank_v}(p) + 1.
$$
poiché $|E_{T,p}|$ è uguale al numero di `$1$' presenti in $\mathbf{v}$ con indice
strettamente minore a $p$, concludendo, chiaramente se e solo se
$\mathbf{is\_leaf}(p) = \mathbf{v}[p] = 1$, che\footnote{In quanto segue si omette
	la biiezione $node$, dando per implicito il rapporto tra enumerazione e nodi.}
$$
	\forall p \in \{0, \cdots, 2n\} ~~ \mathbf{left\_child}(p) = 2 \mathbf{rank_v}(p) + 1
$$
$$
	\forall p \in \{0, \cdots, 2n\} ~~ \mathbf{right\_child}(p) = 2 \mathbf{rank_v}(p) + 2
$$
Per risalire al genitore a partire da un figlio, ossia da $q$ risalire a $p$,
basta osservare che $p$ è un nodo tale per cui
$$
	\begin{cases}
		2\mathbf{rank_v}(p) + 1 = q & \text{se } q \text{ è il figlio di sinistra} \\
		2\mathbf{rank_v}(p) + 2 = q & \text{se } q \text{ è il figlio di destra}
	\end{cases}
	\implies
	\begin{cases}
		\mathbf{rank_v}(p) + \frac{1}{2} = \frac{q}{2} \\
		\mathbf{rank_v}(p) + 1 = \frac{q}{2}
	\end{cases}
	\implies
	\begin{cases}
		\mathbf{rank_v}(p) = \frac{q}{2}  - \frac{1}{2} \\
		\mathbf{rank_v}(p) = \frac{q}{2} - 1
	\end{cases}
$$
ma possiamo concludere che
$$
	\mathbf{rank_v}(p) = \lfloor \frac{q}{2} - \frac{1}{2} \rfloor
$$
siccome a seconda che $q$ sia pari o dispari l'equazione sarà sempre verificata,
ossia questa uguaglianza è vera se e solo se è almeno una delle due precedenti;
se $\frac{q}{2} - \frac{1}{2}$ è intero, allora è vera la prima equazione, mentre
se non è intero si ottiene $\frac{q}{2} - 1$, verificando la seconda.
Applichiamo ad entrami i membri l'operazione di $\mathbf{select}$:
$$
	\mathbf{select_v(rank_v}(p)) = \mathbf{select_v}(\lfloor \frac{q}{2} - \frac{1}{2} \rfloor)
	\implies p = \mathbf{select_v}(\lfloor \frac{q}{2} - \frac{1}{2} \rfloor)
$$

\subsubsection{Complessità in spazio}
Per rappresentare un albero con $n$ nodi interni utilizziamo un vettore $b$ che ha tanti
bit tanti quanti sono i nodi dell'albero, ossia $2n +1$. Oltre a questo, si utilizza
lo spazio utilizzato dalle strutture di rank e select, ossia
$$
	D_n = 2n +1 + o(2n+1) = 2n + 1 + o(n)
$$
con un risultato per $Z_n$ pari a $2n + O(\log_1(n))$, la differenza è
$$
	D_n - Z_n = o(n)
$$
pertanto la struttura è succinta con accesso in tempo costante.

\subsubsection{Alberi binari con dati}
Se i dati si trovano su ogni tipo di nodo dell'albero (sia interni che interni) i dati
ancillari si possono mantenere in un ulteriore vettore della stessa lunghezza.
Se, invece, i dati ancillari si trovano esclusivamente sui nodi interni,
la situazione è leggermente più complicata: il vettore dei dati avrà
lunghezza pare ai al numero di nodi interni e sarà necessario ottenere il numero
ordinale del nodo interno, utilizzando quindi la funzione rank.
Alternativamente, se si vogliono salvare dati sulle foglie, si dovrà utilizzare
una tecnica in grado di contare il \texttt{rank} per gli $0$.

\section{Struttura di Elias-Fano per sequenze monotone}
La struttura (o codifica, rappresentazione) di Elias-Fano che andiamo
a presentare occupa sulle sequenze monotone di interi, ossia
una sequenza di $n$ numeri interni
$$
	0 \leq x_0, x_1, \cdots, x_{n+1}
$$
ordinati, ossia
$$
	\forall i, j ~ i < j \implies x_i < x_j
$$
e tali per cui
$$
	\exists u ~ \forall i ~ x_i < u
$$
dove $u$ è chiamato \textit{dimensione dell'universo}.
L'ADT che vogliamo rappresentare richiede una primitiva
che, presentato un indice $i$, restituisce l'$i$-esimo elemento della sequenza,
ossia
$$
	\forall i ~ \mathbf{project}(i) = x_i
$$
La rappresentazione di seqenze monotone di interi è molto utile.
Per esempio, per rappresentare un grafo, si può attribuire ad ogni
nodo un numero: l'idea classica è quella di utilizzare poi, per esempio,
liste di adiacenza per elencare i nodi raggiungibili da un certo noto
considerato. I nodi adiacenti possono essere ordinati in ordine crescente,
ottenendo una sequenza esattamente monotona.
Un altro esempio di utilizzo di sequenze monotone è negli indici inversi.


\subsection{Rappresentazione}
La rappresentazione banale sarebbe utilizare un vettore di $n$ numeri
utilizzando per ognuno $\log_2(u)$ bit; la rappresentazione di Elias-Fano
utilizza meno spazio basandosi sulla seguente idea: si definisce
$$
	l = \max\{0, \lfloor\log_2(\frac{u}{n})|\rfloor\}
$$
e gli $l$ bit meno significativi di ogni $x_i$ vengono salvati
esplicitamente, mentre i rimanenti bit vengono rappresentati
in un altro modo. Consideremo sempre il caso in cui $l\neq 0$, in
quanto $l = 0 \iff u < n$, facendo quindi una moderata
assunzione di sparsità.
\begin{table}[h]
	\centering
	\begin{tabular}{c}
		\begin{lstlisting}
	 0 0 0 1 parte superiore
	 . . . .
	 1 1 1 1 		
	 ------- limite l 
	 0 1 1 1 			
	 . . . .
	 1 0 0 1 parte inferiore
    \end{lstlisting}
	\end{tabular}
	\caption{Divisione di ogni $x_i$.}
\end{table}

\subsubsection{Implementazione e complessità in spazio}
Le parti inferiori vengono `estratte' definendo
$$
	\forall i \in \{0, \cdots, n\} ~ l_i = x_i \mod 2^l
$$
utilizzando $n \cdot l$ bit in totale.  Ciò che rimane è la parte superiore, ossia
$s_i = \lfloor{\frac{x_{i}}{2^l}}\rfloor$; definiamo una sequenza di differenze
$$
	d_i = \lfloor{\frac{x_i}{2^l}}\rfloor - \lfloor{\frac{x_{i-1}}{2^l}}\rfloor
$$
assumendo $x_{-1} = 0$. Stiamo sfruttando il fatto che la sequenza sia non
decrescente, quindi $\forall i ~ d_i \geq 0$.

Ogni differenza della sequenza viene rappresentata in unario, utilizzando $d_i$ bit
`$0$' seguiti dal bit `$1$' per rappresentare il valore $d_i$; la sequenza
così rappresentata viene salvata in un vettore $\mathbf{d}$ equipaggiato
con funzioni di rango e selezione. Questa struttura occupa in spazio
una quantità al più
$$
	\sum_{i = 0}^{n - 1}(d_i + 1) = n + \sum_{i=0}^{n-1} d_i
	= n + \sum_{i = 0}^{n-1} (\lfloor{\frac{x_i}{2^l}}\rfloor - \lfloor{\frac{x_{i-1}}{2^l}}\rfloor) \text{ bit}
$$
che identifica una serie telescopica; pertanto, il consumo in spazio è
$$
	n - (\lfloor{\frac{x_{n-1}}{2^l}}\rfloor -\lfloor{\frac{x_{-1}}{2^l}}\rfloor) \leq n + \frac{u}{2^l}
	= n + \frac{u}{2^{\lfloor \log_2(\frac{u}{n})\rfloor}} \text{ bit}
$$
Se $u/n$ è una potenza di $2$, allora il consumo è
$$
	n + \frac{u}{(u/n)} = 2n \text{ bit}
$$
altrimenti è al più
$$
	n + \frac{u}{2^{\log_2(\frac{u}{n}) - 1 }} = n + \frac{u}{2^{\log_2(u/n)}1/2} = n + \frac{2n}{u/n} = 3n \text{ bit}
$$
Considerando entrambe le parti, questa struttura occupa in spazio $l\cdot n$ bit
per la parte inferiore e $2n$ o $3n$ bit per la parte superiore; in totale,
quindi, si consumano $(l+2)n$ o $(l+3)n$ bit.
Ricordando che stiamo considerando sempre $l = \lfloor \log_2(u/n) \rfloor$, consideriamo
$$
	\lceil \log_2(u/n) \rceil =
	\begin{cases}
		l & u/n \text{ è una potenza di } 2 \\
		l +1                                \\
	\end{cases}
$$
e concludiamo
$$
	D_n = (2 + \lceil \log_2(u/n) \rceil) n \text{ bit}
$$
che però non tiene conto dello spazio occupato dalla struttura di rango e selezione,
che infatti sono necessiarie: supponiamo di volere la posizione dell'$i$-esimo $1$ in $\mathbf{d}$:
$$
	\mathbf{select_d}(i) = d_0 + d_1 + \cdots + d_{i-1} + i
$$
quindi
$$
	\mathbf{select_d}(i) - i = \sum_{j = 0}^i \lfloor{\frac{x_{j}}{2^l}}\rfloor -\lfloor{\frac{x_{j-1}}{2^l}}\rfloor  = \lfloor \frac{x_i}{2^l} \rfloor
$$
di conseguenza
$$
	x_i =  \lfloor \frac{x_i}{2^l} \rfloor 2^l + (x_i ~ \mathrm{mod}{2^l}) = (\mathbf{select_d}(i) - i ) 2^l + l_i
$$
Contando anche le strutture di rank e select, che quindi sono necessarie, si occupa uno spazio
$$
	D_n = (2 + \lceil \log_2(u/n) \rceil)n + o(n) \text{ bit}
$$

\subsection{Lower bound per le strutture di Elias-Fano}
La parte difficile dello studio di questa struttura è il calcolo del lower bound:
dobbiamo considerare tutte le sequenze monotone
$$
	0 \leq x_0 \leq \cdots \leq x_{n-1} \le u
$$
Quante sono, una volta fissati $n$ e $u$? Esse sono in biiezione con i multinsiemi di cardinalità
$n$ sottoinsiemi di $\{0,1, \cdots, u-1\}$. Uno di questi multinsiemi si può vedere come
$$
	c_0, c_1, \cdots, c_{u-1}
$$
dove ogni $c_i$ è il numero di occorrenze del valore $i$ nel multinsieme, ossia
il numero di soluzioni intere non negative dell'equazione
$$
	c_0 + c_1, \cdots, c_{u-1} = n
$$
e possiamo calcolare il numero di sequenze monotone calcolando il numero
di possibili soluzioni per questa equazione fissati $n$ e $u$.

Per esempio, si immagini di avere un universo di dimensione $u = 7$ e $n = 5$ numeri
estratti da $\{0, 1, \cdots, 7\}$: la sequenza
$$
	0 \leq 1 \leq 3 \leq 3 \leq 5 \leq 6 \le 7
$$
la cui equazione associata è
$$
	c_0 + c_1 + c_2 + \cdots + c_6 = 5
$$
è rappresentabile, in termini di $c_i$, come
$$
	c_0 = 0, c_1 = 1, c_2 = 0, c_3 = 2, c_4 = 0, c_5 = 1, c_6 = 1
$$
che è una soluzione dell'equazione precedente.

\subsubsection{Metodo stars and bars}
Per contare le possibili soluzioni si può utilizzare la tecnica \textit{stars and bars}, in
cui si utilizzano delle stringhe costruite utilizzando $n$ stelline e $u-1$ barrette, piazzate
in qualsiasi posizione:
$$
	| \star || \star \star || \star | \star
$$

Il numero di soluzioni è uguale al numero di stringhe costruite in tale modo: i valori nella sequenza
sono $n$ e vanno `distribuiti' in $u$ spazi, delineati dalle $u-1$ barrette. L'interpretazione della
stringa appena mostrata è $ c_0 = 0, c_1 = 1, c_2 = 0, c_3 = 2, c_4 = 0, c_5 = 1, c_6 = 1 $, come
nell'esempio precedente. I simboli utilizzabili sono $(u-1) + n$, pertanto le possibili stringhe
costruibili con questi simboli sono  ${u + n -1}\choose{u - 1}$.
Questo fornisce il lower bound desiderato:
$$
	Z_n = \log_2{{u + n -1}\choose{u - 1}} = \log_2{{u + n - 1}\choose{n}} \text{ bit}
$$
e introduciamo l'approssimazione
$$
	{{a}\choose{b}} \approx b \log_2(\frac{a}{b}) + (a - b)\log_2(\frac{a}{a-b})
$$
che ci permette di scrivere
\begin{align*}
	Z_n = \log_2{{u + n -1}\choose{u - 1}} & \approx n \log_2(\frac{u + n -1}{n}) + (u -1) \log_2(\frac{u + n - 1}{u - 1}) =        \\
	                                       & = n \log_2(\frac{u + n - 1}{n}) = n \log_2(\frac{u}{n}(1 + \frac{n}{u} - \frac{1}{u})) \\
	                                       & = n \log_2(\frac{u}{n}) + n \log_2(1 + \frac{n}{u} - \frac{1}{u}) \text{ bit}
\end{align*}

\noindent
Un'altra proprietà che utilizziamo è
$$
	x \approx \ln(1 + x)
$$
quindi
\begin{align*}
	Z_n & = n \log_2(\frac{u}{n}) + n \ln(1 + \frac{n}{u}-\frac{1}{u})\frac{1}{\ln(2)}
	\approx n \log_2(\frac{u}{n}) + n (\frac{n}{u}-\frac{1}{u})\frac{1}{\ln(2)}               \\
	    & = n \log_2(\frac{u}{n}) + \frac{1}{u}\frac{1}{\ln(2)} - \frac{n}{u}\frac{1}{\ln(2)}
	\approx n \log_2(\frac{u}{n}) \text{ bit}
\end{align*}

L'utilizzo in spazio calcolato precedentemente è
$$
	D_n = 2n + n \lceil \log_2(\frac{u}{n}) \rceil + o(n)
$$
In questo frangente, per raffinare l'analisi, conviene considerare la differenza
col lower bound non sull'intera struttura, bensì sul singolo elemento.
Definiamo, quindi, il lower bound per l'elemento
$$
	\bar{Z_n} \approx \log_2(\frac{u}{n})
$$
e
$$
	\bar{D_n} = 2+ \lceil \log_2(\frac{u}{n}) \rceil + o(n) = \bar{Z_n} + O(1)
$$
la struttura è `quasi' implicita, dove l'incertezza deriva dal numero di
approssimazioni fatte e sull'assunzione di sparsità; in pratica, ciò che succede
in pratica è che questa uguaglianza vale quando $n \leq \sqrt{u}$.

\section{Struttura per parentesi ben formate}
\subsection{Linguaggi di Dyck}
Un linguaggio di Dyck $L$ è un linguaggio sull'alfabeto $D={(,)}$ tale che $L\subseteq D^*$
e una stringa $w \in D^*$ appartiene a $L$ se e solo se
\begin{enumerate}
	\item $|w|_{(} = |w|_{)}$; e
	\item $\forall w_1, w_2 ~~ w = w_1w_2 \implies|w_1|_{(} \geq |w_2|_{)}$
\end{enumerate}
Un modo interessante per studiare una parola $w$ di Dyck è studiarne la
\textit{funzione di eccesso}, definita
$$
	E_w(i) = |\{j | j \le i ~ w_j = ( ~\}| - |\{j | j \leq i ~ w_j = ) ~\}|
$$
\begin{figure}[h]
	\centering
	\begin{tikzpicture}
		\draw[->] (-1, 0) -- (8.50, 0) node[right] {$x$};
		\draw[->] (-1, 0) -- (-1, 4) node[above] {$y$};

		\draw (0,0) node[circle,fill,inner sep=1pt, label=above:{(}] {} -- (1,1) node[circle,fill,inner sep=1pt]{};
		\draw (1,1) node[circle,fill,inner sep=1pt, label=above:{(}] {} -- (2,1) node[circle,fill,inner sep=1pt]{};
		\draw (2,1) node[circle,fill,inner sep=1pt, label=above:{)}] {} -- (3,1) node[circle,fill,inner sep=1pt]{};
		\draw (3,1) node[circle,fill,inner sep=1pt, label=above:{(}] {} -- (4,2) node[circle,fill,inner sep=1pt]{};
		\draw (4,2) node[circle,fill,inner sep=1pt, label=above:{(}] {} -- (5,2) node[circle,fill,inner sep=1pt]{};
		\draw (5,2) node[circle,fill,inner sep=1pt, label=above:{)}] {} -- (6,1) node[circle,fill,inner sep=1pt]{};
		\draw (6,1) node[circle,fill,inner sep=1pt, label=above:{)}] {} -- (7,0) node[circle,fill,inner sep=1pt, label=above:{)}]{};
	\end{tikzpicture}
	\caption{Funzione di eccesso per la parola $(()(()))$.}
	\label{fig:func_excess}
\end{figure}

Questa funzione parte da $0$, può tornare a $0$ e, se la parola è effettivamente nel
linguaggio, termina a $0$. Ci sono due tipi di stringhe di Dyck: quelle per cui
la funzione non raggiunge nessuno $0$ tranne quello iniziale e quello finale,
chiamate \textit{fortemente bilanciate}, e quelle per cui questa funzione raggiunge
uno $0$ diverso da quello iniziale e quello finale, quindi $w = w_1w_2$ è
\textit{debolmente bilanciata} e tale per cui almeno uno tra $w_1$ e $w_2$ è
fortemente bilanciata.

Ci sono molti motivi per cui le sequenze di parentesi sono interessanti. In particolare,
le espressioni ben parentesizzate sono in biiezione sia con le foreste di alberi binari che con gli
alberi non binari. In qualche modo, esattamente come abbiamo visto una rappresentazione
per gli alberi binari in precedenza, ora stiamo analizzando una struttura per alberi
generali. Le sequenze di parentesi aperte e chiuse le penseremo come stringhe di bit, cioè
una stringa
$$
	(()(()())(()))
$$
è rappresentata come una stringa
$$
	11011010011000
$$
la cui funzione di eccesso è rappresentata in \cref{fig:func_excess_example}.

\begin{figure}[h]
	\centering
	\begin{tikzpicture}
		\draw[->] (-1, 0) -- (14.5, 0) node[right] {$x$};
		\draw[->] (-1, 0) -- (-1, 4) node[above] {$y$};

		\draw (0,0) node[label=above:{(}]	{} -- 	(1,1) node[circle,fill,inner sep=1pt]{};
		\draw (1,1) node[label=above:{(}] 	{} -- 	(2,1) node[circle,fill,inner sep=1pt]{};
		\draw (2,1) node[label=above:{)}] 	{} -- 	(3,1) node[circle,fill,inner sep=1pt]{};
		\draw (3,1) node[label=above:{(}] 	{} -- 	(4,2) node[circle,fill,inner sep=1pt]{};
		\draw (4,2) node[label=above:{(}] 	{} -- 	(5,2) node[circle,fill,inner sep=1pt]{};
		\draw (5,2) node[label=above:{)}] 	{} -- 	(6,2) node[circle,fill,inner sep=1pt]{};
		\draw (6,2) node[label=above:{(}] 	{} -- 	(7,2) node[circle,fill,inner sep=1pt]{};
		\draw (7,2) node[label=above:{)}] 	{} -- 	(8,1) node[circle,fill,inner sep=1pt]{};
		\draw (8,1) node[label=above:{)}] 	{} -- 	(9,1) node[circle,fill,inner sep=1pt]{};
		\draw (9,1) node[label=above:{(}] 	{} -- 	(10,2) node[circle,fill,inner sep=1pt]{};
		\draw (10,2) node[label=above:{(}]  	{} -- 	(11,2) node[circle,fill,inner sep=1pt]{};
		\draw (11,2) node[label=above:{)}]  	{} -- 	(12,1) node[circle,fill,inner sep=1pt]{};
		\draw (12,1) node[label=above:{)}]  	{} -- 	(13,0) node[circle,fill,inner sep=1pt,label=above:{)}]{};
	\end{tikzpicture}
	\caption{Funzione di eccesso per la parola $(()(()())(()))$.}
	\label{fig:func_excess_example}
\end{figure}

\subsection{L'ADT stringa bilanciata}
L'ADT che vogliamo descrivere per le stringhe di parentesi ben bilanciate ha le primitive
$$
	\forall p \in \mathbb{N} ~~ \mathbf{find\_open}(p) = \text{parentesi aperta corrispondente alla chiusa in posizione } p
$$
$$
	\forall p \in \mathbb{N} ~~ \mathbf{find\_close}(p) = \text{parentesi chiusa corrispondente all'aperta in posizione } p
$$
$$
	\forall p \in \mathbb{N} ~~ \mathbf{enclose}(p) = \text{prima parentesi aperta che racchiude la parentesi in posizione } p
$$

Pensando ad una soluzione banale, una $\mathbf{find\_open}$ in una qualche posizione si può
calcolare innanzitutto realizzando che la parentesi aperta corrispondente ad una
parentesi chiusa è sicuramente prima della chiusa stessa; basta quindi cercare all'indietro fermandosi
alla prima parentesi aperta che ha la stessa funzione di eccesso. Analogamente si fa per $\mathbf{find\_close}$.
La cosa più semplice, quindi, sarebbe calcolare anticipatamente la funzione di eccesso e usarla
facendo ricerche lineari, utilizzando uno spazio $n \log_2(n)$.
Faremo meglio creando una struttura succinta con tempo di interrogazione logaritmico.

\subsubsection{Rappresentazione}
La prima operazione sulle parentesi da eseguire nella costruzione della struttura è dividere, nuovamente,
la parola in blocchi di lunghezza $l$, creando $k = \lceil n/l \rceil$ blocchi. In ogni blocco
ci sono parentesi chiuse e aperte e, in alcuni casi, le parentesi hanno la loro corrispondente
dentro il blocco stesso: in questa situazione, definiamo la parentesi \textit{vicina}, mentre
tutte le altri sono definite \textit{lontane}; in generale, tutte le parentesi aperte lontane
si chiuderanno in un qualche blocco successivo.

In ogni blocco ci sono alcune parentesi vicine e varie parentesi lontane. Può succedere
che una o più parentesi lontane si chiudano nello stesso blocco: definiamo \textit{pioniera}
la lontana aperta (benché si possa fare lo stesso ragionamento, a specchio, per le chiuse)
che è la prima del blocco al quale appartiene a chiudersi in un qualsiasi altro blocco successivo.
Un esempio di queste divisioni è in \cref{fig:paren_blocks}, dove le parentesi gialle sono le vicine,
quelle blu sono lontane e quelle rosse sono pioniere.

\begin{figure}[h]

	\centering

	\tikzset{every picture/.style={line width=0.75pt}} %set default line width to 0.75pt        

	\begin{tikzpicture}[x=0.75pt,y=0.75pt,yscale=-1,xscale=1]
		%uncomment if require: \path (0,300); %set diagram left start at 0, and has height of 300

		%Shape: Rectangle [id:dp12028039856705786] 
		\draw  [fill={rgb, 255:red, 208; green, 2; blue, 27 }  ,fill opacity=1 ] (160,170) -- (170,170) -- (170,190) -- (160,190) -- cycle ;

		%Shape: Rectangle [id:dp6017350169887977] 
		\draw  [fill={rgb, 255:red, 247; green, 240; blue, 148 }  ,fill opacity=1 ] (170,170) -- (180,170) -- (180,190) -- (170,190) -- cycle ;

		%Shape: Rectangle [id:dp1862637830525412] 
		\draw  [fill={rgb, 255:red, 247; green, 240; blue, 148 }  ,fill opacity=1 ] (180,170) -- (190,170) -- (190,190) -- (180,190) -- cycle ;

		%Shape: Rectangle [id:dp7375538410378665] 
		\draw  [fill={rgb, 255:red, 208; green, 2; blue, 27 }  ,fill opacity=1 ] (190,170) -- (200,170) -- (200,190) -- (190,190) -- cycle ;

		%Shape: Rectangle [id:dp42832713437851455] 
		\draw  [fill={rgb, 255:red, 139; green, 181; blue, 229 }  ,fill opacity=1 ] (200,170) -- (210,170) -- (210,190) -- (200,190) -- cycle ;

		%Shape: Rectangle [id:dp3771120606312378] 
		\draw  [fill={rgb, 255:red, 139; green, 181; blue, 229 }  ,fill opacity=1 ] (210,170) -- (220,170) -- (220,190) -- (210,190) -- cycle ;

		%Shape: Rectangle [id:dp19892421904849122] 
		\draw  [fill={rgb, 255:red, 247; green, 240; blue, 148 }  ,fill opacity=1 ] (220,170) -- (230,170) -- (230,190) -- (220,190) -- cycle ;

		%Shape: Rectangle [id:dp0006276055923941648] 
		\draw  [fill={rgb, 255:red, 247; green, 240; blue, 148 }  ,fill opacity=1 ] (230,170) -- (240,170) -- (240,190) -- (230,190) -- cycle ;

		%Shape: Rectangle [id:dp16538357452442487] 
		\draw  [fill={rgb, 255:red, 139; green, 181; blue, 229 }  ,fill opacity=1 ] (240,170) -- (250,170) -- (250,190) -- (240,190) -- cycle ;

		%Shape: Rectangle [id:dp06112380309966958] 
		\draw  [fill={rgb, 255:red, 208; green, 2; blue, 27 }  ,fill opacity=1 ] (250,170) -- (260,170) -- (260,190) -- (250,190) -- cycle ;

		%Shape: Rectangle [id:dp06951356935410424] 
		\draw  [fill={rgb, 255:red, 247; green, 240; blue, 148 }  ,fill opacity=1 ] (260,170) -- (270,170) -- (270,190) -- (260,190) -- cycle ;

		%Shape: Rectangle [id:dp885838800749071] 
		\draw  [fill={rgb, 255:red, 247; green, 240; blue, 148 }  ,fill opacity=1 ] (270,170) -- (280,170) -- (280,190) -- (270,190) -- cycle ;

		%Shape: Rectangle [id:dp37752499309076915] 
		\draw  [fill={rgb, 255:red, 139; green, 181; blue, 229 }  ,fill opacity=1 ] (280,170) -- (290,170) -- (290,190) -- (280,190) -- cycle ;

		%Shape: Rectangle [id:dp7109616333523956] 
		\draw  [fill={rgb, 255:red, 139; green, 181; blue, 229 }  ,fill opacity=1 ] (290,170) -- (300,170) -- (300,190) -- (290,190) -- cycle ;

		%Straight Lines [id:da856450147041028] 
		\draw    (210,160) -- (210,200) ;
		%Straight Lines [id:da0823905364290688] 
		\draw    (260,160) -- (260,200) ;
		%Straight Lines [id:da8657156752341498] 
		\draw    (205.14,190.29) -- (205.14,204.36) -- (216.43,204.36) -- (216.43,192.93) ;
		\draw [shift={(216.43,189.93)}, rotate = 90] [fill={rgb, 255:red, 0; green, 0; blue, 0 }  ][line width=0.08]  [draw opacity=0] (8.93,-4.29) -- (0,0) -- (8.93,4.29) -- cycle    ;
		%Straight Lines [id:da4613663625082196] 
		\draw    (195.57,189.79) -- (195.57,212.07) -- (244.57,212.07) -- (244.57,193.64) ;
		\draw [shift={(244.57,190.64)}, rotate = 90] [fill={rgb, 255:red, 0; green, 0; blue, 0 }  ][line width=0.08]  [draw opacity=0] (8.93,-4.29) -- (0,0) -- (8.93,4.29) -- cycle    ;
		%Straight Lines [id:da9063548393327128] 
		\draw    (165.43,190.21) -- (165.43,217.64) -- (284.43,217.64) -- (284.3,194.5) ;
		\draw [shift={(284.29,191.5)}, rotate = 89.69] [fill={rgb, 255:red, 0; green, 0; blue, 0 }  ][line width=0.08]  [draw opacity=0] (8.93,-4.29) -- (0,0) -- (8.93,4.29) -- cycle    ;
		%Straight Lines [id:da09579212789908009] 
		\draw    (255.29,170.21) -- (255.29,152.07) -- (296.14,152.07) -- (296.14,166.21) ;
		\draw [shift={(296.14,169.21)}, rotate = 270] [fill={rgb, 255:red, 0; green, 0; blue, 0 }  ][line width=0.08]  [draw opacity=0] (8.93,-4.29) -- (0,0) -- (8.93,4.29) -- cycle    ;

		% Text Node
		\draw (161,172) node [anchor=north west][inner sep=0.75pt]   [align=left] {(};
		% Text Node
		\draw (171,172) node [anchor=north west][inner sep=0.75pt]   [align=left] {(};
		% Text Node
		\draw (181,172) node [anchor=north west][inner sep=0.75pt]   [align=left] {)};
		% Text Node
		\draw (211,172) node [anchor=north west][inner sep=0.75pt]   [align=left] {)};
		% Text Node
		\draw (201,172) node [anchor=north west][inner sep=0.75pt]   [align=left] {(};
		% Text Node
		\draw (191,172) node [anchor=north west][inner sep=0.75pt]   [align=left] {(};
		% Text Node
		\draw (271,172) node [anchor=north west][inner sep=0.75pt]   [align=left] {)};
		% Text Node
		\draw (261,172) node [anchor=north west][inner sep=0.75pt]   [align=left] {(};
		% Text Node
		\draw (251,172) node [anchor=north west][inner sep=0.75pt]   [align=left] {(};
		% Text Node
		\draw (241,172) node [anchor=north west][inner sep=0.75pt]   [align=left] {)};
		% Text Node
		\draw (231,172) node [anchor=north west][inner sep=0.75pt]   [align=left] {)};
		% Text Node
		\draw (221,172) node [anchor=north west][inner sep=0.75pt]   [align=left] {(};
		% Text Node
		\draw (291,172) node [anchor=north west][inner sep=0.75pt]   [align=left] {)};
		% Text Node
		\draw (281,172) node [anchor=north west][inner sep=0.75pt]   [align=left] {)};
	\end{tikzpicture}

	\caption{Divisione in blocchi di una stringa parentesizzata.}
	\label{fig:paren_blocks}
\end{figure}

Assumiamo che $w$ sia la parola da rappresentare e assumiamo che $|w| = n$.
Per rappresentare la parola memorizziamo, oltre a $w$ stessa, un vettore $\mathbf{p}$ di $n$ bit
con $1$ nelle posizioni delle pioniere; il vettore $\mathbf{E}$, che per ogni blocco $i$
da l'eccesso all'inizio del blocco e ha un elemento per ogni blocco; il vettore $\mathbf{M}$, che per ogni
blocco $i$ mantiene la posizione della parentesi corrispondente all'$i$-esima pioniera e, infine, il vettore
$\mathbf{O}$ che per ogni blocco $i$ mantiene la posizione della prima aperta a sinistra dell'inizio del blocco
avente eccesso $x-1$, dove $x$ è il minimo eccesso del blocco. Per la rappresentazione, definiamo $l = \log_2(n)$.
\begin{theorem}
	Se ci sono $k$ blocchi, vi sono al massimo $2^k - 3$ coppie di pionieri.
\end{theorem}
\begin{proof}
	Costruiamo un grafo $G = (V, E)$ dove $V$ sono i blocchi ed esiste un lato tra un blocco $x$ e $y$ se
	e solo se $x$ contiene una pioniera cha ha in $y$ la sua corrispondente.
	Dimostriamo per induzione su $k$: vi sono due casi.
	\begin{itemize}
		\item se l'insieme di blocchi è separabile, ossia esiste una posizione nella parola sulla quale
		      ``non passano archi'', la parola è debolmente bilanciate ed è scomponibile in due diverse
		      parole ben formate. Allora, per ipotesi induttiva, il numero di pioniere nella prima
		      parte è al più $2p - 3$ e il numero di pioniere nella seconda parte è al più $2(k - p + 1) - 3$;
		      allora il numero di pioniere è al più
		      $$
			      2p - 3 + 2 k - 2p + 2 - 3 = 2k - 4 \leq 2k - 3
		      $$
		\item se l'insieme di blocchi non è separabile, ossia la parola è fortemente
		      bilanciate e non è scomponibile in due parole diverse, si prende la
		      prima coppia di parentesi lontane che si trova in $w$ e si rimuove. La parola risultante,
		      che chiamiamo $w'$, gode della proprietà che si vuole dimostrare. Il numero
		      di pioniere in $w$ è quindi al più la somma delle pioniere in $w'$ e la pioniera rimossa
		      $$
			      2k - 4  = 2k - 3
		      $$

	\end{itemize}
\end{proof}

\begin{figure}[h]
	\centering
	\begin{subfigure}{0.45\textwidth}
		\centering
		\tikzset{every picture/.style={line width=0.75pt}} %set default line width to 0.75pt        
		\begin{tikzpicture}[x=0.75pt,y=0.75pt,yscale=-1,xscale=1]
			%uncomment if require: \path (0,300); %set diagram left start at 0, and has height of 300

			%Shape: Rectangle [id:dp5751448069992374] 
			\draw   (100,130) -- (170,130) -- (170,150) -- (100,150) -- cycle ;
			%Shape: Rectangle [id:dp8956649728081978] 
			\draw   (190,130) -- (230,130) -- (230,150) -- (190,150) -- cycle ;
			%Shape: Rectangle [id:dp8836798951210239] 
			\draw   (250,130) -- (290,130) -- (290,150) -- (250,150) -- cycle ;
			%Shape: Rectangle [id:dp07169651712981018] 
			\draw   (300,130) -- (380,130) -- (380,150) -- (300,150) -- cycle ;
			%Straight Lines [id:da8396678932783305] 
			\draw    (110,130) -- (110,100) -- (260,100) -- (260,128) ;
			\draw [shift={(260,130)}, rotate = 270] [color={rgb, 255:red, 0; green, 0; blue, 0 }  ][line width=0.75]    (10.93,-3.29) .. controls (6.95,-1.4) and (3.31,-0.3) .. (0,0) .. controls (3.31,0.3) and (6.95,1.4) .. (10.93,3.29)   ;
			%Straight Lines [id:da8376287767119306] 
			\draw    (120,130) -- (120,110) -- (220,110) -- (220,128) ;
			\draw [shift={(220,130)}, rotate = 270] [color={rgb, 255:red, 0; green, 0; blue, 0 }  ][line width=0.75]    (10.93,-3.29) .. controls (6.95,-1.4) and (3.31,-0.3) .. (0,0) .. controls (3.31,0.3) and (6.95,1.4) .. (10.93,3.29)   ;
			%Straight Lines [id:da7343053938900628] 
			\draw    (200,150) -- (200,170) -- (260,170) -- (260,152) ;
			\draw [shift={(260,150)}, rotate = 90] [color={rgb, 255:red, 0; green, 0; blue, 0 }  ][line width=0.75]    (10.93,-3.29) .. controls (6.95,-1.4) and (3.31,-0.3) .. (0,0) .. controls (3.31,0.3) and (6.95,1.4) .. (10.93,3.29)   ;
			%Straight Lines [id:da02843704523195245] 
			\draw    (280,130) -- (280,110) -- (330,110) -- (330,128) ;
			\draw [shift={(330,130)}, rotate = 270] [color={rgb, 255:red, 0; green, 0; blue, 0 }  ][line width=0.75]    (10.93,-3.29) .. controls (6.95,-1.4) and (3.31,-0.3) .. (0,0) .. controls (3.31,0.3) and (6.95,1.4) .. (10.93,3.29)   ;
			%Straight Lines [id:da4532725638041315] 
			\draw  [dash pattern={on 3.75pt off 3pt on 7.5pt off 1.5pt}]  (270,90) -- (270,180) ;
			%Straight Lines [id:da17944476616348748] 
			\draw    (100,180) -- (100,190) -- (290,190) -- (290,180) ;
			%Straight Lines [id:da1753890360224526] 
			\draw    (250,200) -- (250,210) -- (380,210) -- (380,200) ;

			% Text Node
			\draw (181,192.4) node [anchor=north west][inner sep=0.75pt]    {$p$};
			% Text Node
			\draw (281,212.4) node [anchor=north west][inner sep=0.75pt]    {$k\ -\ p\ +\ 1$};
		\end{tikzpicture}
		\subcaption{Caso $1$.}
	\end{subfigure}
	\hfill
	\begin{subfigure}{0.45\textwidth}

		\tikzset{
			pattern size/.store in=\mcSize,
			pattern size = 5pt,
			pattern thickness/.store in=\mcThickness,
			pattern thickness = 0.3pt,
			pattern radius/.store in=\mcRadius,
			pattern radius = 1pt}
		\makeatletter
		\pgfutil@ifundefined{pgf@pattern@name@_y43ms7xnt}{
			\pgfdeclarepatternformonly[\mcThickness,\mcSize]{_y43ms7xnt}
			{\pgfqpoint{0pt}{-\mcThickness}}
			{\pgfpoint{\mcSize}{\mcSize}}
			{\pgfpoint{\mcSize}{\mcSize}}
			{
				\pgfsetcolor{\tikz@pattern@color}
				\pgfsetlinewidth{\mcThickness}
				\pgfpathmoveto{\pgfqpoint{0pt}{\mcSize}}
				\pgfpathlineto{\pgfpoint{\mcSize+\mcThickness}{-\mcThickness}}
				\pgfusepath{stroke}
			}}
		\makeatother

		% Pattern Info

		\tikzset{
			pattern size/.store in=\mcSize,
			pattern size = 5pt,
			pattern thickness/.store in=\mcThickness,
			pattern thickness = 0.3pt,
			pattern radius/.store in=\mcRadius,
			pattern radius = 1pt}
		\makeatletter
		\pgfutil@ifundefined{pgf@pattern@name@_mifsyi5bu}{
			\pgfdeclarepatternformonly[\mcThickness,\mcSize]{_mifsyi5bu}
			{\pgfqpoint{0pt}{0pt}}
			{\pgfpoint{\mcSize+\mcThickness}{\mcSize+\mcThickness}}
			{\pgfpoint{\mcSize}{\mcSize}}
			{
				\pgfsetcolor{\tikz@pattern@color}
				\pgfsetlinewidth{\mcThickness}
				\pgfpathmoveto{\pgfqpoint{0pt}{0pt}}
				\pgfpathlineto{\pgfpoint{\mcSize+\mcThickness}{\mcSize+\mcThickness}}
				\pgfusepath{stroke}
			}}
		\makeatother
		\tikzset{every picture/.style={line width=0.75pt}} %set default line width to 0.75pt        

		\begin{tikzpicture}[x=0.75pt,y=0.75pt,yscale=-1,xscale=1]
			%uncomment if require: \path (0,300); %set diagram left start at 0, and has height of 300

			%Shape: Rectangle [id:dp5339690558371227] 
			\draw   (120,150) -- (190,150) -- (190,170) -- (120,170) -- cycle ;
			%Straight Lines [id:da850734740747954] 
			\draw    (140,150) -- (140,120) -- (280,120) -- (280,148) ;
			\draw [shift={(280,150)}, rotate = 270] [color={rgb, 255:red, 0; green, 0; blue, 0 }  ][line width=0.75]    (10.93,-3.29) .. controls (6.95,-1.4) and (3.31,-0.3) .. (0,0) .. controls (3.31,0.3) and (6.95,1.4) .. (10.93,3.29)   ;
			%Straight Lines [id:da9735874841202442] 
			\draw    (150,150) -- (150,130) -- (240,130) -- (240,148) ;
			\draw [shift={(240,150)}, rotate = 270] [color={rgb, 255:red, 0; green, 0; blue, 0 }  ][line width=0.75]    (10.93,-3.29) .. controls (6.95,-1.4) and (3.31,-0.3) .. (0,0) .. controls (3.31,0.3) and (6.95,1.4) .. (10.93,3.29)   ;
			%Straight Lines [id:da863983070784418] 
			\draw    (220,170) -- (220,190) -- (280,190) -- (280,172) ;
			\draw [shift={(280,170)}, rotate = 90] [color={rgb, 255:red, 0; green, 0; blue, 0 }  ][line width=0.75]    (10.93,-3.29) .. controls (6.95,-1.4) and (3.31,-0.3) .. (0,0) .. controls (3.31,0.3) and (6.95,1.4) .. (10.93,3.29)   ;
			%Straight Lines [id:da6442133803721162] 
			\draw    (300,150) -- (300,130) -- (350,130) -- (350,148) ;
			\draw [shift={(350,150)}, rotate = 270] [color={rgb, 255:red, 0; green, 0; blue, 0 }  ][line width=0.75]    (10.93,-3.29) .. controls (6.95,-1.4) and (3.31,-0.3) .. (0,0) .. controls (3.31,0.3) and (6.95,1.4) .. (10.93,3.29)   ;
			%Shape: Rectangle [id:dp881091443202814] 
			\draw   (190,150) -- (260,150) -- (260,170) -- (190,170) -- cycle ;
			%Shape: Rectangle [id:dp2693088037465775] 
			\draw   (260,150) -- (330,150) -- (330,170) -- (260,170) -- cycle ;
			%Shape: Rectangle [id:dp6243374270107382] 
			\draw   (330,150) -- (400,150) -- (400,170) -- (330,170) -- cycle ;
			%Straight Lines [id:da11664171795224454] 
			\draw    (150,170) -- (150,190) -- (210,190) -- (210,172) ;
			\draw [shift={(210,170)}, rotate = 90] [color={rgb, 255:red, 0; green, 0; blue, 0 }  ][line width=0.75]    (10.93,-3.29) .. controls (6.95,-1.4) and (3.31,-0.3) .. (0,0) .. controls (3.31,0.3) and (6.95,1.4) .. (10.93,3.29)   ;
			%Straight Lines [id:da43843554948583285] 
			\draw    (130,150) -- (130,140.71) -- (130,110) -- (390,110) -- (390,148) ;
			\draw [shift={(390,150)}, rotate = 270] [color={rgb, 255:red, 0; green, 0; blue, 0 }  ][line width=0.75]    (10.93,-3.29) .. controls (6.95,-1.4) and (3.31,-0.3) .. (0,0) .. controls (3.31,0.3) and (6.95,1.4) .. (10.93,3.29)   ;
			%Shape: Rectangle [id:dp45540530016238834] 
			\draw  [pattern=_y43ms7xnt,pattern size=2.0999999999999996pt,pattern thickness=0.75pt,pattern radius=0pt, pattern color={rgb, 255:red, 0; green, 0; blue, 0}][dash pattern={on 0.84pt off 2.51pt}] (130,150) -- (120,150) -- (120,170) -- (130,170) -- cycle ;
			%Shape: Rectangle [id:dp21031590903119124] 
			\draw  [pattern=_mifsyi5bu,pattern size=2.0999999999999996pt,pattern thickness=0.75pt,pattern radius=0pt, pattern color={rgb, 255:red, 0; green, 0; blue, 0}][dash pattern={on 0.84pt off 2.51pt}] (400,150) -- (390,150) -- (390,170) -- (400,170) -- cycle ;
		\end{tikzpicture}

		\subcaption{Caso $2$.}
	\end{subfigure}
	\caption{Dimostrazione induttiva del numero di pioniere in una parola ben parentesizzata.}
	\label{fig:proof_pioneers}
\end{figure}

Quindi, lo spazio occupato da queste strutture è:
$$
	\begin{aligned}
		 & w          &  & n                                                                                          \\
		 & p          &  & n + o(n) \text{ (sarà necessario introdurre la struttura di rango)}                        \\
		 & \mathbf{E} &  & k  \log_2(n)                                                                               \\
		 & \mathbf{O} &  & k  \log_2(n)                                                                               \\
		 & \mathbf{M} &  & \text{pioniere} \cdot \log_2(n) < 2(2k - 3) \log_2(n) = (4k - 6) \log_2(n) \le 4k\log_2(n)
	\end{aligned}
$$
Sommando, lo spazio occupato è $D_n = 2n + o(n) + 6k \log_2(n) = 2n + 6n + o(n) = 8n + o(n)$ bit.

\subsubsection{Implementrazione}
Per implementare la funzione $\mathbf{find\_close}$ che, trovata una parentesi aperta in posizione $p$, restituisce
la posizione in cui si chiude, si deve:
\begin{enumerate}
	\item calcolare gli eccessi di tutte le posizioni del blocco al quale appartiene $p$ con $E$ (tempo logaritmico);
	\item se $p$ è vicina, ossia esiste un'altra posizione nel blocco con lo stesso eccesso,
	      la funzione è completa. Altrimenti, $p$ è la posizione di una parentesi lontana e
	      $j = \mathbf{rank_p}(p)$ è l'indice della pioniera che precede $p$; si può usare $M[j] = p'$ per
	      calcolare la posizione in cui si chiude la pioniera che precede $p$, che diciamo essere in un
	      blocco $b$. Scorrendo indietro, per trovare la chiusa corrispondente, basta trovare la
	      posizione con eccesso uguale all'eccesso di $p$.
\end{enumerate}

Tutto questo richiede tempo $\log_2(n)$ e analogamente per l'implementazione di $\mathbf{find\_open}$.
Per l'implementazione di $\mathbf{enclosed}$, che cerca la prima parentesi aperta che racchiude la parentesi
in posizione $p$, ipotizziamo di aver già trovato la corrispondente $\bar{p}$ con una find e ammettiamo che
l'eccesso di entrambe sia $e$. Inizialmente, si cerca alla sinistra della posizione della parentesi aperta
se c'è una posizione con eccesso $e-1$, che segna l'aperta precedente. Se, nel blocco al quale appartiene
l'aperta, tutte le posizioni hanno eccesso $\ge e-1$, si cerca per una chiusa di eccesso $e-1$ nella parte
a destra del blocco in cui giace la chiusa. Se anche in questo caso non si trova, significa che la parentesi
che si cerca è di eccesso $e-1$ che giace nel blocco a sinistra a quello della parentesi aperta, che ha
come valore di eccesso minore di tale blocco. In tal caso, basta usare il vettore $\mathbf{O}$ per trovare
la posizione della parentesi cercata.

\subsection{Lower bound per parentesi ben bilanciate}
\subsubsection{Foreste ordinate}
Una foresta ordinata è una sequenza ordinata (per numero di nodi) di alberi ordinati (corrispondenti
ad alberi radicati).
\begin{figure}[h]
	\centering
	\begin{tikzpicture}
		\draw (0,1)	node[fill,circle, inner sep=1pt] {} -- (0,0) node[fill,circle,inner sep=1pt] {};

		\draw (2,1)	node[fill,circle, inner sep=1pt] {} -- (2,0) node[fill,circle,inner sep=1pt] {};
		\draw (2,1)	node[fill,circle, inner sep=1pt] {} -- (1,0) node[fill,circle,inner sep=1pt] {};
		\draw (2,1)	node[fill,circle, inner sep=1pt] {} -- (3,0) node[fill,circle,inner sep=1pt] {};

		\draw (5,1)	node[fill,circle, inner sep=1pt] {} -- (4,0) node[fill,circle,inner sep=1pt] {};
		\draw (4,0)	node[fill,circle, inner sep=1pt] {} -- (4,-1) node[fill,circle,inner sep=1pt] {};
		\draw (5,1)	node[fill,circle, inner sep=1pt] {} -- (5,0) node[fill,circle,inner sep=1pt] {};
		\draw (5,1)	node[fill,circle, inner sep=1pt] {} -- (6,0) node[fill,circle,inner sep=1pt] {};
		\draw (6,0)	node[fill,circle, inner sep=1pt] {} -- (6,-1) node[fill,circle,inner sep=1pt] {};
	\end{tikzpicture}
	\caption{Foresta ordinata di alberi.}
	\label{fig:ordered_forest}
\end{figure}
Formalmente, definiamo induttivamente una foresta ordinata:
\begin{itemize}
	\item $\langle \rangle$ indica la lista vuota di alberi ed è una foresta ordinata;
	\item se $\langle T_1, \cdots, T_k\rangle$ sono alberi, allora $\langle T_1, \cdots, T_k\rangle$ è una
	      foresta ordinata
	\item se $F$ è una foresta ordinata, $tree(F)$ è un albero radicato in un
	      nuovo nodo che ha come figli tutti gli alberi di $F$.
\end{itemize}

\subsubsection{Isomorfismo tra foreste ordinate e alberi binari}
Definiamo una funzione $\phi$ che associa ad ogni foresta ordinata un preciso albero binario.
Induttivamente,
$$
	\phi(\langle \rangle) = \cdot
$$
ossia $\phi$ di una foresta vuota è un albero costituito da una singola radice.
$$
	\phi(\langle tree(F), T_1, \cdots, T_k>\rangle) = (\phi(F), \phi(\langle T_1, \cdots, T_k\rangle))
$$

\begin{figure}[h]
	\centering
	\tikzset{every picture/.style={line width=0.75pt}} %set default line width to 0.75pt        

	\begin{tikzpicture}[x=0.75pt,y=0.75pt,yscale=-1,xscale=1]
		%uncomment if require: \path (0,300); %set diagram left start at 0, and has height of 300

		%Straight Lines [id:da1467409507063523] 
		\draw    (144.93,41.07) -- (144.93,71.07) ;
		\draw [shift={(144.93,71.07)}, rotate = 90] [color={rgb, 255:red, 0; green, 0; blue, 0 }  ][fill={rgb, 255:red, 0; green, 0; blue, 0 }  ][line width=0.75]      (0, 0) circle [x radius= 3.35, y radius= 3.35]   ;
		\draw [shift={(144.93,41.07)}, rotate = 90] [color={rgb, 255:red, 0; green, 0; blue, 0 }  ][fill={rgb, 255:red, 0; green, 0; blue, 0 }  ][line width=0.75]      (0, 0) circle [x radius= 3.35, y radius= 3.35]   ;
		%Shape: Circle [id:dp48951213847038366] 
		\draw  [fill={rgb, 255:red, 0; green, 0; blue, 0 }  ,fill opacity=1 ] (154.57,54.79) .. controls (154.57,52.93) and (156.07,51.43) .. (157.93,51.43) .. controls (159.78,51.43) and (161.29,52.93) .. (161.29,54.79) .. controls (161.29,56.64) and (159.78,58.14) .. (157.93,58.14) .. controls (156.07,58.14) and (154.57,56.64) .. (154.57,54.79) -- cycle ;

		%Straight Lines [id:da5368285010745462] 
		\draw    (226.43,55) -- (208.31,76.91) -- (201.86,84.71) ;
		\draw [shift={(226.43,55)}, rotate = 129.59] [color={rgb, 255:red, 0; green, 0; blue, 0 }  ][fill={rgb, 255:red, 0; green, 0; blue, 0 }  ][line width=0.75]      (0, 0) circle [x radius= 3.35, y radius= 3.35]   ;
		%Shape: Triangle [id:dp5019688866391238] 
		\draw   (201.86,84.71) -- (236.86,124.71) -- (166.86,124.71) -- cycle ;
		%Shape: Triangle [id:dp12657672899601724] 
		\draw   (273.64,93.29) -- (318.57,145) -- (228.71,145) -- cycle ;
		%Straight Lines [id:da934069022901869] 
		\draw    (226.43,55) -- (273.64,93.29) ;
		%Straight Lines [id:da2898443038516424] 
		\draw    (277.71,123) -- (277.86,139.14) ;
		\draw [shift={(277.86,139.14)}, rotate = 89.49] [color={rgb, 255:red, 0; green, 0; blue, 0 }  ][fill={rgb, 255:red, 0; green, 0; blue, 0 }  ][line width=0.75]      (0, 0) circle [x radius= 3.35, y radius= 3.35]   ;
		\draw [shift={(277.71,123)}, rotate = 89.49] [color={rgb, 255:red, 0; green, 0; blue, 0 }  ][fill={rgb, 255:red, 0; green, 0; blue, 0 }  ][line width=0.75]      (0, 0) circle [x radius= 3.35, y radius= 3.35]   ;
		%Straight Lines [id:da8113066852584067] 
		\draw    (354.98,55.5) -- (336.86,77.41) -- (330.4,85.21) ;
		\draw [shift={(354.98,55.5)}, rotate = 129.59] [color={rgb, 255:red, 0; green, 0; blue, 0 }  ][fill={rgb, 255:red, 0; green, 0; blue, 0 }  ][line width=0.75]      (0, 0) circle [x radius= 3.35, y radius= 3.35]   ;
		%Straight Lines [id:da7241289121021388] 
		\draw    (354.98,55.5) -- (402.19,93.79) ;
		%Shape: Circle [id:dp6494084453696087] 
		\draw  [fill={rgb, 255:red, 0; green, 0; blue, 0 }  ,fill opacity=1 ] (325.1,87.83) .. controls (325.1,86.18) and (326.44,84.83) .. (328.1,84.83) .. controls (329.76,84.83) and (331.1,86.18) .. (331.1,87.83) .. controls (331.1,89.49) and (329.76,90.83) .. (328.1,90.83) .. controls (326.44,90.83) and (325.1,89.49) .. (325.1,87.83) -- cycle ;
		%Straight Lines [id:da4037066529287967] 
		\draw    (403.26,95.5) -- (385.15,117.41) -- (377.98,125.79) ;
		\draw [shift={(403.26,95.5)}, rotate = 129.59] [color={rgb, 255:red, 0; green, 0; blue, 0 }  ][fill={rgb, 255:red, 0; green, 0; blue, 0 }  ][line width=0.75]      (0, 0) circle [x radius= 3.35, y radius= 3.35]   ;
		%Shape: Triangle [id:dp04881880314765141] 
		\draw   (445.69,115.21) -- (480.69,155.21) -- (410.69,155.21) -- cycle ;

		%Shape: Triangle [id:dp12559883893437063] 
		\draw   (377.98,125.79) -- (422.9,177.5) -- (333.05,177.5) -- cycle ;
		%Straight Lines [id:da48087885685526743] 
		\draw    (379.19,155.5) -- (379.33,171.64) ;
		\draw [shift={(379.33,171.64)}, rotate = 89.49] [color={rgb, 255:red, 0; green, 0; blue, 0 }  ][fill={rgb, 255:red, 0; green, 0; blue, 0 }  ][line width=0.75]      (0, 0) circle [x radius= 3.35, y radius= 3.35]   ;
		\draw [shift={(379.19,155.5)}, rotate = 89.49] [color={rgb, 255:red, 0; green, 0; blue, 0 }  ][fill={rgb, 255:red, 0; green, 0; blue, 0 }  ][line width=0.75]      (0, 0) circle [x radius= 3.35, y radius= 3.35]   ;
		%Straight Lines [id:da02388870469608051] 
		\draw    (403.26,95.5) -- (445.69,115.21) ;
		%Straight Lines [id:da6273665141516217] 
		\draw    (536.48,54.83) -- (518.36,76.74) -- (511.9,84.55) ;
		\draw [shift={(536.48,54.83)}, rotate = 129.59] [color={rgb, 255:red, 0; green, 0; blue, 0 }  ][fill={rgb, 255:red, 0; green, 0; blue, 0 }  ][line width=0.75]      (0, 0) circle [x radius= 3.35, y radius= 3.35]   ;
		%Straight Lines [id:da5905284436508569] 
		\draw    (536.48,54.83) -- (583.69,93.12) ;
		%Straight Lines [id:da2030740854922638] 
		\draw    (584.76,94.83) -- (566.65,116.74) -- (559.48,125.12) ;
		\draw [shift={(584.76,94.83)}, rotate = 129.59] [color={rgb, 255:red, 0; green, 0; blue, 0 }  ][fill={rgb, 255:red, 0; green, 0; blue, 0 }  ][line width=0.75]      (0, 0) circle [x radius= 3.35, y radius= 3.35]   ;
		%Straight Lines [id:da49066689108397754] 
		\draw    (584.76,94.83) -- (627.19,114.55) ;
		%Straight Lines [id:da38720603069737713] 
		\draw    (557.87,127.72) -- (539.76,149.63) -- (532.59,158.01) ;
		\draw [shift={(557.87,127.72)}, rotate = 129.59] [color={rgb, 255:red, 0; green, 0; blue, 0 }  ][fill={rgb, 255:red, 0; green, 0; blue, 0 }  ][line width=0.75]      (0, 0) circle [x radius= 3.35, y radius= 3.35]   ;
		%Straight Lines [id:da1048947282900694] 
		\draw    (560.76,129.5) -- (603.19,149.21) ;
		%Shape: Circle [id:dp32417892244375524] 
		\draw  [fill={rgb, 255:red, 0; green, 0; blue, 0 }  ,fill opacity=1 ] (533.82,160.09) .. controls (533.82,158.2) and (532.3,156.67) .. (530.41,156.67) .. controls (528.53,156.67) and (527,158.2) .. (527,160.09) .. controls (527,161.97) and (528.53,163.5) .. (530.41,163.5) .. controls (532.3,163.5) and (533.82,161.97) .. (533.82,160.09) -- cycle ;
		%Shape: Circle [id:dp9675952466326133] 
		\draw  [fill={rgb, 255:red, 0; green, 0; blue, 0 }  ,fill opacity=1 ] (513.6,86.53) .. controls (513.6,84.65) and (512.07,83.12) .. (510.19,83.12) .. controls (508.3,83.12) and (506.78,84.65) .. (506.78,86.53) .. controls (506.78,88.42) and (508.3,89.94) .. (510.19,89.94) .. controls (512.07,89.94) and (513.6,88.42) .. (513.6,86.53) -- cycle ;

		%Shape: Circle [id:dp5658104800173653] 
		\draw  [fill={rgb, 255:red, 0; green, 0; blue, 0 }  ,fill opacity=1 ] (609.38,151.2) .. controls (609.38,149.31) and (607.85,147.79) .. (605.97,147.79) .. controls (604.08,147.79) and (602.55,149.31) .. (602.55,151.2) .. controls (602.55,153.08) and (604.08,154.61) .. (605.97,154.61) .. controls (607.85,154.61) and (609.38,153.08) .. (609.38,151.2) -- cycle ;
		%Shape: Circle [id:dp12193907009892957] 
		\draw  [fill={rgb, 255:red, 0; green, 0; blue, 0 }  ,fill opacity=1 ] (634.27,115.64) .. controls (634.27,113.76) and (632.74,112.23) .. (630.86,112.23) .. controls (628.97,112.23) and (627.44,113.76) .. (627.44,115.64) .. controls (627.44,117.53) and (628.97,119.06) .. (630.86,119.06) .. controls (632.74,119.06) and (634.27,117.53) .. (634.27,115.64) -- cycle ;


		% Text Node
		\draw (125.12,44.91) node [anchor=north west][inner sep=0.75pt]    {$\phi ( \ \ \ \ \ \ \ ) \ =\ \ \ \ \ \ \ \ \ \ \ \ \ \ \ \ \ \ \ \ =\ \ \ \ \ \ \ \ \ \ \ \ \ \ \ \ \ \ \ \ \ \ \ \ \ \ \ \ \ \ \ \ \ \ \ \ \ \ \ =\ $};
		% Text Node
		\draw (427.26,134.33) node [anchor=north west][inner sep=0.75pt]    {$\phi ( \langle \rangle )$};
		% Text Node
		\draw (183.43,103.83) node [anchor=north west][inner sep=0.75pt]    {$\phi ( \langle \rangle )$};
		% Text Node
		\draw (257.75,120.69) node [anchor=north west][inner sep=0.75pt]    {$\phi ( \ \ )$};
		% Text Node
		\draw (360.33,153.19) node [anchor=north west][inner sep=0.75pt]    {$\phi ( \ \ )$};


	\end{tikzpicture}
	\caption{Esempio di calcolo isomorfismo tra foreste e alberi binari.}
	\label{fig:iso_forest_bintree_example}
\end{figure}



\begin{figure}[h]
	\centering



	\tikzset{every picture/.style={line width=0.75pt}} %set default line width to 0.75pt        

	\begin{tikzpicture}[x=0.75pt,y=0.75pt,yscale=-1,xscale=1]
		%uncomment if require: \path (0,300); %set diagram left start at 0, and has height of 300

		%Straight Lines [id:da7390170116429094] 
		\draw    (220,90) -- (330,120) ;
		\draw [shift={(330,120)}, rotate = 15.26] [color={rgb, 255:red, 0; green, 0; blue, 0 }  ][fill={rgb, 255:red, 0; green, 0; blue, 0 }  ][line width=0.75]      (0, 0) circle [x radius= 3.35, y radius= 3.35]   ;
		\draw [shift={(220,90)}, rotate = 15.26] [color={rgb, 255:red, 0; green, 0; blue, 0 }  ][fill={rgb, 255:red, 0; green, 0; blue, 0 }  ][line width=0.75]      (0, 0) circle [x radius= 3.35, y radius= 3.35]   ;
		%Straight Lines [id:da039244879101243746] 
		\draw    (220,90) -- (280,120) ;
		\draw [shift={(280,120)}, rotate = 26.57] [color={rgb, 255:red, 0; green, 0; blue, 0 }  ][fill={rgb, 255:red, 0; green, 0; blue, 0 }  ][line width=0.75]      (0, 0) circle [x radius= 3.35, y radius= 3.35]   ;
		\draw [shift={(220,90)}, rotate = 26.57] [color={rgb, 255:red, 0; green, 0; blue, 0 }  ][fill={rgb, 255:red, 0; green, 0; blue, 0 }  ][line width=0.75]      (0, 0) circle [x radius= 3.35, y radius= 3.35]   ;
		%Straight Lines [id:da9807403722275164] 
		\draw    (220,90) -- (240,120) ;
		\draw [shift={(240,120)}, rotate = 56.31] [color={rgb, 255:red, 0; green, 0; blue, 0 }  ][fill={rgb, 255:red, 0; green, 0; blue, 0 }  ][line width=0.75]      (0, 0) circle [x radius= 3.35, y radius= 3.35]   ;
		\draw [shift={(220,90)}, rotate = 56.31] [color={rgb, 255:red, 0; green, 0; blue, 0 }  ][fill={rgb, 255:red, 0; green, 0; blue, 0 }  ][line width=0.75]      (0, 0) circle [x radius= 3.35, y radius= 3.35]   ;
		%Straight Lines [id:da03267007977703551] 
		\draw    (220,90) -- (210,120) ;
		\draw [shift={(210,120)}, rotate = 108.43] [color={rgb, 255:red, 0; green, 0; blue, 0 }  ][fill={rgb, 255:red, 0; green, 0; blue, 0 }  ][line width=0.75]      (0, 0) circle [x radius= 3.35, y radius= 3.35]   ;
		\draw [shift={(220,90)}, rotate = 108.43] [color={rgb, 255:red, 0; green, 0; blue, 0 }  ][fill={rgb, 255:red, 0; green, 0; blue, 0 }  ][line width=0.75]      (0, 0) circle [x radius= 3.35, y radius= 3.35]   ;
		%Straight Lines [id:da3117293837981189] 
		\draw    (220,90) -- (140,120) ;
		\draw [shift={(140,120)}, rotate = 159.44] [color={rgb, 255:red, 0; green, 0; blue, 0 }  ][fill={rgb, 255:red, 0; green, 0; blue, 0 }  ][line width=0.75]      (0, 0) circle [x radius= 3.35, y radius= 3.35]   ;
		\draw [shift={(220,90)}, rotate = 159.44] [color={rgb, 255:red, 0; green, 0; blue, 0 }  ][fill={rgb, 255:red, 0; green, 0; blue, 0 }  ][line width=0.75]      (0, 0) circle [x radius= 3.35, y radius= 3.35]   ;
		%Straight Lines [id:da5543598400532516] 
		\draw    (220,90) -- (180,120) ;
		\draw [shift={(180,120)}, rotate = 143.13] [color={rgb, 255:red, 0; green, 0; blue, 0 }  ][fill={rgb, 255:red, 0; green, 0; blue, 0 }  ][line width=0.75]      (0, 0) circle [x radius= 3.35, y radius= 3.35]   ;
		\draw [shift={(220,90)}, rotate = 143.13] [color={rgb, 255:red, 0; green, 0; blue, 0 }  ][fill={rgb, 255:red, 0; green, 0; blue, 0 }  ][line width=0.75]      (0, 0) circle [x radius= 3.35, y radius= 3.35]   ;
		%Straight Lines [id:da6616928412892461] 
		\draw    (140,120) -- (130,140) ;
		\draw [shift={(130,140)}, rotate = 116.57] [color={rgb, 255:red, 0; green, 0; blue, 0 }  ][fill={rgb, 255:red, 0; green, 0; blue, 0 }  ][line width=0.75]      (0, 0) circle [x radius= 3.35, y radius= 3.35]   ;
		\draw [shift={(140,120)}, rotate = 116.57] [color={rgb, 255:red, 0; green, 0; blue, 0 }  ][fill={rgb, 255:red, 0; green, 0; blue, 0 }  ][line width=0.75]      (0, 0) circle [x radius= 3.35, y radius= 3.35]   ;
		%Straight Lines [id:da5001878304038733] 
		\draw    (140,120) -- (110,140) ;
		\draw [shift={(110,140)}, rotate = 146.31] [color={rgb, 255:red, 0; green, 0; blue, 0 }  ][fill={rgb, 255:red, 0; green, 0; blue, 0 }  ][line width=0.75]      (0, 0) circle [x radius= 3.35, y radius= 3.35]   ;
		\draw [shift={(140,120)}, rotate = 146.31] [color={rgb, 255:red, 0; green, 0; blue, 0 }  ][fill={rgb, 255:red, 0; green, 0; blue, 0 }  ][line width=0.75]      (0, 0) circle [x radius= 3.35, y radius= 3.35]   ;
		%Straight Lines [id:da35297443192918876] 
		\draw    (140,120) -- (150,140) ;
		\draw [shift={(150,140)}, rotate = 63.43] [color={rgb, 255:red, 0; green, 0; blue, 0 }  ][fill={rgb, 255:red, 0; green, 0; blue, 0 }  ][line width=0.75]      (0, 0) circle [x radius= 3.35, y radius= 3.35]   ;
		\draw [shift={(140,120)}, rotate = 63.43] [color={rgb, 255:red, 0; green, 0; blue, 0 }  ][fill={rgb, 255:red, 0; green, 0; blue, 0 }  ][line width=0.75]      (0, 0) circle [x radius= 3.35, y radius= 3.35]   ;
		%Straight Lines [id:da30077845664351066] 
		\draw    (110,140) -- (100,160) ;
		\draw [shift={(100,160)}, rotate = 116.57] [color={rgb, 255:red, 0; green, 0; blue, 0 }  ][fill={rgb, 255:red, 0; green, 0; blue, 0 }  ][line width=0.75]      (0, 0) circle [x radius= 3.35, y radius= 3.35]   ;
		\draw [shift={(110,140)}, rotate = 116.57] [color={rgb, 255:red, 0; green, 0; blue, 0 }  ][fill={rgb, 255:red, 0; green, 0; blue, 0 }  ][line width=0.75]      (0, 0) circle [x radius= 3.35, y radius= 3.35]   ;
		%Straight Lines [id:da5436015840206943] 
		\draw    (110,140) -- (120,160) ;
		\draw [shift={(120,160)}, rotate = 63.43] [color={rgb, 255:red, 0; green, 0; blue, 0 }  ][fill={rgb, 255:red, 0; green, 0; blue, 0 }  ][line width=0.75]      (0, 0) circle [x radius= 3.35, y radius= 3.35]   ;
		\draw [shift={(110,140)}, rotate = 63.43] [color={rgb, 255:red, 0; green, 0; blue, 0 }  ][fill={rgb, 255:red, 0; green, 0; blue, 0 }  ][line width=0.75]      (0, 0) circle [x radius= 3.35, y radius= 3.35]   ;
		%Straight Lines [id:da3418649793206039] 
		\draw    (240,120) -- (230,140) ;
		\draw [shift={(230,140)}, rotate = 116.57] [color={rgb, 255:red, 0; green, 0; blue, 0 }  ][fill={rgb, 255:red, 0; green, 0; blue, 0 }  ][line width=0.75]      (0, 0) circle [x radius= 3.35, y radius= 3.35]   ;
		\draw [shift={(240,120)}, rotate = 116.57] [color={rgb, 255:red, 0; green, 0; blue, 0 }  ][fill={rgb, 255:red, 0; green, 0; blue, 0 }  ][line width=0.75]      (0, 0) circle [x radius= 3.35, y radius= 3.35]   ;
		%Straight Lines [id:da5537357380053466] 
		\draw    (240,120) -- (250,140) ;
		\draw [shift={(250,140)}, rotate = 63.43] [color={rgb, 255:red, 0; green, 0; blue, 0 }  ][fill={rgb, 255:red, 0; green, 0; blue, 0 }  ][line width=0.75]      (0, 0) circle [x radius= 3.35, y radius= 3.35]   ;
		\draw [shift={(240,120)}, rotate = 63.43] [color={rgb, 255:red, 0; green, 0; blue, 0 }  ][fill={rgb, 255:red, 0; green, 0; blue, 0 }  ][line width=0.75]      (0, 0) circle [x radius= 3.35, y radius= 3.35]   ;
		%Straight Lines [id:da596511628614987] 
		\draw    (330,120) -- (330,140) ;
		\draw [shift={(330,140)}, rotate = 90] [color={rgb, 255:red, 0; green, 0; blue, 0 }  ][fill={rgb, 255:red, 0; green, 0; blue, 0 }  ][line width=0.75]      (0, 0) circle [x radius= 3.35, y radius= 3.35]   ;
		\draw [shift={(330,120)}, rotate = 90] [color={rgb, 255:red, 0; green, 0; blue, 0 }  ][fill={rgb, 255:red, 0; green, 0; blue, 0 }  ][line width=0.75]      (0, 0) circle [x radius= 3.35, y radius= 3.35]   ;
		%Straight Lines [id:da9692313360351018] 
		\draw    (250,140) -- (250,160) ;
		\draw [shift={(250,160)}, rotate = 90] [color={rgb, 255:red, 0; green, 0; blue, 0 }  ][fill={rgb, 255:red, 0; green, 0; blue, 0 }  ][line width=0.75]      (0, 0) circle [x radius= 3.35, y radius= 3.35]   ;
		\draw [shift={(250,140)}, rotate = 90] [color={rgb, 255:red, 0; green, 0; blue, 0 }  ][fill={rgb, 255:red, 0; green, 0; blue, 0 }  ][line width=0.75]      (0, 0) circle [x radius= 3.35, y radius= 3.35]   ;
		%Straight Lines [id:da3464424144057545] 
		\draw    (230,140) -- (230,160) ;
		\draw [shift={(230,160)}, rotate = 90] [color={rgb, 255:red, 0; green, 0; blue, 0 }  ][fill={rgb, 255:red, 0; green, 0; blue, 0 }  ][line width=0.75]      (0, 0) circle [x radius= 3.35, y radius= 3.35]   ;
		\draw [shift={(230,140)}, rotate = 90] [color={rgb, 255:red, 0; green, 0; blue, 0 }  ][fill={rgb, 255:red, 0; green, 0; blue, 0 }  ][line width=0.75]      (0, 0) circle [x radius= 3.35, y radius= 3.35]   ;
		%Straight Lines [id:da9449001958581409] 
		\draw    (460,80) -- (510,100) ;
		\draw [shift={(510,100)}, rotate = 21.8] [color={rgb, 255:red, 0; green, 0; blue, 0 }  ][fill={rgb, 255:red, 0; green, 0; blue, 0 }  ][line width=0.75]      (0, 0) circle [x radius= 3.35, y radius= 3.35]   ;
		\draw [shift={(460,80)}, rotate = 21.8] [color={rgb, 255:red, 0; green, 0; blue, 0 }  ][fill={rgb, 255:red, 0; green, 0; blue, 0 }  ][line width=0.75]      (0, 0) circle [x radius= 3.35, y radius= 3.35]   ;
		%Straight Lines [id:da5101057369964663] 
		\draw    (460,80) -- (440,100) ;
		\draw [shift={(440,100)}, rotate = 135] [color={rgb, 255:red, 0; green, 0; blue, 0 }  ][fill={rgb, 255:red, 0; green, 0; blue, 0 }  ][line width=0.75]      (0, 0) circle [x radius= 3.35, y radius= 3.35]   ;
		\draw [shift={(460,80)}, rotate = 135] [color={rgb, 255:red, 0; green, 0; blue, 0 }  ][fill={rgb, 255:red, 0; green, 0; blue, 0 }  ][line width=0.75]      (0, 0) circle [x radius= 3.35, y radius= 3.35]   ;
		%Straight Lines [id:da7168760346038819] 
		\draw    (440,100) -- (480,120) ;
		\draw [shift={(480,120)}, rotate = 26.57] [color={rgb, 255:red, 0; green, 0; blue, 0 }  ][fill={rgb, 255:red, 0; green, 0; blue, 0 }  ][line width=0.75]      (0, 0) circle [x radius= 3.35, y radius= 3.35]   ;
		\draw [shift={(440,100)}, rotate = 26.57] [color={rgb, 255:red, 0; green, 0; blue, 0 }  ][fill={rgb, 255:red, 0; green, 0; blue, 0 }  ][line width=0.75]      (0, 0) circle [x radius= 3.35, y radius= 3.35]   ;
		%Straight Lines [id:da7536172846092781] 
		\draw    (480,120) -- (520,140) ;
		\draw [shift={(520,140)}, rotate = 26.57] [color={rgb, 255:red, 0; green, 0; blue, 0 }  ][fill={rgb, 255:red, 0; green, 0; blue, 0 }  ][line width=0.75]      (0, 0) circle [x radius= 3.35, y radius= 3.35]   ;
		\draw [shift={(480,120)}, rotate = 26.57] [color={rgb, 255:red, 0; green, 0; blue, 0 }  ][fill={rgb, 255:red, 0; green, 0; blue, 0 }  ][line width=0.75]      (0, 0) circle [x radius= 3.35, y radius= 3.35]   ;
		%Straight Lines [id:da06546573584424464] 
		\draw    (520,140) -- (560,160) ;
		\draw [shift={(560,160)}, rotate = 26.57] [color={rgb, 255:red, 0; green, 0; blue, 0 }  ][fill={rgb, 255:red, 0; green, 0; blue, 0 }  ][line width=0.75]      (0, 0) circle [x radius= 3.35, y radius= 3.35]   ;
		\draw [shift={(520,140)}, rotate = 26.57] [color={rgb, 255:red, 0; green, 0; blue, 0 }  ][fill={rgb, 255:red, 0; green, 0; blue, 0 }  ][line width=0.75]      (0, 0) circle [x radius= 3.35, y radius= 3.35]   ;
		%Straight Lines [id:da31121333233660786] 
		\draw    (560,160) -- (600,180) ;
		\draw [shift={(600,180)}, rotate = 26.57] [color={rgb, 255:red, 0; green, 0; blue, 0 }  ][fill={rgb, 255:red, 0; green, 0; blue, 0 }  ][line width=0.75]      (0, 0) circle [x radius= 3.35, y radius= 3.35]   ;
		\draw [shift={(560,160)}, rotate = 26.57] [color={rgb, 255:red, 0; green, 0; blue, 0 }  ][fill={rgb, 255:red, 0; green, 0; blue, 0 }  ][line width=0.75]      (0, 0) circle [x radius= 3.35, y radius= 3.35]   ;
		%Straight Lines [id:da9734645903114482] 
		\draw    (600,180) -- (640,200) ;
		\draw [shift={(640,200)}, rotate = 26.57] [color={rgb, 255:red, 0; green, 0; blue, 0 }  ][fill={rgb, 255:red, 0; green, 0; blue, 0 }  ][line width=0.75]      (0, 0) circle [x radius= 3.35, y radius= 3.35]   ;
		\draw [shift={(600,180)}, rotate = 26.57] [color={rgb, 255:red, 0; green, 0; blue, 0 }  ][fill={rgb, 255:red, 0; green, 0; blue, 0 }  ][line width=0.75]      (0, 0) circle [x radius= 3.35, y radius= 3.35]   ;
		%Straight Lines [id:da3400373607542443] 
		\draw    (480,120) -- (470,140) ;
		\draw [shift={(470,140)}, rotate = 116.57] [color={rgb, 255:red, 0; green, 0; blue, 0 }  ][fill={rgb, 255:red, 0; green, 0; blue, 0 }  ][line width=0.75]      (0, 0) circle [x radius= 3.35, y radius= 3.35]   ;
		\draw [shift={(480,120)}, rotate = 116.57] [color={rgb, 255:red, 0; green, 0; blue, 0 }  ][fill={rgb, 255:red, 0; green, 0; blue, 0 }  ][line width=0.75]      (0, 0) circle [x radius= 3.35, y radius= 3.35]   ;
		%Straight Lines [id:da8071946852447195] 
		\draw    (520,140) -- (510,160) ;
		\draw [shift={(510,160)}, rotate = 116.57] [color={rgb, 255:red, 0; green, 0; blue, 0 }  ][fill={rgb, 255:red, 0; green, 0; blue, 0 }  ][line width=0.75]      (0, 0) circle [x radius= 3.35, y radius= 3.35]   ;
		\draw [shift={(520,140)}, rotate = 116.57] [color={rgb, 255:red, 0; green, 0; blue, 0 }  ][fill={rgb, 255:red, 0; green, 0; blue, 0 }  ][line width=0.75]      (0, 0) circle [x radius= 3.35, y radius= 3.35]   ;
		%Straight Lines [id:da9920119862914486] 
		\draw    (600,180) -- (590,200) ;
		\draw [shift={(590,200)}, rotate = 116.57] [color={rgb, 255:red, 0; green, 0; blue, 0 }  ][fill={rgb, 255:red, 0; green, 0; blue, 0 }  ][line width=0.75]      (0, 0) circle [x radius= 3.35, y radius= 3.35]   ;
		\draw [shift={(600,180)}, rotate = 116.57] [color={rgb, 255:red, 0; green, 0; blue, 0 }  ][fill={rgb, 255:red, 0; green, 0; blue, 0 }  ][line width=0.75]      (0, 0) circle [x radius= 3.35, y radius= 3.35]   ;
		%Straight Lines [id:da6764843933308277] 
		\draw    (560,160) -- (550,180) ;
		\draw [shift={(550,180)}, rotate = 116.57] [color={rgb, 255:red, 0; green, 0; blue, 0 }  ][fill={rgb, 255:red, 0; green, 0; blue, 0 }  ][line width=0.75]      (0, 0) circle [x radius= 3.35, y radius= 3.35]   ;
		\draw [shift={(560,160)}, rotate = 116.57] [color={rgb, 255:red, 0; green, 0; blue, 0 }  ][fill={rgb, 255:red, 0; green, 0; blue, 0 }  ][line width=0.75]      (0, 0) circle [x radius= 3.35, y radius= 3.35]   ;
		%Straight Lines [id:da9822121216527463] 
		\draw    (550,180) -- (530,200) ;
		\draw [shift={(530,200)}, rotate = 135] [color={rgb, 255:red, 0; green, 0; blue, 0 }  ][fill={rgb, 255:red, 0; green, 0; blue, 0 }  ][line width=0.75]      (0, 0) circle [x radius= 3.35, y radius= 3.35]   ;
		\draw [shift={(550,180)}, rotate = 135] [color={rgb, 255:red, 0; green, 0; blue, 0 }  ][fill={rgb, 255:red, 0; green, 0; blue, 0 }  ][line width=0.75]      (0, 0) circle [x radius= 3.35, y radius= 3.35]   ;
		%Straight Lines [id:da3765788262191845] 
		\draw    (550,180) -- (560,200) ;
		\draw [shift={(560,200)}, rotate = 63.43] [color={rgb, 255:red, 0; green, 0; blue, 0 }  ][fill={rgb, 255:red, 0; green, 0; blue, 0 }  ][line width=0.75]      (0, 0) circle [x radius= 3.35, y radius= 3.35]   ;
		\draw [shift={(550,180)}, rotate = 63.43] [color={rgb, 255:red, 0; green, 0; blue, 0 }  ][fill={rgb, 255:red, 0; green, 0; blue, 0 }  ][line width=0.75]      (0, 0) circle [x radius= 3.35, y radius= 3.35]   ;
		%Straight Lines [id:da057732075656578896] 
		\draw    (420,140) -- (460,160) ;
		\draw [shift={(460,160)}, rotate = 26.57] [color={rgb, 255:red, 0; green, 0; blue, 0 }  ][fill={rgb, 255:red, 0; green, 0; blue, 0 }  ][line width=0.75]      (0, 0) circle [x radius= 3.35, y radius= 3.35]   ;
		\draw [shift={(420,140)}, rotate = 26.57] [color={rgb, 255:red, 0; green, 0; blue, 0 }  ][fill={rgb, 255:red, 0; green, 0; blue, 0 }  ][line width=0.75]      (0, 0) circle [x radius= 3.35, y radius= 3.35]   ;
		%Straight Lines [id:da6702517354459413] 
		\draw    (460,160) -- (500,180) ;
		\draw [shift={(500,180)}, rotate = 26.57] [color={rgb, 255:red, 0; green, 0; blue, 0 }  ][fill={rgb, 255:red, 0; green, 0; blue, 0 }  ][line width=0.75]      (0, 0) circle [x radius= 3.35, y radius= 3.35]   ;
		\draw [shift={(460,160)}, rotate = 26.57] [color={rgb, 255:red, 0; green, 0; blue, 0 }  ][fill={rgb, 255:red, 0; green, 0; blue, 0 }  ][line width=0.75]      (0, 0) circle [x radius= 3.35, y radius= 3.35]   ;
		%Straight Lines [id:da6027456046960759] 
		\draw    (440,100) -- (420,140) ;
		\draw [shift={(420,140)}, rotate = 116.57] [color={rgb, 255:red, 0; green, 0; blue, 0 }  ][fill={rgb, 255:red, 0; green, 0; blue, 0 }  ][line width=0.75]      (0, 0) circle [x radius= 3.35, y radius= 3.35]   ;
		\draw [shift={(440,100)}, rotate = 116.57] [color={rgb, 255:red, 0; green, 0; blue, 0 }  ][fill={rgb, 255:red, 0; green, 0; blue, 0 }  ][line width=0.75]      (0, 0) circle [x radius= 3.35, y radius= 3.35]   ;

		% Text Node
		\draw (70,80) node [anchor=north west][inner sep=0.75pt]    {$\phi ( \ \ \ \ \ \ \ \ \ \ \ \ \ \ \ \ \ \ \ \ \ \ \ \ \ \ \ \ \ \ \ \ \ \ \ \ \ \ \ \ \ \ \ \ \ \ \ \ \ \ \ \ \ \ \ \ \ \ \ \ \ \ \ \ \ \ \ \ \ \ ) \ =\ $};
		% Text Node
		\draw (215,67) node [anchor=north west][inner sep=0.75pt]   [align=left] {a};
		% Text Node
		\draw (133,95) node [anchor=north west][inner sep=0.75pt]   [align=left] {b};
		% Text Node
		\draw (163,116) node [anchor=north west][inner sep=0.75pt]   [align=left] {c};
		% Text Node
		\draw (196,117) node [anchor=north west][inner sep=0.75pt]   [align=left] {d};
		% Text Node
		\draw (223,107) node [anchor=north west][inner sep=0.75pt]   [align=left] {e};
		% Text Node
		\draw (264,116) node [anchor=north west][inner sep=0.75pt]   [align=left] {f};
		% Text Node
		\draw (334,101) node [anchor=north west][inner sep=0.75pt]   [align=left] {g};
		% Text Node
		\draw (91,122) node [anchor=north west][inner sep=0.75pt]   [align=left] {h};
		% Text Node
		\draw (131,141) node [anchor=north west][inner sep=0.75pt]   [align=left] {i};
		% Text Node
		\draw (158,132) node [anchor=north west][inner sep=0.75pt]   [align=left] {j};
		% Text Node
		\draw (214,130) node [anchor=north west][inner sep=0.75pt]   [align=left] {k};
		% Text Node
		\draw (258,132) node [anchor=north west][inner sep=0.75pt]   [align=left] {l};
		% Text Node
		\draw (311,132) node [anchor=north west][inner sep=0.75pt]   [align=left] {m};
		% Text Node
		\draw (81,152) node [anchor=north west][inner sep=0.75pt]   [align=left] {n};
		% Text Node
		\draw (118,161) node [anchor=north west][inner sep=0.75pt]   [align=left] {o};
		% Text Node
		\draw (211,152) node [anchor=north west][inner sep=0.75pt]   [align=left] {p};
		% Text Node
		\draw (258,152) node [anchor=north west][inner sep=0.75pt]   [align=left] {q};
		% Text Node
		\draw (458,52) node [anchor=north west][inner sep=0.75pt]   [align=left] {a};
		% Text Node
		\draw (428,82) node [anchor=north west][inner sep=0.75pt]   [align=left] {b};
		% Text Node
		\draw (481,102) node [anchor=north west][inner sep=0.75pt]   [align=left] {c};
		% Text Node
		\draw (521,122) node [anchor=north west][inner sep=0.75pt]   [align=left] {d};
		% Text Node
		\draw (561,141) node [anchor=north west][inner sep=0.75pt]   [align=left] {e};
		% Text Node
		\draw (601,161) node [anchor=north west][inner sep=0.75pt]   [align=left] {f};
		% Text Node
		\draw (641,181) node [anchor=north west][inner sep=0.75pt]   [align=left] {g};
		% Text Node
		\draw (538,162) node [anchor=north west][inner sep=0.75pt]   [align=left] {k};
		% Text Node
		\draw (521,182) node [anchor=north west][inner sep=0.75pt]   [align=left] {p};
		% Text Node
		\draw (564,182) node [anchor=north west][inner sep=0.75pt]   [align=left] {l};
		% Text Node
		\draw (398,132) node [anchor=north west][inner sep=0.75pt]   [align=left] {h};
		% Text Node
		\draw (438,152) node [anchor=north west][inner sep=0.75pt]   [align=left] {i};
		% Text Node
		\draw (478,171) node [anchor=north west][inner sep=0.75pt]   [align=left] {j};
		% Text Node
		\draw (411,182) node [anchor=north west][inner sep=0.75pt]   [align=left] {...};
	\end{tikzpicture}
	\caption{Funzionamento, in generale, dell'isomorfismo tra foreste ordinate e alberi binari con
		la tecnica `first-child next-sibling`.}
	\label{fig:iso_forest_bintree}
\end{figure}

Ciò che abbiamo dimostrato è che le foreste ordinate sono tante quante gli alberi binari:
$\phi$ `manda' una foresta ordinata con $n$ nodi interni in un albero binario con altrettanti nodi.

\subsubsection{Isomorfismo tra foreste ordinate e parole di Dyck.}
Definiamo $\psi: D_{2n} \rightarrow F_n$ un isomorfismo tra parole ben parentesizzate di lunghezza $2n$ e
foreste ordinate con $n$ nodi induttivamente:
$$
	\psi(\epsilon) = \langle \rangle
$$
$$
	\psi(w_1 \cdots w_k) =  \langle\psi(w_1), \cdots, \psi(w_k)\rangle
$$
e
$$
	\psi((w)) = \langle \text{ albero radicato con figli tutti e soli gli alberi della foresta} \phi(w)  \rangle
$$

\noindent
Sappiamo che $D_{2n} \approxeq F_n \approxeq B_n$ e sappiamo quanti sono i possibili alberi binari
con $n$ nodi sono $|B_n| = C_n  \approx 2n + o (\log_2(n))$; di conseguenza,
le parole di Dyck di lunghezza $2n$ hanno come information-theoretical lower bound
$$
	Z_n = n + o(\log_2(n)) \text{ bit}
$$
Quindi
$$
	D_n - Z_n =  8n + o(n) - n - o(\log_2(n)) = 7n - o(n) - o(\log_2(n))  = O(n) \implies D_n \approx Z_n + O(Z_n)
$$
pertanto la struttura descritta è compatta.

% lezione 15-12-2021
\section{Struttura per hash minimali perfetti}
\subsection{Funzioni di hash}
Le funzioni di hash compaiono in molti contesti diversi e un'applicazione
`famose' sono le \textit{tabelle di hash}.
Nel modo più generale, dato un universo $U$ infinito o \textit{molto grande} e un numero $m \in \mathbb{N}$,
che è il \textit{numero di bucket}, una funzione di hash è
$$
	h: U \rightarrow m
$$

le funzioni di hash, che denominiamo $H_{U,m}$ è infinito se $U$ è infinito, mentre è finito
se $U$ è finito, in particolare $|H_{U,m}| = m^{|U|}$.
La funzione $h$ deve avere alcune proprietà: prendiamo come esempio proprio le tabelle di hash,
utilizzate per memorizzare un sottoinsieme $S \subseteq U$, per esempio delle stringhe su un certo
alfabeto $\Sigma = \{a, b, c, d\}$. Fissato un $m \in \mathbb{N}$, la funzione è
$$
	h: \Sigma^* \leadsto m
$$
vedendo i simboli in $\Sigma$ come un'enumerazione $a = 0, b = 1, c =2, d = 3$,
una funzione di hash potrebbe semplicemente sommare i simboli di una stringa e
operarne il modulo in $m$, in modo che il valore risultante sia $0 \leq h \le m-1 $.

Lo scopo è mantenere una tabella, spesso realizzata come un vettore di list di
elementi, i cui indici sono esattamente gli hash possibili che la funzione $h$
calcola. In questo modo, è possibile accedere in modo (quasi) diretto alle
parole nella tabella\footnote{Vi sono principalmente due  modi per costruire
	tabelle di hash: \textit{open addressing} e \textit{separate chaining}.
	In questo frangente stiamo analizzando la versione con separate chaining.}.
In termini pratici, partendo da una tabella vuota, quando
si vuole inserire una stringa $s = "foo"$ nella tabella, si utilizza la funzione
di hash per calcolare $h(s) = h_1$. Il valore calcolato sarà l'indice della
tabella dove inserire il riferimento alla stringa.

\begin{figure}[h]
	\centering
	\tikzset{every picture/.style={line width=0.75pt}} %set default line width to 0.75pt        

	\begin{tikzpicture}[x=0.75pt,y=0.75pt,yscale=-1,xscale=1]
		%uncomment if require: \path (0,300); %set diagram left start at 0, and has height of 300

		%Shape: Rectangle [id:dp9579578997678553] 
		\draw   (70,100) -- (110,100) -- (110,120) -- (70,120) -- cycle ;
		%Shape: Rectangle [id:dp027628855530104857] 
		\draw   (70,80) -- (110,80) -- (110,100) -- (70,100) -- cycle ;
		%Shape: Rectangle [id:dp8143559001814539] 
		\draw   (70,140) -- (110,140) -- (110,160) -- (70,160) -- cycle ;
		%Shape: Rectangle [id:dp38276393364199424] 
		\draw   (70,120) -- (110,120) -- (110,140) -- (70,140) -- cycle ;
		%Straight Lines [id:da801581038753237] 
		\draw    (110,90) -- (137,90) ;
		\draw [shift={(140,90)}, rotate = 180] [fill={rgb, 255:red, 0; green, 0; blue, 0 }  ][line width=0.08]  [draw opacity=0] (8.93,-4.29) -- (0,0) -- (8.93,4.29) -- cycle    ;
		%Shape: Rectangle [id:dp3050844046907898] 
		\draw   (140,80) -- (180,80) -- (180,100) -- (140,100) -- cycle ;
		%Shape: Rectangle [id:dp41513696416167833] 
		\draw   (170,80) -- (180,80) -- (180,100) -- (170,100) -- cycle ;

		%Shape: Rectangle [id:dp43239961582118536] 
		\draw   (210,80) -- (250,80) -- (250,100) -- (210,100) -- cycle ;
		%Shape: Rectangle [id:dp38503115889664563] 
		\draw   (240,80) -- (250,80) -- (250,100) -- (240,100) -- cycle ;

		%Straight Lines [id:da3303160944704562] 
		\draw    (180,90) -- (207,90) ;
		\draw [shift={(210,90)}, rotate = 180] [fill={rgb, 255:red, 0; green, 0; blue, 0 }  ][line width=0.08]  [draw opacity=0] (8.93,-4.29) -- (0,0) -- (8.93,4.29) -- cycle    ;
		%Straight Lines [id:da047149376903893425] 
		\draw    (110,149) -- (137,149) ;
		\draw [shift={(140,149)}, rotate = 180] [fill={rgb, 255:red, 0; green, 0; blue, 0 }  ][line width=0.08]  [draw opacity=0] (8.93,-4.29) -- (0,0) -- (8.93,4.29) -- cycle    ;
		%Shape: Rectangle [id:dp591261104910037] 
		\draw   (140,139) -- (180,139) -- (180,159) -- (140,159) -- cycle ;
		%Shape: Rectangle [id:dp4443187367264515] 
		\draw   (170,139) -- (180,139) -- (180,159) -- (170,159) -- cycle ;


		% Text Node
		\draw (58,85) node [anchor=north west][inner sep=0.75pt]   [align=left] {0};
		% Text Node
		\draw (58,105) node [anchor=north west][inner sep=0.75pt]   [align=left] {1};
		% Text Node
		\draw (54,125) node [anchor=north west][inner sep=0.75pt]   [align=left] {$\displaystyle \cdots $};
		% Text Node
		\draw (45,145) node [anchor=north west][inner sep=0.75pt]   [align=left] {m-1};
		% Text Node
		\draw (141,82) node [anchor=north west][inner sep=0.75pt]   [align=left] {foo};
		% Text Node
		\draw (211,82) node [anchor=north west][inner sep=0.75pt]   [align=left] {bar};
		% Text Node
		\draw (141,141) node [anchor=north west][inner sep=0.75pt]   [align=left] {baz};
	\end{tikzpicture}
	\caption{Esempio di una tabella di hash con separate chaining.}
	\label{fig:hashtable_example}
\end{figure}

Possono accadere dei conflitti, ossia due stringhe $s_1$ e $s_2$ tali che
$h(s_1) = h(s_2)$; per ovviare a questo problema si descrive $M$ come una mappa di liste, ossia
ad un indice $h_1$ di $M$ corrisponde una lista di stringhe.

\subsubsection{Requisiti}
La tabella $M$ funziona con qualsiasi funzione di hash $h$ (anche $h(\cdot) = 0$),
ma l'efficienza di $M$ cambia al suo variare, fino al denegenerare in una lista.
Per far funzionare ``bene'' $M$, $h$ deve essere
\begin{itemize}
	\setlength\itemsep{0pt}
	\item $h$ sia veloce da calcolare; e
	\item $h$ divide l'universo $U$ in \textit{buckets} in modo tale che le
	      controimmagini siano più o meno \textit{grandi uguali}.
	      In altri termini, se guardiamo $\Sigma^*$ e guardiamo come
	      $h$ ne partiziona gli elementi, vorremmo che i blocchi (o bucket)
	      della partizione siano più o meno grandi uguali e non ci sia
	      qualche blocco che contiene molti più elementi degli
	      altri.
	      % disegno pagina 1 15-12-2021 buckets nell'universo. 
\end{itemize}
Questi requisiti sono esprimibili formalmente ovviamente, ma solitamente oltre
a questi requisiti di base ve ne sono altri che dipendono dall'utilizzo
che si vuole fare della funzione; nel nostro caso, facciamo le seguenti
assunzioni: la prima, chiamata \textit{full randomness assumption},
è che possiamo estrarre uniformemente una funzione $h$ dall'insieme $\mathcal{H}_{U,m}$.

La seconda è che $h$ sia calcolabile in tempo e spazio costante, inoltre occupa
spazio costante in termini di codice (non usa array arbitrariamente grandi, ...).
Questa assunzione è normalmente inattuabile: si pensi al caso in cui $U = \Sigma^*$:
si dovrebbero considerare tutte le possibili funzioni dalle stringhe ai naturali
minori di $m$, che sono infinite e tra le quali ce ne sono anche di non calcolabili.

\subsubsection{Rappresentazione}
Ciò che si può fare è invece considerare un universo limitato con un
upper bound dipendende da un qualche intero $k$.
Chiaramente si perde della `randomness', ma il vantaggio è che molto spesso
i risultati che si ottengono sotto l'assunzione di completa casualità si possono
portare anche sotto $U$ così definito, magari con qualche approssimazione.

Supponiamo quindi di avere $U = \Sigma^{\leq k}$; vogliamo scrivere delle
funzioni $h: U \rightarrow m$ e un modo per descriverle è il seguente.
Si usa un array $\mathbf{p}$ chiamato \textit{array dei pesi} contenente $k$ valori,
inizializzandolo a valori pseudocasuali nell'insieme $\{0, \cdots, m-1\}$.
Quando si vuole calcolare l'hash di una stringa $s = "foo"$ si considera il valore
di ogni lettera della stringa e si moltiplica per il peso di quel carattere
in senso posizionale, ossia il primo peso è associato al primo carattere della stringa,
il secondo peso al secondo carattere e così via; quindi si sommano i risultati e
si computa il modulo $m$. Per esempio, se
$$
	a = 0, b = 1, c = 2, \cdots, f = 5, \cdots, o = 14, \cdots
$$
e
$$
	\mathbf{p} = [15, 86, 90, \cdots]
$$
Il calcolo dell'hash è
$$
	h("foo") = ((5 * 15) + (5 * 86) + (14*90)) \mod m
$$
ottendo un valore che si calcola in un tempo lineare nella lunghezza della stringa
diverso per ogni possibile scelta dei pesi. Cioè, non stiamo
scegliendo tra tutte le possibili funzioni di hash, bensì scegliamo tra le funzioni
di hash caratterizzabili dal vettore $\mathbf{p}$, ossia in $m^k$ modi.
Queste funzioni non occupano molto spazio e si calcolano in un tempo non
costante ma logaritmico nella dimensione dell'universo.

\subsection{Relazione con i grafi}
\subsubsection{Sequenza di peeling di un grafo}
Si supponga di avere un grafo $G = (V,E)$ non orientato. Una \textbf{sequenza di peeling} è
una sequenza di coppie di archi e vertici in cui appaiono tutti i lati e uno dei
due vertici incidenti a tale lato. Ogni vertice che appare è chiamato \textit{hinge}
della sequenza. La sequenza deve essere tale per cui nessun vertice hinge $x_i$
è apparso nei lati che precedono $i$.

\begin{figure}[htpb]
	\begin{center}
		\begin{subfigure}[b]{0.45\textwidth}
			\begin{center}
				\begin{tikzpicture}
					\node [circle,fill,label=above:{A},inner sep=1pt](A) at (0,0) {};
					\node [circle,fill,label=above:{B},inner sep=1pt](B) at (1,-1) {};
					\node [circle,fill,label=above:{C},inner sep=1pt](C) at (2,0) {};
					\node [circle,fill,label=above:{D},inner sep=1pt](D) at (4,0) {};
					\node [circle,fill,label=above:{E},inner sep=1pt](E) at (6,0) {};

					\draw (A) -- (C) node [midway,above] {a};
					\draw (B) -- (C) node [midway,above] {b};
					\draw (C) -- (D) node [midway,above] {c};
					\draw (D) -- (E) node [midway,above] {d};
				\end{tikzpicture}

			\end{center}
		\end{subfigure}
		\begin{subfigure}[b]{0.45\textwidth}
			\begin{center}
				\begin{align*}
					a & = \{(A, C), A\} \\
					b & = \{(B, C), B\} \\
					c & = \{(C, D), D\} \\
					d & = \{(D, E), E\}
				\end{align*}
			\end{center}
		\end{subfigure}
	\end{center}
	\caption{Un grafo e una sequenza di peeling valida.}%
	\label{fig:example_peeling}
\end{figure}

Prendendo come esempio il grafo in \cref{fig:example_peeling}, per verificare una
sequenza di peeling si deve innanzitutto verificare che tutti i lati
appaiano nella sequenza; quindi, si verifica che ogni vertice non sia mai
comparso prima: per esempio, avendo scelto il lato $(A,B)$ invece di $(A,C)$,
non si sarebbe potuto scegliere il vertice $B$ da associare al lato $(B,C)$.

Non tutti i grafi ammettono una sequenza di peeling:
\begin{theorem}
	Un grafo $G$ ammette una sequenza di peeling se e solo se è aciclico.
\end{theorem}
\begin{proof}

	$\implies$ Per assurdo, si supponga che
	$\langle\{e_0, x_0\}, \cdots, \{e_{m-1}, x_{m-1}\}\rangle$
	sia una sequenza di peeling e che esista un ciclo sui vertici
	$y_1, y_2, \cdots, y_k$ e i relativi lati $\{y_1, y_2\}, \cdots, \{y_{k-1}, y_k\}$.
	Sia $\bar{i}$ l'indice massimo della sequenza del ciclo.
	Inserendo tutti i lati del ciclo nella sequenza, quando si arriva ad
	inserire l'ultimo lato del ciclo, non ci sarà modo di scegliere un nodo
	che ancora non appare nella sequenza.

	$\impliedby$ Per induzione su $|E|$. Omessa.
	%(esercizio). Hint: si parte dai lati più esterni. 
\end{proof}

\subsubsection{Ipergrafi}
Vogliamo ora generalizzare la nozione di sequenza di peeling agli ipergrafi.

\paragraph{Definizione}
Un $r$-ipergrafo è $G = (V, E)$ di vertici e \textit{iperlati} dove ogni lato è un
insieme di $r$ vertici, ossia $E \subseteq {V \choose r}$. Un esempio è
in \cref{fig:example:hypergraph}.

\begin{figure}[htpb]
	\begin{center}
		\begin{tikzpicture}[scale=1, transform shape]
			\node (f) at (0,0) {$F$};
			\node (g) at (1,0) {$G$};
			\node (e) at (2,0) {$E$};
			\node (d) at (2,1) {$D$};
			\node (a) at (2,2) {$A$};
			\node (b) at (3,2) {$B$};

			\node [draw, rounded corners,inner sep=0pt,fit=(f) (g)] {};
			\node [draw, dashed, rounded corners,inner sep=5pt,fit=(g) (e)] {};
			\node [draw, rounded corners,inner sep=0pt,fit=(a) (d) (e)] {};
			\node [draw, dashed, rounded corners,inner sep=5pt,fit=(a) (b) (d)] (fd){};
		\end{tikzpicture}
	\end{center}
	\caption{Esempio di ipergrafo.}%
	\label{fig:example:hypergraph}
\end{figure}

Non esiste una nozione di aciclicità per ipergrafi, mentre esiste una nozione di
sequenza di peeling: una sequenza di peeling per un $r$-ipergrafo è una sequenza
dei suoi iperlati ai quali si associa un hinge in modo tale che non sia mai apparso
negli iperlati precedenti.
Per questo motivo non si generalizza la nozione di aciclicità bensì quella di sequenza di peeling;
si dice che un ipergrafo è aciclico se e solo se ammette una sequenza di peeling.


\subsection{Tecnica MWHC}
\subsubsection{Funzioni statiche}
Il nostro obiettivo è memorizzare funzioni statiche.
Dato un universo $U$, un sottoinsieme fissato $X \subseteq U$ e $r \in \mathbb{N}$
vogliamo memorizzare una funzione
$$
	f: X \rightarrow 2^r
$$
di nuovo, si immagini $U$ come l'insieme dei caratteri ASCII;
una funzione $f$ può essere come quella in \cref{tab:example:static_func},
in cui $X$ è l'insieme di quelle tre stringhe: a noi non interessa
memorizzare stringhe diverse da quelle.

\begin{table}[htpb]
	\begin{tabular}{c|c c}
		x              & $f(x)$          &    \\
		\hline                                \\
		Paolo Boldi    & \texttt{00111}  & 7  \\
		Anna Zuppi     & \texttt{101000} & 20 \\
		Giovanni Galli & \texttt{101110} & 23
	\end{tabular}
	\centering
	\caption{Esempio di funzione statica.}
	\label{tab:example:static_func}
\end{table}

Memorizzare una funzione significa che vogliamo ricavare una struttura
dati $D$ tale che permette di valutare un input in modo da ottenere
il valore di $f(x)$ ossia, per esempio, $"Paolo Boldi" \mapsto 00111$ e così via, mentre
il comportamento inteso per gli input non appartenenti all'insieme $X$
che vogliamo memorizzare è irrilevante.
La funzione $f$ si definisce \textit{statica} poiché è paragonabile
alla struttura \textit{dizionario} nei linguaggi di programmazione come
Python, benché questa struttura sia immodificabile e
non ci sia modo di sapere se una certa chiave è presente o meno.

\subsubsection{Rappresentazione}
Per costruire una rappresentazione succinta si utilizza la tecnica
MWHC\footnote{Majewski, Worwald, Havas, Czech}.
Inizialmente si fissa un $m$ intero e si scelgono uniformemente due funzioni hash
$$
	h_1, h_2 : U \rightarrow m
$$
Assumiamo che $\forall x \in X ~ h_1(x) \neq h_2(x)$.
Costruiamo un grafo i cui vertici sono i numeri $0, \cdots, m-1$ e i lati
corrispondono agli elementi dell'insieme $X$, con l'idea che alla chiave $x$
corrisponde il lato $(h_1(x), h_2(x))$.

Seguendo l'esempio in \cref{tab:example:static_func}, immaginiamo che
le due funzioni calcolino l'hash delle stringhe in $X$ come descritto
in \cref{tab:static_hashes}, supponendo $m = 17$ e $r = 5$. 
Si può ora costruire un grafo costruito
come in \cref{fig:static_graph}: grazie all'assunzione precedente,
ossia la differenza dei due hash per ogni chiave, sappiamo che nel
grafo non vi sono cappi. In caso accada, si generano due nuove
funzioni $h_1$ e $h_2$. Inoltre, vogliamo che il grafo sia
aciclico e che a nessun lato corrispondano due o più chiavi diverse: anche
in questi casi si possono generare due nuove funzioni hash.

\begin{table}[htpb]
	\begin{tabular}{c | c | c}
		x              & $h_1(x)$ & $h_2(x)$ \\
		\hline                               \\
		Paolo Boldi    & 3        & 7        \\
		Anna Zuppi     & 14       & 13       \\
		Giovanni Galli & 13       & 2
	\end{tabular}
	\centering
	\caption{Calcolo di $h_1$ e $h_2$ sulle stringhe nell'insieme $X$.}
	\label{tab:static_hashes}
\end{table}
\begin{figure}[htpb]
	\begin{center}
		\begin{tikzpicture}
			\node [circle,fill,label=above:{0},inner sep=1pt](0) at (0,0) {};
			\node [circle,fill,label=above:{1},inner sep=1pt](1) at (1,1) {};
			\node [circle,fill,label=above:{2},inner sep=1pt](2) at (2,1) {};
			\node [circle,fill,label=above:{3},inner sep=1pt](3) at (3,0.5) {};
			\node [circle,fill,label=above:{7},inner sep=1pt](7) at (5,0.5) {};
			\node [circle,fill,label=right:{13},inner sep=1pt](13) at (2,-1) {};
			\node [circle,fill,label=above:{14},inner sep=1pt](14) at (-1.5,-1) {};
			\node [circle,fill,label=above:{$\cdots$},inner sep=1pt](d) at (4,-1) {};

			\draw (14) -- (13) node [midway,above] {zoppi - $20$};
			\draw (13) -- (2) node [midway,above, label={[rotate=-90]above:{galli - $23$}}] {};
			\draw (3) -- (7) node [midway,above] {boldi - $7$};
		\end{tikzpicture}

	\end{center}
	\caption{Grafo associato per la costruzione di una struttura per funzioni statiche.}%
	\label{fig:static_graph}
\end{figure}
Trasformeremo ora il grafo in un sistema di equazioni: ogni vertice è una variabile $w_0, w_1, \cdots, w_{m-1}$ e ad ogni lato corrisponde
l'equazione
$$
	\forall x \in X ~~ (w_{h_1(x)} + w_{h_2(x)}) \mod 2^r = f(x)
$$
Se il grafo è aciclico (che è un'assunzione) il sistema è risolvibile. 
L'esistenza di una sequenza di peeling (che esiste se e solo se il grafo è aciclico) significa 
che si possono ordinare le equazioni in modo che una delle due variabili non sia mai 
comparsa prima: questo significa che la soluzione si può trovare ordinando le equazioni
e assegnando il valore che rende vera l'equazione alla variabile che non è ancora apparsa. 
Quindi, per esempio
$$
\begin{cases}
    (w_{14} + w_{13}) \mod 2^5 = 20 \\
    (w_{2} + w_{13}) \mod 2^5 = 23 \\
    (w_{3} + w_{7}) \mod 2^5 = 7 \\
\end{cases}
$$
Usando come hinge $w_{14} = 20$, $w_{2} = 23$ e $w_{7} = 7$ si risolve il sistema. 

\subsubsection{Implementazione}
Una volta memorizzate le soluzioni $w_i$ del sistema in un vettore $\mathbf{w}$, 
si può calcolare $f(x)$ per ogni $x \in X$ come 
$$
f(x) = (w_{h_1(x)} + w_{h_2(x)}) \mod 2^m 
$$
Bisogna però decidere la grandezza di $m$, ossia il numero di vertici (o il numero di bucket, o il numero 
massimo che può risultare da un hash): è facile vedere che se $m$ è ``troppo piccolo'' è molto imporbabile 
che riescano a soddisfare le condizioni, ossia si troveranno cicli, collisioni e così via. Se si 
sceglie ``troppo grande'', la struttura che si memorizza occupa molto spazio. 
Precisamente, questo tradeoff dipende da quanto è probabile che il grafo generato sia aciclico. 
\begin{theorem}
    Se $m > (2.09 \cdot |X|)$, il grafo è ``quasi sempre'' aciclico. Il numero atteso di tentativi 
    di generazione di coppie di funzioni hash è circa $2$.
\end{theorem}
\begin{proof}
    Omessa.
\end{proof}

Naturalmente si può anche calcolare la funzione di un qualcosa che non è tra gli input desiderati e 
qualcosa verrà prodotto, ma non avrà un senso ben inteso. 

\subsubsection{Spazio}
Il vettore che dobbiamo memorizzare ha $m$ elementi, ognuno dei quali contiene $r$ bit: in tutto, 
il vettore occupa spazio $m\cdot r$ bit. Definendo $|X| = n$, dal teorema precedente si ha 
$m\cdot r \geq 2.09nr$; lo spazio totale si trova aggiungendo lo spazio per memorizzare 
le funzioni di hash $h_1$ e $h_2$.  

Tutto questo processo può essere eseguito non solo per i grafi (costruiti con $2$ funzioni hash)
ma anche per gli ipergrafi, costruendo un $r$-ipergrafo utilizzando $r$ funzioni hash. Ci si può 
quindi chiedere quale sia la costante che rende ipergrafi con $r > 2$ ``quasi sempre'' aciclici: 
\begin{theorem}
    Per ogni $k$-ipergrafo esiste una costante $\gamma_k$ tale che se $m > \gamma_k n$ allora 
    l'ipergrafo ammette quasi sempre una sequenza di peeling. 
\end{theorem}
\begin{proof}
    Omessa.
\end{proof}

In effetti, le costanti $\gamma_k$ hanno un minimo in $k = 3$: il meglio che si può 
ottenere è quindi utilizzando $3$ funzioni di hash. Il numero di bit che consuma 
$\mathbf{w}$ è quindi $1.23nr$ bit.  

\begin{table}[htpb]
    \centering
    \begin{tabular}{c|c}
	 $k$ &  $\gamma_k$ \\ 
	 \hline 
	 2 & 2.09 \\
	 3 & 1.23 \\
	 $\cdots$ & > 1.23 $\cdots$
    \end{tabular}
    \caption{Costanti $\gamma_k$ per $k$-ipergrafi.}
    \label{tab:gamma_k_hypergraph}
\end{table}

\subsubsection{Lower bound}
Per capire che tipo di struttura stiamo costruendo, dobbiamo calcolare l'information-theoretical 
lower bound. Stiamo memorizzando una funzione da un insieme $X$ fissato ad un insieme $2^r$.  
Le funzioni di questo tipo sono ${2^{r}}^{|X|} = 2^{r|X|} = 2^{rn}$, definendo come prima $|X| = n$. 
Quindi, l'information-theoretical lower bound è 
$$
Z_n = \log_2(2^{rn}) = rn \text{ bit}
$$
e la struttura che abbiamo descritto occupa 
$$
D_n = 1.23nr = O(Z_n) \text{ bit}
$$
pertanto la struttura è compatta.

\subsubsection{Struttura succinta}
Si può osservare che nel vettore $\mathbf{w}$ molte entry sono uguali a $0$: il numero 
di entry diverse da zero è pari al numero di hinge nella sequenza di peeling 
che risolve il sistema generato dal grafo, quindi $\mathbf{w}$ che (assumendo 
l'utlizzo di $3$ funzioni di hash) ha esattamente $m = 1.23n$ elementi, al più
$n$ elementi sono non nulli. 

Quindi, si possono memorizzare unicamente gli elementi non nulli in $\tilde{\mathbf{w}}$ 
utilizzando un ulteriore array $\mathbf{b}$ di $m$ bit tale che 
$$
\mathbf{b}[i] = \begin{cases}
    1 & \mathbf{w}[i] \neq 0 \\
    0 & \mathbf{w}[i] = 0
\end{cases}
$$
definendo quindi il vettore 
$$
\mathbf{w}[i] = \begin{cases}
    0 & \mathbf{b}[i] = 0 \\
    \tilde{\mathbf{w}}[\mathbf{rank_b}[i]] & \mathbf{b}[i] = 1
\end{cases}
$$
In questo modo, la struttura (che è comunque compatta) in totale occupa 
$$
D_n = nr + m = nr + 1.23n = (r + 1.23) n \text{ bit}
$$
e si ha quindi 
$$
(r + 1.23 ) n < 1.23rn \implies (r + 1.23) < 1.23r  \implies r > 5
$$

\noindent
Inizialmente, la tecnica MWHC era stata pensata per un uso ben più specifico, 
ossia memorizzare una funziona di hash minimale perfetta, ossia una funzione
che mappa $n$ chiavi in $n$ bucket privi di collisioni. 
Tuttavia questa tecnica è abbastanza generale da memorizzare qualsiasi funzione e 
si può utilizzare questa tecnica come intesa inizialmente semplicemente associando 
$f(x)$ diversi da $0$ a $n-1$ per ogni elemento di $X$. 

\cleardoublepage

\appendix
\chapter{Laboratorio 1: Cammini disgiunti tramite algoritmo basato su pricing}


\begin{minted}[mathescape, fontsize=\small, xleftmargin=0.5em]{python}
import networkx as nx
import math
from networkx.algorithms.shortest_paths.weighted import dijkstra_path
\end{minted}


Definiamo una funzione per generare grafi in una  forma precisa, ossia a 
\textit{doppio ventaglio}: 

\begin{minted}[mathescape, fontsize=\small, xleftmargin=0.5em]{python}
def doubleFan(k):
    G = nx.DiGraph()
    G.add_nodes_from(['s', 't', 'x', 'y'])
    G.add_nodes_from(['a' + str(i) for i in range(k)])
    G.add_nodes_from(['b' + str(i) for i in range(k)])
    G.add_edges_from([('s', 'a' + str(i)) for i in range(k)])
    G.add_edges_from([('a' + str(i), 'x') for i in range(k)])
    G.add_edges_from([('y', 'b' + str(i)) for i in range(k)])
    G.add_edges_from([('b' + str(i), 't') for i in range(k)])
    G.add_edge('x', 'y')
    return G
\end{minted}

\begin{minted}[mathescape, fontsize=\small, xleftmargin=0.5em]{python}
nx.draw(doubleFan(2))
\end{minted}
\includegraphics[width=0.3 \linewidth]{labs/figures/lab1_figure3_1.pdf}


\begin{minted}[mathescape, fontsize=\small, xleftmargin=0.5em]{python}
nx.draw(doubleFan(4))
\end{minted}
\includegraphics[width=0.3 \linewidth]{labs/figures/lab1_figure4_1.pdf}



\begin{minted}[mathescape, fontsize=\small, xleftmargin=0.5em]{python}
nx.draw(doubleFan(8))
\end{minted}
\includegraphics[width=0.3 \linewidth]{labs/figures/lab1_figure5_1.pdf}


Definiamo, quindi, una funzione che implementa \textsc{PricedDisjointPaths}: 


\begin{minted}[mathescape, fontsize=\small, xleftmargin=0.5em]{python}
def greedyDisjointPaths(G_original, sourceTargetPairs, c = 1):
    G = G_original.copy()
    result = []
    beta = math.pow(G.number_of_edges(), 1 / (c + 1))
    # Set all lengths to 1 and all congestion to 0
    for u,v,d in G.edges(data = True):
        d['length'] = 1
        d['congestion'] = 0
    # Main cycle
    while True:
        minPath = None
        for pairIndex in range(len(sourceTargetPairs)):
            try:
                source = sourceTargetPairs[pairIndex][0]
                target = sourceTargetPairs[pairIndex][1]
                path = dijkstra_path(G, source, target, 'length')
            except:
                pass
            else:
                pathLength = 0
                for i in range(len(path) - 1):
                    pathLength += G[path[i]][path[i+1]]['length']
                if minPath == None or pathLength < minPathLength:
                    minPath = path
                    minPathLength = pathLength
                    minPathIndex = pairIndex
        if minPath == None:
            break
        result.append(minPath)
        sourceTargetPairs.pop(minPathIndex)
        for i in range(len(path) - 1):
            x1 = path[i]
            x2 = path[i+1]
            G[x1][x2]['length'] *= beta
            G[x1][x2]['congestion'] += 1
            if G[x1][x2]['congestion'] == c:
                G.remove_edge(x1, x2)
    return result
\end{minted}



\begin{minted}[mathescape, fontsize=\small, xleftmargin=0.5em]{python}
g = doubleFan(2)
nx.draw(g, with_labels = True)
\end{minted}
\includegraphics[width=0.5 \linewidth]{labs/figures/lab1_figure7_1.pdf}



\begin{minted}[mathescape, fontsize=\small, xleftmargin=0.5em]{python}
greedyDisjointPaths(g, [('s', 't')]*11, c = 10)
\end{minted}
\begin{minted}[fontsize=\small, xleftmargin=0.5em, mathescape, frame = leftline]{text}
[['s', 'a0', 'x', 'y', 'b0', 't'],
 ['s', 'a1', 'x', 'y', 'b1', 't'],
 ['s', 'a0', 'x', 'y', 'b0', 't'],
 ['s', 'a1', 'x', 'y', 'b1', 't'],
 ['s', 'a0', 'x', 'y', 'b0', 't'],
 ['s', 'a1', 'x', 'y', 'b1', 't'],
 ['s', 'a0', 'x', 'y', 'b0', 't'],
 ['s', 'a1', 'x', 'y', 'b1', 't'],
 ['s', 'a0', 'x', 'y', 'b0', 't'],
 ['s', 'a1', 'x', 'y', 'b1', 't']]
\end{minted}



\begin{minted}[mathescape, fontsize=\small, xleftmargin=0.5em]{python}
g = doubleFan(4)
nx.draw(g, with_labels = True)
\end{minted}
\includegraphics[width=0.5 \linewidth]{labs/figures/lab1_figure9_1.pdf}



\begin{minted}[mathescape, fontsize=\small, xleftmargin=0.5em]{python}
greedyDisjointPaths(g, [('s', 't')]*11, c = 10)
\end{minted}
\begin{minted}[fontsize=\small, xleftmargin=0.5em, mathescape, frame = leftline]{text}
[['s', 'a0', 'x', 'y', 'b0', 't'],
 ['s', 'a1', 'x', 'y', 'b1', 't'],
 ['s', 'a2', 'x', 'y', 'b2', 't'],
 ['s', 'a3', 'x', 'y', 'b3', 't'],
 ['s', 'a0', 'x', 'y', 'b0', 't'],
 ['s', 'a1', 'x', 'y', 'b1', 't'],
 ['s', 'a2', 'x', 'y', 'b2', 't'],
 ['s', 'a3', 'x', 'y', 'b3', 't'],
 ['s', 'a0', 'x', 'y', 'b0', 't'],
 ['s', 'a1', 'x', 'y', 'b1', 't']]
\end{minted}


\input{labs/lab2.tex}

\backmatter
% if necessary, insert here your bibliography
\bibliographystyle{alpha}
\bibliography{bib/db}

\end{document}
