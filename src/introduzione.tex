\section*{Notazione}
\addcontentsline{toc}{section}{Notazione}


\subsection*{Insiemi}
Indicheremo la cardinalità di un insieme $A$ con $\card{A}$.
Indichiamo con $\N$ l'insieme dei numeri naturali, con $\Z$ i numeri interi (relativi), con $\Q$ i razionali e con $\R$ i reali.
Indichiamo con $S^+$, dove $S$ è uno degli insiemi sopracitati, il sottoinsieme dei numeri positivi di $S$.


\subsection*{Parole e linguaggi}
Se $\Sigma$ è un alfabeto (un insieme di simboli), sia $(\Sigma\star, \cdot, \emptyword)$ il monoide libero\footnote{Per un'introduzione alla relazione tra strutture algebriche e linguaggi (e la loro	chiusura di Kleene), consultare \cite{Sakarovitch:09:automata}.} su $\Sigma$, dove $\Sigma\star$ è l'insieme delle infinite parole su $\Sigma$, $\cdot$ è l'operazione (associativa) di concatenazione di parole e $\emptyword$ la parola vuota (nonché elemento neutro di $\cdot$).
Indichiamo la lunghezza di una parola $w$ con $\len w$.


\subsection*{Funzioni}
Dati due insiemi $A$ e $B$ si definisce
\begin{equation*}
	B^A := \{f : A \to B\}
\end{equation*}
l'insieme di tutte le funzioni che hanno dominio $A$ e codominio $B$.\footnote{La notazione deriva dal fatto che se $A$ e $B$ sono insiemi finiti, la cardinalità di $B^A$ è $\card{B}^{\card{A}}$.}
La notazione di Von Neumann definisce un numero naturale $k$ come l'insieme di cardinalità $k$ dei naturali da $0$ a $k-1$:
$$
	k := \{0,1,\cdots,k-1\}
$$
ad esempio, $0 = \emptyset$, $1 = \{0\}$ e così via. Quindi $2^*$ è l'insieme delle stringhe binarie.
Dato un insieme $A$, l'insieme dei sottoinsiemi di $A$ può essere definito con
$$
	2^A := \{f | f: A \to \{0,1\}\}
$$
in quanto le immagini di una funzione $f_B\in2^A$ rappresentano booleani che descrivono l'appartenenza al sottoinsieme $B$ determinato da $f_B$. $f_B$ si dice funzione caratteristica di $B$
Seguendo la definizione, si ha inoltre
$$
	A^2 = \{f| f: \{0,1\} \to A\}\text.
$$
Ogni funzione in $A^2$ associa a $0$ un elemento di $A$ e a $1$ un elemento di $A$, quindi, a meno di un isomorfismo, l'insieme delle immagini delle funzioni in $A^2$ è $A \times A$.
Come ultimo esempio, si ha che $2^{2^*}$ è l'insieme di tutti i linguaggi binari.


\subsection*{Grafi}
Per i grafi non orientati usiamo i nomi vertice e lato, mentre per i grafi orientati nodo e arco. In un grafo $G=(V,E)$, $V$ è l'insieme dei vertici e $E\subseteq\binom{V}{2}$ è l'insieme dei lati.
Dato un lato $v=\set{a,b}$, si dice che $v$ incide sui vertici $a$ e $b$. I vertici su cui non incide alcun lato vengono chiamati vertici isolati.
In un grafo qualsiasi, un cammino da $a$ a $b$ è una sequenza di archi $(a,x_1),(x_1,x_2),\dots,(x,b)$ e la sua lunghezza è il numero di lati che lo compongono. Nella nostra notazione un cammino può percorrere lo stesso vertice più volte. Ci interesseranno tipicamente, però, i cammini semplici (cioè dove ciò non accade). Due vertici $a$ e $b$ sono connessi (si scrive $a\leadsto b$) se e solo se esiste un cammino da $a$ a $b$. Data la relazione di equivalenza
\begin{equation*}
	a\sim b \iff a\leadsto b
\end{equation*}
le classi di equivalenza si dicono componenti connesse. Un circuito è un cammino da un vertice $a$ ad $a$ (cioè chiuso) di lunghezza maggiore o uguale a $3$.

\subsubsection*{Visite di grafi}
Una visita di un grafo consiste nello svolgere una qualche operazione su tutti vertici di un grafo o su tutti quelli connessi a un seme iniziale $s$. Per fare ciò si usa una struttura dati $D$ chiamata dispenser. La scelta di una struttura dati per il dispenser ha come conseguenza il tipo di visita. I nodi sconosciuti si considerano bianchi, quelli visti ma non visitati grigi e quelli visitati neri. L'algoritmo \ref{alg:VisitaGrafo} rappresenta una visita generica di un grafo $G=(V,E)$.

\begin{algorithm}
	\DontPrintSemicolon
\SetKwData{D}{D}
colora tutti i vertici di bianco\;
colora $s$ di grigio\;
$\D\asn\set{s}$\;
\While{$D\neq\emptyset$}{
	estrai $x$ da $D$\;
	visita $x$\;
	colora $x$ di nero\;
	\For{$y$ vicini di $x$}{
		\If{$y$ è bianco}{
			colora $y$ di grigio\;
			$D\asn D\cup\set{y}$\;
		}
	}
}

	\caption{Visita generica di un grafo $G=(V,E)$}
	\label{alg:VisitaGrafo}
\end{algorithm}
