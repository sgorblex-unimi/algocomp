\chapter{Algoritmi probabilistici}
Il modello di calcolo di riferimento per gli algoritmi deterministici, inclusi quelli approssimanti visti fin'ora, è quello della macchina di Turing, in cui a un input corrisponde un output (\cref{fig:mdtdet}).
\begin{figure}[ht]
	\centering
	\begin{tikzpicture}[x=0.75pt,y=0.75pt,yscale=-1,xscale=1]
	\draw   (280,152) .. controls (280,145.37) and (285.37,140) .. (292,140) -- (358,140) .. controls (364.63,140) and (370,145.37) .. (370,152) -- (370,188) .. controls (370,194.63) and (364.63,200) .. (358,200) -- (292,200) .. controls (285.37,200) and (280,194.63) .. (280,188) -- cycle ;
	\draw    (170,170) -- (277,170) ;
	\draw [shift={(280,170)}, rotate = 180] [fill={rgb, 255:red, 0; green, 0; blue, 0 }  ][line width=0.08]  [draw opacity=0] (8.93,-4.29) -- (0,0) -- (8.93,4.29) -- cycle    ;
	\draw    (370,170) -- (477,170) ;
	\draw [shift={(480,170)}, rotate = 180] [fill={rgb, 255:red, 0; green, 0; blue, 0 }  ][line width=0.08]  [draw opacity=0] (8.93,-4.29) -- (0,0) -- (8.93,4.29) -- cycle    ;
	\draw (311,162) node [anchor=north west][inner sep=0.75pt]   [align=left] {MdT};
	\draw (214,152) node [anchor=north west][inner sep=0.75pt]   [align=left] {input};
	\draw (404,152) node [anchor=north west][inner sep=0.75pt]   [align=left] {output};
\end{tikzpicture}

	\caption{Macchina di Turing deterministica}
	\label{fig:mdtdet}
\end{figure}

Un diverso modello di calcolo è quello della macchina di Turing probabilistica (\cref{fig:mdtprob}), che ha accesso a una sorgente aleatoria, cioè un input secondario di bit casuali.

\begin{figure}[ht]
	\centering
	\begin{tikzpicture}[x=0.75pt,y=0.75pt,yscale=-1,xscale=1]
	\draw   (280,152) .. controls (280,145.37) and (285.37,140) .. (292,140) -- (358,140) .. controls (364.63,140) and (370,145.37) .. (370,152) -- (370,188) .. controls (370,194.63) and (364.63,200) .. (358,200) -- (292,200) .. controls (285.37,200) and (280,194.63) .. (280,188) -- cycle ;
	\draw    (170,170) -- (277,170) ;
	\draw [shift={(280,170)}, rotate = 180] [fill={rgb, 255:red, 0; green, 0; blue, 0 }  ][line width=0.08]  [draw opacity=0] (8.93,-4.29) -- (0,0) -- (8.93,4.29) -- cycle    ;
	\draw    (370,170) -- (477,170) ;
	\draw [shift={(480,170)}, rotate = 180] [fill={rgb, 255:red, 0; green, 0; blue, 0 }  ][line width=0.08]  [draw opacity=0] (8.93,-4.29) -- (0,0) -- (8.93,4.29) -- cycle    ;
	\draw    (330,62.29) -- (330,137) ;
	\draw [shift={(330,140)}, rotate = 270] [fill={rgb, 255:red, 0; green, 0; blue, 0 }  ][line width=0.08]  [draw opacity=0] (8.93,-4.29) -- (0,0) -- (8.93,4.29) -- cycle    ;
	\draw (311,162) node [anchor=north west][inner sep=0.75pt]   [align=left] {MdT};
	\draw (214,152) node [anchor=north west][inner sep=0.75pt]   [align=left] {input};
	\draw (404,152) node [anchor=north west][inner sep=0.75pt]   [align=left] {output};
	\draw (347,72) node [anchor=north west][inner sep=0.75pt]   [align=left] {sorgente aleatoria};
	\draw    (178,37) -- (491,37) -- (491,62) -- (178,62) -- cycle  ;
	\draw (181,41) node [anchor=north west][inner sep=0.75pt]   [align=left] { 0 1 0 1 1 1 0 0 1 1 1 1 1 1 0 0 1 0 1 0 1 1 1 1 ...};
\end{tikzpicture}

	\caption{Macchina di Turing probabilistica}
	\label{fig:mdtprob}
\end{figure}

Un algoritmo basato su questo modello si dice probabilistico, in quanto l'output dipende dall'input e dal seme casuale.
L'algoritmo possiede quindi una distribuzione associata, che mappa un input $x$ alla probabilità di ottenere l'output $y$:
\begin{equation*}
	P(Y = y \mid X = x)
\end{equation*}
Gli algoritmi probabilistici possono risolvere problemi di ottimizzazione quanto di decisione e si dividono in due famiglie:
\begin{description}
	\item[Monte Carlo] la correttezza dell'output, cioè la giusta decisione per i problemi di decisione e il calcolo della soluzione ottima per i problemi di ottimizzazione, è probabilistica; il tempo di esecuzione è deterministico;
	\item[Las Vegas] l'output è deterministicamente corretto; il tempo di esecuzione è probabilistico.
\end{description}
È possibile combinare algoritmi approssimanti con algoritmi probabilistici ottenendo algoritmi che approssimano entro una certa soglia dall'ottimo con una certa probabilità.



\section{\MinCut}
\MinCut è il problema di determinare il taglio minimo in un grafo non orientato.
In un grafo non orientato $G=(V,E)$, un taglio è dato dalla partizione di $V$ in due sottoinsiemi $X, X\compl$.
Il costo del taglio è dato dal numero di archi da vertici in $X$ a vertici in $X\compl$.
\MinCut è \NPO-completo.

\popt{\MinCut}
{Grafo non orientato $G=(V,E)$}
{Taglio $X\subseteq V$}
{Determinare il taglio minimo in $G$}
{$X\subset V\mid X\neq\emptyset$}
{$\MIN$}
{$\card{\set{e\in E\mid e\cap X\neq\emptyset\land e\setminus X\neq\emptyset}}$}

\noindent\MinCut è un problema \NPO-completo.


\subsection{L'algoritmo di Karger}
L'algoritmo di Karger è un algoritmo Monte Carlo per il taglio minimo e si basa sull'operazione di contrazione.
Sia $G=(V,E)$ un multigrafo\footnote{Nei multigrafi sono ammessi lati paralleli, cioè più lati che incidono sulla stessa coppia di vertici. Consideriamo quindi $E$ un insieme dotato di molteplicità. Formalmente, all'insieme è associata una funzione che indica la molteplicità di ogni elemento. Useremo comunque per semplicità le nozioni insiemistiche per insiemi standard.} di input. Si consideri il multigrafo $G'=(V',E')$, dove:
\begin{itemize}
	\item $V':=\set{\bar v\mid v\in V}$, dove $\bar v:=\set{v}$;
	\item $E':=\set{\set{\bar u,\bar v}\mid \set{u,v}\in E}$.
\end{itemize}
La contrazione $G'\downarrow e$ del lato $e=\set{\bar u,\bar v}$ (\cref{fig:contrazione}) consiste in una modifica di $V'$ e $E'$ come segue:
\begin{enumerate}
	\item viene aggiunto a $V'$ un nuovo supervertice $\bar z=\bar u\cup\bar v$;
	\item ogni arco $\set{\bar u,\bar w}\in E'$, con $w\neq v$, è sostituito da un arco $\set{\bar z,\bar w}$;
	\item ogni arco $\set{\bar w,\bar v}\in E'$, con $w\neq u$, è sostituito da un arco $\set{\bar w,\bar z}$;
	\item gli archi del tipo $\set{\bar u,\bar v}$ vengono rimossi da $E'$;
	\item i vertici $\bar u,\bar v$ vengono rimossi da $V'$.
\end{enumerate}

% TODO: questa immagine andrebbe arricchita un po': non fa capire come si fondono gli archi
\begin{figure}[ht]
	\centering
	\begin{tikzpicture}[vertex/.style={draw,inner sep=0pt,minimum size=5pt,fill, circle}]
	\node[vertex] at (0, 2)  (u1) {};
	\node[vertex] at (-1, 2)  (u2) {};
	\node[vertex] at (1, 2)  (u3) {};
	\node[vertex,label={0:{$\bar u$}}] at (0, 0)  (a) {};
	\node[vertex,label={0:{$\bar v$}}] at (0, 1)  (b) {};
	\node[vertex] at (0, -1)  (l1) {};
	\node[vertex] at (-1, -1)  (l2) {};
	\node[vertex] at (1, -1)  (l3) {};
	\draw (u1) -- (b);
	\draw (u2) -- (b);
	\draw (u3) -- (b);
	\draw (l1) -- (a);
	\draw (l2) -- (a);
	\draw (l3) -- (a);
	\node at (1.75,0.5) (to) {$\implies$};
	\draw (a) edge node [left] {$e$} (b);
	\node[vertex] at (3.5, 1.5)  (lu1) {};
	\node[vertex] at (2.5, 1.5)  (lu2) {};
	\node[vertex] at (4.5, 1.5)  (lu3) {};
	\node[vertex,label={0:{$\bar u\cup \bar v$}}] at (3.5, 0.5)  (la) {};
	\node[vertex] at (3.5, -0.5)  (ll1) {};
	\node[vertex] at (2.5, -0.5)  (ll2) {};
	\node[vertex] at (4.5, -0.5)  (ll3) {};
	\draw (lu1) -- (la);
	\draw (lu2) -- (la);
	\draw (lu3) -- (la);
	\draw (ll1) -- (la);
	\draw (ll2) -- (la);
	\draw (ll3) -- (la);
\end{tikzpicture}

	\caption{Contrazione $G'\downarrow e$.}
	\label{fig:contrazione}
\end{figure}

In seguito chiameremo semplicemente $G$ il grafo $G'$ derivante dall'input. L'algoritmo \ref{algo:Karger} di Karger effettua contrazioni casuali a partire dal grafo $G$, ottenendo multigrafi $G_0,G_1,\dots$.

\begin{algorithm}[ht]
	\caption{Algoritmo di Karger per \MinCut}
	\label{algo:Karger}
	\SetKwFunction{Unif}{uniformExtraction}
\SetKwFunction{Connected}{isConnected}
\SetKwFunction{FindConnected}{findConnectedComponent}
\SetKwFunction{Choose}{chooseAnElement}
\KwInput{grafo $G=(V,E)$}

\If{$\lnot$\Connected{$G$}}{
	\Return{\FindConnected{$G$}}\;
}
\While{$\card V > 2$}{
	\tcp{Estrai uniformemente a caso un lato e contrailo}
	$e = \Unif{$E$}$\;
	$G = G\downarrow e$\;
}
\tcp{Restituisci uno dei due vertici rimanenti}
\Return{\Choose{$V$}}\;

\end{algorithm}

Chiaramente, l'output dipende dalle scelte dei lati da contrarre, che sono casuali.
Sia $S\star$ il taglio minimo, $k\star$ il numero di lati tagliati da $S\star$ e $G_0,\dots,G_i,\dots$ la sequenza di grafi ottenuti per ogni contrazione operata dall'algoritmo, dove $G_i$ è ottenuto dopo $i$ passi. Si verificano i seguenti fatti:
\begin{oss}\label{oss:kargercontraction}
	\begin{equation*}
		\forall i \quad \card{V_i} = n-i \land \card{E_i} \leq m-i
	\end{equation*}
\end{oss}
\begin{oss}\label{oss:kargercuts}
	Per ogni $i$, ogni taglio in $G_i$ è un taglio in $G$ dello stesso costo.
\end{oss}
\begin{oss}\label{oss:kargermindeg}
	Il grado di ogni vertice di $G_i$ è maggiore o uguale a $k\star$.
\end{oss}

Proviamo ora due lemmi che ci serviranno per dimostrare che l'algoritmo di Karger ottiene la soluzione ottima con buona probabilità.
\begin{lemma}\label{lem:kargeredges}
	$\displaystyle \card{E_i} \geq \frac{(n-i) \cdot k\star}{2}$
\end{lemma}
\begin{proof}
	\begin{align*}
		\sum_{v\in V_i} d_i(v) & \geq \card{V_i} k\star      &  & \text{per \cref{oss:kargermindeg}}      \\
		\sum_{v\in V_i} d_i(v) & \geq (n-i) k\star           &  & \text{per \cref{oss:kargercontraction}} \\
		2\card{E_i}            & \geq (n-i) k\star                                                        \\
		\card{E_i}             & \geq \frac{(n-i) k\star}{2}                                              \\
	\end{align*}
\end{proof}

\begin{lemma}\label{lem:kargerprob_ei}
	Sia $\eve_i$ l'evento "al passo $i$-esimo (i.e. da $G_i$ a $G_{i+1}$) non viene contratto un lato del taglio minimo". Allora:
	\begin{equation*}
		\forall i \quad P(\eve_i \mid \eve_0, \dots, \eve_{i-1}) \geq \frac{n-i-2}{n-i}
	\end{equation*}
\end{lemma}
\begin{proof}
	\begin{align*}
		P(\eve_i \mid \eve_0,\dots,\eve_{i-1}) & = 1-P(\lnot \eve_i \mid \eve_0,\dots,\eve_{i -1})    \\
		                                       & = 1-\frac{k\star}{\card{\eve_i}}                     \\
		                                       & \geq 1-\frac{2k\star}{(n-i)k\star} = 1-\frac{2}{n-i} \\
		                                       & = \frac{n-i-2}{n-i}
	\end{align*}
\end{proof}

\begin{theorem}
	L'algoritmo di Karger emette l'ottimo con probabilità $p \geq \frac{1}{\binom n2}$.
\end{theorem}
\begin{proof}
	L'algoritmo emette l'ottimo se non sono stati contratti archi nel taglio ottimo, ossia se si verifica l'evento $\eve_0\land \eve_1\land\dots\land \eve_{n-3}$. La probabilità di tale evento è
	\begin{align*}
		P(\eve_0\land \eve_1\land\dots\land \eve_{n-3}) & =P(\eve_0)\cdot P(\eve_1\mid \eve_0)\cdots P(\eve_{n-3}\mid \eve_0,\dots,\eve_{n-4})                            \\
		                                                & \geq\frac{n-2}{n}\cdot\frac{n-3}{n-1}\cdots\frac{n-(n-3)-2}{n-(n-3)} \qquad \text{per \cref{lem:kargerprob_ei}} \\
		                                                & =\frac{n-2}{n}\cdot\frac{n-3}{n-1}\cdots\frac{1}{3}                                                             \\
		                                                & =\frac{(n-2)!}{n!/2}=\frac{2}{n(n-1)}=\frac{1}{\binom{n}{2}}
	\end{align*}
\end{proof}

\begin{corollario}
	Eseguendo l'algoritmo di Karger $\ceil{\binom{n}{2}\ln n}$ volte e scegliendo la migliore soluzione si ottiene l'ottimo con probabilità maggiore o uguale a $1-\frac{1}{n}$.
\end{corollario}
\begin{proof}
	Ad ogni esecuzione dell'algoritmo, la probabilità di non trovare l'ottimo è al più
	\begin{equation*}
		1-\frac{1}{\binom{n}{2}}\text.
	\end{equation*}
	Eseguendo l'algoritmo $\ceil{\binom{n}{2}\ln n}$ volte, la probabilità che nessun'esecuzione trovi l'ottimo è al più
	\begin{equation*}
		\left(1-\frac{1}{{\binom{n}{2}}}\right)^{\ceil{\binom{n}{2}\ln n}}\leq \left(1-\frac{1}{{\binom{n}{2}}}\right)^{{\binom{n}{2}}\ln n}\leq\left(\frac{1}{e}\right)^{\ln n}=\frac{1}{n}\text.
	\end{equation*}
\end{proof}



\section{\MinSetCover}
Si faccia riferimento al paragrafo \ref{sec:SetCover} per la formalizzazione di \MinSetCover.

Il problema può essere trasposto in un problema di programmazione lineare intera di variabili $x_1,\dots,x_m$ e vincoli
\begin{equation*}
	\begin{cases}
		x_j \leq 1                   & \quad \forall j\in m \\
		x_j \geq 0                   & \quad \forall j\in m \\
		\sum_{i:u\in S_i} x_i \geq 1 & \quad \forall u\in U
	\end{cases}
\end{equation*}
Questo problema, in quanto equivalente, ha soluzioni ottime di valore $v\star$ coincidenti con quelle per \MinSetCover.

L'algoritmo probabilistico \ref{algo:ProbSetCover} risolve polinomialmente all'ottimo il rilassamento continuo $\hat\pi$ di un'istanza così ottenuta, ottenendo una soluzione $\hat x_1,\dots,\hat x_m$ di valore $\hat v\leq v\star$.
Quindi usa i valori $\hat x_i$ come probabilità per scegliere l'insieme $S_i$.
Questa operazione viene ripetuta $\ceil{k+\ln n}$ volte, dove $n:=\card U$ e $k$ è un parametro.

\begin{algorithm}
	\caption{Algoritmo probabilistico per \MinSetCover.}
	\label{algo:ProbSetCover}
	\KwInput{$S_1,\dots,S_m,w_1,\dots,w_m\in\Q^+,k\in\N^+$}

Sia $\hat\pi$ il problema equivalente di \IntegerLinearProgramming rilassato poi a \LinearProgramming.\;
Siano $\hat x_1,\dots,\hat x_m$ le soluzioni ottime, trovate polinomialmente, per il problema $\hat\pi$.\;
$S\asn\emptyset$\;
\For{$\ceil{k+\ln n}$ volte}{
	\For{$i\asn1,\dots,m$}{
		$S\asn S\cup i$ con probabilità $\hat x_i$.\;
	}
}
\Return{$S$}\;

\end{algorithm}

Per l'analisi dell'algoritmo \ref{algo:ProbSetCover} si richiamano due risultati di probabilità:
\begin{theorem}[disuguaglianza di Markov]\label{thm:markov}
	Per ogni variabile aleatoria $X$ non negativa e per ogni $\beta>0$
	\begin{equation*}
		P[X\geq\beta] \leq \frac{\ev{X}}{\beta}
	\end{equation*}
\end{theorem}
\begin{theorem}[union bound o disuguaglianza di Boole]\label{thm:boole}
	\begin{equation*}
		P\left[\bigcup_i E_i\right] \leq \sum_i P[E_i]
	\end{equation*}
\end{theorem}

\begin{theorem}\label{thm:ammisetcover}
	La probabilità che l'algoritmo \ref{algo:ProbSetCover} produca una soluzione ammissibile è almeno $1-e^{-k}$.
\end{theorem}
\begin{proof}
	La probabilità di trovare una soluzione ammissibile è complementare a quella di non coprire almeno un elemento.
	Chiamiamo $\eve_u$ l'evento dato dalla non copertura dell'elemento $u$.
	La probabilità che l'output sia ammissibile è
	\begin{align*}
		1-P\left[\bigcup_{u\in U} \eve_u\right] & \geq 1-\sum_{u\in U} P[\eve_u]                                     &  & \quad \text{teorema \ref{thm:boole}}                                             \\
		                                        & = 1-\sum_{u\in U}\prod_{i:u\in S_i} P[i\notin I]                   &  & \quad \parbox{5cm}{$u$ non è coperto da nessuno degli insiemi che lo contengono} \\
		                                        & = 1-\sum_{u\in U}\prod_{i:u\in S_i} (1-\hat x_i)^{\ceil{k+\ln n}}                                                                                        \\
		                                        & \geq 1-\sum_{u\in U}\prod_{i:u\in S_i} e^{-\hat x_i\ceil{k+\ln n}} &  & \quad \text{$1-x\leq e^{-x}$}                                                    \\
		                                        & = 1-\sum_{u\in U} e^{-\ceil{k+\ln n}\sum_{i:u\in S_i} \hat x_i}                                                                                          \\
		                                        & \geq 1-\sum_{u\in U} e^{-\ceil{k+\ln n}}                           &  & \quad \text{per vincolo su $\hat x_i$}                                           \\
		                                        & = 1-ne^{-\ceil{k+\ln n}} \geq 1-ne^{-(k+\ln n)}                                                                                                          \\
		                                        & = 1-ne^{-k}e^{-\ln n} = 1-e^{-k} \text.
	\end{align*}
\end{proof}

\begin{theorem}\label{thm:ProbSetCoveralpha}
	Per ogni $\alpha>0$ e dato $k$, sia $v$ la soluzione prodotta dall'algoritmo \ref{algo:ProbSetCover}.
	\begin{equation*}
		P\left[\frac{v}{v\star}\geq\alpha\ceil{k+\ln n}\right] \leq \frac{1}{\alpha}
	\end{equation*}
\end{theorem}
\begin{proof}
	Calcoliamo il valore atteso di $v$ rispetto all'esecuzione dell'intero algoritmo:
	\begin{align*}
		\ev{v} & = \ev{\sum_{i=1}^m w_i [i\in I]}                                                           \\
		       & = \sum_{i=1}^m w_i \ev{i\in I}                                                             \\
		       & = \sum_{i=1}^m w_i P[i\in I]                                                               \\
		       & \leq \sum_{i=1}^m w_i \hat x_i\ceil{k+\ln n}     &  & \quad \text{teorema \ref{thm:boole}} \\
		       & = \ceil{k+\ln n}\hat v \leq \ceil{k+\ln n}v\star
	\end{align*}
	Applicando la disuguaglianza di Markov con $X:=\frac{v}{v\star}$ e $\beta:=\alpha\ceil{k+\ln n}$:
	\begin{equation*}
		P\left[\frac{v}{v\star}\geq\alpha\ceil{k+\ln n}\right] \leq \frac{\ev{\frac{v}{v\star}}}{\alpha\ceil{k+\ln n}} \leq \frac{v\star\ceil{k+\ln n}}{v\star}\cdot\frac{1}{\alpha\ceil{k+\ln n}}=\frac{1}{\alpha} \text.
	\end{equation*}
\end{proof}

\begin{corollario}
	Eseguendo l'algoritmo \ref{algo:ProbSetCover} con $k=3$ la probabilità di ottenere una soluzione ammissibile con fattore di approssimazione $\frac{v}{v\star}\leq 6+2\ln n$ è almeno $45\%$.
\end{corollario}
\begin{proof}~
	\begin{itemize}
		\item La probabilità che l'algoritmo emetta una soluzione non ammissibile è, in virtù del teorema \ref{thm:ammisetcover}, al più $1-e^{-k}$;
		\item La probabilità che l'algoritmo emetta una soluzione con fattore di approssimazione non inferiore a $6+2\ln n$ è, in virtù del teorema \ref{thm:ammisetcover}, al più $1/2$.
	\end{itemize}

	L'algoritmo emette una soluzione ammissibile e con fattore di approssimazione di al più $6+2\ln n$ complementarmente al produrre una soluzione non ammissibile oppure non buona.
	Pertanto, per union bound, la probabilità di ottenere una soluzione ammissibile e buona è almeno $1-(e^{-3}+\frac12)\geq 45\%$.
\end{proof}



\section{\MaxEkSat}
\MaxEkSat è una versione $k$-indicizzata di \MaxSat: date $t$ clausole di $k$ letterali ciascuna, l'obiettivo è massimizzare il numero di clausole soddisfatte.
\MaxEkSat è \NPO-completo se $k\geq3$.

\popt
{\MaxEkSat}
{Formula booleana nelle variabili $x_1,\dots,x_n$ in forma normale congiuntiva con clausole $c_1,\dots,c_t$ di $k$ letterali senza ripetizione di variabili}
{$\pi:\N\to 2$}
{Determinare il numero massimo di clausole che si possono rendere vere}
{Assegnamenti di valori di verità}
{$\MAX$}
{Numero di clausole rese vere}


\subsection{Algoritmo probabilistico}
Un algoritmo probabilistico banale per \MaxEkSat assegna un valore casuale a ogni variabile.
Siano $X_1,\dots,X_n$ variabili aleatorie bernoulliane con probabilità di successo $1/2$.
Sia $T$ la variabile aleatoria corrispondente al numero di clausole rese vere e $C_1,\dots,C_t$ variabili binarie uguali a $1$ se e solo se la clausola relativa è soddisfatta.

\begin{theorem}\label{thm:probassgn}
	Assegnando uniformemente a caso le variabili si ottiene:
	\begin{equation*}
		\ev{T} = \frac{2^k-1}{2^k} t
	\end{equation*}
\end{theorem}
\begin{proof}
	Per la legge del valore atteso totale:
	\begin{align*}
		\ev{T} & = \sum_{b_1\in2}\dots\sum_{b_n\in2} \ev{T\mid X_1=b_1,\dots,X_n=b_n}P[X_1=b_1,\dots,X_n=b_n]                  \\
		       & = \sum_{b_1\in2}\dots\sum_{b_n\in2} \ev{T\mid X_1=b_1,\dots,X_n=b_n}P[X_1=b_1]\cdots P[X_n=b_n]               \\
		       & = \frac{1}{2^n}\sum_{b_1\in2}\dots\sum_{b_n\in2} \ev{T\mid X_1=b_1,\dots,X_n=b_n}                             \\
		       & = \frac{1}{2^n}\sum_{b_1\in2}\dots\sum_{b_n\in2} \ev{C_1+\dots+C_t\mid X_1=b_1,\dots,X_n=b_n}                 \\
		       & = \frac{1}{2^n}\sum_{j=1}^t \left(\sum_{b_1\in2}\dots\sum_{b_n\in2} \ev{C_j\mid X_1=b_1,\dots,X_n=b_n}\right) \\
	\end{align*}
	Il valore atteso di ciascuna variabile $C_j$ dati gli assegnamenti di tutte le variabili è deterministico.
	La loro somma è uguale al numero di assegnamenti che verifica la clausola, cioè il numero di assegnamenti possibili meno quelli che la falsificano, ossia $2^n-2^{n-k}$. Quindi:
	\begin{align*}
		 & = \frac{1}{2^n}\sum_{j=1}^t (2^n-2^{n-k})=\frac{2^n-2^{n-k}}{2^n}t \\
		 & = \frac{2^k-1}{2^k}t
	\end{align*}
\end{proof}

\subsection{Algoritmo derandomizzato}
Il seguente lemma estende il teorema \ref{thm:probassgn} asserendo che il lower bound sul valore atteso può essere preservato se si fissa un determinato assegnamento delle prime $j$ variabili:
\begin{theorem}\label{thm:maxsatderandomexv}
	Per ogni $j=0,\dots,n$ esistono $b_1,\dots,b_j\in2$ tali che
	\begin{equation*}
		\ev{T\mid X_1=b_1,\dots,X_j=b_j} \geq \frac{2^k-1}{2^k}t
	\end{equation*}
\end{theorem}
\begin{proof}
	Per induzione su $j$:
	\begin{itemize}
		\item per $j=0$ la tesi coincide con il teorema \ref{thm:probassgn};
		\item per ipotesi induttiva, per $j$ vale:
		      \begin{equation*}
			      \ev{T\mid X_1=b_1,\dots,X_j=b_j}\geq\frac{2^k-1}{2^k}t \text.
		      \end{equation*}
		      Se per assurdo la tesi fosse falsa in $j+1$, allora varrebbe, sia per $b_{j+1}=0$ sia per $b_{j+1}=1$:
		      \begin{equation}\label{eq:ekpa}
			      \ev{T\mid X_1=b_1,\dots,X_j=b_j,X_{j+1}=b_{j+1}}<\frac{2^k-1}{2^k}t \text.
		      \end{equation}
		      Ma applicando la legge del valore atteso totale all'ipotesi induttiva otteniamo
		      \begin{gather*}
			      \ev{T\mid X_1=b_1,\dots,X_j=b_j} \geq \frac{2^k-1}{2^k}t \\
			      \ev{T\mid X_1=b_1,\dots,X_{j+1}=0}\frac12 + \ev{T\mid X_1=b_1,\dots,X_{j+1}=1}\frac12 \geq \frac{2^k-1}{2^k}t
		      \end{gather*}
		      Che è impossibile se vale (\ref{eq:ekpa}).
	\end{itemize}
\end{proof}

\begin{algorithm}[ht]
	\caption{Algoritmo derandomizzato per \MaxEkSat.}
	\label{algo:DerandomMaxEkSat}
	\SetKw{Continue}{continue}

$D \asn \emptyset$
\For{$i\asn1,\dots,n$}{
	$\Delta_0 \asn 0$\;
	$\Delta_1 \asn 0$\;
	$\Delta D_0 \asn \emptyset$\;
	$\Delta D_1 \asn \emptyset$\;
	\For{$j\asn1,\dots,t$}{
		\If{$j\in D$}{
			\Continue\;
		}
		\If {$x_i\notin c_j\land\neg x_i\notin c_j$}{
			\Continue\;
		}
		$h \asn \card{\set{k\mid k>i\land (x_k\in c_j \lor \neg x_k\in c_j)}}$\;
		\If{$\neg x_i\notin c_j$}{
			$\Delta_0 \asn \Delta_0 - \frac{1}{2^h}$\;
			$\Delta_1 \asn \Delta_1 + \frac{1}{2^h}$\;
			$\Delta D_1 \asn \Delta D_1 \cup \{j\}$\;
		}\Else{
			$\Delta_0 \asn \Delta_0 + \frac{1}{2^h}$\;
			$\Delta_1 \asn \Delta_1 - \frac{1}{2^h}$\;
			$\Delta D_0 \asn \Delta D_0 \cup \{j\}$\;
		}
	}
	$u \asn \arg\max_{0,1}(\Delta_0,\Delta_1)$\;
	$X_i \asn u$\;
	$D \asn D \cup \Delta D_u$\;
}

\end{algorithm}

\begin{theorem}
	L'algoritmo \ref{algo:DerandomMaxEkSat} fornisce una $\frac{2^k}{2^k -1}$-approssimazione per \MaxEkSat.
\end{theorem}

% TODO: il teorema va dimostrato. Vanno fatte ulteriori analisi sull'argomento prima di trascrivere quanto detto a lezione, in particolare la formalizzazione va rivista. Non si può parlare di valori attesi in un algoritmo deterministico, e tantomeno si può dare per scontato che gli assegnamenti successivi siano equiprobabili



\section{Il teorema PCP}
Il teorema PCP, \flang{Probabilistically Checkable Proofs}, è uno dei più importanti teoremi della teoria della complessità dal teorema di Cook-Levin.
Il teorema mette in relazione classi di problemi di decisione riconosciuti da macchine deterministiche con classi riconosciute da macchine con aspetti randomici.


% TODO: decidere se spostare insieme alla definizione di NP nel primo capitolo e includere qui solo i probabilistic checkers
\subsection{Macchine di Turing oracolari}
Le macchine di Turing oracolari sono estensioni delle macchine di Turing che hanno accesso a un oracolo, una stringa $w\in 2\star$ che ha la semantica di certificato.
Una macchina può effettuare una query del bit $w_i$ dell'oracolo inserendo la sua posizione $i$ in un apposito \emph{nastro di query}.
Può poi cambiare stato in funzione del valore del bit restituito.

\begin{figure}[ht]
	\centering
	\begin{tikzpicture}[-arr/.style={-{Latex[scale=1.5]}}]
	\node[draw, rounded corners=5pt,minimum width=2.5cm,minimum height=1.5cm] (tm) {Macchina di Turing};
	\draw[-arr] (-4,0) -- (tm.west)	node[midway,above] {input};
	\draw[-arr] (tm.east) -- (4,0)	node[midway,above] {output};
	\matrix (query) [tape,below=of tm] { 1 & 0 & 1 & 0 & 0 & 1 & 1 & \dots\\};
	\draw[-arr,shorten >=-3.5pt] (tm.south) -- (query.north);
	\node[below=0 of query] {nastro di query};
	\node[draw,rounded corners=3pt,minimum width=1.5cm,minimum height=1cm,right=0.7 of query] (oracolo) {$w$};
	\node[below=0 of oracolo] {oracolo};
	\draw[-arr,shorten <=-3.5pt] (query.east) -- (oracolo.west);
\end{tikzpicture}

	\caption{Macchina di Turing oracolare}
	\label{fig:mdtoracle}
\end{figure}

\begin{defin}
	Un linguaggio $L\subseteq 2^*$ appartiene a $\NP$ se e solo se esiste una macchina di Turing oracolare $V$ tale che:
	\begin{itemize}
		\item $V(x,w)$ termina in un numero di passi polinomiale in $\len x$;
		\item $\forall x\in 2\star \quad \exists w\in 2\star \mid V(x,w)$ accetta $\iff x\in L$.
	\end{itemize}
\end{defin}


\subsection{Verificatori probabilistici}
I verificatori probabilistici (\flang{probabilistic checkers}) estendono ulteriormente le macchine di Turing oracolari potendo accedere a una stringa $r$ di bit casuali giacenti su un apposito nastro.

\begin{figure}[ht]
	\centering
	\begin{tikzpicture}[-arr/.style={-{Latex[scale=1.5]}}]
	\node[draw, rounded corners=5pt,minimum width=2.5cm,minimum height=1.5cm] (tm) {Macchina di Turing};
	\draw[-arr] (-4,0) -- (tm.west)	node[midway,above] {input};
	\draw[-arr] (tm.east) -- (4,0)	node[midway,above] {output};
	\matrix (rand) [tape,above=1 of tm] { 0 & 1 & 0 & 1 & 1 & 1 & 0 & 0 & 1 & 1 & 1 & 1 & 1 & 1 & 0 & 0 & \dots\\};
	\draw[-arr,shorten <=-3.5pt] (rand.south) -- (tm.north);
	\node[above=0 of rand] {sorgente aleatoria};
	\matrix (query) [tape,below=of tm] { 1 & 0 & 1 & 0 & 0 & 1 & 1 & \dots\\};
	\draw[-arr,shorten >=-3.5pt] (tm.south) -- (query.north);
	\node[below=0 of query] {nastro di query};
	\node[draw,rounded corners=3pt,minimum width=1.5cm,minimum height=1cm,right=0.7 of query] (oracolo) {$w$};
	\node[below=0 of oracolo] {oracolo};
	\draw[-arr,shorten <=-3.5pt] (query.east) -- (oracolo.west);
\end{tikzpicture}

	\caption{Verificatore probabilistico}
	\label{fig:probcheck}
\end{figure}

% TODO: approfondire con completeness e soundness? Eventualmente in appendice
I linguaggi binari possono essere classificati in base alle caratteristiche dei verificatori probabilistici che li riconoscono:
\begin{defin}
	Siano $r,q:\N\to\N$ e $L$ un linguaggio.
	$L\in\PCP[r,q]$ se e solo se esiste un verificatore probabilistico $V$ tale che
	\begin{itemize}
		\item $\forall x,R,w$, $V(x,R,w)$ termina in un numero di passi polinomiale in $\len x$;
		\item $\forall x,R,w$, $V(x,R,w)$ effettua al massimo $q(\len x)$ query all'oracolo;
		\item $\forall x,R,w$, $V(x,R,w)$ fa uso di al più $r(\len x)$ bit casuali;
		\item se $x\in L$ allora esiste $w\in 2\star$ tale che $V$ accetta $x$ con probabilità $1$, ossia per ogni stringa casuale $R\in 2^{\leq r(\len x)}$;
		\item se $x\notin L$ allora per qualunque $w$, $V$ rifiuta con probabilità maggiore di $\frac12$, ossia per più di metà delle stringhe casuali $R\in 2^{\leq r(\len x)}$.
	\end{itemize}
	Con abuso di notazione, se $R,Q\subseteq\set{f:\N\to\N}$ allora $\PCP[R,Q]=\bigcup_{r\in R,q\in Q} \PCP[r,q]$.
\end{defin}
\noindent Ad esempio, $\PCP[0,0]=\P$ e, se $\mathrm{Poly}$ è l'insieme delle funzioni al più polinomiali, $\PCP[0,\mathrm{Poly}]=\NP$.

Il teorema PCP limita il nondeterminismo dei verificatori per linguaggi in \NP sostituendolo con un grado di casualità:
\begin{theorem}[PCP \cite{Arora:98:PCP}]\label{thm:pcp}
	$\NP=\PCP[O(\log(n)),O(1)]$.
\end{theorem}


\subsubsection{Verificatori probabilistici normalizzati}
Un verificatore probabilistico è normalizzato se e solo se
\begin{itemize}
	\item estrae tutti i bit casuali all'inizio della computazione;
	\item estratti i bit random effettua tutte le query all'oracolo;
	\item effettuate le query all'oracolo prosegue con il resto della computazione.
\end{itemize}

Si può dimostrare che la classe $\NP=\PCP[O(\log n),O(1)]$ è caratterizzabile unicamente da verificatori probabilistici in forma normale.



\section{Inapprossimabilità}
Grazie al teorema PCP è possibile dimostrare che l'approssimabilità polinomiale di alcuni problemi è limitata.


\subsection{\MaxSat}
\begin{theorem}
	Se $\P\neq\NP$, esiste $\epsilon>0$ tale che \MaxSat non è $(1+\epsilon)$-approssimabile in tempo polinomiale.
\end{theorem}
\begin{proof}
	\newcommand{\Rand}{\mathcal R}
	Sia $L$ \NP-completo. Allora per il teorema PCP esistono $\bar q\in O(1)$ e $r\in O(\log n)$ tali che $L\in\PCP[r,\bar q]$.
	Sia $V$ il verificatore probabilistico caratteristico di $L$.
	% TODO: includere approfondimento riguardo al caso in cui il numero di query cambi in base a input e stringa randomica. Quale epsilon si sceglie in quel caso?
	Per semplicità, si assuma che il verificatore effettui esattamente $q\in\N$ query per ogni input.
	Il verificatore, ricevuto in input $z\in 2\star$, fa uso di una stringa casuale $R\in\Rand\subseteq 2^{r(\len z)}$ e di un oracolo $w$ su cui effettua $q$ query.

	Siano $x^{z,R}_{i_j}$ variabili booleane, dove $i_j$ è la posizione su cui interrogare l'oracolo per la $j$-esima query, con $j\in q$.
	Fissati $z$ ed $R$, il valore di queste variabili è in funzione di $w$, ponendo $x^{z,R}_i:=1\iff w_i=1$.

	Poiché la computazione di $V$ è completamente determinata da $z$, $R$ e $w$, è possibile costruire\footnote{Con una costruzione analoga a quella del teorema di Cook-Levin ciò è possibile in tempo polinomiale.} con le variabili $x^z,R$ una formula booleana $\phi^{z,R}$ che, una volta scelto $w$, abbia valore positivo se e solo se $z$ viene accettato.
	Senza perdita di generalità assumiamo che la formula sia in forma normale congiuntiva e, poiché le variabili sono al più $q$, $\phi^{z,R}$ ha al più $2^q$ clausole con al più $q$ letterali l'una. La dimensione della formula è quindi costante nella lunghezza dell'input.

	Costruiamo una formula $\Phi_z$ che astrae $\phi^{z,R}$ rispetto alla stringa casuale come segue
	\begin{equation*}
		\Phi_z := \bigwedge_{R\in\Rand} \phi^{z,R}
	\end{equation*}
	$\Phi_z$ è in forma normale congiuntiva e, poiché $\Rand$ ha cardinalità lineare nella lunghezza dell'input, si può costruire in tempo e spazio polinomiali.

	Fissato $z$, per definizione di verificatore probabilistico uno e uno solo dei seguenti casi si verifica:
	\begin{itemize}
		\item se $z\in L$, esiste un oracolo $w$ per cui $V$ accetta a prescindere dalla stringa casuale $R$. La formula $\Phi_z$ è quindi interamente soddisfacibile usando come assegnamento i valori dei bit di $w$. Il numero di clausole soddisfatte è $\card\Rand 2^q$;
		\item se $z\notin L$, scelto un qualunque oracolo, $V$ accetta con probabilità inferiore a $1/2$, ossia per meno della metà delle stringhe casuali. È possibile soddisfare meno della metà delle sottoformule $\phi^{z,R}$ della formula $\Phi_z$. In termini di clausole, sono soddisfatte $2^q$ clausole per ognuna delle al più $\frac{\card\Rand}{2}$ sottoformule soddisfatte e al più $2^q-1$ per le altre almeno $\frac{\card\Rand}{2}$, per un totale di $\frac{\card\Rand}{2}2^q+\frac{\card\Rand}{2}(2^q-1)$ clausole.
	\end{itemize}

	Sia $\epsilon:=\frac{1}{2^{q+1}}$.
	Per assurdo, si ammetta che \MaxSat ammetta un algoritmo $(1+\epsilon)$-approssimante e sia $t$ la soluzione prodotta da questo algoritmo sull'istanza che consiste nella formula $\Phi_z$.
	Sia $t\star$ la soluzione ottima su tale istanza.
	\begin{itemize}
		\item se $z\in L$ allora
		      \begin{equation*}
			      t \geq \frac{t\star}{1+\epsilon} = \frac{\card\Rand 2^q}{1+\frac{1}{2^{q+1}}} =: A
		      \end{equation*}
		\item se $z\notin L$ allora
		      \begin{equation*}
			      t \leq t\star < \frac{\card\Rand}{2}2^q+\frac{\card\Rand}{2}(2^q-1) = 2^q\card\Rand-\frac{\card\Rand}{2} =: B
		      \end{equation*}
	\end{itemize}
	Studiando la differenza $A-B$:
	\begin{align*}
		A-B & = \frac{2^q\card\Rand}{1+\frac{1}{2^{q+1}}} - 2^q\card\Rand+\frac{\card\Rand}{2}                           \\
		    & = \card\Rand \frac{2^{q+1} - 2^{q+1}(1+\frac{1}{2^{q+1}}) + (1+\frac{1}{2^{q+1}})}{2(1+\frac{1}{2^{q+1}})} \\
		    & = \card\Rand \frac{2^{q+1} - 2^{q+1}-1 + 1+\frac{1}{2^{q+1}}}{2(1+\frac{1}{2^{q+1}})}                      \\
		    & = \card\Rand \frac{\frac{1}{2^{q+1}}}{2(1 + \frac{1}{2^{q+1}})}                                            \\
		    & > 0 \text.
	\end{align*}

	Poiché $A>B$, è possibile costruire un algoritmo che decida se $z\in L$ in tempo polinomiale come segue:
	\begin{enumerate}
		\item costruire in tempo polinomiale la formula $\Phi_z$ che simula il certificatore $V$;
		\item eseguire l'algoritmo polinomiale che $1+\epsilon$-approssima \MaxSat su $\Phi_z$, ottenendo una soluzione di valore $t$;
		\item se $t\geq A$ allora $z\in L$; se $t\leq B$ allora $z\notin L$.
		      Poiché $L$ è \NP-completo, ciò è impossibile se $\P\neq\NP$.
	\end{enumerate}
\end{proof}


\subsection{\IndependentSet}
Il problema dell'insieme indipendente consiste nel determinare l'insieme indipendente di cardinalità massima in un dato grafo.
Dato un grafo $G=(V,E)$, un insieme indipendente è un insieme $X\subseteq V$ che non contiene vertici a due a due adiacenti, ossia tale che $\forall i,j\in X:\set{i,j}\notin E$.

\popt{\IndependentSet}
{Grafo $G=(V,E)$}
{$X\subseteq V$}
{Determinare l'insieme indipendente di cardinalità massima in $G$}
{$X$ è un insieme indipendente}
{$\MAX$}
{$\card X$}

Grazie al teorema PCP si può limitare l'approssimabilità di \IndependentSet a non meno di $2$:
\begin{theorem}\label{thm:ind_set_inapprox}
	\begin{equation*}
		\P\neq\NP \impl \IndependentSet\notin 2-\APX \text.
	\end{equation*}
\end{theorem}
\begin{proof}
	\newcommand{\Rand}{\mathcal R}
	Sia $L$ \NP-completo. Allora per il teorema PCP esistono $q\in O(1)$ e $r\in O(\log n)$ tali che $L\in\PCP[r,q]$.
	Sia $V$ il verificatore probabilistico caratteristico di $L$.

	Fissato un input $z$ accettato da $V$, sia $\Rand\subseteq 2^{r(\len z)}$ l'insieme delle stringhe casuali ottenibili su input $z$.
	Si dice configurazione di $V$ una tupla $C:=\tuple{R,I,f}$ composta da:
	\begin{itemize}
		\item una stringa $R\in\Rand$ di bit random;
		\item un insieme $I\subseteq\N$ di indici di posizioni dell'oracolo di cui effettuare query, di cardinalità al più $\card I=q(\len z)$;
		\item una funzione $f:I\to 2$ che restituisce il valore di ogni query all'oracolo.
	\end{itemize}
	Due configurazioni $C_1$ e $C_2$ si dicono compatibili se e solo se si verificano le due seguenti condizioni
	\begin{itemize}
		\item $R_{C_1}\neq R_{C_2}$;
		\item $\forall i\in I_{C_1}\cap I_{C_2},\quad f_{C_1}(i)=f_{C_2}(i)$.
	\end{itemize}

	Fissato $z$, sia $G$ un grafo non orientato in cui i vertici sono le configurazioni di $V$ e due vertici sono adiacenti se e solo se le rispettive configurazioni sono incompatibili (cioè non compatibili).
	Il grafo si può costruire in tempo polinomiale in $\len z$ essendo il numero dei suoi vertici nell'ordine di $2^{r(\len z)}\cdot 2^q$.

	\begin{fact}\label{fac:indip1}
		Se $z\in L$, $G$ ha un insieme indipendente di cardinalità maggiore o uguale a $\card\Rand$.
	\end{fact}
	\begin{proof}
		Se $z\in L$, per definizione di verificatore probabilistico esiste un certificato $w$ tale per cui $V$ accetta su ogni stringa casuale.
		Tutte le configurazioni $C$ di $V$ con query compatibili con $w$, ossia tali che $f_C(i)=w_i\forall i\in I_C$ e che spaziano su tutti i valori $R_C\in\Rand$, sono tra loro compatibili, quindi i rispettivi vertici danno luogo a un insieme indipendente.
		Questi sono nel numero delle stringhe casuali, cioè $\card\Rand$.
	\end{proof}

	\begin{fact}\label{fac:indip2}
		Se $z\notin L$, ogni insieme indipendente di $G$ ha cardinalità minore di $\frac{\card\Rand}{2}$.
	\end{fact}
	\begin{proof}
		Per assurdo, si ammetta l'esistenza di un insieme indipendente $X$ con cardinalità di almeno $\frac{\card\Rand}{2}$.
		I vertici in tale insieme corrispondono a configurazioni con diverse sequenze di bit randomici e con query non discordanti.
		Si può quindi costruire una stringa $w$ compatibile con tutte le query di ogni configurazione, cioè tale per cui per ogni configurazione $C\in X$, $f_C(i)=w_i\forall i\in I_C$.
		Poiché la stringa è un oracolo accettante per più di $\frac{\card\Rand}{2}$ stringhe casuali, la macchina accetta con probabilità di almeno $\frac12$, il che è impossibile se $z\notin L$.
	\end{proof}

	Per assurdo, sia $A$ un algoritmo $2$-approssimante per \IndependentSet.
	\begin{itemize}
		\item Se $z\in L$ allora in virtù del fatto \ref{fac:indip1} l'algoritmo emette un output dal valore di almeno $\frac{\card R}{2}$ (nel peggiore dei casi);
		\item se $z\notin L$ allora in virtù del fatto \ref{fac:indip2} l'algoritmo emette un output di valore minore di $\frac{\card R}{2}$ (nel migliore dei casi).
	\end{itemize}
	L'algoritmo potrebbe quindi decidere in tempo polinomiale il linguaggio $L$, il che è impossibile se $\P\neq\NP$.
\end{proof}
